\section{Flow}

Where there are graphs, we tend to see flow networks quite a lot.

\begin{fdef}[Flow Network]
	A directed graph $G=(V,E)$ together with a \textit{capacity} function $c:E\to\Zp$, a designated \textit{source} $s$, and a designated \textit{sink} $t$ is known as a \textit{flow network}.
\end{fdef}

We shall assume that there are no edges entering $s$ and no edges leaving $t$.\\
We can think of $s$ as a source of water, with the intermediate edges being pipes that have some limit to how much they can carry and the intermediate nodes being junctions that transmit water, and $t$ as a reservoir.

The main question we wish to answer is:
\begin{quote}
	What is the maximum rate that can be sent from the source to the sink without violating any capacity constraints?
\end{quote}

This is modelled more concretely using the following definition.

\begin{fdef}[Flow]
	A \textit{flow} in a flow network is a function $f:E\to\R^{\geq 0}$ that satisfies
	\begin{itemize}
		\item Capacity constraints: For each $e\in E$, $f(e)\leq c(e)$.
		\item Flow conservation: For each $v\in V\setminus\{s,t\}$,
		\[ \sum_{\vv{uv}\in E} f(\vv{uv}) = \sum_{\vv{vw}\in E} f(\vv{vw}). \]
	\end{itemize}
	The \textit{value of the flow} $f$, denoted $|f|$, is given by
	\[ |f| = \sum_{\vv{sv}\in E} f(\vv{sv}). \]
\end{fdef}

The flow conservation rule written above is also known as \textit{Kirchhoff's law}. For the sake of brevity, we denote
\begin{align*}
	f^{\ot}(v) &= \sum_{\vv{uv}\in E} f(\vv{uv}) \text{ and} \\
	f^{\to}(v) &= \sum_{\vv{vw}\in E} f(\vv{vw}).
\end{align*}
We then have $|f|=f^{\to}(s)$.\\
For any $U\subseteq V$, we use $f^{\ot}(U)$ and $f^{\to}(U)$ to denote the flow from $V\setminus U$ to $U$ and $U$ to $V\setminus U$ respectively. That is,
\[ f^\to(U) = \sum_{\substack{u\in U, v\in V\setminus U \\ \vv{uv}\in E}} f(\vv{uv}). \]

\begin{lemma}
	For any flow $f$ on a flow network with source $s$ and sink $t$,
	\[ |f| = \sum_{\vv{sv}\in E}f(\vv{sv}) = f^{\to}(s) = \sum_{\vv{vt}\in E}f(\vv{vt}) = f^{\ot}(t). \]
\end{lemma}
\begin{proof}
	We have
	\begin{align*}
		|f| &= f^{\to}(s) + \sum_{v\in V\setminus\{s,t\}} (f^\to(v) - f^\ot(v)) \\
			&= \sum_{v\in V\setminus \{t\}} (f^\to(v) - f^\ot(v)) & (f^\ot(s)=0) \\
			&= f^\to(V\setminus\{t\}) - f^\ot(V\setminus\{t\}) \\
			&= f^\ot(t) - f^\to(t) = f^\ot(t). & (\text{there are no outgoing edges from the sink})
	\end{align*}
\end{proof}

In the \textit{maximum flow problem} (shortened as max-flow), we are given a flow network $G=(V,E)$ along with a capacity function $c:E\to\N$. The output should be the maximum valued flow that can be transferred in the network.\footnote{This is well-defined since the set of flow values is bounded above, so has a supremum. Further, this bound is attained, as can be shown by considering a sequence of flows that converge (in flow value) to the maximum flow. We can then show that the flow in each edge must converge as well for some subsequence (using the Bolzano-Weierstrass Theorem), thus implying the required since the capacity constraint has a weak inequality and not a strong one.}

\begin{fdef}
	Given a directed graph $G=(V,E)$ with source $s$, sink $t$, and a capacity function $c:E\to\N$, an \textit{$(s,t)$-cut} or just \textit{cut} is given by $S,T\subseteq V$ such that
	\begin{itemize}
		\item $s\in S$, $t\in T$,
		\item $S\cup T=V$, and $S\cap T=\emptyset$.
	\end{itemize}
	Given an $(s,t)$-cut $(S,T)$, we define its \textit{capacity}
	\[ \capp(S,T) = \sum_{\substack{u\in S, v\in T \\ \vv{uv}\in E}} c(\vv{uv}). \]
\end{fdef}

That is, the capacity of the cut is essentially the capacity of the edges across the cut.\\
In the \textit{minimum cut problem} (shortened as min-cut), we are given a flow network $G=(V,E)$ along with a capacity function $c:E\to\N$. The output is the cut that has minimum capacity.

The max-flow and min-cut problems are in fact very closely related. When a minimization problem and maximization problem are related, we usually refer to it as a min-max relationship.

\begin{lemma}
	Let $f$ be any flow in a flow network $G$. Let $(S,T)$ be any $(s,t)$-cut in $G$. Then $|f|\leq\capp(S,T)$. In particular, the value of the maximum flow is at most the capacity of the minimum cut.
\end{lemma}
\begin{proof}
	We have
	\begin{align*}
		|f| &= f^\to(s) \\
			&= f^\to(S) - f^\ot(S) \\
			&\leq f^\to(S) \\
			&= \sum_{\substack{u\in S \\ v\in T}} f(\vv{uv}) & (\text{here, }f(\vv{uv})=0\text{ if }\vv{uv}\not\in E) \\
			&\leq \sum_{\substack{u\in S \\ v\in T}} c(\vv{uv})  = \capp(S,T) & (\text{here, }f(\vv{uv})=0\text{ if }\vv{uv}\not\in E). \\
	\end{align*}
\end{proof}