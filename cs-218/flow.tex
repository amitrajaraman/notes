\section{Flow}

\subsection{Introduction}

Where there are graphs, we tend to see flow networks quite a lot.

\begin{fdef}[Flow Network]
	A directed graph $G=(V,E)$ together with a \textit{capacity} function $c:E\to\Zp$, a designated \textit{source} $s$, and a designated \textit{sink} $t$ is known as a \textit{flow network}.
\end{fdef}

We shall assume that there are no edges entering $s$ and no edges leaving $t$.\\
We can think of $s$ as a source of water, with the intermediate edges being pipes that have some limit to how much they can carry and the intermediate nodes being junctions that transmit water, and $t$ as a reservoir.

The main question we wish to answer is:
\begin{quote}
	What is the maximum rate that can be sent from the source to the sink without violating any capacity constraints?
\end{quote}

This is modelled more concretely using the following definition.

\begin{fdef}[Flow]
	A \textit{flow} in a flow network is a function $f:E\to\R^{\geq 0}$ that satisfies
	\begin{itemize}
		\item Capacity constraints: For each $e\in E$, $f(e)\leq c(e)$.
		\item Flow conservation: For each $v\in V\setminus\{s,t\}$,
		\[ \sum_{\vv{uv}\in E} f(\vv{uv}) = \sum_{\vv{vw}\in E} f(\vv{vw}). \]
	\end{itemize}
	The \textit{value of the flow} $f$, denoted $|f|$, is given by
	\[ |f| = \sum_{\vv{sv}\in E} f(\vv{sv}). \]
\end{fdef}

The flow conservation rule written above is also known as \textit{Kirchhoff's law}. For the sake of brevity, we denote
\begin{align*}
	f^{\ot}(v) &= \sum_{\vv{uv}\in E} f(\vv{uv}) \text{ and} \\
	f^{\to}(v) &= \sum_{\vv{vw}\in E} f(\vv{vw}).
\end{align*}
We then have $|f|=f^{\to}(s)$.\\
For any $U\subseteq V$, we use $f^{\ot}(U)$ and $f^{\to}(U)$ to denote the flow from $V\setminus U$ to $U$ and $U$ to $V\setminus U$ respectively. That is,
\[ f^\to(U) = \sum_{\substack{u\in U, v\in V\setminus U \\ \vv{uv}\in E}} f(\vv{uv}). \]

\begin{lemma}
	For any flow $f$ on a flow network with source $s$ and sink $t$,
	\[ |f| = \sum_{\vv{sv}\in E}f(\vv{sv}) = f^{\to}(s) = \sum_{\vv{vt}\in E}f(\vv{vt}) = f^{\ot}(t). \]
\end{lemma}
\begin{proof}
	We have
	\begin{align*}
		|f| &= f^{\to}(s) + \sum_{v\in V\setminus\{s,t\}} (f^\to(v) - f^\ot(v)) \\
			&= \sum_{v\in V\setminus \{t\}} (f^\to(v) - f^\ot(v)) & (f^\ot(s)=0) \\
			&= f^\to(V\setminus\{t\}) - f^\ot(V\setminus\{t\}) \\
			&= f^\ot(t) - f^\to(t) = f^\ot(t). & (\text{there are no outgoing edges from the sink})
	\end{align*}
\end{proof}

\subsection{Max-Flow}

In the \textit{maximum flow problem} (shortened as max-flow), we are given a flow network $G=(V,E)$ along with a capacity function $c:E\to\N$. The output should be the maximum valued flow that can be transferred in the network.\footnote{This is well-defined since the set of flow values is bounded above by the sum of capacities, so has a supremum. Further, this bound is attained, as can be shown by considering a sequence of flows that converge (in value) to the maximum flow value. We can then show that the flow in each edge must converge as well for some subsequence (using the Bolzano-Weierstrass Theorem), thus implying the required since the capacity constraint has a weak inequality and not a strong one.}

\begin{fdef}
	Given a directed graph $G=(V,E)$ with source $s$, sink $t$, and a capacity function $c:E\to\N$, an \textit{$(s,t)$-cut} or just \textit{cut} is given by $S,T\subseteq V$ such that
	\begin{itemize}
		\item $s\in S$, $t\in T$,
		\item $S\cup T=V$, and $S\cap T=\emptyset$.
	\end{itemize}
	Given an $(s,t)$-cut $(S,T)$, we define its \textit{capacity} by
	\[ \capp(S,T) = \sum_{\substack{u\in S, v\in T \\ \vv{uv}\in E}} c(\vv{uv}). \]
\end{fdef}

That is, the capacity of the cut is essentially the capacity of the edges across the cut.\\
In the \textit{minimum cut problem} (shortened as min-cut), we are given a flow network $G=(V,E)$ along with a capacity function $c:E\to\N$. The output is the cut that has minimum capacity.

The max-flow and min-cut problems are in fact very closely related. When a minimization problem and maximization problem are related, we usually refer to it as a min-max relationship.

For a flow $f$ and some edge $e$, we say that $f$ \textit{saturates} $e$ if $f(e)=c(e)$ and \textit{avoids} $e$ if $f(e)=0$.

\begin{lemma}
	\label{ff-algo lemma 1}
	Let $f$ be any flow in a flow network $G$. Let $(S,T)$ be any $(s,t)$-cut in $G$. Then $|f|\leq\capp(S,T)$. In particular, the value of the maximum flow is at most the capacity of the minimum cut. Moreover, $|f|=\capp(S,T)$ if and only if $f$ saturates every edge from $S$ to $T$ and avoids every edge from $T$ to $S$.
\end{lemma}
\begin{proof}
	We have
	\begin{align*}
		|f| &= f^\to(s) \\
			&= f^\to(S) - f^\ot(S) \\
			&\leq f^\to(S) \\
			&= \sum_{\substack{u\in S \\ v\in T}} f(\vv{uv}) & (\text{here, }f(\vv{uv})=0\text{ if }\vv{uv}\not\in E) \\
			&\leq \sum_{\substack{u\in S \\ v\in T}} c(\vv{uv})  = \capp(S,T) & (\text{here, }f(\vv{uv})=0\text{ if }\vv{uv}\not\in E). \\
	\end{align*}
	The second part is easily seen from the above series of inequalities, where equality holds iff $f^\ot(S)=0$ and for each $u\in S$, $v\in T$, $f(\vv{uv})=c(\vv{uv})$.
\end{proof}

Let us try to come up with an algorithm to solve the max-flow problem.\\
Let us start with $f(e)=0$ for all $e\in E$. We then try to ``push'' some flow along an $s$-$t$ path allowed by the capacity constraints. We then ask if this is the maximum flow possible. We try to push more flow along another path. However, we might face a problem wherein a previous flow restricts the new flow we are trying to push. We then want to ``undo'' the previous flow. How do we formalize this notion?

\begin{definition}[Residual Graph]
	Given a graph $G$ with capacity function $c:E\to\N$ and a flow $f$, a \textit{residual graph} $G$ with respect to $f$, denoted $G_f$ is such that
	\begin{itemize}
		\item its vertex set is the same as that of $G$.
		\item if $e$ is an edge in $G$ such that $f(e)<c(e)$, then $G_f$ has the edge $e$ with capacity $c(e)-f(e)$.
		\item if $e=\vv{uv}$ is such that $f(e)>0$, we add an edge $\vv{vu}$ in $G_f$ with capacity $f(e)$ on it. This is called a \textit{backward edge}.
	\end{itemize}
\end{definition}

In the above, ``undoing'' a flow just means sending flow along a backward edge and adding more flow corresponds to sending a flow along a normal edge (with a lowered capacity). This behaves as an intermediate step to finding the optimal flow.

For each edge of $G$, note that there at most two edges in $G_f$. So, the size of $G_f$ is at most twice that of $G$.

Let $\pi$ be any $s$-$t$ path in $G_f$. Denote by $\theta(\pi,f)$ the smallest residual capacity (smallest capacity among the relevant edges in the residual graph) along $\pi$ in $G_f$.

\begin{algorithm}[H]
	\DontPrintSemicolon
	\SetNoFillComment
	\SetKwProg{Fn}{}{}{}
	\SetKwFunction{aug}{Aug}
	\Fn{\aug{$\pi,f$}} {
		$b\gets\theta(\pi,f)$\;
		\ForEach{$e=\vv{uv}\in\pi$} {
			\eIf{$e$ is a forward edge} {
				increase $f(\vv{uv})$ in $G$ by $b$
			} {
				decrease $f(\vv{vu})$ in $G$ by $b$
			}
		}
	}
	\caption{Finding an Augmenting Path}\label{algo: aug path}
\end{algorithm}

\begin{algorithm}[H]
	\DontPrintSemicolon
	\SetNoFillComment
	\SetKwProg{Fn}{}{}{}
	\SetKwFunction{aug}{Aug}
	\For{$e\in E$} {
		$f(e)\gets 0$
	}
	Compute $G_f$\;
	\While{there is an $s$-$t$ path $\pi$ in $G_f$} {
		$f'\gets\aug{$\pi,f$}$\;
		$f\gets f'$\;
		Compute $G_f$
	}
	\Return{$f$}\;
	\caption{Ford and Fulkerson's Algorithm}\label{algo: max flow}
\end{algorithm}

\Cref{algo: max flow} finds a flow with the maximum flow value in the network!\\
There are a couple of questions we must answer.
\begin{itemize}
	\item Why does the algorithm terminate?
	\item Why is the algorithm even correct in the first place?
	\item What is the time complexity?
	\item Does it matter which $s$-$t$ path we choose in the residual graph?
	\item Could we extend this to the case where capacities are not integers?
\end{itemize}

Also, what is the relationship between max-flow and min-cut?

\paragraph{Termination.} Note that at each step of the algorithm, the new flow $f'$ satisfies $|f'|=|f|+\theta(\pi,f)$. Since the capacities are always integral (even in the residual graph), the flow value increases by at least $1$ at each step. It is also easy to see that the maximum flow is at most $C_{\text{max}}$, which is the sum of the capacity of every edge.\\
Therefore, the algorithm terminates.

\begin{lemma}
	The running time of the algorithm is bounded by $\mathcal{O}(C_{\text{max}}|E|)$.
\end{lemma}
\begin{proof}
	In each loop, we require $\mathcal{O}(m)$ time to maintain the residual graph. The time to find the $s$-$t$ path $\pi$ takes $\mathcal{O}(m+n)$ and augmenting takes $\mathcal{O}(n)$ (there are at most $n-1$ vertices on $\pi$). Since the loop runs for at most $C_{\text{max}}$ iterations, the bound follows.\\
	In fact, we can bound it better as $\mathcal{O}(f|E|)$, where $f$ is the maximal flow value.
\end{proof}

\begin{lemma}
	The Ford and Fulkerson algorithm returns a flow with maximum value.
\end{lemma}

\paragraph{Correctness.} Denote by $\mathfrak{f}$ be the flow returned by the Ford-Fulkerson algorithm. Let
\[ A = \{v\in V : \text{there is a path to $v$ from $s$ in $G_{\mathfrak{f}}$ }\} \]
and $B=V\setminus A$. Observe that $s\in A$ and $t\in B$. Further, when the algorithm terminates, there is no path from $s$ to $t$ in $G_\mathfrak{f}$. Since the two sets are disjoint, $(A,B)$ is a cut on $G$ (this makes sense because the two graphs have the same vertex set).

To show optimality, it suffices to show (using \Cref{ff-algo lemma 1}) that $\mathfrak{f}$ saturates every edge from $A$ to $B$ and avoids every edge from $B$ to $A$.
\begin{itemize}
	\item Suppose instead that there is an edge $\vv{uv}\in E$ such that $u\in A$, $v\in B$, and $\mathfrak{f}(\vv{uv}) < c(\vv{uv})$. However, then, $\vv{uv}$ is a forward edge in $G_\mathfrak{f}$, so $v\in A$. This is a contradiction and therefore, $\mathfrak{f}$ saturates every edge from $A$ to $B$.
	\item On the other hand, suppose there is an edge $\vv{uv}\in E$ with $u\in B$ and $v\in A$ with $\mathfrak{f}(\vv{uv})>0$. Then $\vv{vu}$ will be a backward edge in $G_\mathfrak{f}$, so $u\in A$. This is again a contradiction, so $\mathfrak{f}$ saturates every edge from $B$ to $A$.
\end{itemize}

Our analysis shows that the $(A,B)$ cut has capacity equal to $|\mathfrak{f}|$. Therefore, $|\mathfrak{f}|$ is the maximal flow in $G$ and further, $\capp(A,B)$ is the minimum capacity cut in $G$ (this follows from the fact that we have attained this flow value, so it cannot exceed the capacity of any cut).

\paragraph{Importance of chosen paths.} The chosen path turns out to be quite important. An example to highlight this fact can be found \href{https://en.wikipedia.org/wiki/Ford%E2%80%93Fulkerson_algorithm#Integral_example}{here}.

\paragraph{Non-integral costs.} It is not too difficult to show that there is no problem if the costs are rational -- the flow increases by at least $1/q$, where $q$ is the maximum denominator of any cost when written as a fraction.\\
However, it need not terminate when the costs are irrational. Further, it need not even converge to the correct value. A standard counterexample to show this can be found \href{https://en.wikipedia.org/wiki/Ford%E2%80%93Fulkerson_algorithm#Non-terminating_example}{here}.

However, there have been far more efficient algorithms for determining max-flow.

The Edmond-Karp algorithm augments along the path $\pi$ with largest $\theta(\pi, f)$. This in fact runs in $\mathcal{O}(|E|^2\log |E| \log |f|)$. Augmenting along the shortest path (in terms of number of edges) gives rise to an algorithm that runs in $\mathcal{O}(|V||E|^2)$.

Following this, there were numerous algorithms. Most recently, the ``compact networks'' method proposed by Orlin in 2012 runs in $\mathcal{O}(|V||E|)$ using dynamic trees.\\
It is possible to get much faster algorithms if the capacities are all unit. For example, there exists an $\mathcal{O}(\min\{|V|^{2/3}, |E|^{1/2}\}|E|)$ algorithm.

\subsection{Applications}

Let us look at some applications of the max-flow min-cut theorem, which states that the value of the max-flow is equal to the capacity of the min-cut.

\subsubsection{Edge Disjoint Paths}
\label{subsubsec: edge-disjoint paths}

Let $G=(V,E)$ be a directed graph and $s,t\in V$ be designated vertices. Two $s$-$t$ paths $\pi$ and $\pi'$ are said to be \textit{edge disjoint} if they do not share any edges.

The edge disjoint paths problem asks to determine the maximum number of disjoint paths between the source and sink nodes $s$ and $t$ in a directed graph $G$.

We claim that this quantity is equal to the maximum flow in the graph if each edge is assigned a capacity of $1$.

\begin{lemma}
	The max flow value of $G$ is at least $k$ iff $G$ has $k$ edge disjoint paths.\\
	It follows that the max flow value of $G$ is $k$ iff the maximum number of edge disjoint paths in $G$ is $k$.
\end{lemma}
\begin{proof}
	Suppose $G$ has $k$ edge disjoint paths. Consider the flow that sends a flow of $1$ along each of these $k$ disjoint paths from $s$ to $t$ ($f(e)=1$ if $e$ belongs to one of the $k$ paths). This gives a flow of value $k$.\\

	On the other hand, suppose the max flow value of $G$ is $k$. Then, there is an integral flow $f$ of value $k$ in $G$ (the flow output by the Ford and Fulkerson algorithm).\\
	By tracing out the edges of flow $1$, we can generate $k$ edge disjoint paths. We prove this by induction on the number of edges that carry non-zero flow on them.\footnote{More precisely, we are proving that given any flow in the graph, there is a decomposition of the edges with non-zero flow into a set of edge-disjoint paths.}\\
	If there are $0$ edges, then there are no disjoint paths and there is nothing to prove.\\
	Say there are $t>0$ edges with non-zero flow. Then, let $\pi$ be an acyclic path from $s$ to $t$ that walks along $1$-edges (why does such a path exist?). Consider the flow $f'$ that is the same as $f$ except that $f(e)=0$ for any $e\in\pi$. Now, since $|f'|<|f|=t$, we may consider the resulting edge disjoint paths. We wish to show that $\pi$ and any of these paths are disjoint. This is trivial however, by definition, and the claim follows.
\end{proof}

\subsubsection{Network Connectivity}

Let $G=(V,E)$ be a directed graph and $s,t$ be the source and sink. We say that $F\subseteq E$ \textit{disconnects} $s$ from $t$ if the removal of $F$ from the graph disconnects $t$ and $s$.

The network connectivity problem asks to determine what the smallest $F\subseteq E$ that disconnects $s$ from $t$ is.

\begin{theorem}[Menger's Theorem]
	The maximum number of edge-disjoint $s$-$t$ paths in a graph is equal to the minimum number of edges whose removal disconnects $t$ from $s$.
\end{theorem}
\begin{proof}
	Suppose the graph has capacities as defined in \Cref{subsubsec: edge-disjoint paths}). The first quantity, as we have seen, is equal to the max-flow. The quantity on the right on the other hand, is equal to the capacity of the min-cut (by definition). The required follows.
\end{proof}