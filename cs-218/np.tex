\section{NP Hardness and Reductions}

Suppose we have $n$ jobs that take time $S_1,\ldots,S_n$ each and two processors $P_1$ and $P_2$. We wish to schedule the jobs on these two processors such that the overall time taken (the time when the last job completes) is minimized. How do we do this?\\

One idea is that we initially schedule $S_1$ and $S_2$ and when one of the jobs ends, we schedule the next job $S_3$ on the processor that is freed. We repeat this, scheduling jobs whenever processors become free.\\
This does not work -- consider the set of times $\{1,1,100\}$. Our scheduling produces time $101$ (we schedule the first two jobs first), but it is quite easy to see that it can be done in time $100$.\\

The issue seems to be that the ordering of the jobs is problematic. So instead, schedule the jobs in non-increasing order and repeat the above algorithm. Does this work?\\
No, it does not. Consider the set of jobs $\{3,3,2,2,2\}$ -- our algorithm gives a time of $7$ but a time of $6$ can be attained.\\

It in fact turns out that it is impossible (or at least extremely hard) to get a polynomial time algorithm. How do we show that this problem is ``hard'' though? What does ``hard'' even mean?\\

\subsection{Reduction}

Reduction essentially means using other algorithms for problems we wish to solve. For example, consider the problem of determining the existence of a perfect matching in a graph $G=(V,E)$.\\
On the other hand, consider the problem that finds a maximum matching in a graph $G=(V,E)$.\\
It is not too difficult to see that the first problem can be ``reduced'' to the second -- if we find a maximum matching in the graph, we can check whether it has $|V|/2$ vertices and thus solve the first problem.\\
Can we come up with a reduction that goes the other way round? Before doing this in fact, can we reduce the problem of finding the \textit{size} of a maximum matching in a graph to the first problem? Yes, we can.
\begin{itemize}
	\item If $G$ has a perfect matching, then return $|V|/2$.
	\item Let $G_1$ be a graph with vertex set $V\cup\{v\}$, where $v$ is a fresh vertex that has an edge to every vertex in $V$. If $G_1$ has a perfect matching, then $G$ has a matching of size $(|V|-1)/2$ (Why?), which must be a maximum matching if this is the case.
	\item Continuing on, create $G_{i+1}$ by adding a new vertex that is connected to every vertex of $G_{i}$. Then if $k$ is the minimum index such that $G_k$ has a perfect matching, $G$ has a matching of size $(|V|-k)/2$ -- there cannot be any edges within the new vertices in a perfect matching because this would imply the existence of a perfect matching of $G_i$ for some $i<k$ (Why?). Further, this is a maximum matching. If there was a matching of size $(|V|-k+2)/2$, then $G_{k-2}$ must have a matching (Why?).
\end{itemize}
Since $i$ can go up to $n$, we are calling the perfect matching algorithm only $n$ times. This can be decreased to $\log n$ by using a binary search-like scheme.\\
Now that we have done this, the original problem is not too difficult. Suppose the size of a maximum matching is $l$. We can remove an edge and ask if the remaining graph has a matching of size $l$ (that is, if the size of a maximum matching is still equal to $l$). If no, we discard the edge. The remaining graph will just be a matching, which gives us a maximum matching in the original graph.