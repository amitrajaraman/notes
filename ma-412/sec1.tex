\section{Introduction}

%%% LECTURE 1

\subsection{Some basic definitions}

	Consider the equation $X^2 + 1 = 0$. Clearly, this equation has no roots over $\R$. Consider the set
	\[ \C = \{ (a,b) : a,b\in\R \} = \R^2, \]
	and define addition and subtraction over $\C$ as
	\begin{align*}
		(a,b) + (c,d) &= (a+c,b+d) \\
		(a,b) \cdot (c,d) &= (ac-bd,ad+bc).
	\end{align*}
	It is easy to show that $(\C,+,\cdot)$ is a field with additive identity $(0,0)$ and multiplicative identity $(1,0)$. Further observe that $\R$ is a subfield of $\C$ -- consider the field homomorphism $\R \to \C$ defined by $a \mapsto (a,0)$.\\
	Now, we denote $\iota = (0,1)$, and write $(a,b)$ as $a+b\iota$.\\

	Observe that the equation $X^2 + 1 = 0$ \emph{does} have roots over $\C$ since it can be written as $(X+\iota)(X-\iota)$. For the sake of completeness, we also note that the multiplicative identity of $a+\iota b$ is
	\[ \frac{1}{a+\iota b} = \frac{a - \iota b}{a^2 + b^2} = \frac{a}{a^2+b^2} - \frac{b}{a^2+b^2}\iota. \]

	When writing $z = a + b\iota$ where $a,b\in\R$, we write $a = \Re z$ (the real part of $z$) and $b = \Im z$ (the imaginary part of $z$). We also define the absolute value $|z| = (a^2 + b^2)^{1/2}$ of $z$, and the \emph{conjugate} $\overline{z} = a - \iota b$ of $z$. We clearly have
	\begin{align*}
		z\overline{z} &= |z|^2 \\
		\Re z &= \frac{z+\overline{z}}{2} \\
		\Im z &= \frac{z-\overline{z}}{2\iota}.
	\end{align*}
	It is easy to check that
	\[ \overline{z+w} = \overline{z} + \overline{w} \text{ and } \overline{z\cdot w} = \overline{z}\cdot \overline{w}. \]
	We also have
	\begin{align*}
		\left| \frac{z}{w} \right| &= \frac{|z|}{|w|} \\
		|\overline{z}| = |z|.
	\end{align*}

	\begin{exercise}
		Check that the set
		\[ M = \begin{pmatrix} \alpha & \beta \\ -\beta & \alpha \end{pmatrix} : \alpha,\beta\in\R \]
		with matrix addition and multiplication is a field isomorphic to $\C$.
	\end{exercise}

	To close out the tedious part of things, we have
	\begin{align}
		|z+w|^2 &= |z|^2 + |w|^2 + 2 \Re(z\overline{w}) \nonumber \\
		|z+w| &\le |z| + |w| \label{eqn: triangle inequality}
	\end{align}

	\Cref{eqn: triangle inequality} is referred to as the \emph{triangle inequality}.

\subsection{Polar representations and roots}

	Consider $z = x + \iota y \in \C$. We may then define
	\[ x = r\cos\theta \;\;\;\;\; y = r\sin\theta, \]
	where $|z| = r$ and the angle $\theta$ is called the \emph{argument} of $z$ as is denoted $\theta = \arg z$. We typically restrict $\theta$ to $(-\pi,\pi]$.\\
	We denote $\cis\theta = \cos\theta + \iota\sin\theta$. Therefore, we have
	\[ z = |z| \cis(\arg z). \]
	Observe that rather conveniently,
	\[ \cis\theta_1 \cdot \cis\theta_2 = \cis(\theta_1 + \theta_2). \]
	Therefore, inductively,
	\[ z_1 z_2 \cdots z_n = \left(\prod_{i} |z_i|\right) \cis\left(\sum_i \arg z_i \right). \]
	In particular,
	\[ z^n = r^n \cis(n\theta) \]
	for any $n > 0$. If $z \ne 0$ (equivalently, $r\ne0$), the above holds for all $n \in \Z$.\\
	In the case where $r=1$, we have
	\begin{equation}
		\label{eqn: de moivre's}
		(\cos\theta + \iota\sin\theta)^n = \cos(n\theta) + \iota\sin(n\theta)
	\end{equation}
	\Cref{eqn: de moivre's} is referred to as \emph{de Moivre's Formula}.\\

	Let us consider the equation $z^n = a$. This equation has $n$ roots of the form
	\[ z = |a|^{1/n} \cis\left( \frac{2k\pi + \arg a}{n} \right) \]
	for $k = 0,1,\ldots,n-1$.

	A \emph{line} in the complex plane is a set of the form
	\[ L = \{ z = a+tb : t \in \R \}, \]
	for some fixed $a,b\in\C$, where $b$ is a \emph{directional} vector whose absolute value may be assumed to be $1$. Since $b \ne 0$, we equivalently have
	\[ L = \left\{ z : \Im\left( \frac{z-a}{b} \right) = 0 \right\}. \]
	We can also define the half-planes
	\begin{align*}
		H_a &= \left\{ z : \Im\left( \frac{z-a}{b} \right) > 0 \right\} \\
		K_a &= \left\{ z : \Im\left( \frac{z-a}{b} \right) < 0 \right\}.
	\end{align*}
	Note that $H_a = a + H_0$, where the addition is Minkowski addition:
	\[ H_a = \{ a + z : z \in H_0 \}. \]

\subsection{The extended plane}
	
	Define $\C_\infty = \C \cup \{\infty\}$ and let $S = \{(x_1,x_2,x_3) : x_1^2 + x_2^2 + x_3^2 = 1\}$ be the unit sphere in $\R^3$. We shall show a bijection from $\C_\infty$ to $S$.\\
	Let $N = (0,0,1)$ be the `north pole' of $S$, and orient $\C$ (as $\R^2$) in the horizontal plane in a manner such that $\C$ cuts $S$ along the equator. For $z = x + \iota y \in \C$, let us define the corresponding point $Z = (x_1,x_2,x_3) \in S$. We shall draw a line connecting $z$ to $N$, and let $Z$ be the point of intersection (other than $N$) of this line with $S$. Finally, we shall map $\infty$ to $N$.\\
	Let us define this more explicitly. The line through $N$ and $z$ is
	\[ L = \{ tN + (1-t)z : t \in \R \}. \]
	Then, letting $z = (x,y,0)$, we have
	\[ t^2 + (1-t)^2 |z|^2 = 1. \]
	So,
	\[ |z|^2 = \frac{1-t^2}{(1-t)^2} = \frac{1+t}{1-t} \]
	and
	\[ t = \frac{|z|^2-1}{|z|^2+1}. \]
	Therefore, we map $z$ to
	\[ Z = \left( \frac{2 \Re z}{|z|^2+1} , \frac{2 \Im z}{|z|^2+1} , \frac{|z|^2-1}{|z|^2+1}. \right) \in S. \]
	Based on this, we can define a distance metric between points in $\C_\infty$. For $z,z' \in \C_\infty$ mapping to $Z,Z' \in S$, we let $d(z,z')$ be the Euclidean distance between $Z,Z'$ in $\R^3$. More explicitly,
	\begin{align*}
		d(z,z')^2 &= (x_1 - x_1')^2 + (x_2 - x_2')^2 + (x_3 - x_3')^2 \\
			&= 2 - 2(x_1x_1' + x_2x_2' + x_3x_3') \\
			&= \frac{2|z-z'|}{\left((|z|^2+1) (|z'|^2+1)\right)^{1/2}}
	\end{align*}
	when $z,z'\in\C$ and if $z' = \infty$ (so $Z' = (0,0,1)$), we have
	\[ d(z,z') = \frac{4}{|z|^2 + 1} \]

	This correspondence between points of $S$ and $\C_\infty$ is called the \emph{stereographic projection}. 

	\begin{exercise}
		If $P$ is a plane in $\R^3$ and $\Lambda = P \cap S$ is a circle on $S$, show that the projection of $\Lambda$ on $\C$ under the stereographic projection is a circle as well (possibly a circle of infinite radius, namely a line).
	\end{exercise}

\subsection{Power series}

	In this section, we begin discussing convergence of series in $\C$ and related properties.

	\begin{fdef}
		If $a_n \in \C$ for every $n \ge 0$, the series $\sum_{n=0}^\infty a_n$ is said to \emph{converge} to $z$ iff for all $\epsilon>0$, there exists $N \in \N$ such that
		\[ \left| \sum_{n=0}^m a_n - z  \right| < \epsilon \]
		for all $m \ge N$.\\
		The series $\sum_{n=0}^\infty a_n$ is said to converge \emph{absolutely} if $\sum_{n=0}^\infty |a_n|$ converges.
	\end{fdef}

	\begin{theorem}
		$\C$ is complete. That is, every Cauchy sequence in $\C$ is convergent.
	\end{theorem}
	\begin{proof}
		Suppose $\{x_n + \iota y_n\}$ is a Cauchy sequence in $\C$, where $x_n, y_n \in \R$ for each $n$. We then have the existence of $N \in \N$ such that for all $m,k > N$, $|(x_m - x_k) + \iota(y_m - y_k)| < \epsilon$. Consequently, $|x_m - x_k| < \epsilon$ and $|y_m - y_k| < \epsilon$. However, since $\R$ is complete, this implies that $(x_n)$ and $(y_n)$ are convergent, completing the proof. 
	\end{proof}	

	\begin{theorem}
		If $\sum a_n$ converges absolutely, $\sum a_n$ converges.
	\end{theorem}
	\begin{proof}
		Let $\epsilon > 0$, $z_n = \sum_{i=0}^n a_i$, and $S_n = \sum_{i=0}^n |a_i|$. Because $\C$ is complete, it suffices to show that $(z_n)$ is Cauchy. \\
		Since $\sum |a_n|$ is convergent, there exists $N \in \N$ such that $|S_m - S_k| < \epsilon$ for all $m,k > N$. Supposing $m > k$, we have
		\[ S_m - S_k = \sum_{i=k+1}^m |a_i|. \]
		So,
		\begin{align*}
			|z_m - z_k| &= \left| \sum_{i=k+1}^m a_i \right| \\
				&\le \sum_{i=k+1}^m |a_i| < \epsilon,
		\end{align*}
		completing the proof.
	\end{proof}

	\begin{exercise}
		\label{ex: conv of summ zn}
		Show that $\sum_{n=0}^\infty z^n$ converges iff $|z| < 1$.
	\end{exercise}

	\begin{ftheo}
		For a given power series $\sum_{n=0}^\infty a_n (z-a)^n$, define the number $R$ ($0\le R\le \infty$) by
		\[ \frac{1}{R} = \limsup_{n\to\infty} |a_n|^{1/n}. \]
		Then,
		\begin{itemize}
			\item[(a)] If $|z-a| < R$, the series converges absolutely.
			\item[(b)] If $|z-a| > R$, the terms of the series become unbounded and the series diverges.
			\item[(b)] If $0 < r < R$, the series converges uniformly on the set $\{ z : |z-a| \le r \}$.
		\end{itemize}
	\end{ftheo}

	This $R$ is referred to as the \emph{radius of convergence} of the power series.

	\begin{proof}
		\phantom{easter}
		\begin{itemize}
			\item[(a)]
			We assume without loss of generality that $a = 0$. If $|z| < R$, there exists $r$ with $|z| < r < R$. By the definition of $R$, for all $\epsilon>0$, there exists $N \in \N$ such that
			\[ \frac{1}{R} - \epsilon < \sup_{k \ge n} |a_k|^{1/k} < \frac{1}{R} + \epsilon \]
			for all $n > N$. If we take $\epsilon = 1/r - 1/R$, it follows that $|a_n|^{1/n} < 1/r$ for all $n > N$. That is, for all $n > N$, $|a_n| < 1/r^n$ and so 
			\[ |a_n z^n| < \left(\frac{|z|}{r}\right)^n. \]
			Therefore, $\sum_{n=N}^\infty a_n z^n$ is dominated by $\sum_{n=N}^\infty (|z|/r)^n$. Now however, we can just use the result of \Cref{ex: conv of summ zn} to conclude absolute convergence since $|z|/r < 1$.
			
			\item[(b)]
			Let $|z| > R$ and choose $r$ with $|z| > r > R$. For $\epsilon > 0$, there exists $N \in \N$ such that
			\[ \frac{1}{R} - \epsilon < \sup_{k \ge n} |a_k|^{1/k} \text{ for all $n > N$}. \]
			Choosing $\epsilon = 1/R - 1/r$,
			\[ |a_n|^{1/n} > 1/r \]
			for infinitely many $n \in \N$. It follows that $|a_n z^n| > (|z|/r)^n$ for infinitely many $n \in \N$. Since $|z|/r > 1$, these terms become unbounded and therefore the series diverges.

			\item[(c)]
			Now, suppose $r<R$ and choose $\rho$ such that $r<\rho<R$. Similar to the argument in (a), we get that
			\[ |a_n| < \frac{1}{\rho^n} \text{ for all $n\ge N$.} \]
			If $|z| \le r$, $|a_n z^n| \le (r/\rho)^n$ and $r/\rho < 1$. The Weierstrass $M$-test then gives that the power series converges uniformly on $\{z : |z| \le r\}$. \qedhere
		\end{itemize}
	\end{proof}

	It should be noted that we cannot conclude anything when $|z-a|=R$.

	\begin{theorem}
		If $\sum a_n (z-a)^n$ is a power series with radius of convergence $R$, then if it exists,
		\[ \lim_{n\to\infty} \left| \frac{a_n}{a_{n+1}} \right| = R. \] 
	\end{theorem}
	\begin{proof}
		Again, assume that $a = 0$ and let $\alpha = \lim |a_n / a_{n+1}|$, which we assume exists. Suppose that $|z| < \alpha$ and take $r \in \R$ such that $|z| < r < \alpha$. For all $\epsilon > 0$, there exists $N \in \N$ such that for $n \ge N$,
		\[ \alpha - \epsilon < \left|\frac{a_n}{a_{n+1}}\right| < \alpha + \epsilon. \]
		Taking $\epsilon = \alpha - r$, $|a_n / a_{n+1}| > r$ for all $n \ge N$. Let $B = |a_N| r^N$. Then,
		\[ a_{N+1} r^{N+1} = |a_{N+1}| r \cdot r^N < |a_N| r^N = B. \]
		Similarly, we get that $|a_n| r^n < B$ for all $n \ge N$. Therefore,
		\[ |a_n z^n| < B \left( \frac{|z|}{r} \right)^n \]
		for all $n \ge N$. Thus, the sequence converges absolutely since $|z| < r$.\\
		Since $r < \alpha$ was arbitrary, this implies that $\alpha \le R$.\\

		On the other hand, if $|z| > \alpha$, take $r \in \R$ such that $|z| > r > \alpha$. Taking $\epsilon = r - \alpha$, we get $N \in \N$ such that
		\[ \left|\frac{a_n}{a_{n+1}}\right| < r \]
		for all $n \ge N$. Letting $B = |a_N| r^N$ again, we once more obtain that $|a_n| r^n > B$ for all $n \ge N$. This gives that
		\[ |a_n z^n| > B \left( \frac{|z|}{r} \right)^n  \]
		for all $n \ge N$, and since $|z| > r$, the sequence diverges (we may assume that $B \ne 0$ by making $N$ larger if required to ensure that $a_N \ne 0$ -- if this is not possible, the problem is trivial since it means that $(a_n)$ is eventually $0$). Since the choice of $r$ was arbitrary, this implies that $R \le \alpha$, completing the proof.
	\end{proof}

	Now, consider the series
	\[ \sum_{n=0}^\infty \frac{z^n}{n!}. \]
	The radius of convergence of this series is $\infty$. So, it converges for any complex number $z$, and convergence is uniform on every compact subset of $\C$.\\
	The above defines a function $\exp:\C\to\C$.\\
	We also denote $e^z = \exp(z)$.

	\begin{fdef}[Differentiability]
		If $G$ is an open set in $\C$ and $f : G \to \C$, then $f$ is said to be \emph{differentiable} at a point $a \in G$ if the limit
		\[ \lim_{h\to 0} \frac{f(a+h) - f(a)}{h} \]
		exists. If it exists, the value of this limit is denoted $f'(a)$ and is called the \emph{derivative} of $f$ at $a$.
	\end{fdef}

	If $f$ is differentiable at each point of $G$, we say that $f$ is differentiable on $G$. Note that if $f$ is differentiable on $G$, then $f' : G \to \C$ is a function. If $f'$ is continuous, $f$ is said to be \emph{continuously differentiable}.

	\begin{theorem}
		If $f : G \to \C$ is differentiable at a point $a \in G$, $f$ is continuous at $a$.
	\end{theorem}
	\begin{proof}
		The proof of this is direct:
		\begin{align*}
			\lim_{z \to a} |f(z) - f(a)| &= \left( \lim_{z \to a} \frac{|f(z)-f(a)|}{|z-a|} \right) \cdot \lim_{z\to a} |z-a| \\
				&= f'(a) \cdot 0 = 0. \qedhere
		\end{align*}
	\end{proof}

	\begin{fdef}
		A function $f : G \to \C$ is said to be \emph{analytic} if $f$ is continuously differentiable on $G$.
	\end{fdef}

	Let $f,g$ be analytic on $G$ and $\Omega$ respectively, and suppose that $f(G) \subseteq \Omega$. Then, $g \circ f$ is analytic on $G$ and
	\[ (g \circ f)'(z) = g'(f(z)) \cdot f'(z) \]
	for all $z \in G$. This is called the \emph{chain rule}.\\

	We shall show later that if $f$ is differentiable then its derivative is continuous, and so $f$ is analytic.

	\begin{ftheo}
		\label{theo: power series rad conv}
		Let $f(z) = \sum_{n=0}^\infty a_n (z-a)^n$ have radius of convergence $R > 0$. Then
		\begin{itemize}
			\item[(a)] For each $k \ge 1$, the series
			\[ \sum_{n=k}^\infty n(n-1)\cdots(n-k+1) a_n (z-a)^{n-k} \]
			has radius of convergence $R$.

			\item[(b)] The function $f$ is infinitely differentiable on $B(a,R)$ (the open ball of radius $R$ centered at $a$), and further, $f^{(k)}(z)$ is given by the series in (a) for all $k \ge 1$ and $|z-a| < R$.

			\item[(c)] For $n \ge 0$, $a_n = \frac{1}{n!} f^{(n)}(a)$. 
		\end{itemize}
	\end{ftheo}

	\begin{proof}
		Assume that $a = 0$.
		\begin{itemize}
			\item[(a)] Note that it suffices to prove the result for $k = 1$ (Why?). To show this, it is enough to show that
			\[ \limsup_{n\to\infty} |a_n|^{1/n} = \limsup_{n\to\infty} |na_n|^{1/(n-1)} \]
			First, it is not difficult to show that $\lim_{n\to\infty} n^{1/(n-1)} = 1$. It may be shown that for any sequences $(c_n),(d_n)$ in $\R$ where $c_n \ge 0$, if $\lim c_n = c$ and $\limsup d_n = d$, then $\limsup c_n d_n = cd$. Therefore, we are done if we show that $\limsup_{n\to\infty} |a_n|^{1/n} = \limsup_{n\to\infty} |a_n|^{1/(n-1)}$.
			\[ \sum_{n=0}^\infty a_n z^n = a_0 + z\sum_{n=0}^\infty a_{n+1} z^n. \]
			Let $R'$  be the radius of convergence of $\sum_{n=0}^\infty a_{n+1} z^n$. We want to show that $R' = R$. \\
			If $|z| < R'$, then 
			\[ \sum |a_n z^n| \le |a_0| + |z| \sum_{n=0}^\infty |a_{n+1} z^n| < \infty, \]
			so $R' \le R$. On the other hand, if $|z| < R$ and $z \ne 0$,
			\[ \sum |a_{n+1} z^n| < \frac{1}{|z|} \left(\sum |a_n z^n| + |a_0|\right) < \infty, \]
			so $R \le R'$ and we are done.

			\item[(b)] Once again, it suffices to prove the result for $k = 0$. For $|z| < R$ and $g(z) = \sum_{n=1}^\infty n a_n z^{n-1}$,
			\[ s_n(z) = \sum_{k=0}^n a_k z^k \text{ and } R_n(z) = \sum_{k=n+1}^\infty a_k z^k, \]
			fix a point $w \in B(0,R)$ and $r$ such that $|w| < r < R$. We wish to show that $f'(w)$ exists and is equal to $g(w)$. Let $\delta > 0$ be arbitrary with $\overline{B(w,\delta)} \subseteq B(0,r)$. Letting $z \in B(w,\delta)$, we have
			\[ \frac{f(z)-f(w)}{z-w} - g(w) = \frac{s_n(z)-s_n(w)}{z-w} - s_n'(w) + s_n'(w) - g(w) + \frac{R_n(z) - R_n(w)}{z-w}. \]
			We have
			\[ |z^k - w^k| = |z-w||z^{k-1} + z^{k-2}w + \cdots + w^{k-1}| \le |z-w| kr^{k-1}. \]
			Therefore,
			\[ \left| \frac{R_n(z) - R_n(w)}{z-w} \right| = \left| \sum_{k=n+1}^\infty a_k \frac{z^k - w^k}{z-w} \right| \le \sum_{k=n+1}^\infty |a_k| k r^{k-1}. \]
			Since $r < R$, $\sum_{k=1}^\infty |a_k| k r^{k-1}$ converges and so for any $\epsilon > 0$, there exists $N_1 \in N$ such that for $n \ge N_1$,
			\[ \left| \frac{R_n(z) - R_n(w)}{z-w} \right| < \epsilon/3. \]
			Since $\lim s_n'(w) = g(w)$, there exists $N_2 \in \N$ such that
			\[ |s_n'(w) - g(w)| < \epsilon/3 \]
			for $n \ge N_2$. Choose $n \ge \max(N_1, N_2)$. Then, there exists $\delta > 0$ such that whenever $0 < |z-w| < \delta$,
			\[ \left| \frac{s_n(z) - s_n(w)}{z-w} - s_n'(w) \right| < \epsilon/3. \]
			Putting all these together, we get the desideratum.

			\item[(c)] This is straightforward using the explicit expression for $f^{(k)}(a)$. \qedhere
		\end{itemize}
	\end{proof}

	If the series $f(z) = \sum_{n=0}^\infty a_n (z-a)^n$ has radius of convergence $R > 0$, then $f$ is analytic on $B(a,R)$. Therefore, $\exp$ is analytic on $\C$.\\
	%%% LECTURE 4
	Further, letting $g = \exp$,
	\[ g'(z) = \sum_{n=1}^\infty \frac{n}{n!} z^{n-1} = \sum_{n=1}^\infty \frac{1}{(n-1)!} z^{n-1} = g(z). \]

	Define the functions $\cos$ and $\sin$ using power series as
	\begin{align*}
		\cos z &= 1 - \frac{z^2}{2!} + \frac{z^4}{4!} + \cdots + (-1)^k \frac{z^{2k}}{(2k)!} + \cdots \\
		\sin z &= z - \frac{z^3}{3!} + \frac{z^5}{5!} + \cdots + (-1)^k \frac{z^{2k+1}}{(2k+1)!} + \cdots
	\end{align*}
	Note that
	\[ \cos z = \frac{e^{\iota z} + e^{-\iota z}}{2} \text{ and } \sin z = \frac{e^{\iota z} - e^{-\iota z}}{2\iota}. \]
	Therefore,
	\[ e^{\iota z} = \cos z + \iota \sin z. \]
	In particular, if $z = \theta \in \R$,
	\[ e^{\iota \theta} = \cos \theta + \iota \sin \theta. \]
	It is direct to show next that $\cos^2 z + \sin^2 z = 1$ for $z \in \C$.

	\begin{definition}
		A function $f$ is said to be \emph{periodic} with period $c$ if $f(z) = f(z+c)$ for all $z\in\C$.
	\end{definition}

	$e^{z}$ is periodic with period $2\pi\iota$.\\

	Similar to $\cos$ and $\sin$, one can define the function $\log$ as
	\[ \log (1 + z) = z - \frac{z^2}{2} + \frac{z^3}{3} - \frac{z^4}{4} + \cdots. \]
	$\log z$ is defined only when $|z-1| < 1$. Further note that we cannot define $\log$ as the inverse of $\exp$ (as we do over the reals) since $\exp$ is not injective here.\\
	We would like to define $\log$ such that $w = \exp z$ when $z = \log w$. Since $\exp$ is non-zero, also suppose that $w \ne 0$. If $z = x + \iota y$, then $|w| = e^x$ and $\arg w = y + 2\pi k\iota$ for some $k\in\Z$. Therefore, the solution set for $e^z = w$ is
	\[ \{ \log |w| + \iota (\arg w + 2\pi k) : k \in \Z \}. \]

	\begin{fdef}
		If $G$ is an open connected set in $\C$ and $f : G \to \C$ is a continuous function such that $z = \exp(f(z))$ for all $z \in G$, then $f$ is a \emph{branch of the logarithm}.
	\end{fdef}

	\begin{lemma}
		If $G \subseteq \C$ is open and connected and $f$ is a branch of the logarithm on $G$, then the totality of the branches of $\log z$ are the functions $f(z) + 2\pi k\iota$ for $k \in \Z$.
	\end{lemma}
	\begin{proof}
		If $g(z) = f(z) + 2\pi k\iota$ for some $k \in \Z$, then $\exp(g(z)) = \exp(f(z)) = z$, so $g$ is also a branch of the logarithm.\\
		On the other hand, suppose that $g$ is a branch of the logarithm. For $z\in G$, $\exp(f(z)) = \exp(g(z)) = z$, so $g(z) = f(z) + 2\pi k\iota$. However, note that this $k$ depends on $z$. We must show that the same $k$ works for all $z$. Indeed, $h(z) = (g(z) - f(z))/2\pi\iota$ is continuous on $G$ and $h(G) \subseteq \Z$, so the required follows.
	\end{proof}


	Now, let $G = \C \setminus \R_{\le 0}$. Clearly, $G$ is connected and each $z \in G$ can be uniquely denoted by $|z| e^{\iota\theta}$, where $-\pi < \theta < \pi$. For $\theta$ in this range, define
	\[ f(r e^{\iota\theta}) = \log r + \iota\theta. \]
	This is a branch of the logarithm on $G$, and is commonly referred to as the \emph{principal branch}.

	\begin{theorem}
		Let $G,\Omega$ be open subsets of $\C$. Suppose that $f : G \to \C$ and $g : \Omega \to \C$ are continuous such that $g(f(z)) = z$ for all $z \in G$. If $G$ is differentiable and $g'(z) \ne 0$, $f$ is differentiable and
		\[ f'(z) = \frac{1}{g'(f(z))}. \]
		If $g$ is analytic, so is $f$.
	\end{theorem}
	\begin{proof}
		Fix $a \in G$ and let $h \in \C \setminus \{0\}$ with $a+h \in G$. Since $g(f(a)) = a \ne a+h = g(f(a+h))$, $f(a) \ne f(a+h)$.
		Also,
		\[ 1 = \frac{g(f(a+h)) - g(f(a))}{h} = \frac{g(f(a+h)) - g(f(a))}{f(a+h)-f(a)} \cdot \frac{f(a+h)-f(a)}{h}. \]
		Take the limit of either side as $h \to 0$. The first fraction is equal to $g'(f(a))$ since $\lim_{h\to 0} (f(a+h) - f(a)) = 0$, and therefore $\lim_{h\to0} (f(a+h) - f(a))/h = f'(a)$ exists, and $1 = g'(f(a)) \cdot f'(a)$. The required follows.\\
		If $g$ is analytic, then $g'$ is continuous so $f$ is analytic.
	\end{proof}

	\begin{corollary}
		Any branch of the logarithm function is analytic and has derivative $z \mapsto 1/z$.
	\end{corollary}

	Given a branch of the logarithm $f$ on an open connected set $G$ and fixed $b \in \C$, define $g(z) = \exp(bf(z))$. If $b \in \Z$, $g(z) = z^b$. In general, this defines a branch of $z^b$ ($b\in\C$) for any open connected set on which there is a branch of $\log z$.\\
	If we write $z^b$ as a function, it is implicitly understood that the $f$ in $\exp(bf(z))$ is the principal branch of the logarithm. Since $\log$ is analytic, so is $z \mapsto z^b$.



	%%% LECTURE OF 19-01-2022 (MISSED)

\subsection{Cauchy-Riemann Equations}

	Let $f:G\to\C$ be analytic and let
	\[ u(x,y) = \Re(f(x+\iota y)), v(x,y) = \Im(f(x+\iota y)) \]
	for $x+\iota y \in G$. Let us evaluate the limit
	\[ f'(z) = \lim_{h\to 0} \frac{f(z+h)-f(z)}{h}. \]
	in two different ways.\\
	First, if we let $h\to 0$ through real values, we get
	\[ f'(z) = \frac{\partial u}{\partial x}(x,y) + \iota \frac{\partial v}{\partial x}(x,y). \]
	Along the imaginary axis, we get
	\[ f'(z) = -\iota \frac{\partial u}{\partial y}(x,y) + \frac{\partial v}{\partial y}(x,y). \]
	Therefore,
	\[ \frac{\partial u}{\partial x} = \frac{\partial v}{\partial y} \text{ and } \frac{\partial u}{\partial y} = - \frac{\partial v}{\partial x}. \]

	Supposing that $u$ and $v$ have continuous second derivative (we shall later show that they are infinitely differentiable), we have that
	\[ \frac{\partial^2 u}{\partial x^2} = \frac{\partial^2 v}{\partial x \partial y} \text{ and } \frac{\partial^2 u}{\partial y^2} = - \frac{\partial^2 v}{\partial y \partial x}. \]
	Therefore, since the second derivatives are continuous,
	\begin{equation}
		\label{eqn-harmonic}
		\frac{\partial^2 u}{\partial x^2} + \frac{\partial^2 u}{\partial y^2} = 0.
	\end{equation}
	A function $u$ with continuous second partial derivatives satisfying \Cref{eqn-harmonic} is said to be \emph{harmonic}. Similarly, $v$ is also harmonic.

	\begin{ftheo}
		Let $u$, $v$ be real-valued functions defined on an open connected set (a \emph{region}) $G$ and suppose that they have continuous second partial derivatives. Then, $f:G\to\C$ defined by $f(z) = u(z) + \iota v(z)$ is analytic iff $u$ and $v$ satisfy the Cauchy-Riemann equations.
	\end{ftheo}
	\begin{proof}
		We have already shown the forward direction.\\
		For the other direction, let $z = x + \iota y \in G$ and $B(z,r) \subseteq G$. Let $h = s + \iota t \in B(0,r)$. Our goal is to show that for all $\epsilon > 0$, there exists $\delta > 0$ such that
		\[ \left| \frac{f(z+h) - f(z) - f'(z)h}{h} \right| < \epsilon \]
		for all $h \in B(0,\delta)$ for some $f'(z) \in \C$.
		Note that
		\[ u(x+s,y+t) - u(x,y) = \left( u(x+s,y+t) - u(x,y+t) \right) + \left( u(x,y+t) - u(x,y) \right). \]
		Now, for fixed $t \in (-r,r)$, $s \mapsto u(x+s,y+t)$ is a differentiable function on $(-r,r)$. We apply the mean value theorem to conclude that there exist $s_1,t_1 \in (-r,r)$ for each $s+\iota t \in B(0,r)$ such that $|s_1| < |s|$, $|t_1| < |t|$, and
		\begin{align*}
			u(x+s,y+t) - u(x,y+t) &= u_x(x+s_1,y+t)s \\
			u(x,y+t) - u(x,y) &= u_y(x,y+t_1)t.
		\end{align*}
		Now, let
		\[ \varphi(s,t) = \left( u(x+s,y+t) - u(x,y) \right) - \left( u_x(x,y)s + u_y(x,y)t \right). \]
		We get that
		\[ \varphi(s,t) = \left(s u_x(x+s_1,y+t) - s u_x(x,y)\right) + \left(t u_y(x,y+t_1) - t u_y(x,y)\right). \]
		So,
		\[ \frac{\varphi(s,t)}{s+\iota t} = \frac{s}{s+\iota t} \left(u_x(x+s_1,y+t) - u_x(x,y)\right) + \frac{t}{s+\iota t} \left(u_y(x,y+t_1) - u_y(x,y)\right) \]
		and on taking the limit of both sides as $s+\iota t \to 0$, we can use the fact that $|s| \le |s+\iota t|$, $|t| \le |s+\iota t|$, $|s_1| < |s|$, $|t_1| < t$, and the continuity of $u_x$, $u_y$, to conclude that
		\[ \lim_{s+\iota t \to 0} \frac{\varphi(s,t)}{s+\iota t} = 0. \]
		Therefore,
		\[ u(x+s,y+t) - u(x,y) = u_x(x,y)s + u_y(x,y) t + \varphi(s,t). \]
		We get a similar equation for $v$ as well, with a function $\psi$ (in place of $\varphi$). Combining the two,
		\begin{align*}
			\frac{f(z+s+\iota t) - f(z)}{s+\iota t} &= \frac{u(x+s,y+t)-u(x,y)}{s+\iota t} + \iota \frac{v(x+s,y+t) - v(x,y)}{s+\iota t} \\
				&= \frac{s u_x(x,y) + t u_y(x,y) + \varphi(s,t) + \iota \left( s v_x(x,y) + t v_y(x,y) + \psi(s,t) \right)}{s+\iota t} \\
				&= \frac{u_x(x,y) (s+\iota t) + \iota v_x(x,y) (s+\iota t) + \varphi(s,t) + \iota \psi(s,t)}{s+\iota t},
		\end{align*}
		where we used Cauchy-Riemann equations in the final step and thus,
		\[ \lim_{s+\iota t \to 0} \frac{f(z+s+\iota t) - f(z)}{s+\iota t} = u_x(x,y) + \iota v_x(x,y), \]
		completing the proof. Since $u_x$ and $v_x$ are continuous, $f'$ is continuous and $f$ is analytic.
	\end{proof}

	A next question is: given some $u$ such that
	\[ \frac{\partial^2 u}{\partial x^2} + \frac{\partial^2 u}{\partial y^2} = 0, \]
	when does there exist harmonic $v$ such that $u + \iota v$ is analytic? Such a $v$ is referred to as a \emph{harmonic conjugate} of $u$.

	It turns out that the answer is not always. Indeed, $u(x,y) = \log((x^2+y^2)^{1/2})$ on $\C\setminus\{0\}$, despite being harmonic, does not have a harmonic conjugate.

	\begin{ftheo}
		\label{theo: open disk harmonic conjugate}
		Let $G$ be either the entirety of $\C$ or some open disk. If $u : G \to \R$ is a harmonic function, then $u$ has a harmonic conjugate.
	\end{ftheo}
	\begin{proof}
		Let $G = B(0,R)$ for some $0 < R \le \infty$ and let $u : G \to \R$ be analytic. Define
		\[ v(x,y) = \int_0^y u_x(x,t) \dif t + \varphi(x) \]
		so that $u_x = v_y$. We shall determine $\varphi$ such that $v_x = - u_y$. Differentiating with respect to $x$, we get
		\begin{align*}
			v_x(x,y) &= \int_0^y u_{xx}(x,t) \dif t + \varphi'(x) \\
				&= - \int_0^y u_{yy}(x,t) \dif t + \varphi'(x) \\
				&= - u_y(x,y) + u_y(x,0) + \varphi'(x).
		\end{align*}
		Therefore, $\varphi'(x) = - u_y(x,0)$, and the function
		\[ v(x,y) = \int_0^y u_x(x,t) \dif t - \int_0^x u_y(s,0) \dif s \]
		is a harmonic conjugate of $u$.
	\end{proof}

	The above proof requires that the entire segments $[(0,0),(x,0)]$ $[(x,0),(x,y)]$ are contained in $G$, which is true when we are on a disk.

\subsection{Transformations}

	Consider the two hyperbolas defined by
	\begin{align*}
		x^2-y^2 &= c \\
		2xy &= d,
	\end{align*}
	where $c,d \ne 0$.\\
	This gives
	\[ y = \pm \sqrt{ \frac{-c \pm \sqrt{d^2+c^2}}{2} }. \]
	Consider the functions
	\begin{align*}
		u(x,y) &= x^2 - y^2 \\
		v(x,y) &= 2xy.
	\end{align*}
	The two hyperbolas above are mapped by this $f = u + \iota v$ to the straight lines $u=c$ and $v=d$.\\

	\begin{fdef}
		A \emph{path} in a region $G \subseteq \C$ is a continuous function $\gamma:[a,b]\to G$ for some interval $[a,b]$ in $\R$. If $\gamma'(t)$ exissts for each $t \in [a,b]$ and $\gamma':[a,b]\to\C$ is continuous, then $\gamma$ is said to be \emph{smooth}. $\gamma$ is said to be \emph{piecewise smooth} if there is a partition $a=t_0 < t_1 < \cdots t_{n-1} < t_n = b$ of $[a,b]$ such that $\gamma$ is smooth on each subinterval $[t_{i-1},t_i]$ for $1\le i\le n$.\\
		For a path $\gamma : [a,b] \to \C$, $\gamma([a,b])$ is sometimes referred to as the \emph{trace} of $\gamma$ and denoted $\{\gamma\}$.
	\end{fdef}

	By the existence of $\gamma'$, we mean that the two-sided limit
	\[ \lim_{h\to 0} \frac{\gamma(t+h)-\gamma(t)}{h} \]
	exists for $t \in (a,b)$ and the right and left sided limits exist for $t = a,b$ respectively. This is equivalent to saying that $\Re \gamma$ and $\Im \gamma$ have derivatives.

	Suppose $\gamma : [a,b] \to G$ is a smooth path and for some $t_0 \in (a,b)$, $\gamma'(t_0) \ne 0$. Then, $\gamma$ has a \emph{tangent line} at the point $z_0 = \gamma(t_0)$. This line goes through the point $z_0$ in the direction of the vector $\gamma'(t_0)$, that is, the slope of the line is $\tan(\arg \gamma'(t_0))$.\\

	If $\gamma_1$ and $\gamma_2$ are two smooth paths with $\gamma_1(t_1) = \gamma_2(t_2) = z_0$ and $\gamma_1'(t_1),\gamma_2'(t_2) \ne 0$, then define the \emph{angle} between the paths $\gamma_1,\gamma_2$ at $z_0$ to be $\arg(\gamma_2'(t_2)) - \arg(\gamma_1'(t_1))$.\\

	Suppose $\gamma$ is a smooth path in $G$ and $f:G\to\C$ is analytic. Then, $\sigma = f \circ \gamma$ is also a smooth path and $\sigma'(t) = f'(\gamma(t)) \cdot \gamma'(t)$. Further, if $z_0$ is a fixed point of $f$ with $\gamma(t_0) = z_0$,
	\[ \arg(\sigma'(t_0)) - \arg(\gamma'(t_0)) = \arg(f'(z_0)). \]
	Let $\gamma_1,\gamma_2$ be smooth paths with $\gamma_1(t_1) = \gamma_2(t_2) = z_0$ with non-zero derivatives at $t_1,t_2$ respectively, and let $\sigma_1 = f \circ \gamma_1, \sigma_2 = f \circ \gamma_2$. Further suppose that the two paths $\gamma_1,\gamma_2$ are not tangent to each other at $z_0$. Then,
	\[ \arg(\gamma_2'(t_2)) - \arg(\gamma_1'(t_1)) = \arg(\sigma_2'(t_2)) - \arg(\sigma_1'(t_1)). \]
	This says that the angle between two paths are preserved after applying an analytic function to both. A function $f$ satisfying this is said to have the \emph{angle-preserving property}.

	\begin{fdef}
		A function $f : G \to \C$ which has the angle-preserving property and also has 
		\[ \lim_{z\to a} \left|\frac{f(z)-f(a)}{z-a}\right| \]
		existing for all $a \in G$ is called a \emph{conformal map}.
	\end{fdef}

	It turns out that a function $f$ is a conformal map if and only if it is analytic and $f'(z) \ne 0$ for all $z$ (How?).

	\begin{fdef}
		A mapping of the form
		\[ S(z) = \frac{az+b}{cz+d} \]
		is called a \emph{linear fractional transformation}. If we further have that $ad-bc \ne 0$, then $S(z)$ is called a \emph{M\"{o}bius transformation}.
	\end{fdef}

	We have
	\[ S'(z) = \frac{ad-bc}{(cz+d)^2}. \]

	If $w = S(z)$, it is relatively simple to show that
	\[ z = S^{-1}(w) = \frac{dw-b}{-cw+a}. \]
	Therefore, the inverse of a M\"{o}bius transformation is a M\"{o}bius transformation. The composition of two M\"{o}bius transformations is a M\"{o}bius transformation as well.\\
	Also observe that the coefficients $a,b,c,d$ for a given M\"{o}bius transformation are not unique since we can multiply them by a constant. We may also extend $S$ to $\C_\infty$ with $S(\infty) = a/c$ and $S(-d/c) = \infty$.\\

	$S(z) = z+a$ is called a \emph{translation}, $S(z) = az$ with $a\ne 0$ is called a \emph{dilation}, $S(z) = e^{\iota\theta} z$ is called a \emph{rotation}, and $S(z) = 1/z$ is called the \emph{inversion}. We shall see later that any M\"{o}bius transformation is a composition of these five types of transformations.

	What are the fixed points of a M\"{o}bius transformation $S$? $S(z) = z$ gives
	\[ cz^2 + (a-d)z + b = 0. \]
	Therefore, a M\"{o}bius transformation has at most two fixed points unless $S(z) = z$ for all $z \in \C_\infty$.\\

	Let $a,b,c\in\C_\infty$ be distinct with $S(a) = \alpha$, $S(b) = \beta$, $S(c) = \gamma$. Let $T$ be another M\"{o}bius transformation with $T(a) = \alpha$, $T(b) = \beta$, $T(c) = \gamma$. Then $T^{-1} \circ S$ has three (distinct) fixed points, and therefore $S = T$.\\
	Therefore, any M\"{o}bius transformation is uniquely determined by its value at any three distinct points.\\

	Let $z_2,z_3,z_4 \in \C_\infty$ be distinct. Define $S:\C_\infty\to\C_\infty$ by
	\[
	S(z) =
	\begin{cases}
		\frac{(z-z_3)/(z-z_4)}{(z_2-z_3)/(z_2-z_4)}, & z_2,z_3,z_4 \in \C, \\
		\frac{z_2-z_4}{z-z_4}, & z_3 = \infty, \\
		\frac{z-z_3}{z_2-z_3}, & z_4 = \infty. \\
	\end{cases}
	\]
	In any case, $S(z_2) = 1$, $S(z_3) = 0$, $S(z_4) = \infty$, and $S$ is the only transformation having this property.

	\begin{definition}
		If $z_1 \in \C_\infty$, then $(z_1,z_2,z_3,z_4)$ is referred to as the {\emph{cross-ratio}} of $z_1,z_2,z_3,z_4$ and is the image of $z_1$ under the M\"{o}bius transformation described above, which is the unique M\"{o}bius transformation taking $z_2$ to $1$, $z_3$ to $0$, and $z_4$ to $\infty$.
	\end{definition}

	For example, $(z_2,z_2,z_3,z_4) = 1$ and $(z,1,0,\infty) = z$.\\
	If $M$ is any M\"{o}bius transformation with $M(w_2) = 1$, $M(w_3) = 0$, $M(w_4) = \infty$, then $M(z) = (z,w_2,w_3,w_4)$ for all $z \in \C_\infty$.

	\begin{theorem}
		\label{thm: cross-product mobius equal}
		If $z_2,z_3,z_4$ are distinct points and $T$ is any M\"{o}bius transformation, then
		\[ (z_1,z_2,z_3,z_4) = (Tz_1,Tz_2,Tz_3,Tz_4). \]
	\end{theorem}
	\begin{proof}
		Let $S(z) = (z,z_2,z_3,z_4)$. If $M = ST^{-1}$, then
		\[ M(T(z_2)) = 1, \quad M(T(z_3)) = 0, \quad M(T(z_4)) = \infty. \]
		Therefore, $M = (z,Tz_2,Tz_3,Tz_4)$. That is,
		\[ ST^{-1}z = (z,Tz_2,Tz_3,Tz_4) \]
		for all $z \in \C_\infty$. Setting $z = Tz_1$ yields the required.
	\end{proof}

	\begin{lemma}
		If $\{z_2,z_3,z_4\},\{w_2,w_3,w_4\} \subseteq \C_\infty$, then there exists a unique M\"{o}bius transformation $S$ with $Sz_i = w_i$ for each $i$.
	\end{lemma}
	
	We omit the proof of the above.

	\begin{lemma}
		\label{thm: mobius real to circle}
		Let $\{z_1,z_2,z_3,z_4\} \subseteq \C_\infty$. Then, $(z_1,z_2,z_3,z_4)$ is real iff the four points lie on a circle.
	\end{lemma}
	\begin{proof}
		Define $S:\C_\infty\to\C_\infty$ by $Sz = (z,z_2,z_3,z_4)$. We are done if we show that $S^{-1}(\R_\infty)$ is a circle (since a circle is uniquely determined by three distinct points on it).\\
		Let $S(z) = (az+b)/(cz+d)$.\\

		First, let us show that $S^{-1}(\R_\infty) \subseteq \Gamma$ for a circle $\Gamma$ in $\C_\infty$. Let $w \in S^{-1}(\R_\infty)$. Then, $Sw = \overline{Sw}$ so
		\[ \frac{aw+b}{cw+d} = \frac{\overline{a}\overline{w} + \overline{b}}{\overline{c}\overline{w} + \overline{d}}. \]
		This gives that
		\begin{equation}
			\tag{$*$}
			\label{eqn: mobius circle real}
			(a\overline{c} - \overline{a}c)|w|^2 + (a\overline{d} - \overline{b}c)w + (b\overline{c} - d\overline{a})\overline{w} + (b\overline{d} - \overline{b}d) = 0.
		\end{equation}
		If $a\overline{c}$ is real, we get that
		\[ \Im\left((a\overline{d}-\overline{b}c)w + b\overline{d}\right) = 0, \]
		which is a circle through $\infty$ (a line).\\
		If on the other hand $a\overline{c}$ is not real, then (\ref{eqn: mobius circle real}) becomes
		\[ 2\iota\underbrace{\Im(a\overline{c})}_{\alpha \ne 0}|w|^2 + (a\overline{d} - b\overline{c})w + (b\overline{c} - \overline{a}d)\overline{w} + (b\overline{d} - \overline{b}d) = 0. \]
		Dividing by $2\iota\alpha$,
		\[ |w|^2 + \frac{(a\overline{d} - b\overline{c})w}{2\iota\alpha} + \frac{(b\overline{c} - \overline{a}d)\overline{w}}{2\iota\alpha} + \frac{(b\overline{d} - \overline{b}d)}{2\iota\alpha} = 0. \]
		Since $\alpha$ is real,
		\[ \overline{\frac{(b\overline{c} - \overline{a}d)\overline{w}}{2\iota\alpha}} = \frac{(a\overline{d} - b\overline{c})w}{2\iota\alpha} \]
		and
		\[ \frac{(b\overline{d} - \overline{b}d)}{2\iota\alpha} \]
		is real. This gives
		\[ |w|^2 + \overline{\gamma}w + \gamma\overline{w} - \delta = 0 \]
		for some $\gamma \in \C, \delta \in \R$. This is equivalent to $|w+\gamma| = (|\gamma|^2 + \delta)^{1/2}$, which is the equation of a circle\footnote{it may be checked that $|\gamma|^2 + \delta$ is a positive real by substituting their values.}.\\

		Letting $T = S^{-1}$ and $\Gamma$ be the circle obtained in the previous part of the proof, we must now show that $T(\R_\infty) = \Gamma$. Since $\R_\infty$ is connected and compact and $T$ is a homeomorphism, $T(\R_\infty)$ is a closed arc, say $\Gamma_1$, of $\Gamma$. If $\Gamma_1 \ne \Gamma$, let $z_1,z_2$ be the endpoints of this arc. If $T(\infty) = z_3$ which is distinct from $z_1,z_2$, then $\R_\infty \setminus \{\infty\}$ is connected but $\Gamma_1 \setminus \{z_1\}$ is disconnected, which is a contradiction. So, suppose $T(\infty) = z_1$. Then, $\R_\infty \setminus \{\infty,T^{-1}(z_2)\}$ is disconnected but $\Gamma_1 \setminus \{z_1,z_2\}$ is connected, yielding a contradiction once more and completing the proof.
	\end{proof}

	Next, we give a more general version of the above.
	
	\begin{ftheo}
		A M\"{o}bius transformation takes circles to circles.
	\end{ftheo}

	Note that \Cref{thm: mobius real to circle} follows from this since $\R_\infty$ is a circle (of infinite radius) in $\C_\infty$.

	\begin{proof}
		Let $\Gamma$ be a circle in $\C_\infty$ and $S$ a M\"{o}bius transformation. Let $z_2,z_3,z_4$ be three distinct points on $\Gamma$, and set $w_j = Sz_j$ for each $j$. We claim that $S(\Gamma)$ is the circle $\Gamma'$ determined by $w_2,w_3,w_4$. Indeed,
		\[ (z,z_2,z_3,z_4) = (Sz,w_2,w_3,w_4) \]
		for any $z$, and if $z \in \Gamma$, the LHS is real by \Cref{thm: mobius real to circle}, and using the same theorem on the RHS completes the proof.
	\end{proof}

	\begin{fdef}
		Let $\Gamma$ be a circle through $z_2,z_3,z_4$. The points $z,z^* \in \C_\infty$ are said to be \emph{symmetric} with respect to $\Gamma$ if
		\[ (z^*,z_2,z_3,z_4) = \overline{(z,z_2,z_3,z_4)}. \]
	\end{fdef}

	\begin{remark}
		The above definition only depends on $\Gamma$, not the choice of $z_2,z_3,z_4$.
	\end{remark}

	%%% LECTURE 8

	Observe that $z$ is symmetric with respect to itself with respect to $\Gamma$ if and only if $z \in \Gamma$. Indeed, it implies that $(z,z_2,z_3,z_4)$ is real, which by \Cref{thm: mobius real to circle} implies that $z \in \Gamma$.\\
	
	What does it mean for $z,z^*$ to be symmetric?\\
	If $\Gamma$ is a straight line, $z,z^*$ are symmetric with respect to $\Gamma$ iff their perpendicular bisector is equal to $\Gamma$. That is, the line joining $z,z^*$ is perpendicular to $\Gamma$ and they are the same distance from $\Gamma$ (but on opposite sides). Indeed, choosing $z_4 = \infty$, we get that
	\[ \frac{z^* - z_3}{z_2 - z_3} = \frac{\overline{z} - \overline{z_3}}{\overline{z_2} - \overline{z_3}}, \]
	so
	\[ |z - z_3| = |z^* - z_3| \]
	for all $z_3 \in \Gamma$.\\
	Now, suppose that $\Gamma = \{ z : |z-a| = R \}$ for some $0 < R < \infty$. We extensively use \Cref{thm: cross-product mobius equal} and the five types of M\"{o}bius translations in the following sequence of equations. Then,
	\begin{align*}
		(z^*,z_2,z_3,z_4) &= \overline{(z,z_2,z_3,z_4)} \\
			&= \overline{(z-a,z_2-a,z_3-a,z_4-a)} \\
			&= \left(\overline{z}-\overline{a},\frac{R^2}{z_2 - a},\frac{R^2}{z_3 - a},\frac{R^2}{z_4 - a}\right) \\
			&= \left(\frac{R^2}{\overline{z}-\overline{a}},z_2-a,z_3-a,z_4-a\right) \\
			&= \left( \frac{R^2}{\overline{z} - \overline{a}} + a , z_2, z_3, z_4 \right).
	\end{align*}

	Therefore, $z^* = a + \frac{R^2}{\overline{z}-\overline{a}}$, that is,
	\[ (z^* - a) (\overline{z} - \overline{a}) = R^2. \]
	Since
	\[ \frac{z^* - a}{z - a} = \frac{R^2}{|z-a|^2} > 0 \]
	is real, it follows that $z^*$ is on the ray $\{a + t(z-a) : 0 < t < \infty\}$. We also have that
	\[ |z^* - a| |z - a| = R^2, \]
	so one can easily obtain $z^*$ from $z$ or vice-versa. 

	\begin{lemma}[Symmetry Principle]
		If a M\"{o}bius transformation takes a circle $\Gamma_1$ to the circle $\Gamma_2$, then any pair of points symmetric with respect to $\Gamma_1$ is mapped to a pair of points symmetric with respect to $\Gamma_2$.
	\end{lemma}
	\begin{proof}
		The proof of this is near-direct.
		\begin{align*}
			(Tz,Tz_2,Tz_3,Tz_4) &= (z^*,z_2,z_3,z_4) \\
				&= \overline{(z,z_2,z_3,z_4)} \\
				&= \overline{(Tz,Tz_2,Tz_3,Tz_4)}. \qedhere
		\end{align*}
	\end{proof}


	\begin{fdef}
		If $\Gamma$ is a circle, then an \emph{orientation} for $\Gamma$ is an ordered triple $(z_1,z_2,z_3)$ of points in $\Gamma$.
	\end{fdef}
	An orientation is used to represent a ``direction'' of the circle, where we ``go'' from $z_1$ to $z_2$ to $z_3$.\\

	Let $\Gamma = \R$ and $z_1,z_2,z_3 \in \R$. Also put $Tz = (z,z_1,z_2,z_3)$. Since $T(\R_\infty) = \R_\infty$, $a,b,c,d$ can be chosen to be reals. Then,
	\begin{align*}
		Tz &= \frac{az+b}{cz+d} \\
			&= \frac{az+b}{|cz+d|^2} (c\overline{z}+d) \\
			&= \frac{1}{|cz+d|^2} \left( ac |z|^2 + bd + bc\overline{z} + adz \right).
	\end{align*}
	So,
	\[ \Im(z,z_1,z_2,z_3) = \frac{ad-bc}{|cz+d|^2} \Im z \]
	and thus, $\{ z : \Im (z,z_1,z_2,z_3) > 0 \}$ is either the upper or lower half-plane depending on whether $ad-bc$ is negative or positive. Note that $ad-bc$ is the determinant of $\begin{pmatrix}a & b \\ c & d\end{pmatrix}$.\\

	Let $\Gamma$ be an arbitrary circle and suppose that $z_1,z_2,z_3 \in \Gamma$. Then, for any M\"{o}bius transformation $S$,
	\begin{align*}
		\{ z : \Im(z,z_1,z_2,z_3) > 0 \} &= \{ z : \Im(Sz,Sz_1,Sz_2,Sz_3) > 0 \} \\
			&= S^{-1} \{ z : \Im(z,Sz_1,Sz_2,Sz_3) > 0 \}.
	\end{align*}
	So, if $S$ is chosen to map $\Gamma$ to $\R_\infty$, then the above set is equal to $S^{-1}$ of either the upper or lower halfspace.\\

	\begin{definition}
		If $z_1,z_2,z_3$ is an orientation of $\Gamma$, we denote the \emph{right side} and \emph{left side} of $\Gamma$ (with respect to $(z_1,z_2,z_3)$) to be
		\[ \{ z : \Im(z,z_1,z_2,z_3) > 0 \} \text{ and } \{z : \Im(z,z_1,z_2,z_3) < 0\} \]
		respectively.
	\end{definition}

	\begin{theorem}[Orientation Principle]
		Let $\Gamma_1,\Gamma_2$ be circles in $\C_\infty$ such that $T\Gamma_1 = \Gamma_2$ for some M\"{o}bius transformation $T$. Let $(z_1,z_2,z_3)$ be an orientation of $\Gamma_1$. Then, $T$ takes the right side (resp. left side) of $\Gamma_1$ with respect to the orientation $(z_1,z_2,z_3)$ to the right side (resp. left side) of $\Gamma_2$ with respect to the orientation $(Tz_1,Tz_2,Tz_3)$.
	\end{theorem}
	The proof of the above is left as an exercise to the reader.\\

	Since $(z,1,0,\infty) = z$ by definition, the right side of $\R_\infty$ with respect to the orientation $(1,0,\infty)$ is the upper half-plane.\\

	\begin{exercise}
		Find an analytic function $f : G \to \C$ where $G = \{ z : \Re z > 0 \}$, such that $f(G) = \{ z : |z| < 1 \}$.
	\end{exercise}

	Similar to the above exercise, one may show that
	\[ g(z) = \frac{e^z - 1}{e^z + 1} \]
	maps the infinite strip $\{ z : |\Im z| < \pi/2 \}$ to the open unit disk $D$.