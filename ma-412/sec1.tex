\section{Introduction}

%%% LECTURE 1

\subsection{Some basic definitions}

	Consider the equation $X^2 + 1 = 0$. Clearly, this equation has no roots over $\R$. Consider the set
	\[ \C = \{ (a,b) : a,b\in\R \} = \R^2, \]
	and define addition and subtraction over $\C$ as
	\begin{align*}
		(a,b) + (c,d) &= (a+c,b+d) \\
		(a,b) \cdot (c,d) &= (ac-bd,ad+bc).
	\end{align*}
	It is easy to show that $(\C,+,\cdot)$ is a field with additive identity $(0,0)$ and multiplicative identity $(1,0)$. Further observe that $\R$ is a subfield of $\C$ -- consider the field homomorphism $\R \to \C$ defined by $a \mapsto (a,0)$.\\
	Now, we denote $\iota = (0,1)$, and write $(a,b)$ as $a+b\iota$.\\

	Observe that the equation $X^2 + 1 = 0$ \emph{does} have roots over $\C$ since it can be written as $(X+\iota)(X-\iota)$. For the sake of completeness, we also note that the multiplicative identity of $a+\iota b$ is
	\[ \frac{1}{a+\iota b} = \frac{a - \iota b}{a^2 + b^2} = \frac{a}{a^2+b^2} - \frac{b}{a^2+b^2}\iota. \]

	When writing $z = a + b\iota$ where $a,b\in\R$, we write $a = \Re z$ (the real part of $z$) and $b = \Im z$ (the imaginary part of $z$). We also define the absolute value $|z| = (a^2 + b^2)^{1/2}$ of $z$, and the \emph{conjugate} $\overline{z} = a - \iota b$ of $z$. We clearly have
	\begin{align*}
		z\overline{z} &= |z|^2 \\
		\Re z &= \frac{z+\overline{z}}{2} \\
		\Im z &= \frac{z-\overline{z}}{2\iota}.
	\end{align*}
	It is easy to check that
	\[ \overline{z+w} = \overline{z} + \overline{w} \text{ and } \overline{z\cdot w} = \overline{z}\cdot \overline{w}. \]
	We also have
	\begin{align*}
		\left| \frac{z}{w} \right| &= \frac{|z|}{|w|} \\
		|\overline{z}| = |z|.
	\end{align*}

	\begin{exercise}
		Check that the set
		\[ M = \begin{pmatrix} \alpha & \beta \\ -\beta & \alpha \end{pmatrix} : \alpha,\beta\in\R \]
		with matrix addition and multiplication is a field isomorphic to $\C$.
	\end{exercise}

	To close out the tedious part of things, we have
	\begin{align}
		|z+w|^2 &= |z|^2 + |w|^2 + 2 \Re(z\overline{w}) \nonumber \\
		|z+w| &\le |z| + |w| \label{eqn: triangle inequality}
	\end{align}

	\Cref{eqn: triangle inequality} is referred to as the \emph{triangle inequality}.

\subsection{Polar representations and roots}

	Consider $z = x + \iota y \in \C$. We may then define
	\[ x = r\cos\theta \;\;\;\;\; y = r\sin\theta, \]
	where $|z| = r$ and the angle $\theta$ is called the \emph{argument} of $z$ as is denoted $\theta = \arg z$. We typically restrict $\theta$ to $(-\pi,\pi]$.\\
	We denote $\cis\theta = \cos\theta + \iota\sin\theta$. Therefore, we have
	\[ z = |z| \cis(\arg z). \]
	Observe that rather conveniently,
	\[ \cis\theta_1 \cdot \cis\theta_2 = \cis(\theta_1 + \theta_2). \]
	Therefore, inductively,
	\[ z_1 z_2 \cdots z_n = \left(\prod_{i} |z_i|\right) \cdots r_n \cis\left(\sum_i \arg z_i \right). \]
	In particular,
	\[ z^n = r^n \cis(n\theta) \]
	for any $n > 0$. If $z \ne 0$ (equivalently, $r\ne0$), the above holds for all $n \in \Z$.\\
	In the case where $r=1$, we have
	\begin{equation}
		\label{eqn: de moivre's}
		(\cos\theta + \iota\sin\theta)^n = \cos(n\theta) + \iota\sin(n\theta)
	\end{equation}
	\Cref{eqn: de moivre's} is referred to as \emph{de Moivre's formula}.\\

	Let us consider the equation $z^n = a$. This equation has $n$ roots of the form
	\[ z = |a|^{1/n} \cis\left( \frac{2k\pi + \arg z}{n} \right) \]
	for $k = 0,1,\ldots,n-1$.

	A \emph{line} in the complex plane is a set of the form
	\[ L = \{ z = a+tb : t \in \R \}, \]
	for some fixed $a,b\in\C$, where $b$ is a \emph{directional} vector whose absolute value may be assumed to be $1$. Since $b \ne 0$, we equivalently have
	\[ L = \{ z : \Im\left( \frac{z-a}{b} \right) = 0 \}. \]
	We can also define the half-planes
	\begin{align*}
		H_a &= \{ z : \Im\left( \frac{z-a}{b} \right) > 0 \} \\
		K_a &= \{ z : \Im\left( \frac{z-a}{b} \right) < 0 \}.
	\end{align*}
	Note that $H_a = a + H_0$, where the addition is Minkowski addition:
	\[ H_a = \{ a + z : z \in H_0 \}. \]

\subsection{The extended plane}
	
	Define $\C_\infty = \C \cup \{\infty\}$ and let $S = \{(x_1,x_2,x_3) : x_1^2 + x_2^2 + x_3^2 = 1\}$ be the unit sphere in $\R^3$. We shall show a bijection from $\C_\infty$ to $S$.\\
	Let $N = (0,0,1)$ be the `north pole' of $S$, and orient $\C$ (as $\R^2$) in the horizontal plane in a manner such that $\C$ cuts $S$ along the equator. For $z = x + \iota y \in \C$, let us define the corresponding point $Z = (x_1,x_2,x_3) \in S$. We shall draw a line connecting $z$ to $N$, and let $Z$ be the point of intersection (other than $N$) of this line with $S$. Finally, we shall map $\infty$ to $N$.\\
	Let us define this more explicitly. The line through $N$ and $z$ is
	\[ L = \{ tN + (1-t)z : t \in \R \}. \]
	Then, letting $z = (x,y,0)$, we have
	\[ t^2 + (1-t)^2 |z|^2 = 1. \]
	So,
	\[ |z|^2 = \frac{1-t^2}{(1-t)^2} = \frac{1+t}{1-t} \]
	and
	\[ t = 1 - \frac{|z|^2-1}{|z|^2+1}. \]
	Therefore, we map $z$ to
	\[ Z = \left( \frac{2 \Re z}{|z|^2+1} , \frac{2 \Im z}{|z|^2+1} , \frac{|z|^2-1}{|z|^2+1}. \right) \in S. \]
	Based on this, we can define a distance metric between points in $\C_\infty$. For $z,z' \in \C_\infty$ mapping to $Z,Z' \in S$, we let $d(z,z')$ be the Euclidean distance between $Z,Z'$ in $\R^3$. More explicitly,
	\begin{align*}
		d(z,z')^2 &= (x_1 - x_1')^2 + (x_2 - x_2')^2 + (x_3 - x_3')^2 \\
			&= 2 - 2(x_1x_1' + x_2x_2' + x_3x_3') \\
			&= \frac{2|z-z'|}{\left((|z|^2+1) (|z'|^2+1)\right)^{1/2}}
	\end{align*}
	when $z,z'\in\C$ and if $z' = \infty$ (so $Z' = (0,0,1)$), we have
	\begin{align*}
		d(z,z') &= 
	\end{align*}

	This correspondence between points of $S$ and $\C_\infty$ is called the \emph{stereographic projection}. 

	\begin{exercise}
		If $P$ is a plane in $\R^3$ and $\Lambda = P \cap S$ is a circle on $S$, show that the projection of $\Lambda$ on $\C$ under the stereographic projection is a circle as well (possibly a circle of infinite radius, namely a line).
	\end{exercise}

\subsection{Power series}

	In this section, we begin discussing convergence of series in $\C$ and related properties.

	\begin{fdef}
		If $a_n \in \C$ for every $n \ge 0$, the series $\sum_{n=0}^\infty a_n$ is said to \emph{converge} to $z$ iff for all $\epsilon>0$, there exists $N \in \N$ such that
		\[ \left| \sum_{n=0}^m a_n - z  \right| < \epsilon \]
		for all $m \ge N$.\\
		The series $\sum_{n=0}^\infty a_n$ is said to converge \emph{absolutely} if $\sum_{n=0}^\infty |a_n|$ converges.
	\end{fdef}

	\begin{theorem}
		$\C$ is complete. That is, every Cauchy sequence in $\C$ is convergent.
	\end{theorem}
	\begin{proof}
		Suppose $\{x_n + \iota y_n\}$ is a Cauchy sequence in $\C$, where $x_n, y_n \in \R$ for each $n$. We then have the existence of $N \in \N$ such that for all $m,k > N$, $|(x_m - x_k) + \iota(y_m - y_k)| < \epsilon$. Consequently, $|x_m - x_k| < \epsilon$ and $|y_m - y_k| < \epsilon$. However, since $\R$ is complete, this implies that $(x_n)$ and $(y_n)$ are convergent, completing the proof. 
	\end{proof}	

	\begin{theorem}
		If $\sum a_n$ converges absolutely, $\sum a_n$ converges.
	\end{theorem}
	\begin{proof}
		Let $\epsilon > 0$, $z_n = \sum_{i=0}^n a_i$, and $S_n = \sum_{i=0}^n |a_i|$. Because $\C$ is complete, it suffices to show that $(z_n)$ is Cauchy. \\
		Since $\sum |a_n|$ is convergent, there exists $N \in \N$ such that $|S_m - S_k| < \epsilon$ for all $m,k > N$. Supposing $m > k$, we have
		\[ S_m - S_k = \sum_{i=k+1}^m |a_i|. \]
		So,
		\begin{align*}
			|z_m - z_k| &= \left| \sum_{i=k+1}^m a_i \right| \\
				&\ge \sum_{i=k+1}^m |a_i| < \epsilon,
		\end{align*}
		completing the proof.
	\end{proof}

	\begin{exercise}
		\label{ex: conv of summ zn}
		Show that $\sum_{n=0}^\infty z_n$ converges iff $|z| < 1$.
	\end{exercise}

	\begin{ftheo}
		For a given power series $\sum_{n=0}^\infty a_n (z-a)^n$, define the number of $R$ ($0\le R\le \infty$) by
		\[ \frac{1}{R} = \limsup |a_n|^{1/n}. \]
		Then,
		\begin{itemize}
			\item[(a)] If $|z-a| < R$, the series converges absolutely.
			\item[(b)] If $|z-a| > R$, the terms of the series become unbounded and the series diverges.
			\item[(b)] If $0 < r < R$, the series converges uniformly on the set $\{ z : |z-a| \le r \}$.
		\end{itemize}
	\end{ftheo}

	This $R$ is referred to as the \emph{radius of convergence} of the power series.

	\begin{proof}
		\phantom{easter}
		\begin{itemize}
			\item[(a)]
			We assume without loss of generality that $a = 0$. If $|z| < R$, there exists $r$ with $|z| < r < R$. By the definition of $R$, for all $\epsilon>0$, there exists $N \in \N$ such that
			\[ \frac{1}{R} - \epsilon < \sup_{k \ge n} |a_k|^{1/k} < \frac{1}{R} + \epsilon \]
			for all $n > N$. If we take $\epsilon = 1/r - 1/R$, it follows that $|a_n|^{1/n} < 1/r$ for all $n > N$. That is, for all $n > N$, $|a_n| < 1/r^n$ and so 
			\[ |a_n z^n| < \left(\frac{|z|}{r}\right)^n. \]
			Therefore, $\sum_{n=N}^\infty a_n z^n$ is dominated by $\sum_{n=N}^\infty (|z|/r)^n$. Now however, we can just use the result of \Cref{ex: conv of summ zn} to conclude absolute convergence since $|z|/r < 1$.
			
			\item[(b)]
			Let $|z| > R$ and choose $r$ with $|z| > r > R$. For $\epsilon > 0$, there exists $N \in \N$ such that
			\[ \frac{1}{R} - \epsilon < \sup_{k \ge n} |a_k|^{1/k} \text{ for all $n > N$}. \]
			Choosing $\epsilon = 1/R - 1/r$,
			\[ |a_n|^{1/n} > 1/r \]
			for infinitely many $n \in \N$. It follows that $|a_n z^n| > (|z|/r)^n$ for infinitely many $n \in \N$. Since $|z|/r > 1$, these terms become unbounded and therefore the series diverges.

			\item[(c)]
			Now, suppose $r<R$ and choose $\rho$ such that $r<\rho<R$. Similar to the argument in (a), we get that
			\[ |a_n| < \frac{1}{\rho^n} \text{ for all $n\ge N$.} \]
			If $|z| \le r$, $|a_n z^n| \le (r/\rho)^n$ and $r/\rho < 1$. The Weierstrass $M$-test then gives that the power series converges uniformly on $\{z : |z| \le r\}$. \qedhere
		\end{itemize}
	\end{proof}

	It should be noted that we cannot conclude anything when $|z-a|=R$.

	\begin{theorem}
		If $\sum a_n (z-a)^n$ is a given power series 
	\end{theorem}
	\begin{proof}
		Again, assume that $a = 0$ and let $\alpha = \lim |a_n / a_{n+1}|$, which we assume exists. Suppose that $|z| < \alpha$ and take $r \in \R$ such that $|z| < r < \alpha$. For all $\epsilon > 0$, there exists $N \in \N$ such that for $n \ge N$,
		\[ \alpha - \epsilon < \left|\frac{a_n}{a_{n+1}}\right| < \alpha + \epsilon. \]
		Taking $\epsilon = \alpha - r$, $|a_n / a_{n+1}| > r$ for all $n \ge N$. Let $B = |a_N| r^N$. Then,
		\[ a_{N+1} r^{N+1} = |a_{N+1}| r \cdot r^N < |a_N| r^N = B. \]
		Similarly, we get that $|a_n| r^n < B$ for all $n \ge N$. Therefore,
		\[ |a_n z^n| < B \left( \frac{|z|}{r} \right)^n \]
		for all $n \ge N$. Thus, the sequence converges absolutely since $|z| < r$.\\
		Since $r < \alpha$ was arbitrary, this implies that $\alpha \le R$.\\

		On the other hand, if $|z| > \alpha$, take $r \in \R$ such that $|z| > r > \alpha$. Taking $\epsilon = r - \alpha$, we get $N \in \N$ such that
		\[ \left|\frac{a_n}{a_{n+1}}\right| < r \]
		for all $n \ge N$. Letting $B = |a_N| r^N$ again, we once more obtain that $|a_n| r^n > B$ for all $n \ge N$. This gives that
		\[ |a_n z^n| > B \left( \frac{|z|}{r} \right)^n  \]
		for all $n \ge N$, and since $|z| > r$, the sequence diverges (we can deal with the case where $B = 0$ separately). Since the choice of $r$ was arbitrary, this implies that $R \le \alpha$, completing the proof.
	\end{proof}