\section{Integration}

\subsection{Basic definitions}

	\subsubsection{Integrals of real functions}

		First, let us recall the definition of the Riemann integral\footnote{technically the Darboux integral?} of functions on $\R$.

		\begin{fdef}[Riemann Integral]
			Let $[a,b]$ be a given interval. A \emph{partition} $\mathcal{P}$ of $[a,b]$ is a finite set of points $x_0,x_1,\ldots,x_n$ where
			\[ a = x_0 \le x_1 \le \cdots \le x_{n-1} \le x_n = b.  \]
			We also write $\Delta x_i = x_i - x_{i-1}$ for $i = 1,2,\ldots,n$.\\
			For a bounded real function $f$ on $[a,b]$ and each partition $\mathcal{P}$ of $[a,b]$, we set
			\[ M_i = \sup_{x_{i-1} \le x \le x_i} f(x), \qquad m_i = \inf_{x_{i-1} \le x \le x_i} f(x). \]
			Further, set
			\[ U(\mathcal{P},f) = \sum_{i=1}^{n} M_i \Delta x_i, \qquad L(\mathcal{P},f) = \sum_{i=1}^n m_i \Delta x_i \]
			as the upper and lower Riemann sum respectively,
			and finally,
			\[ \overline{\int_a^b} f \dif x = \inf_{\mathcal{P}} U(\mathcal{P},f), \qquad \underline{\int_a^b} f \dif x = \sup_{\mathcal{P}} L(\mathcal{P},f) \]
			as the upper and lower Riemann integrals of $f$.\\
		\end{fdef}

		Next, we define the slightly more general Riemann-Stieltjes integral. Note that this is the same as the usual Riemann integral when $\alpha$ is the identity function.

		\begin{fdef}[Riemann-Stieltjes Integral]
			Let $\alpha : [a,b] \to \R$ be a monotonically increasing function on $[a,b]$. Corresponding to each partition $\mathcal{P}$ of $[a,b]$, write $\Delta \alpha_i = \alpha(x_i) - \alpha(x_{i-1})$. Clearly, $\Delta \alpha_i \ge 0$ for each $i$.\\
			For any real function $f$ which is bounded on $[a,b]$, we put
			\[ U(\mathcal{P},f,\alpha) = \sum_{i=1}^n M_i \Delta \alpha_i, \qquad L(\mathcal{P},f,\alpha) = \sum_{i=1}^n m_i \Delta \alpha_i, \]
			where $M_i,m_i$ are defined as in the definition of the Riemann integral. We then define
			\[ \overline{\int_a^b} f \dif \alpha = \inf_{\mathcal{P}} U(\mathcal{P},f,\alpha), \qquad \underline{\int_a^b} f \dif \alpha = \sup_{\mathcal{P}} L(\mathcal{P},f,\alpha). \]
			If these two are equal, we say that $f$ is \emph{Riemann-Stieltjes integrable} with respect to $\alpha$ on $[a,b]$ and denote the common value as $\int_a^b f \dif \alpha$.
		\end{fdef}

		We also remark that
		\[ \int_a^b f \dif \alpha = \lim_{\max \Delta \alpha_k \to 0} \sum_{k=1}^n f(\tau_k) \Delta \alpha_k, \]
		where $x_{k-1} \le \tau_k \le x_k$ for each $k$.\\
		More generally, we define the \emph{mesh} of $\mathcal{P}$ with respect to $\alpha$ as
		\[ \norm{\mathcal{P}} = \max\{ \Delta\alpha_i : 1 \le i \le n \}. \]
		So for all $\epsilon > 0$, there exists $\delta > 0$ such that for any partition $\mathcal{P}$ of $[a,b]$ with $\norm{P} < \delta$, then
		\[ \left| \sum_{k=1}^n f(\tau_k) \Delta\alpha_k - \int_a^b f \dif \alpha \right| < \epsilon \]
		for any choice of points $x_{k-1} \le \tau_k \le x_k$.

	\subsubsection{Riemann-Stieltjes integrals of complex-valued functions}

		\begin{fdef}
			A function $\gamma : [a,b] \to \C$ for $[a,b] \subseteq \R$ is said to be of \emph{bounded variation} if there exists $M > 0$ such that for any partition $\mathcal{P} = \{ a = t_0 < t_1 < \cdots < t_{m-1} < t_m = b \}$ of $[a,b]$,
			\[ v(\gamma;\mathcal{P}) = \sum_{k=1}^m |\gamma(t_k) - \gamma(t_{k-1})| \le M. \]
			The \emph{total variation} $V(\gamma)$ of $\gamma$ is defined by
			\[ V(\gamma) = \sup \{ v(\gamma;\mathcal{P}) : \mathcal{P}\text{ is a partition of }[a,b] \}. \]
			Clearly, $V(\gamma) \le M < \infty$.
		\end{fdef}

		\begin{lemma}
			Let $\gamma : [a,b] \to \C$ be of bounded variation. Then,
			\begin{enumerate}
				\item If $\mathcal{P},\mathcal{Q}$ are partitions of $[a,b]$ with $\mathcal{P} \subseteq \mathcal{Q}$, then $v(\gamma;\mathcal{P}) \le v(\gamma;\mathcal{Q})$.
				\item If $\sigma : [a,b] \to \C$ is also of bounded variation and $\alpha,\beta\in\C$, then $\alpha\gamma + \beta\sigma$ is of bounded variation and
				\[ V(\alpha\gamma + \beta\sigma) \le |\alpha| V(\gamma) + |\beta| V(\sigma). \]
			\end{enumerate}
		\end{lemma}
		We omit the proof of the above, which is direct on using the triangle inequality on the definition of $v(\gamma;\mathcal{P})$.

		\begin{lemma}
			If $\gamma : [a,b] \to \C$ is piecewise smooth, $\gamma$ is of bounded variation and
			\[ V(\gamma) = \int_a^b |\gamma'(t)| \dif t. \]
		\end{lemma}
		\begin{proof}
			It suffices to show the required in the case where $\gamma$ is smooth, since in general we can consider the refinement of any partition that splits along the pieces along which $\gamma$ is smooth.\\
			The right hand side is well-defined since $\gamma'$ is continuous. Let $\mathcal{P} = \{ a = t_0 < t_1 < \cdots < t_{m-1} < t_m = b \}$. By definition,
			\begin{align*}
				v(\gamma,\mathcal{P}) &= \sum_{k=1}^m |\gamma(t_k) - \gamma(t_{k-1})| \\
					&= \sum_{k=1}^m \left| \int_{t_{k-1}}^{t_k} \gamma'(t) \dif t \right| \\
					&\le \sum_{k=1}^m \int_{t_{k-1}}^{t_k} |\gamma'(t)| \dif t = \int_a^b |\gamma'(t)| \dif t.
			\end{align*}
			Therefore, $V(\gamma) \le \int_a^b |\gamma'(t)| \dif t$, so $\gamma$ is of bounded variation.\\
			Since $\gamma'$ is continuous, it is uniformly continuous. So, if $\epsilon > 0$, we may choose $\delta_1 > 0$ such that
			\[ |s-t| < \delta_1 \implies |\gamma'(s) - \gamma'(t)| < \epsilon. \]
			Also, let $\delta_2 > 0$ such that if $\norm{P} < \delta_2$, then
			\[ \left| \int_a^b |\gamma'(t)| \dif t - \sum_{k=1}^m |\gamma'(\tau_k)| (t_k - t_{k-1}) \right| < \epsilon, \]
			where $\tau_k$ is any point in $[t_{k-1},t_k]$. Therefore,
			\begin{align*}
				\int_a^b |\gamma'(t)| \dif t &\le \epsilon + \sum_{k=1}^{m} |\gamma'(t_k)| (t_k - t_{k-1}) \\
					&= \epsilon + \sum_{k=1}^m \left| \int_{t_{k-1}}^{t_k} \gamma'(\tau_k) \dif t \right| \\
					&\le \epsilon + \sum_{k=1}^m \left| \int_{t_{k-1}}^{t_k} (\gamma'(\tau_k) - \gamma'(t)) \dif t \right| + \sum_{k=1}^m \left| \int_{t_{k-1}}^{t_k} \gamma'(t) \dif t \right|.
			\end{align*}
			If $\norm{P} < \delta = \min(\delta_1,\delta_2)$, then $|\gamma'(\tau_k) - \gamma'(t)| < \epsilon$ for all $t \in [t_{k-1},t_k]$ and
			\begin{align*}
				\int_a^b |\gamma'(t) \dif t| &\le \epsilon + \epsilon(b-a) + \sum_{k=1}^{m} |\gamma(t_k) - \gamma(t_{k-1})| \\
					&= \epsilon(1 + b-a) + V(\gamma;P) \le \epsilon(1 + b-a) + V(\gamma),
			\end{align*}
			so we are done since $1+b-a > 0$ is finite and $\epsilon$ can be made arbitrarily small.
		\end{proof}

		\begin{ftheo}
			Let $\gamma : [a,b] \to \C$ be of bounded variation and suppose that $f : [a,b] \to \C$ is continuous. Then, there exists a (unique) complex number $\mathcal{I}$ such that for every $\epsilon > 0$, there exists $\delta > 0$ such that when $\mathcal{P} = \{ t_0 < t_1 < \cdots < t_m \}$ is a partition of $[a,b]$ with $\norm{P} = \max_{1 \le k \le m} (t_k - t_{k-1}) < \delta$,
			\[ \left| \mathcal{I} - \sum_{k=1}^m f(\tau_k) (\gamma(t_k) - \gamma(t_{k-1})) \right| < \epsilon \]
			for any choice of points $\tau_k$ with $t_{k-1} \le \tau_k \le t_k$.\\
			This $\mathcal{I}$ is called the integral of $f$ with respect to $\gamma$ over $[a,b]$ and is denoted by
			\[ \mathcal{I} = \int_a^b f \dif \gamma = \int_a^b f(t) \dif\gamma(t). \]
		\end{ftheo}
		\begin{proof}
			First of all, note that it suffices to consider the case where $\gamma$ is real-valued, since we can write $\gamma = \gamma_1 + \iota \gamma_2$, where $\gamma_1,\gamma_2$ are real-valued, to get two integrals $\mathcal{I}_1,\mathcal{I}_2$ (for $\gamma_1,\gamma_2$ respectively), and finally use the triangle inequality to get $\mathcal{I} = \mathcal{I}_1 + \iota\mathcal{I}_2$.\\
			Since $f$ is continuous, it is uniformly continuous. We can (inductively) find positive numbers $\delta_1 > \delta_2 > \cdots $ such that if $|s-t| < \delta_m$, $|f(s) - f(t)| < 1/m$. For each $M \ge 1$, let $\mathcal{P}_m$ be the collection of all partitions $P$ of $[a,b]$ with $\norm{P} \le \delta_m$, so $\mathcal{P}_1 \supseteq \mathcal{P}_2 \supseteq \cdots \supseteq \mathcal{P}_m \supseteq \cdots$. Finally, define $F_m$ to be the closure of the set
			\[ \left\{ \sum_{k=1}^n f(\tau_k) (\gamma(t_k) - \gamma(t_{k-1})) : P \in \mathcal{P}_m \text{ and } t_{k-1} \le \tau_k \le t_k \right\}. \]
			Because $\mathcal{P}_1 \supseteq \mathcal{P}_2 \supseteq \cdots$, it follows trivially that
			\[ F_1 \supseteq F_2 \supseteq \cdots. \]
			We claim that
			\begin{align}
				\diam F_m &\le \frac{2}{m} V(\gamma). \label{eqn: diam-bound}
			\end{align}
			If we do this, then Cantor's Theorem (since $\C$ is complete) implies that there is precisely one complex number $\mathcal{I}$ such that $\mathcal{I} \in F_m$ for all $m \ge 1$. Then, for any $\epsilon > 0$, we may let $m > (2/\epsilon)V(\gamma)$ so $\epsilon > (2/m) V(\gamma) \ge \diam F_m$. Since $\mathcal{I} \in F_m$, $F_m \subseteq B(\mathcal{I},\epsilon)$. Therefore, $\delta = \delta_m$ gets the job done.\\
			So, we must show that
			\[ \diam \left\{ f(\tau_k) \left( \gamma(t_k) - \gamma(t_{k-1}) \right) : P \in \mathcal{P}_m \text{ and } t_{k-1} \le \tau_k \le t_k \right\} \le \frac{2}{m} V(\gamma). \]
			To do this, if $P = \{ t_0 < \cdots < t_n \}$ is a partition, denote by $S(P)$ a sum of the form $\sum f(\tau_k) \left( \gamma(t_k)  - \gamma(t_{k-1})\right)$ where $t_{k-1} \le \tau_k \le t_k$ for each $k$. Fixing $m \ge 1$, let $P \in \mathcal{P}_m$. If $P \subseteq Q$ (so $Q \in \mathcal{P}_m$ as well), then
			\[ |S(P) - S(Q)| < \frac{1}{m} V(\gamma). \]
			We only show this in the case where $Q$ is obtained from $P$ by adding a single extra partition point (the general case follows similarly). Let $Q = \{ t_0 < t_1 < \cdots < t_{p-1} < t^* < t_p < \cdots t_n \}$. If $t_{p-1} \le \sigma \le t^*$ and $t^* \le \sigma' \le t_p$. Then,
			\[ S(Q) = \sum_{k \ne p} f(\sigma_k) ( \gamma(t_k) - \gamma(t_{k-1}) ) + f(\sigma) \left( \gamma(t^*) - \gamma(t_{p-1}) \right) + f(\sigma') \left( \gamma(t_p) - \gamma(t^*) \right). \]
			Then, using the definition of $\delta_m$,
			\begin{align*}
				\left| S(P) - S(Q) \right| &= \left| \sum_{k\ne p} \left( f(\tau_k) - f(\sigma_k) \right) \left( \gamma(t_k) - \gamma(t_{k-1}) \right) \right. \\
				&\qquad\left. + f(\tau_p) (\gamma(t_p) - \gamma(t_{p-1})) - f(\sigma)(\gamma(t^*) - \gamma(t_{p-1})) - f(\sigma')( \gamma(t_p) - \gamma(t^*) ) \right| \\
				&\le \frac{1}{m} \sum_{k \ne p} |\gamma(t_k) - \gamma(t_{k-1}| + \left| \left( f(\tau_p) - f(\sigma) \right) \left( \gamma(t^*) - \gamma(t_{p-1}) \right) + \left( f(\tau_p) - f(\sigma') \right) \left( \gamma(t_p) - \gamma(t^*) \right) \right| \\
				&\le \frac{1}{m} \sum_{k \ne p} \left| \gamma(t_k) - \gamma(t_{k-1}) \right| + \frac{1}{m} \left| \gamma(t^*) - \gamma(t_{p-1}) \right| + \frac{1}{m} \left| \gamma(t_p) - \gamma(t^*) \right| \\
				&\le \frac{1}{m} V(\gamma).
			\end{align*}

			Next, let $P,R$ be any two partitions in $\mathcal{P}_m$, and $Q = P \cup R$ a partition that contains $P$ and $R$. Using the first part,
			\[ |S(P) - S(Q)| \le |S(P) - S(Q)| + |S(Q) - S(R)| \le \frac{2}{m} V(\gamma). \]
			It follows that the diameter of the set of interest is at most $(2/m) V(\gamma)$, completing the proof.
		\end{proof}

		\begin{ftheo}
			Let $f,g$ be continuous functions on $[a,b]$ and let $\gamma,\sigma$ be functions of bounded variation on $[a,b]$. Then for any scalars $\alpha,\beta$,
			\begin{align*}
				\int_a^b (\alpha f + \beta g) \dif \gamma &= \alpha \int_a^b f \dif \gamma + \beta \int_a^b g \dif \gamma \\
				\int_a^b f \dif( \alpha\gamma + \beta\sigma ) &= \alpha\int_a^b f \dif \gamma + \beta\int_a^b f \dif \sigma.
			\end{align*}
		\end{ftheo}

		% \begin{proof}
		% 	It follows near-directly that $\int_a^b \alpha f \dif \gamma = \alpha \int_a^b f \dif \gamma$ (just use $|\alpha|\epsilon$ instead of $\epsilon$ in the definition). So, to show \Cref{eqn: scalar-addition-closure of fn int}, it suffices to show it for $\alpha = \beta = 1$. To do this, use $\epsilon/2$ in the definitions of integrability, and take $\delta = \min\{\delta_1,\delta_2\}$ (where $\delta_1,\delta_2$ are the values obtained from the integrability of $f,g$).\\
		% 	Similarly, $\int_a^b f \dif (\alpha \gamma) = \alpha \int_a^b f \dif \gamma$ and $\int_a^b f \dif(\gamma+\sigma) = \int_a^b f \dif \gamma + \int_a^b f \dif \sigma$.
		% \end{proof}

		\begin{prop}
			\label{lemma: can split integral}
			Let $\gamma : [a,b] \to \C$ be of bounded variation and let $f : [a,b] \to \C$ be continuous. If $a = t_0 < t_1 < \cdots < t_{n-1} < t_n = b$, then
			\[ \int_a^b f \dif \gamma = \sum_{k=1}^n \int_{t_{k-1}}^{t_k} f \dif \gamma. \]
		\end{prop}

		We omit the proofs of the above.

		\begin{ftheo}
			If $\gamma$ is piecewise smooth and $f : [a,b] \to \C$ is continuous, then $\int_a^b f \dif \gamma = \int_a^b f(t) \gamma'(t) \dif t$.
		\end{ftheo}
		\begin{proof}
			It suffices to consider the case where $\gamma$ is smooth by \Cref{lemma: can split integral}. Also, by looking at the real and imaginary parts of $\gamma$ separately, it suffices to consider the case where $\gamma$ is real-valued on $[a,b]$. Let $\epsilon > 0$ and choose $\delta > 0$ such that if $P = \{ a = t_0 < t_1 < \cdots < t_n = b \}$ has $\norm{P} < \delta$, then
			\[ \left| \int_a^b f \dif \gamma - \sum_{k=1}^n f(\tau_k) (\gamma(t_k) - \gamma(t_{k-1})) \right| < \epsilon/2 \]
			and
			\[ \left| \int_a^b f(t) \gamma'(t) \dif t - \sum_{k=1}^n f(\tau_k) \gamma'(\tau_k) (t_k - t_{k-1}) \right| < \epsilon/2 \]
			for any $t_{k-1} \le \tau_k \le t_k$ for each $k$.\\
			Applying the mean value theorem on $\gamma$ (this requires that $\gamma$ be real-valued), one gets that there exists $\tau_k \in [t_{k-1},t_k]$ for each $k$ such that
			\[ \gamma'(\tau_k) = \frac{\gamma(t_k) - \gamma(t_{k-1})}{t_k - t_{k-1}}. \]
			Using these $\tau_k$ specifically, 
			\[ \left| \int_a^b f \dif \gamma - \sum_{k=1}^n f(\tau_k) \gamma'(\tau_k) (t_k - t_{k-1}) \right| < \epsilon/2, \]
			so
			\[ \left| \int_a^b f \dif \gamma - \int_a^b f(t) \gamma'(t) \dif t \right| < \epsilon, \]
			completing the proof.
		\end{proof}

\subsection{Integrals On Curves}

	\begin{fdef}
		$\gamma:[a,b]\to\C$ is called a \emph{rectifiable path} if it is continuous and of bounded variation. Note that if $\gamma$ is piecewise smooth, then it is rectifiable and its length is
		\[ \int_a^b |\gamma'(t)| \dif t = V(\gamma). \]
	\end{fdef}

	\begin{fdef}
		If $\gamma : [a,b] \to \C$ is a rectifiable path and $f$ is a function continuous on $\{\gamma\}$, then the \emph{(line) integral} of $f$ along $\gamma$ is
		\[ \int_a^b f(\gamma(t)) \dif \gamma(t). \]
		This line integral is also denoted as
		\[ \int_\gamma f = \int_\gamma f(z) \dif z. \]
	\end{fdef}

	For example, if $\gamma:[0,2\pi] \to \C$ as $\gamma(t) = e^{\iota t}$,
	\[ \int_\gamma \frac{1}{z} \dif z = \int_0^{2\pi} e^{-\iota t} (\iota e^{\iota t}) \dif t = 2 \pi \iota. \]
	and
	\[ \int_\gamma z^m \dif z = \int_0^{2\pi} e^{\iota m t} (\iota e^{\iota t}) \dif t = \iota \int_0^{2\pi} \cos((m+1)t) \dif t - \int_0^{2\pi} \sin((m+1)t) \dif t = 0. \]

	\begin{theorem}
		If $\gamma : [a,b] \to \C$ is a rectifiable path and $\varphi : [c,d] \to [a,b]$ is a continuous non-decreasing function with $\varphi(c) = a, \varphi(d) = b$, then for any function $f$ continuous on $\gamma$,
		\[ \int_\gamma f = \int_{\gamma \circ \varphi} f. \]
	\end{theorem}

	\begin{remark}
		The above uses the fact that $\gamma \circ \varphi$ is also rectifiable (Why is this true?).
	\end{remark}

	\begin{proof}
		Let $\epsilon > 0$ and choose $\delta_1 > 0$ such that for a partition $\{ s_0 < s_1 < \cdots  < s_n \}$ of $[c,d]$ with $(s_{k} - s_{k-1}) < \delta_1$ and any $s_{k-1} \le \sigma_k \le s_k$,
		\[ \left| \int_{\gamma \circ \varphi} f - \sum_{k=1}^{n} f((\gamma \circ \varphi)(s_k)) - f((\gamma \circ \varphi)(s_{k-1})) \right| < \epsilon/2. \]
		Similarly, choose $\delta_2 > 0$ such that for a partition $\{ t_0 < t_1 < \cdots < t_m \}$ of $[a,b]$ with $(t_k - t_{k-1}) < \delta_2$ and $t_{k-1} \le \tau_k \le t_k$,
		\[ \left| \int_{\gamma} f - \sum_{k=1}^{m} f(\gamma(t_k)) - f(\gamma(t_{k-1})) \right| < \epsilon/2. \]
		Since $\varphi$ is uniformly continuous on $[c,d]$, there exists $\delta > 0$ less than $\delta_1$ such that $|\varphi(s) - \varphi(t)| < \delta_2$ whenever $|s-t| < \delta$. So, if $\{ s_0 < s_1 < \cdots < s_n \}$ is a partition of $[c,d]$ with $(s_k - s_{k-1}) < \delta < \delta_1$ and $t_k = \varphi(s_k)$, then $\{ t_0 < t_1 < \cdots < t_n \}$ is a partition of $[a,b]$ with $(t_k - t_{k-1}) < \delta_2$. If $s_{k-1} \le \sigma_k \le s_k$ and $\tau_k = \varphi(\sigma_k)$, then we can use the two earlier inequalities to conclude that
		\[ \left| \int_\gamma f - \int_{\gamma \circ \varphi} f \right| < \epsilon, \]
		completing the proof.
	\end{proof}

	\begin{definition}
		Let $\gamma : [a,b] \to \C$ be a rectifiable path, and for $a \le t \le b$, set $|\gamma|(t) = V(\gamma;[a,t])$. That is,
		\[ |\gamma|(t) = \sup\left\{ \sum_{k=1}^{n} |\gamma(t_k) - \gamma(t_{k-1})| : \{ t_0 < t_1 < \cdots < t_n \}\text{ is a partition of }[a,t] \right\}. \]
		Clearly, $|\gamma|$ is increasing on $[a,b]$ and of bounded variation. In fact, $V(|\gamma|;[a,b]) = |\gamma|(b) - |\gamma|(a)$. If $f$ is continuous on $[a,b]$, define
		\[ \int f |{\dif} {z}| = \int_a^b f(\gamma(t)) \dif |\gamma|(t). \]
	\end{definition}

	\begin{ftheo}
		Let $\gamma : [a,b] \to \C$ be a rectifiable curve and suppose that $f$ is a function continuous on $\{\gamma\}$. Then,
		\begin{equation}
			\label{eqn: 2.2}
			\int_\gamma f = - \int_{-\gamma} f
		\end{equation}
		where $(-\gamma)(t) = \gamma(a+b-t)$,
		\begin{equation}
			\label{eqn: 2.3}
			\left| \int_\gamma f \right| \le \int_\gamma |f| |{\dif} {z}| \le V(\gamma) \sup\{ |f(z)| : z \in \{\gamma\} \},
		\end{equation}
		and for $c \in \C$,
		\begin{equation}
			\label{eqn: 2.4}
			\int_\gamma f(z) \dif z = \int_{\gamma + c} f(z-c) \dif z.
		\end{equation}
	\end{ftheo}
	\begin{proof}
		\Cref{eqn: 2.2,eqn: 2.4} follow near-directly from the definition, so we prove only \Cref{eqn: 2.3}. Let $\epsilon > 0$. Then, there exists $\delta > 0$ such that if $P = \{ t_0 < t_1 < \cdots t_n \}$ is a partition of $[a,b]$ with $\norm{P} < \delta$, then
		\[ \left| \left| \int_\gamma f(z) \dif z \right| - \left| \sum_{k=1}^n f(\gamma(\tau_k)) (\gamma(t_k) - \gamma(t_{k-1})) \right| \right| \le \left| \int_\gamma f(z) \dif z - \sum_{k=1}^n f(\gamma(\tau_k)) (\gamma(t_k) - \gamma(t_{k-1})) \right| < \epsilon/2 \]
		for any $t_{k-1} \le \tau_k \le t_k$.
		That is,
		\begin{align*}
			\left| \int_\gamma f(z) \dif z \right| &< \left| \sum_{k=1}^n f(\gamma(\tau_k)) (\gamma(t_k) - \gamma(t_{k-1})) \right| + \epsilon/2 \\
			&\le  \sum_{k=1}^n \left| f(\gamma(\tau_k)) \right| \left|\gamma(t_k) - \gamma(t_{k-1}) \right| + \epsilon/2.
		\end{align*}
		We may also assume that for this same $\delta$,
		\[ \sum_{k=1}^n |f(\gamma(t_k))| (|\gamma|(t_k) - |\gamma|(t_{k-1})) < \int_\gamma |f(z)| |{\dif} {z}| + \epsilon/2 . \]
		Recall that $|\gamma|(t)$ is an increasing function. So,
		\[ |\gamma|(t_k) - |\gamma|(t_{k-1}) \ge |\gamma(t_k) - \gamma(t_{k-1})| \]
		Therefore,
		\begin{align*}
			\left| \int_\gamma f(z) \dif z \right| &< \sum_{k=1}^n |f(\gamma(\tau_k))| \left( |\gamma|(t_k) - |\gamma|(t_{k-1}) \right) + \epsilon/2 \\
				&< \int_\gamma |f(z)| |{\dif} {z}| + \epsilon.
		\end{align*}
		It follows that
		\[ \left| \int_\gamma f(z) \dif z \right| \le \int_\gamma |f(z)| |{\dif} {z}|. \]
		To conclude the proof, note that
		\[ \int_\gamma |{\dif} {z}| = |\gamma|(b) - |\gamma|(a) = |\gamma|(b) = V(\gamma), \]
		so
		\[ \int_\gamma |f(z)| |{\dif} {z}| \le V(\gamma) \sup_{z \in \{\gamma\}} |f(z)|. \]
	\end{proof}

	\begin{flem}
		\label{lemma: polygonal}
		If $G$ is an open set in $\C$, $\gamma : [a,b] \to G$ is a rectifiable path, and $f : G \to \C$ is continuous, then for every $\epsilon > 0$ there exists a polygonal path $\Gamma$ in $G$ such that $\Gamma(a) = \gamma(a)$, $\Gamma(b) = \gamma(b)$, and
		\[ \left| \int_\gamma f - \int_\Gamma f \right| < \epsilon \]
	\end{flem}
	\begin{proof}
		We prove the result in the case where $G$ is an open disk. In the general case where $G$ need not be a disk, since $\{\gamma\}$ is compact, there exists a number $r$ with $0 < r < d(\{\gamma\},\partial G)$. Choose $\delta > 0$ such that $|\gamma(s) - \gamma(t)| < r$ when $|s-t| < \delta$. The idea is that we shall take several smaller disks and stitch together the polygonal paths on each of these sections.\\
		If $P = \{ t_0 < t_1 < \cdots < t_n \}$ is a partition of $[a,b]$ with $\norm{P} < \delta$, then $|\gamma(t_k) - \gamma(t_{k-1})| < r$ for $t_{k-1} \le t \le t_k$. That is, if $\gamma_k : [t_{k-1}, t_k] \to G$ is defined by $\gamma_k(t) = \gamma(t)$, then $\{\gamma_k\} \subseteq B(\gamma(t_{k-1}),r)$ for $1 \le k \le n$. Getting a polygonal path $\Gamma_k$ for each $k$ such that
		\[ \left| \int_{\gamma_k} f - \int_{\Gamma_k} f \right| < \epsilon/n, \]
		defining $\Gamma(t) = \Gamma_k(t)$ on $[t_{k-1},t_k]$ does the job.
		\\

		Now, let us prove the result in the disk case.\\
		Because $\{\gamma\}$ is a compact set, $d = d(\{\gamma\},\partial G) > 0$. It follows that if $G = B(c,r)$, then $\{\gamma\} \subseteq B(c,\rho)$ where $\rho = r - d/2$.\\
		Now, observe that $f$ is uniformly continuous on $\overline{B}(c,\rho) \subseteq G$. Thus, we may assume without loss of generality that $f$ is uniformly continuous on $G$. Now, choose $\delta > 0$ such that if $|z-w| < \delta$, then $|f(z) - f(w)| < \epsilon$. If $\gamma : [a,b] \to G$, then $\gamma$ is uniformly continuous so there is a partition $P = \{t_0 < t_1 < \cdots < t_n\}$ of $[a,b]$ such that if $t_{k-1} \le s,t \le t_k$, $|\gamma(s) - \gamma(t)| < \delta$, and such that for $t_{k-1} \le \tau_k \le t_k$,
		\[ \left| \int_\gamma f - \sum_{k=1}^n f(\gamma(\tau_k)) (\gamma(t_k) - \gamma(t_{k-1})) \right| < \epsilon. \]
		Now, define $\Gamma : [a,b] \to G$ by
		\[ \Gamma(t) = \frac{(t_k - t) \gamma(t_{k-1}) + (t - t_{k-1}) \gamma(t_k)}{t_k - t_{k-1}} \]
		if $t_{k-1} \le t \le t_k$. This is the polygonal path we shall consider. Indeed,
		\begin{align*}
			\Gamma(t) - \gamma(\tau_k) &= \frac{t_k - t}{t_k - t_{k-1}} (\gamma(t_{k-1}) - \gamma(\tau_k)) + \frac{t - t_{k-1}}{t_k - t_{k-1}} (\gamma(t_k) - \gamma(\tau_k)),
		\end{align*}
		so
		\begin{align*}
			|\Gamma(t) - \gamma(\tau_k)| &\le \left| \frac{t_k - t}{t_k - t_{k-1}} \right| \left| \gamma(t_{k-1}) - \gamma(\tau_k) \right| + \left| \frac{t - t_{k-1}}{t_k - t_{k-1}} \right| \left| \gamma(t_k) - \gamma(\tau_k) \right| \\
				&\le \left| \gamma(t_{k-1}) - \gamma(\tau_k) \right| + \left| \gamma(t_k) - \gamma(\tau_k) \right| < 2\delta.
		\end{align*}
		Thus,
		\begin{align*}
			\int_\Gamma f &= \int_a^b f(\Gamma(t)) \Gamma'(t) \dif t \\
				&= \sum_{k-1}^n \frac{\gamma(t_k) - \gamma(t_{k-1})}{t_k - t_{k-1}} \int_{t_{k-1}}^{t_k} f(\Gamma(t)) \dif t
		\end{align*}
		and
		\begin{align*}
			\left| \int_\gamma f - \int_\Gamma f \right| &= \left| \int_\gamma f - \sum_{k=1}^n f(\gamma(\tau_k)) (\gamma(t_k) - \gamma(t_{k-1})) \right| + \left| \sum_{k=1}^n f(\gamma(\tau_k)) (\gamma(t_k) - \gamma(t_{k-1})) - \int_\Gamma f \right| \\
				&\le \epsilon + \left| \sum_{k=1}^n f(\gamma(\tau_k)) (\gamma(t_k) - \gamma(t_{k-1})) - \int_\Gamma f \right| \\
				&\le \epsilon + \sum_{k=1}^n \frac{|\gamma(t_k) - \gamma(t_{k-1})|}{t_k - t_{k-1}} \int_{t_{k-1}}^{t_k} |f(\Gamma(t)) - f(\gamma(\tau_k))| \dif t \\
				&\le \epsilon + \epsilon \sum_{k=1}^n |\gamma(t_k) - \gamma(t_{k-1})| \\
				&\le \epsilon (1 + V(\gamma)),
		\end{align*}
		which can be made arbitrarily small, thus completing the proof.
	\end{proof}

	The following can be thought of as an analogue of the Fundamental Theorem of Calculus for complex functions.

	\begin{ftheo}
		\label{theo: ftc-like}
		Let $G$ be open in $\C$ and $\gamma$ be a rectifiable path in $G$ with initial and end points $\alpha,\beta$ respectively. If $f : G \to \C$ is a continuous function with a primitive $F : G \to \C$ ($F$ is differentiable and $F' = f$), then
		\[ \int_\gamma f = F(\beta) - F(\alpha). \]
	\end{ftheo}
	\begin{proof}
		When $\gamma : [a,b] \to \C$ is piecewise smooth,
		\begin{align*}
			\int_\gamma f &= \int_a^b f(\gamma(t)) \gamma'(t) \dif t \\
				&= \int_a^b F'(\gamma(t)) \gamma'(t) \dif t \\
				&= \int_a^b (F \circ \gamma)'(t) \dif t \\
				&= (F \circ \gamma) (b) - (F \circ \gamma) (a) & \text{(by the Fundamental Theorem of Calculus)} \\
				&= F(\beta) - F(\alpha).
		\end{align*}
		In general, we may use \Cref{lemma: polygonal}. For $\epsilon > 0$, let $\Gamma$ be a polygonal path of the described form. Since $\Gamma$ is piecewise smooth, $\int_\Gamma f = F(\beta) - F(\alpha)$, so
		\[ \left| \int_\gamma f - (F(\beta) - F(\alpha)) \right| < \epsilon. \]
		Since $\epsilon$ was chosen arbitrarily, the desideratum follows.
	\end{proof}

	The fundamental theorem of calculus says that each continuous function has a primitive. However, this is not true for functions of complex variables. For example, letting $f(z) = |z|^2$, if $F$ is a primitive of $f$, then $F$ is analytic. So, if $F = U + \iota V$, $x^2 + y^2 = F'(x+\iota y)$. Consequently,
	\begin{align*}
		\dpd{U}{x} &= \dpd{V}{y} = x^2 + y^2 \\
		\dpd{U}{y} &= \dpd{V}{x} = 0.
	\end{align*}
	However, $\pd{U}{y} = 0$ implies that $U(x,y) = u(x)$ for some function $u$, which implies that $u'(x) = x^2 + y^2$, a contradiction.

\subsection{Power series representation of analytic functions}

	Recall the following result which we had used in the proof of \Cref{theo: open disk harmonic conjugate}.

	\begin{ftheo}
		Let $\varphi : [a,b] \times [c,d] \to \C$ be a continuous function and defined $g : [c,d] \to \C$ by
		\[ g(t) = \int_a^b \varphi(s,t) \dif s. \]
		Then $g$ is continuous. Moreover, if $\pd{\varphi}{t}$ exists and is a continuous function on $[a,b] \times [c,d]$, then $g$ is continuously differentiable and
		\[ g'(t) = \int_a^b \dpd{\varphi}{t}(s,t) \dif s. \]
	\end{ftheo}

	This is referred to as the Leibniz rule.\\

	For example, this may be used to prove that if $|z| < 1$,
	\[ \int_0^{2\pi} \frac{e^{\iota s}}{e^{\iota s} - z} = 2 \pi. \]
	To do so, let $\varphi(s,t) = e^{\iota s} / (e^{\iota s} - tz)$ for $0 \le t \le 1$ and $0 \le s \le 2\pi$. Observe that $\varphi$ is continuously differentiable since $|z| < 1$. Thus,
	\[ g(t) = \int_0^{2\pi} \varphi(s,t) \dif s \]
	is continuously differentiable. Since $\varphi(s,0) = 1$, $g(0) = 2\pi$. Now,
	\begin{align*}
		g'(t) &= \int_0^{2\pi} \dpd{\varphi}{t}(s,t) \dif s \\
			&= \int_0^{2\pi} \frac{ze^{\iota s}}{(e^{\iota s} - tz)^2} \dif s.
	\end{align*}
	For fixed $t$, $\Phi(s) = z\iota/(e^{\iota s} - tz)$ satisfies
	\[ \Phi'(s) = -\frac{\iota z}{(e^{\iota s} - tz)^{2}} \cdot \iota e^{\iota s} = \frac{ze^{\iota s}}{(e^{\iota s} - tz)^2}. \]
	Therefore, $g'(t) = \Phi(2\pi) - \Phi(0) = 0$, so $g$ is a constant and $g(t) = g(0) = 2\pi$ for any $t$, $1$ in particular.

	\begin{ftheo}
		Let $f : G \to \C$ be analytic and suppose that $\overline{B(a,r)} \subseteq G$ for some $r > 0$. If $\gamma(t) = a + re^{\iota t}$ for $ 0 \le t \le 2\pi$, then
		\[ f(z) = \frac{1}{2\pi\iota} \int_\gamma \frac{f(w)}{w - z} \dif w \]
		for $|z-a| < r$.
	\end{ftheo}
	\begin{proof}
		Defining $G_1 = \left\{ (z-a)/r : z \in G \right\}$ and $g(z) = f(a + r z)$, it suffices to consider the case where $a = 0$ and $r = 1$.\\
		Fix $z$ with $|z| < 1$. It must be shown that
		\[ f(z) = \int_{2\pi\iota} \int_\gamma \frac{f(w)}{w-z} \dif w = \frac{1}{2\pi} \int_0^{2\pi} \frac{f(e^{\iota s}) e^{\iota s}}{e^{\iota s} - z} \dif s. \]
		That is, we want to show that
		\begin{align*}
			 0 &= \int_0^{2\pi} \frac{f(e^{\iota s}) e^{\iota s}}{e^{\iota s} - z} \dif s - 2 \pi f(z) \\
			 	&= \int_0^{2\pi} \left(\frac{f(e^{\iota s}) e^{\iota s}}{e^{\iota s} - z} - f(z)\right) \dif s.
		\end{align*}
		For this, let
		\[ \varphi(s,t) = \frac{f(z + t(e^{\iota s} - z)) e^{\iota s}}{e^{\iota s} - z} - f(z) \]
		for $0 \le t \le 1$ and $0 \le s \le 2\pi$, and
		\[ g(t) = \int_0^{2\pi} \varphi(s,t) \dif s. \]
		We wish to show that $g(1) = 0$. Observe that
		\[ g(0) = \int_0^{2\pi} \frac{f(z) e^{\iota s}}{e^{\iota s} - z} - f(z) \dif s = f(z) \int_0^{2\pi} \frac{e^{\iota s}}{e^{\iota s} - z} \dif s - 2\pi f(z) = 0. \]
		Also,
		\begin{align*}
			g'(t) &= \int_0^{2\pi} \dpd{\varphi}{t}(s,t) \dif s \\
				&= \int_0^{2\pi} \frac{e^{\iota s}}{e^{\iota s} - z} f'(z + t(e^{\iota s} - z)) (e^{\iota s} - z) \dif s \\
				&= \int_0^{2\pi} e^{\iota s} f'(z + t(e^{\iota s - z})) \dif s \\
				&= \frac{1}{t} f(z + t(e^{\iota s} - z)) \Biggr|_{s=0}^{s=2\pi} \\
				&= 0,
		\end{align*}
		completing the proof.
	\end{proof}

	If $|z-a| < r$ and $w$ is such that $|w-a| = r$, then
	\[ \frac{1}{w-z} = \frac{1}{w-a} \cdot \frac{1}{1 - \frac{z-a}{w-a}} = \frac{1}{w-a} \sum_{i=0}^{\infty} \left( \frac{z-a}{w-a} \right)^i. \]
	since $|z-a| < |w-a|$.\\
	Now, multiplying by $f(w)/2\pi\iota$ and integrating around the circle $\gamma$ defined by $|w-a|=r$, we get that
	\[ f(z) = \int_\gamma \frac{f(w)}{2\pi\iota} \sum_{i=0}^\infty \frac{(z-a)^{i}}{(w-a)^{i+1}} \dif w. \]
	But how do we simplify the right hand side? We do not know (\emph{yet}) that the integral and summation may be switched. So, let us get to showing this.

	\begin{lemma}
		Let $\gamma$ be a rectifiable curve in $\C$ and suppose that $F_n$ and $F$ are continuous functions on $\{\gamma\}$. If $(F_n)$ uniformly converges to $F$ on $\{\gamma\}$, then
		\[ \int_\gamma = \lim_{n\to\infty} \int_\gamma F_n. \]
	\end{lemma}
	\begin{proof}
		Let $\epsilon > 0$ and let $N \in \N$ such that
		\[ |F_n(w) - F(w)| < \frac{\epsilon}{V(\gamma)} \]
		for $n \ge N$. This implies that
		\[ \left| \int_\gamma F_n - \int_\gamma F \right| \le V(\gamma) \sup_{w} |F_n(w) - F(w)| \le \epsilon \]
		for $n \ge N$, completing the proof.
	\end{proof}

	\begin{ftheo}
		Let $f$ be analytic on $B(a,R)$. Then,
		\[ f(z) = \sum_{n=0}^\infty a_n(z-a)^n \]
		for all $|z-a| < R$, where $a_n = f^{(n)}(a)/n!$ and this series has radius of convergence at least $R$.
	\end{ftheo}
	\begin{proof}
		Let $0 < r < R$ such that $\overline{B(a,r)} \subseteq B(a,R)$. Let $\gamma(t) = a + re^{\iota t}$ ($0\le t\le 2\pi$). Since $|z-a| < r$, if $M = \max\{|f(w)| : |w-a| = r\}$,
		\[ \frac{|f(w)||z-a|^n}{|w-a|^{n+1}} \le \frac{M}{r} \left( \frac{|z-a|}{r} \right)^n. \]
		Since $|z-a| < r$,
		\[ \sum_{n=0}^{\infty} f(w) \frac{(z-a)^n}{(w-a)^{n+1}} \]
		converges uniformly for $w$ on $\{\gamma\}$. By the discussion before the previous lemma together with the lemma itself,
		\begin{equation}
			\label{eqn: unsimplified taylor}
			\tag{$*$}
			f(z) = \sum_{n=0}^{\infty} \left(\frac{1}{2\pi\iota} \int_\gamma \frac{f(w)}{(w-a)^{n+1}}\right) (z-a)^n.
		\end{equation}
		Since
		\[ a_n = \frac{1}{2\pi\iota} \int_\gamma \frac{f(w)}{(w-a)^{n+1}}. \]
		is independent of $z$, \eqref{eqn: unsimplified taylor} converges for $|z-a| < R$. However, we now know from \Cref{theo: power series rad conv}(c) that $a_n = f^{(n)}(a)/n!$, completing the proof.
	\end{proof}

	\begin{corollary}
		If $f$ is analytic,
		\[ f^{(n)}(a) = \frac{1}{2\pi\iota} \int_\gamma \frac{f(w)}{(w-a)^{n+1}} \dif w \]
		where $\gamma = a + re^{\iota t}$ and $r < R$, the radius of convergence of the series.
	\end{corollary}

	\begin{corollary}
		If $f:G\to\C$ is analytic, then $f$ is infinitely differentiable.
	\end{corollary}
	Indeed, this follows directly from the fact that
	\[ f^{(n)}(a) = \frac{n!}{2\pi\iota} \int_\gamma \frac{f(w)}{(w-a)^{n+1}} \dif w \]
	where $\gamma(t) = a+re^{\iota t}$ for $0 \le t \le 2\pi$.\\

	\begin{corollary}[Cauchy's Estimate]
		\label{theo: cauchys estimate}
		Let $f$ be analytic on $B(a,R)$ and suppose $|f(z)| \le M$ for all $z \in B(a,R)$. Then
		\[ |f^{(n)}(a)| \le \frac{n!M}{R^n}. \]
	\end{corollary}
	Indeed, the above applies with $r < R$ so we get that
	\[ |f^{(n)}(a)| \le \frac{n!}{2\pi} \int_\gamma \frac{|f(w)|}{|w-a|^{n+1}} |{\dif} {w}| \le \frac{n!}{2\pi} \cdot \frac{M}{r^{n+1}} \cdot 2\pi r = \frac{n!M}{r^n}. \]
	Since $r < R$ is arbitrary, we may let $r \to R^{-}$.

	\begin{prop}
		Let $f$ be analytic on the disk $B(a,R)$ and suppose that $\gamma$ is a closed rectifiable curve in $B(a,R)$. Then $\int_\gamma f = 0$.
	\end{prop}
	\begin{proof}
		Due to \Cref{theo: ftc-like}, it suffices to show that $f$ has a primitive. We know that
		\[ f(z) = \sum_{n=0}^{\infty} a_n (z-a)^n \]
		for $|z-a| < R$, where $a_n = f^{(n)}(a)/n!$. Consider the function
		\[ F(z) = (z-a) \sum_{n=0}^{\infty} \frac{a_n}{n+1} (z-a)^{n}. \]
		Since $\lim_{n\to\infty} (n+1)^{1/n} = 1$, this power series has the same radius of convergence as $\sum a_n (z-a)^n$. Therefore, $F$ is defined on $B(a,R)$. Moreover, $F'(z) = f(z)$ for $|z-a| < R$ by \Cref{theo: power series rad conv}(b), completing the proof.
	\end{proof}

	\begin{fdef}
		An \emph{entire} function is a function which is defined and analyitc on the whole complex plane $\C$.
	\end{fdef}

	\begin{prop}
		If $f$ is entire, then it has a power series expansion with infinite radius of convergence.
	\end{prop}
	Therefore, entire functions may be considered as polynomials of ``infinite degree''. Polynomials of finite non-zero degree are typically unbounded. These two insights lead to the following result.

	\begin{ftheo}[Liouville's Theorem]
		\label{liouvilles theorem}
		If $f$ is a bounded entire function, then $f$ is constant.
	\end{ftheo}
	\begin{proof}
		Suppose that $|f(z)| \le M$ for all $z \in \C$. We shall show that $f'(z) = 0$ for all $z \in \C$. By \nameref{theo: cauchys estimate}, since $f$ is analytic on any disk $B(z,R)$, $|f'(z)| \le M/R$. However, $R$ is arbitrary so $f'(z) = 0$ for any $z \in \C$.
	\end{proof}

	\begin{ftheo}[Fundamental Theorem of Algebra]
		If $p$ is a non-constant polynomial with coefficients in $\C$, then there exists $a \in \C$ with $p(a) = 0$.
	\end{ftheo}
	\begin{proof}
		Suppose $p(z) \ne 0$ for all $z \in \C$. Consider the entire function $f(z) = 1/p(z)$. This function is then bounded as $p(z)$ goes to $\infty$ as $z$ goes to infinity. By \nameref{liouvilles theorem}, $f$ (and thus $p$) is constant, which is a contradiction.
	\end{proof}

	Due to the above, $\C$ is an algebraically closed field.

	\begin{corollary}
		If $p(z)$ is a polynomial and its roots are $(p_j)$ with multiplicity $k_j$ (for $1\le j\le m$), then $p(z) = C (z-a_1)^{k_1} (z-a_2)^{k_2} \cdots (z-a_m)^{k_m}$ for some constant $C$, where $\sum k_j$ is the degree of $p$.
	\end{corollary}

	It is not too difficult to show that if $p(z)$ is a non-constant polynomial, then $p$ is a surjective analytic function on $\C$. However, we know that the map  $z \mapsto e^z$ is an entire function but there is no $b \in C$ such that $e^b = 0$. So, power series (``polynomials of infinite degree'') cannot be thought of in the same way as ordinary polynomials (of finite degree). However, we shall see later that given a non-constant entire function $f$, there exists at most one $a \in \C$ that is not in the image of $f$. This is referred to as Little Picard's Theorem.

	\begin{ftheo}
		Let $G$ be a connected open set and $f : G \to \C$ be analytic. Then, the following are equivalent statements.
		\begin{enumerate}[label=(\alph*)]
			\item $f$ is identically zero.
			\item There exists $a \in \C$ such that for all $n \ge 0$, $f^{(n)}(a) = 0$.
			\item $\{ z \in G : f(z) = 0 \}$ has a limit point in $G$.
		\end{enumerate}
	\end{ftheo}
	\begin{proof}
		Clearly, (a) implies (b) and (c).\\
		Next, let us show that (c) implies (b). Let $a \in G$ be a limit point of the zero set of $f$. Let $R > 0$ such that $B(a,R) \subseteq G$. Since $a$ is a limit point of $z$ and $f$ is continuous, $f(a) = 0$. Let $n \ge 1$ such that $f^{(k)}(a) = 0$ for $k < n$ and $f^{(n)}(a) \ne 0$. Expanding $f$ as a power series about $a$ gives that
		\[ f(z) = \sum_{k=n}^\infty a_k (z-a)^{k} \]
		for $|z-a| < R$ and $a_n \ne 0$. Let
		\[ g(z) = \sum_{k=n}^\infty a_k (z-a)^{k-n}. \]
		Since $g$ is continuous in $B(a,R)$ and $g(a) \ne 0$, let $r<R$ such that $g(z) \ne 0$ when $|z-a| < r$. Since $a$ is a limit point of $z$, there exists $b$ with $f(b) = 0$ and $0 < |a-b| < r$. This gives $0 = (b-a)^n g(b)$, so $g(b) = 0$, a contradiction. Therefore, no such $n$ can be found and (b) is true.\\
		Finally, let us show that (b) implies (a). Let
		\[ A = \{ z \in G : f^{(n)}(z) = 0 \text{ for all } n \ge 0 \}. \]
		By the definition of (b), $A \ne \ emptyset$. We shall show that $A$ is both open and closed in $G$, and by the connectedness of $G$ is follows that $A$ is the entirety of $G$. Showing that $A$ is closed is direct -- if $z \in \overline{A}$ and $(z_k)$ a sequence such that $z_k \to z$, then since each $f^{(k)}$ is continuous, $f^{(n)}(z) = \lim f^{(n)}(z_k) = 0$ for all $n \ge 0$, and so $z \in A$. On the other hand, if $a \in A$, we can write $f(z) = \sum_{n = 0}^\infty \frac{f^{(n)}(a)}{n!} (z-a)^n = 0$ on $B(a,R)$ (for some $R > 0$), so $B(a,R) \subseteq A$ and $A$ is open, completing the proof.
	\end{proof}

	\begin{corollary}
		If $f,g$ are analytic on a region $G$, then $f \equiv g$ iff $\{ z \in G : f(z) = g(z) \}$ has a limit point in $G$.
	\end{corollary}

	\begin{corollary}
		If $f$ is non-trivial and analytic on an open connected set $G$, then each zero of $f$ has finite multiplicity. More explicitly, for each $a \in G$ with $f(a) = 0$, there is an integer $n \ge 1$ and an analytic function $g : G \to \C$ such that $g(a) \ne 0$ and $f(z) = (z-a)^n g(z)$ for all $z \in G$.
	\end{corollary}
	\begin{proof}
		It is clear that there exists a largest $n \ge 1$ such that $f^{(k)}(a) = 0$ for all $k \le n-1$.
	\end{proof}

	\begin{corollary}
		If $f : G \to \C$ is non-trivial and analytic, and $a \in G$ with $f(a) = 0$, then there exists $R > 0$ such that $B(a,R) \subseteq G$, and $f(z) \ne 0$ for all $0 < |z-a| < R$. 
	\end{corollary}
	The above follows from the fact that the zeros of $f$ are isolated.

	\begin{ftheo}[Maximum Modulus Theorem]
		\label{theo: maximum modulus theorem}
		If $G$ is a region and $f:G\to\C$ is an analytic function such that there is a point $a\in G$ with $|f(a)| \ge |f(z)|$ for all $z \in G$, then $f$ is constant.
	\end{ftheo}
	That is, if $|f|$ attains its maximum, $f$ is constant.
	\begin{proof}
		Let $\overline{B(a,r)} \subseteq G$ and $\gamma(t) = a + re^{\iota t}$ for $0 \le t \le 2\pi$. Then,
		\begin{align*}
			f(a) &= \frac{1}{2\pi\iota} \int_\gamma \frac{f(w)}{w-a} \dif w \\
				&= \frac{1}{2\pi} \int_0^{2\pi} f(a + re^{\iota t}) \dif t.
		\end{align*}
		Therefore,
		\[ |f(a)| \le \frac{1}{2\pi} \int_0^{2\pi} |f(a+re^{\iota t})| \dif t \le |f(a)|. \]
		Therefore,
		\[ 0 = \int_0^{2\pi} \left( |f(a)| - |f(a+re^{\iota t})| \right) \dif t. \]
		Since the integrand is continuous is non-negative, $|f(a)| = |f(a+re^{\iota t})|$ for all $t \in [0,2\pi]$. If $f(a) = 0$, we are clearly done. Otherwise, since $r$ was arbitrary, $f$ maps any disk $B(a,R)$ to the circle $|z| = |f(a)|$. It may then be shown using the Cauchy-Riemann equations that $f$ is constant on $B(a,R)$ and is equal to $f(a)$ for all $|z-a| < R$. Therefore, $f(z) = f(a)$ for all $z \in G$ since the zeros of $f-f(a)$ are not isolated.
	\end{proof}

	Recall that
	\[ \int_\gamma \frac{1}{z-a} \dif z  = 2 \pi \iota n \]
	if $\gamma(t) = a + e^{\iota n t}$ for $t \in [0,2\pi]$. However, this property is not peculiar to the path $\gamma$, as shown by the following result.

	\begin{ftheo}
		If $\gamma : [0,1] \to \C$ is a closed rectifiable curve and $a \not\in \{\gamma\}$, then
		\[ \frac{1}{2\pi\iota} \int_\gamma \frac{1}{z-a} \dif z \]
		is an integer.
	\end{ftheo}
	\begin{proof}
		Using \Cref{lemma: polygonal}, we may assume that $\gamma$ is piecewise smooth (Why?). \\
		Let us assume that $\gamma$ is smooth.
		Define $g : [0,1] \to \C$ by
		\[ g(t) = \int_0^t \frac{\gamma'(s)}{\gamma(s) - a} \dif s. \]
		Then, $g(0) = 0$ and $g(1) = \int_\gamma 1/(z-a) \dif z$. We also have that
		\[ g'(t) = \frac{\gamma'(t)}{\gamma(t) - a} \]
		for $0 \le t \le 1$. This gives that
		\[ \od{}{t} \left(e^{-g(t)}(\gamma(t) - a)\right) = e^{-g(t)} \gamma'(t) - g'(t) e^{-g(t)} (\gamma(t)-a) = 0. \]
		Therefore,
		\[ e^{-g(0)} (\gamma(0) - a) = e^{-g(1)} (\gamma(1) - a). \]
		Because $\gamma(0) = \gamma(1)$ (the curve is closed) and $g(0) = 0$, $g(1) = 2\pi\iota n$ for some integer $n$.
		In the case where $\gamma$ is piecewise-smooth, we can define $g$ by integrating over each of the smooth intervals and the result follows near-identically.
	\end{proof}

	\begin{fdef}
		If $\gamma$ is a closed rectifiable curve in $\C$ then for $a \not\in \{\gamma\}$,
		\[ n(\gamma;a) = \frac{1}{2\pi\iota} \int_\gamma \frac{1}{z-a} \dif z \]
		is called the \emph{index} of $\gamma$ with respect to the point $a$. It is also sometimes referred to as the \emph{winding number} of $\gamma$ around $a$.
	\end{fdef}
	Recall the definition of $(-\gamma)$ from \eqref{eqn: 2.2}, also denoted $\gamma^{-1}$. If $\gamma$ and $\sigma$ are curves on $[0,1]$ with $\gamma(1) = \sigma(0)$, $\gamma+\sigma$ is the curve
	\[ (\gamma+\sigma)(t) = \begin{cases} \gamma(2t), & 0 \le t \le 1/2, \\ \sigma(2t-1), & 1/2 \le t \le 1. \end{cases} \]

	\begin{prop}
		If $\sigma,\gamma$ are closed rectifiable curves with the same initial (and final) points, then
		\begin{equation}
			n(\gamma;a) = -n(-\gamma;a)
		\end{equation}
		for all $a \not\in \{\gamma\}$ and
		\begin{equation}
			n(\gamma+\sigma;a) = n(\gamma;a) + n(\sigma;a)
		\end{equation}
		for all $a \not\in \{\sigma\} \cup \{\gamma\}$.
	\end{prop}
	We omit the proof of the above.\\

	The reason for $n(\cdot;\cdot)$ being called the winding number is clear from what happens in the case of a circle. For $a + e^{2\pi\iota n t}$, then $n(\gamma;a) = n$ is the number of times this curve ``winds'' or ``wraps'' around $a$. In fact, if $|b-a| < 1$, $n(\gamma;b) = n$ and if $|b-a| > 1$, $n(\gamma;b) = 0$.\\

	Recall that the components of a set are its maximal connected subsets.

	\begin{ftheo}
		Let $\gamma$ be a closed rectifiable curve in $\C$. Then $n(\gamma;a)$ is constant for $a$ belonging to a component of $G = \C \setminus \{\gamma\}$. Also, $n(\gamma;a) = 0$ for $a$ belonging to the unbounded component of $G$.
	\end{ftheo}
	\begin{remark}
		Since $\{\gamma\}$ is compact, the connected set $\{ z : |z| > R \} \subseteq G$ for sufficiently large $R$, so $\gamma$ has precisely one unbounded component.
	\end{remark}
	\begin{proof}
		Define $f : G \to \C$ by $f(a) = n(\gamma;a)$.
		If we manage to show that $f$ is continuous on $G$, we are done since the image of this map is a subset of the integers and each component is connected by definition, so $f$ is constant on each component.\\
		Recall that components of $G$ are open. Fix $a \in G$ and let $r = d(a , \{\gamma\}) > 0$. If $|a-b| < \delta \le r/2$ (we shall fix $\delta$ more precisely later), then
		\begin{align*}
			|f(a) - f(b)| &= \frac{1}{2\pi} \left| \int_\gamma \left( \frac{1}{z-a} - \frac{1}{z-b} \right) \dif z \right| \\
				&\le \frac{|a-b|}{2\pi} \int_\gamma \frac{1}{|z-a||z-b|} |{\dif} {z}|.
		\end{align*}
		By definition, $|z-a| \ge r$ for any $a \in \{\gamma\}$ and $|z-b| \ge |z-a| - |a-b| \ge r/2$. So,
		\begin{align*}
			|f(a) - f(b)| &\le \frac{|a-b|}{2\pi} \int_\gamma \frac{2}{r^2} |{\dif} {z}| \\
				&\le \frac{\delta}{\pi r^2} V(\gamma).
		\end{align*}
		For a given $\epsilon > 0$, setting $\delta = \min\{r/2, \epsilon\pi r^2 / V(\gamma)\}$ does the job, completing the first part of the proof.\\

		It remains to show that $\lim_{a\to\infty}f(a) = 0$ (Why does this imply the required?). Let $U$ be the unbounded component of $G$. For a given $R > 0$, let $a \in U$ such that $d(a;\gamma) > R$. Then,
		\[ |f(a)| = \frac{1}{2\pi} \int_\gamma \left|\frac{1}{z-a}\right| |{\dif} {z}| \le \frac{1}{2\pi R} \int_\gamma |{\dif} {z}| = \frac{V(\gamma)}{2\pi R}. \]
		$R$ can be made arbitrarily large (as $a\to\infty$), so we are done.
	\end{proof}
