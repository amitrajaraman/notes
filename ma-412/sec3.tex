\section{Integrals along closed curves}

	\subsection{Winding Number}

		Recall that
		\[ \int_\gamma \frac{1}{z-a} \dif z  = 2 \pi \iota n \]
		if $\gamma(t) = a + e^{\iota n t}$ for $t \in [0,2\pi]$. However, this property is not peculiar to the path $\gamma$, as shown by the following result.

		\begin{ftheo}
			If $\gamma : [0,1] \to \C$ is a closed rectifiable curve and $a \not\in \{\gamma\}$, then
			\[ \frac{1}{2\pi\iota} \int_\gamma \frac{1}{z-a} \dif z \]
			is an integer.
		\end{ftheo}
		\begin{proof}
			Using \Cref{lemma: polygonal}, we may assume that $\gamma$ is piecewise smooth (Why?). \\
			Let us assume that $\gamma$ is smooth.
			Define $g : [0,1] \to \C$ by
			\[ g(t) = \int_0^t \frac{\gamma'(s)}{\gamma(s) - a} \dif s. \]
			Then, $g(0) = 0$ and $g(1) = \int_\gamma 1/(z-a) \dif z$. We also have that
			\[ g'(t) = \frac{\gamma'(t)}{\gamma(t) - a} \]
			for $0 \le t \le 1$. This gives that
			\[ \od{}{t} \left(e^{-g(t)}(\gamma(t) - a)\right) = e^{-g(t)} \gamma'(t) - g'(t) e^{-g(t)} (\gamma(t)-a) = 0. \]
			Therefore,
			\[ e^{-g(0)} (\gamma(0) - a) = e^{-g(1)} (\gamma(1) - a). \]
			Because $\gamma(0) = \gamma(1)$ (the curve is closed) and $g(0) = 0$, $g(1) = 2\pi\iota n$ for some integer $n$.
			In the case where $\gamma$ is piecewise-smooth, we can define $g$ by integrating over each of the smooth intervals and the result follows near-identically.
		\end{proof}

		\begin{fdef}
			If $\gamma$ is a closed rectifiable curve in $\C$ then for $a \not\in \{\gamma\}$,
			\[ n(\gamma;a) = \frac{1}{2\pi\iota} \int_\gamma \frac{1}{z-a} \dif z \]
			is called the \emph{index} of $\gamma$ with respect to the point $a$. It is also sometimes referred to as the \emph{winding number} of $\gamma$ around $a$.
		\end{fdef}
		Recall the definition of $(-\gamma)$ from \eqref{eqn: 2.2}, also denoted $\gamma^{-1}$. If $\gamma$ and $\sigma$ are curves on $[0,1]$ with $\gamma(1) = \sigma(0)$, $\gamma+\sigma$ is the curve
		\[ (\gamma+\sigma)(t) = \begin{cases} \gamma(2t), & 0 \le t \le 1/2, \\ \sigma(2t-1), & 1/2 \le t \le 1. \end{cases} \]

		\begin{prop}
			If $\sigma,\gamma$ are closed rectifiable curves with the same initial (and final) points, then
			\begin{equation}
				n(\gamma;a) = -n(-\gamma;a)
			\end{equation}
			for all $a \not\in \{\gamma\}$ and
			\begin{equation}
				n(\gamma+\sigma;a) = n(\gamma;a) + n(\sigma;a)
			\end{equation}
			for all $a \not\in \{\sigma\} \cup \{\gamma\}$.
		\end{prop}
		We omit the proof of the above.\\

		The reason for $n(\cdot;\cdot)$ being called the winding number is clear from what happens in the case of a circle. For $a + e^{2\pi\iota n t}$, then $n(\gamma;a) = n$ is the number of times this curve ``winds'' or ``wraps'' around $a$. In fact, if $|b-a| < 1$, $n(\gamma;b) = n$ and if $|b-a| > 1$, $n(\gamma;b) = 0$.\\

		Recall that the components of a set are its maximal connected subsets.

		\begin{ftheo}
			\label{theo: winding number constant on components}
			Let $\gamma$ be a closed rectifiable curve in $\C$. Then $n(\gamma;a)$ is constant for $a$ belonging to a component of $G = \C \setminus \{\gamma\}$. Also, $n(\gamma;a) = 0$ for $a$ belonging to the unbounded component of $G$.
		\end{ftheo}
		\begin{remark}
			Since $\{\gamma\}$ is compact, the connected set $\{ z : |z| > R \} \subseteq G$ for sufficiently large $R$, so $\gamma$ has precisely one unbounded component.
		\end{remark}
		\begin{proof}
			Define $f : G \to \C$ by $f(a) = n(\gamma;a)$.
			If we manage to show that $f$ is continuous on $G$, we are done since the image of this map is a subset of the integers and each component is connected by definition, so $f$ is constant on each component.\\
			Recall that components of $G$ are open. Fix $a \in G$ and let $r = d(a , \{\gamma\}) > 0$. If $|a-b| < \delta \le r/2$ (we shall fix $\delta$ more precisely later), then
			\begin{align*}
				|f(a) - f(b)| &= \frac{1}{2\pi} \left| \int_\gamma \left( \frac{1}{z-a} - \frac{1}{z-b} \right) \dif z \right| \\
					&\le \frac{|a-b|}{2\pi} \int_\gamma \frac{1}{|z-a||z-b|} |{\dif} {z}|.
			\end{align*}
			By definition, $|z-a| \ge r$ for any $a \in \{\gamma\}$ and $|z-b| \ge |z-a| - |a-b| \ge r/2$. So,
			\begin{align*}
				|f(a) - f(b)| &\le \frac{|a-b|}{2\pi} \int_\gamma \frac{2}{r^2} |{\dif} {z}| \\
					&\le \frac{\delta}{\pi r^2} V(\gamma).
			\end{align*}
			For a given $\epsilon > 0$, setting $\delta = \min\{r/2, \epsilon\pi r^2 / V(\gamma)\}$ does the job, completing the first part of the proof.\\

			It remains to show that $\lim_{a\to\infty}f(a) = 0$ (Why does this imply the required?). Let $U$ be the unbounded component of $G$. For a given $R > 0$, let $a \in U$ such that $d(a;\gamma) > R$. Then,
			\[ |f(a)| = \frac{1}{2\pi} \int_\gamma \left|\frac{1}{z-a}\right| |{\dif} {z}| \le \frac{1}{2\pi R} \int_\gamma |{\dif} {z}| = \frac{V(\gamma)}{2\pi R}. \]
			$R$ can be made arbitrarily large (as $a\to\infty$), so we are done.
		\end{proof}

		Now, one would expect to see that for a ``nice'' $f$ defined on a nice region $G$, for closed rectifiable paths $\gamma$, $\int_\gamma f$ is zero. Indeed, this is evidenced by how we saw that $n(\gamma;a)$ is zero on the unbounded component of $\C\setminus\{\gamma\}$. Even before that, we had seen that $\int_\gamma f = 0$ if $f : G \to \C$ is analytic, $\gamma$ is a closed rectifiable curve, and $G = B(a,R)$.\\
		It turns out that the last of the above statements is true for a more general class of regions, not just disks. It is not true on any region however, since the winding number of a path about a point can be nonzero. It turns out that this winding number situation is the only real problematic case, and we shall see in \nameref{cauchys theorem v1} that this ``general class of regions'' is the set of regions without any ``hole''.\\
		On the other hand, one may ask the question: for a fixed domain $G$ and $f$ analytic on $G$, for what $\gamma$ inside $G$ is $\int_\gamma f = 0$?

		\begin{flem}
			\label{lemma 2.31: Fm}
			Let $\gamma$ be a rectifiable curve and suppose $\varphi : \{\gamma\} \to \C$ is continuous. Then, for each $m \ge 1$, defining
			\[ F_m(z) = \int_\gamma \frac{\varphi(w)}{(w-z)^m}, \]
			$F_m$ is analytic on $\C\setminus\{\gamma\}$ and $F_m'(z) = m F_{m+1}(z)$.
		\end{flem}
		Note that this matches the power series expansion for a general function we had got earlier, where $a_n$, which is related to the $n$th derivative of $f$ at that point, was evaluated as an integral of the above form.
		\begin{proof}
			Let us first show that $F_m$ is continuous for each $m$. Let $a \in \C\setminus\{\gamma\}$. We have
			\begin{align}
				F_m(z) - F_m(a) &= \int_\gamma \varphi(w) \left( \frac{1}{(w-z)^m} - \frac{1}{(w-a)^m} \right) \dif w \nonumber \\
					&= \int_\gamma \left( \frac{1}{w-z} - \frac{1}{w-a} \right) \sum_{k=1}^{m} \frac{1}{(w-z)^{m-k} (w-a)^{k-1}} \dif w \nonumber \\
					&= \int_\gamma (z-a) \sum_{k=1}^{m} \frac{1}{(w-z)^{m-k+1} (w-a)^{k}} \dif w \label{eqn: 2.7}
			\end{align}
			So,
			\[ |F_m(z) - F_m(a)| \le \int_\gamma |\varphi(w)| |z-a| \sum_{k=1}^m \frac{1}{|w-z|^{m+1-k}|w-a|^{k}} |{\dif} {w}| \]
			Since $\varphi$ is continous on $\{\gamma\}$ and $\{\gamma\}$ is compact, there exists $M > 0$ such that $|\varphi(w)| \le M$ for all $w \in \{\gamma\}$. Because $a\not\in\{\gamma\}$, $r = d(a,\{\gamma\}) > 0$. Let $\delta \le r/2$. Then, for $z \in \C\setminus\{\gamma\}$ with $|z-a| < \delta$, we have that $|w-z| \ge r$ and $|w-a| \ge |w-z| - |z-a| \ge r/2$. So,
			\begin{align*}
				|F_m(z) - F_m(a)| &\le \int_\gamma |\varphi(w)| |z-a| \sum_{k=1}^m \frac{1}{|w-z|^{m+1-k}|w-a|^{k}} |{\dif} {w}| \\
					&= M\delta \int_\gamma \sum_{k=1}^m \frac{1}{|w-z|^{m+1-k}|w-a|^{k}} |{\dif} {w}| \\
					&\le M\delta \int_\gamma \sum_{k=1}^m \frac{1}{(r/2)^{m+1}} |{\dif}{w}| \\
					&= \delta \cdot Mm \left(\frac{2}{r}\right)^{m+1} V(\gamma).
			\end{align*}
			Taking $\delta$ appropriately small, we are done with the first part of the proof.\\
			Now, let us show the differentiability of $F_m$. Rewriting \eqref{eqn: 2.7},
			\[ \frac{F_m(z) - F_m(a)}{z-a} = \sum_{k=1}^m \int_\gamma \frac{\varphi(w)(w-a)^{-k}}{(w-z)^{m+1-k}} \dif w. \]
			The limit of this as $z\to a$ is clearly well-defined, so $F_m$ is differentiable. Because $a\not\in\gamma$ by definition, each of the $m$ integrands above is a continuous function of $w$.\\
			Therefore,
			\begin{align*}
				\lim_{z\to a} \frac{F_m(z) - F_m(a)}{z-a} &= \sum_{k=1}^m \int_\gamma \frac{\varphi(w)(w-a)^{-k}}{(w-a)^{m+1-k}} \dif w \\
					&= \sum_{k=1}^m \int_\gamma \frac{\varphi(w)}{(w-a)^{m+1}} \dif w = m F_{m+1}(a). \qedhere
			\end{align*}
		\end{proof}

		\begin{definition}
			If $\gamma$ is a closed rectifiable curve and $G$ is a region, we say that $\gamma$ is \emph{homologous} to $0$ on $G$ and write $\gamma \approx 0$ in $G$ if $n(\gamma;w) = 0$ for all $w \in \C\setminus G$.
		\end{definition}

		\begin{ftheo}[Cauchy's Integral Formula, version 1]
			\label{cauchy integral formula v1}
			Let $G$ be an open subset of $\C$ and $f : G \to \C$ be analytic. If $\gamma \approx 0$ in $G$, then for $a \in G\setminus\{\gamma\}$,
			\[ \frac{1}{2\pi\iota} \int_\gamma \frac{f(z)}{z-a} \dif z = n(\gamma;a) f(a). \]
			In particular, if $n(\gamma;a) = 0$, the integral on the left is zero.
		\end{ftheo}
		\begin{proof}
			Define $\varphi : G \times G \to \C$ as
			\[ \varphi(z,w) =
				\begin{cases}
					\frac{f(z) - f(w)}{z-w}, & z\ne w \\
					f'(z), & z=w.
				\end{cases}
			\]
			Observe that if we show that $\int_\gamma \varphi(z,w) \dif z = 0$, then
			\[ f(z) \int_\gamma \frac{1}{w-z} \dif w = \int_\gamma \frac{f(w)}{w-z} \dif z, \]
			which implies the required since the left-hand side is just $2\pi\iota n(\gamma;z) f(z)$.\\
			It is not too difficult to show that $\varphi$ is continuous $G\times G$ (this uses the continuity of $f'$!).\\
			Fix some $w \in G$. We shall first show that $\psi_w$ that maps $z \mapsto \varphi(z,w)$ is analytic on $G$. First, let us check at $a \ne w$. We have
			\begin{align*}
			 	\lim_{h\to 0} \frac{\varphi(a+h,w) - \varphi(a,w)}{h} &= \lim_{h\to 0} \frac{1}{h} \left( \frac{f(a+h) - f(w)}{a+h-w} - \frac{f(a) - f(w)}{a-w} \right) \\
			 		&= \lim_{h \to 0} \frac{1}{h} \left( \frac{(a-w) (f(a+h) - f(w)) - (a-w) f(a) - hf(a) + (a+h-w)f(w)}{(a+h+w)(a-w)} \right) \\
			 		&= \lim_{h \to 0} \frac{1}{h} \left( \frac{(a-w)(f(a+h)-f(a)) - h(f(a)-f(w))}{(a+h-w)(a-w)} \right) \\
			 	\psi_w'(a) &= \frac{f'(a)}{a-w} - \frac{f(a)-f(w)}{(a-w)^2}.
			\end{align*} 
			Since $f$ is analytic, $\psi_w$ is analytic on $G \setminus \{w\}$.\\
			For $a=w$ on the other hand,
			\begin{align}
				\lim_{h\to 0} \frac{\varphi(w+h,w) - \varphi(w,w)}{h} &= \lim_{h\to 0} \frac{1}{h} \left( \frac{f(w+h) - f(w)}{h} - f'(w) \right) \nonumber \\
					&= \lim_{h\to 0} \frac{f(w+h) - f(w) - hf'(w)}{h^2} \label{eqn: 2.8} \\
				\psi_w'(w) &= \frac{1}{2} f''(w), \nonumber
			\end{align}
			where the final step is direct on using the fact that $f$ has a power series expansion on some $B(w,r)$ for small $r$.\\
			Checking that $\psi_w$ is analytic at $G$ is not too difficult on using the power series expansion of $f$ about $w$ (we in fact get a limit similar to \eqref{eqn: 2.8}).\\
			So, we now have that $\psi_w$ is analytic. Define
			\[ H = \{ w \in \C : n(\gamma;w) = 0 \}. \]
			By \Cref{theo: winding number constant on components}, $H$ is open. Moreover, $G \cup H = \C$. Define $g:\C\to\C$ by
			\[
				g(z) =
				\begin{cases}
					\int_\gamma \psi_w(z) \dif w, & z \in G \\
					\int_\gamma \frac{f(w)}{w-z} \dif w, & z \in H.
				\end{cases}
			\]
			We shall show that $g$ is bounded and entire, and thus constant. If we then show that $\lim_{z\to 0} g(z) = 0$ (this involves only the second part of the definition of $g$), we have $g(z) = 0$ on $G$ as well, which is exactly what we want.\\
			Boundedness of the first part is straightforward as $G$ may be assumed to be bounded. For the second part,
			\[ \int_\gamma \frac{|f(w)|}{|w-z|} |{\dif}{w}| \le M \int_\gamma \frac{1}{|w-z|} |{\dif}{w}|, \]
			where $M$ is the supremum of $f$. However, the integral is clearly bounded, and the integrand (so the integral) may even be made infinitely small as $z \to \infty$. If we show now that $g$ is entire, then $g$ is zero everywhere on $\C$ and we are home.
		\end{proof}

		\begin{ftheo}[Cauchy's Integral Formula, version 2]
			\label{cauchys integral formula v2}
			Let $G$ be an open subset of $\C$ and $f : G \to \C$ be analytic. If $\gamma_1,\ldots,\gamma_m$ are closed rectifiable curves in $G$ such that $\sum_k n(\gamma_k;w) = 0$ for $w \in \C\setminus G$, then for $a \in G\setminus\bigcup_k\{\gamma_k\}$,
			\[ \sum_{k=1}^m \frac{1}{2\pi\iota} \int_{\gamma_i} \frac{f(z)}{z-a} \dif z = f(a) \sum_{k=1}^m n(\gamma_k;a). \]
		\end{ftheo}
		The idea of the proof is very similar to that of \nameref{cauchy integral formula v1}, with the only difference being that we define
		\[ H = \{ z \in \C : \sum_k n(\gamma_k;z) = 0 \} \]
		and
		\[
			g(z) = 
			\begin{cases}
				\sum_{k=1}^m \int_{\gamma_k} \frac{f(w)}{w-z} \dif w, & z \in H, \\
				\sum_{k=1}^m \varphi(z,w) \dif w, & z \in G.
			\end{cases}
		\]

		% \begin{fcor}[Cauchy's Theorem, version 1]
		\begin{fcor}[Cauchy's Theorem]
			\label{cauchys theorem v1}
			Let $G$ be an open subset of $\C$ and $f : G \to \C$ be analytic. If $(\gamma_k)_{k=1}^m$ are closed rectifiable curves in $G$ such that $\sum_k n(\gamma_k;w) = 0$ for $w \in \C \setminus G$, then
			\[ \sum_{k=1}^m \int_{\gamma_k} f = 0. \]
		\end{fcor}

		The above follows directly from \nameref{cauchys integral formula v2} on setting $g(z) = f(z) (z-a)$ for some $a \in G \setminus \bigcup_k \{\gamma_k\}$. Indeed, such an $a$ exists since $\bigcup_k \{\gamma_k\}$ is a finite union of compact sets so is closed and bounded, but $G$ is open (if it is closed, it must be $\C$, which is not bounded).\\
		In fact, we may even prove \Cref{cauchys integral formula v2} from \Cref{cauchys theorem v1} by using it on the analytic function
		\[
			g(z) =
			\begin{cases}
				\frac{f(z)-f(a)}{z-a}, & z \ne a, \\
				f'(a), & z = a.
			\end{cases}
		\]

		Going back to \Cref{lemma 2.31: Fm}, we have that
		\[ F(z) = n(\gamma;z) = \frac{1}{2\pi\iota} \int_\gamma \frac{f(w)}{w-z} \dif w \]
		for $z \in G \setminus \{\gamma\}$. The result there says that
		\[ F^{(m)}(a) = m! \frac{1}{2\pi\iota} \int_\gamma \frac{f(w)}{(w-a)^{m+1}} \dif w. \]
		Further,
		\[ F^{(m)}(a) = n(\gamma;a) f^{(m)}(a) \]
		since $n(\gamma,\cdot)$ is constant on components.

		\begin{theorem}
			Let $G$ be an open subset of $\C$ and $f : G \to \C$ be analytic. If $(\gamma_k))_{k=1}^m$ are closed rectifiable curves in $G$ such that $\sum_k n(\gamma_k;w) = 0$ for $w \in \C \setminus G$, then for $a \in G \setminus \bigcup \{\gamma_k\}$ and $r \ge 1$,
			\[ f^{(r)}(a) \sum_{k=1}^m n(\gamma_k;a) = r! \sum_{k=1}^m \frac{1}{2\pi\iota} \int_{\gamma_k} \frac{f(w)}{(w-a)^{r+1}} \dif w. \]
		\end{theorem}

		\begin{ftheo}[Morera's Theorem]
			\label{moreras theorem}
			Let $G$ be a region and $f : G \to \C$ be continuous such that for any triangular path $T$ in $G$, $\int_T f = 0$. Then $f$ is analytic.
		\end{ftheo}
		Above, a triangular path is a closed polygonal curve that consists of three ``edges''. That is, it looks like a triangle.
		\begin{proof}
			It is enough to show that $f$ is analytic on each open disk contained inside $G$, so assume wlog that $G$ is an open disk $B(a,R)$.
			We are done if we find a primitive $F$ of $f$. Indeed, this would mean that $F$, and thus $f$, is analytic.\\
			For $z \in G$, define
			\[ F(z) = \int_{[a,z]} f, \]
			where $[a,z]$ is the segment joining $a$ and $z$. More concretely, $[a,z]$ is the curve given by
			\[ \gamma(t) = a + t(z-a) \]
			for $t \in [0,1]$.\\
			Fix some $z_0 \in G$. We shall show that $F'(z_0) = f(z_0)$ For any $z \in G$,
			\[ F(z) = \int_{[a,z_0]} f + \int_{[z_0,z]} f = F(z_0) + \int_{[z_0,z]} f . \]
			Then,
			\begin{align*}
				\frac{F(z) - F(z_0)}{z-z_0} &= \frac{1}{z-z_0} \int_{[z_0,z]} f \\
					&= f(z_0) + \frac{1}{z-z_0} \int_{z_0,z} f(w) - f(z_0) \dif w.
			\end{align*}
			Now, fixing $\epsilon > 0$, use the continuity of $f$ to get $\delta > 0$ such that if $|z_0 - w| < \delta$, then $|f(z_0) - f(w)| < \epsilon$. Then, when $|z_0 - z| < \delta$,
			\begin{align*}
				\left| \frac{F(z) - F(z_0)}{z-z_0} - f(z_0) \right| \le \frac{1}{|z-z_0|} \int_{[z_0,z]} |f(w) - f(z_0)| |{\dif}{w}| \\
					&\le \frac{1}{|z-z_0|} \int_{[z_0,z]} \epsilon |{\dif}{w}| \\
					&= \epsilon,
			\end{align*}
			completing the proof.
		\end{proof}

		Recall \Cref{cor: finite multiplicity}. Similarly, if $a_1,\ldots,a_k$ are the zeroes of $f$ (repeated according to multiplicity), then
		\[ f(z) = (z-a_1)(z-a_2)\cdots(z-a_k) g(z), \]
		where $g(a_i) \ne 0$ for all $i$. Therefore,
		\[ \frac{f'(z)}{f(z)} = \frac{1}{z-a_1} + \frac{1}{z-a_2} + \cdots + \frac{1}{z-a_k} + \frac{g'(z)}{g(z)}. \]

		\begin{prop}
			Let $G$ be a region and $f$ an analytic function on $G$ with finitely many zeroes $a_1,\ldots,a_k$ (repeated according to multiplicity). If $\gamma$ is a closed rectifiable curve in $G$ that does not pass through any $a_j$ and $\gamma \approx 0$ on $G$, then
			\[ \frac{1}{2\pi\iota} \int_\gamma \frac{f'(z)}{f(z)} \dif z = \sum_j n(\gamma;a_j). \]
		\end{prop}

		We do not prove the above since it follows directly from the prior discussion and \Cref{cauchys theorem v1}.

		\begin{corollary}
			\label{cor 2.38}
			Let $G$ be a region and $f$ an analytic function on $G$ with finitely many points $a_1,\ldots,a_k$ with $f(a_i) = \alpha$. If $\gamma$ is a closed rectifiable curve in $G$ that does not pass through any $a_j$ and $\gamma \approx 0$ on $G$, then
			\[ \frac{1}{2\pi\iota} \int_\gamma \frac{f'(z)}{f(z)-\alpha} \dif z = \sum_j n(\gamma;a_j). \]
		\end{corollary}

		The above follows directly on applying the previous proposition to $f-\alpha$.\\

		Recall that if $\gamma:[0,1] \to G$ is rectifiable and $f:G\to\C$ is analytic, then $\sigma = (f\circ\gamma)$ is rectifiable. \\
		Suppose we further have that $\gamma$ is closed and smooth and $\gamma \approx 0$ in $G$. Let $\alpha \in \C \setminus \{\sigma\}$. Then,
		\[ n(\sigma;\alpha) = \frac{1}{2\pi\iota} \int_\sigma \frac{1}{w-\alpha} \dif w = \frac{1}{2\pi\iota} \int_0^1 \frac{f'(\gamma(t))\gamma'(t)}{f(\gamma(t)) - \alpha} \dif t = \frac{1}{2\pi\iota} \int_\gamma \frac{f'(z)}{f(z)-\alpha}, \]
		which we just evaluated above. As might be expected, this is in fact true for any closed rectifiable $\gamma$.\\

		What about the scenario of \Cref{cor 2.38} where $f$ has infinitely many zeroes in $G$? By \Cref{theo: f identically zero}, any limit point of the zero set $Z(f)$ of $f$ is in $\partial G$. Since $G$ is open, $\partial G \cap G = \emptyset$. \\
		We shall show that
		\[ Z(f) \cap \underbrace{\{ z \in \C : n(\gamma;z) \ne 0 \}}_{H} \subseteq G \]
		is finite, so \Cref{cor 2.38} is still true, since all but finitely many of the terms are zero.\\
		Observe that $H$ is closed and bounded in $\C$, and is thus compact. Further note that since the zeroes of $f$ are isolated, $Z(f) \cap H$ is a discrete closed subset of $H$. But any discrete compact set is finite(!), so we are done.\\
		Therefore, even in the infinite zero case of \Cref{cor 2.38}, we have
		\[ \frac{1}{2\pi\iota} \int_\gamma \frac{f'(z)}{f(z)} \dif z = \sum n(\gamma;a), \]
		where the sum is over all points in $Z(f)$, taken with multiplicity.


		\begin{prop}
			Let $f : G \to \C$ be analytic and non-constant, $\alpha \in f(G) \setminus \{\sigma\}$, and $\gamma \approx 0$ in $G$ is such that $n(\gamma;a) = 1$ for all $a \in f^{-1}(\alpha)$ (this also assumes that $\{\gamma\}$ does not contain any such $a$). Then, $f(G)$ contains the component of $\C\setminus\{\sigma\}$ containing $\alpha$, where $\sigma = f \circ \gamma$.
		\end{prop}
		\begin{proof}
			Let $\beta$ belong to the mentioned component. We must show show the existence of a $z \in G$ with $f(z) = \beta$.\\
			By \Cref{theo: winding number constant on components}, $n(\sigma;\alpha) = n(\sigma;\beta)$. The first quantity is equal to $m = \sum_k n(\gamma;z_k(\alpha))$, and the second is equal to $\frac{1}{2\pi\iota} \int_\gamma \frac{f'(z)}{f(z)-\beta} \dif z$, where $z_k(\alpha)$ are the finitely many points inside $G$ for which $n(\gamma;z_k(\alpha)) \ne 0$ and $f(z_k(\alpha)) = \alpha$. Note that $m \ne 0$ by the $n(\gamma;a) = 1$ condition. If $\beta \not\in f(G)$, then the function $z \mapsto f'(z) / (f(z) - \beta)$ is analytic on $G$. Since $\gamma \approx 0$ on $G$, this must then be zero by \nameref{cauchys theorem v1} so we have arrived at a contradiction, proving the claim.
		\end{proof}

		Further note that above, we get that the number $m$ of points $z_k(\alpha)$ (taken with multiplicity) is equal to the number of point $z_k(\beta)$.

		\begin{lemma}
			\label{lemma: open mapping lemma}
			Suppose $f$ is analytic and non-constant on $B(a,R)$. If $f-\alpha$ has a zero at $a$ of order $m$, then there exist $\epsilon,\delta > 0$ such that for $0 < |\zeta-\alpha| < \delta$, the equation $f-\zeta$ has exactly $m$ simple roots in $B(a,\epsilon)$.
		\end{lemma}

		This also implies that $f(B(a,\epsilon)) \supseteq B(\alpha,\delta)$.

		\begin{ftheo}[Open Mapping Theorem]
			\label{open mapping theorem}
			Let $G$ be a region and let $f$ be a non-constant analytic function on $G$. Then for any open $U \subseteq G$, $f(U)$ is open in $\C$.
		\end{ftheo}
		\begin{proof}
			Fix $\alpha \in f(U)$. We shall demonstrate the existence of $\delta > 0$ such that $B(\alpha,\delta) \subseteq U$. Since $\alpha \in f(U)$, let $a \in U$ such that $f(a) = \alpha$. There also exists $R > 0$ such that $B(a,R) \subseteq U$. The result directly follows on using the remark after the previous lemma.
		\end{proof}

		\begin{ftheo}[Goursat's Theorem]
			\label{goursats theorem}
			Let $G\subseteq \C$ be open and $f:G\to\C$ be differentiable. Then, $f$ is analytic.
		\end{ftheo}
		\begin{proof}
			It suffices to consider the case where $G$ is an open disk. By \nameref{moreras theorem}, it suffices to show that $\int_T f = 0$ for any triangular curve $T$ in $G$. Fix some such $T = [a,b,c,a]$, and let $\triangle$ be the closed set formed by its convex hull. Joining the midpoints of each side of $T$, we get four triangles $(\triangle_i)_{i=1}^4$ as follows. Let $T_i = \partial \triangle_i$ be paths having the following `directions'. Then,
			\[ \int_T f = \sum_{i=1}^4 \int_{T_i} f. \]
			Let $T^{(1)} \in \{T_i\}_{i=1}^4$ such that
			\[ \left| \int_{T^{(1)}} f \right| = \max_i \left| \int_{T_i} f \right|. \]
			Observe that $\ell(T_i) = (1/2) \ell(T)$ and $\diam(\triangle_i) = (1/2) \diam(\triangle)$. Now,
			\[ \left| \int_T f \right| \le 4 \left| \int_{T^{(1)}} f \right|. \]
			We may perform the same process on $T^{(1)}$ to get $T^{(2)}$, and in general on $T^{(i)}$ to get $T^{(i+1)}$. This sequence of triangles is such that if $\triangle^{(n)} = \conv(T^{(n)})$, then
			\[ \triangle^{(1)} \supseteq \triangle^{(2)} \supseteq \cdots \triangle^{(n)} \supseteq \cdots, \]
			\[ \left| \int_{T^{(n)}} f \right| \le 4 \left| \int_{T^{(n+1)}} f \right|, \]
			\[ \ell(T^{(n+1)}) = \frac{1}{2} \ell(T^{(n)}), \text{ and } \diam(\triangle^{(n+1)}) = \frac{1}{2} \diam(\triangle^{(n)}). \]
			This yields that
			\begin{align*}
				\left| \int_T f \right| &\le 4^n \left| \int_{T^{(n)}} f \right|, \\
				\ell(T^{(n)}) &= \frac{1}{2^n} \ell(T), \\
				\diam(\triangle^{(n)}) &= \frac{1}{2^n} \diam(\triangle).
			\end{align*}
			Using Cantor's Theorem, since $\C$ is complete, $\bigcap_n \triangle^{(n)}$ is a singleton, say $\{z_0\}$. Fix $\epsilon > 0$. Because $f$ is differentiable at $z_0$, let $\delta > 0$ such that $B(z_0,\delta) \subseteq G$ and
			\[ \left| \frac{f(z) - f(z_0)}{z-z_0} - f'(z_0) \right| < \epsilon \]
			whenever $|z-z_0| < \delta$. Now, choose $n$ such that $\diam(\triangle^{(n)}) < \delta$. Since $z_0 \in \triangle^{(n)}$, $\triangle^{(n)} \subseteq B(z_0,\delta)$. We now have that
			\begin{align*}
				\left| \int_T f \right| &\le 4^n \left| \int_{T^{(n)}} f \right| \\
					&= 4^n \left| \int_T f(z) - f(z_0) - f'(z_0) (z-z_0) \dif z \right| \\
					&\le 4^n \int_T \left| f(z) - f(z_0) - f'(z_0) (z-z_0) \right| |{\dif} z| \\
					&\le 4^n \int_T \epsilon |z-z_0| |{\dif} z| \\
					&\le 4^n \int_T \epsilon \diam(\triangle^{(n)}) |{\dif} z| \\
					&= 4^n \epsilon \diam(\triangle^{(n)}) \ell(T^{(n)}) \\
					&= \epsilon \diam(\triangle) \ell(T).
			\end{align*} 
			Since $\epsilon$ can be made arbitrarily small, we are done.
		\end{proof}

	\subsection{Homotopy}

		\begin{fdef}
			Let $\gamma_1,\gamma_2:[0,1] \to \C$ be two closed rectifiable curves. Then, a \emph{homotopy} between $\gamma_1,\gamma_2$ is a continuous function $\Gamma:[0,1]\times[0,1] \to \C$ such that
			\begin{enumerate}
				\item $\Gamma(s,0) = \gamma_1(s)$,
				\item $\Gamma(s,1) = \gamma_2(s)$, and
				\item $\Gamma(0,t) = \Gamma(1,t)$ for any $s,t$.
			\end{enumerate}
			If there exists a homotopy between two curves, they are said to be \emph{homotopic} and we write $\gamma_1 \sim \gamma_2$.
		\end{fdef}

		When a curve is homotopic to a constant curve, we write $\gamma \sim 0$.\\

		In a convex set, any two paths are homotopic. as is seen by the homotopy
		\[ \Gamma(s,t) = t\gamma(s) + (1-t)a \]
		between any curve and a constant curve. Similarly, any two paths are homotopic in a star-shaped set, as can be seen by using the above homotopy with $a$ as the point that is in sight of everything.


		\begin{ftheo}[Homotopic version of Cauchy's theorem]
			If $\gamma_0,\gamma_1$ are closed rectifiable curves in a region $G$ and $\gamma_0 \sim \gamma_1$, then
			\[ \int_{\gamma_0} f = \int_{\gamma_1} f \]
			for every analytic $f$ on $G$.
		\end{ftheo}
		\begin{proof}
			We only prove the result in the case where there exists a homotopy $\Gamma$ between $\gamma_0,\gamma_1$ with continuous second partial derivatives. Then, throughout the unit square $I^2 = [0,1]^2$,
			\[ \md{\Gamma}{2}{s}{\phantom{1}}{t}{\phantom{1}} = \md{\Gamma}{2}{t}{\phantom{1}}{s}{\phantom{1}}. \]
			Define
			\[ g(t) = \int_0^1 f(\Gamma(s,t)) \dpd{\Gamma}{s}(s,t) \dif s. \]
			We have that
			\[ g(0) = \int_0^1 f(\gamma_0(s)) \cdot \dpd{\gamma_0}{s}(s) \dif s = \int_{\gamma_0} f \]
			and similarly, $g(1) = \int_{\gamma_1} f$. If we show that $g$ is constant, we are done. To show this,
			\[ g'(t) = \int_0^1 \left( f'(\Gamma(s,t)) \dpd{\Gamma}{t}(s,t) \dpd{\Gamma}{s}(s,t) + f(\Gamma(s,t)) \md{\Gamma}{2}{t}{\phantom{1}}{s}{\phantom{1}}(s,t) \right) \dif s. \]
			Now,
			\[ \dpd{}{s} \left( (f \circ \Gamma)(s,t) \dpd{\Gamma}{t}(s,t) \right) = f'(\Gamma(s,t)) \dpd{\Gamma}{t}(s,t) \dpd{\Gamma}{s}(s,t) + f(\Gamma(s,t)) \md{\Gamma}{2}{t}{\phantom{1}}{s}{\phantom{1}}(s,t). \]
			Therefore,
			\[ g'(t) = (f \circ \Gamma)(1,t) \dpd{\Gamma}{t}(1,t) - (f \circ \Gamma)(0,t) \dpd{\Gamma}{t}(0,t). \]
			Since $\Gamma(0,t) = \Gamma(1,t)$ for all $t$, this is zero. Therefore, $g$ is constant on $[0,1]$ and $\int_{\gamma_0} f = \int_{\gamma_1} f$.
		\end{proof}

		\begin{corollary}
			If $\gamma \sim 0$, then $\gamma \approx 0$.
		\end{corollary}
		Above, $0$ refers to any constant curve that maps every $t \in [0,1]$ to some fixed $a \in \C$. Similar to how we define $\gamma \approx 0$, we may define $\gamma_1 \approx \gamma_2$ on $G$ in general, asserting that the winding numbers at the relevant points with respect to the two curves are equal.
		\begin{proof}
			Letting $\gamma_0$ be a constant curve, for any $w \not \in G$,
			\begin{align*}
				n(\gamma;w) &= \frac{1}{2\pi\iota} \int_\gamma \frac{1}{z-w} \dif z \\
					&= \frac{1}{2\pi\iota} \int_{\gamma_0} \frac{1}{z-w} \dif z \\
					&= \frac{1}{2\pi\iota} \int_{\gamma_0} \frac{1}{a-w} \dif z = 0.
			\end{align*}
		\end{proof}

		The converse of the above is not true in general.

		\begin{fdef}[Simply connected domain]
			An open set $G$ is said to be \emph{simply connected} if $G$ is connected and every closed curve $\gamma$ on $G$ is homotopic to a constant curve.
		\end{fdef}

		\begin{ftheo}
			$G$ is simply connected if and only if $\C_\infty \setminus G$ is connected.
		\end{ftheo}

		\begin{corollary}
			If $G$ is simply connected, $\int_\gamma f = 0$ for every closed rectifiable curve $\gamma$ in $G$ and analytic function $f$ on $G$.
		\end{corollary}

		Recall that if a function has a primitive, then its integral along any closed rectifiable curve is zero.\\
		Further, the proof of \nameref{moreras theorem} showed that the converse is true on open disks.

		\begin{prop}
			If $G$ is simply connected and $f:G \to \C$ is analytic in $G$, then $f$ has a primitive in $G$.
		\end{prop}
		\begin{proof}
			Fix $a \in G$. Define
			\[ F(z) = \int_{\gamma_z} f(z) \dif z, \]
			where $\gamma_z$ is any rectifiable path from $a$ to $z$. Simple connectedness implies that the value of the above is the same for any choice of $\gamma_z$. Indeed, if $\gamma_1,\gamma_2$ are two such choices, then the path $(\gamma_1 * \gamma_2^{-1}) \sim 0$, so the integral along it is $0$. This integral is just equal to $\int_{\gamma_1} f - \int_{\gamma_2} f$. For $z_0 \in G$, let $R > 0$ such that $B(z_0,R) \subseteq G$. For any $\epsilon > 0$, we must demonstrate a $\delta > 0$ such that whenever $|z-z_0| < \delta$,
			\[ \left| \frac{F(z) - F(z_0)}{z-z_0} - f(z_0) \right| < \epsilon. \]
			It suffices to show this for $z \in B(z_0,R)$. To do this, we can let $\gamma_1$ be a path from $a$ to $z_0$ and $\gamma_2 = [z_0,z]$. Then,
			\begin{align*}
				\frac{F(z) - F(z_0)}{z-z_0} &= \frac{1}{z-z_0} \left( \int_{\gamma_1 * \gamma_2} f - \int_{\gamma_1} f \right) \\
					&= \frac{1}{z-z_0} \int_{[z_0,z]} f.
			\end{align*}
			The rest of the proof follows exactly as that of Morera's Theorem, and we can show that $F' = f$.
		\end{proof}

		\begin{corollary}
			If $G$ is simply connected and $f : G \to \C$ is analytic such that $f(z) \ne 0$ for all $z \in G$, there exists analytic $g : G \to \C$ such that $e^{g(z)} = f(z)$ for all $z \in G$.
		\end{corollary}

		In particular, there exists a branch of the log on any simply connected domain that does not contain $0$.

		\begin{proof}
			Then, there exists an analytic function $h : G \to \C$ such that $h(z) = f'(z) / f(z)$. Now,
			\[ \frac{\dif}{\dif z} (e^{-h(z)} f(z)) = e^{-h(z)} f'(z) - f(z) h'(z) e^{-h(z)} = 0. \]
			Therefore, $e^{-h(z)} f(z)$ is some non-zero constant $\alpha$, so $f(z) = \alpha e^{h(z)}$. We easily get some $\beta$ such that $e^{\beta} = \alpha$, so the analytic function $\beta + h(z)$ does the job.
		\end{proof}