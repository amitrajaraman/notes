\section{Several Bounds}

In this section we describe numerous useful bounds.

\subsection{Bounding Volume using the Entropy Function}

\begin{definition}
    Let $q\geq 2$ and $n\geq r\geq 1$ be integers. Then the \textit{volume} of a Hamming ball of radius $r$ is given by
    $$\Vol_q(r,n)=|B_q(\textbf{0}, r)|=\sum_{i=0}^r\binom{n}{i}(q-1)^i.$$
\end{definition}

\begin{theorem}
\label{volumeBound1}
    Let $q\geq 2$ be an integer and $0\leq p\leq 1-\frac{1}{q}$ be a real. Then
    \begin{enumerate}[(i)]
        \item $\Vol_q(pn,n)\leq q^{H_q(p)n}$.
        \item for large enough $n$, $\Vol_q(pn,n)\geq q^{H_q(p)n-o(n)}$
    \end{enumerate}
\end{theorem}
\begin{proof}
\phantom{owo}
\begin{enumerate}[(i)]
    \item We have
    \begin{align*}
        1 &= \sum_{i=1}^n\binom{n}{i}p^i(1-p)^{n-i} \\
          &\geq \sum_{i=1}^{pn}\binom{n}{i} p^i(1-p)^{n-i} \\
          &= \sum_{i=1}^{pn}\binom{n}{i} (q-1)^i \left(\frac{p}{(1-p)(q-1)}\right)^i (1-p)^n \\
          &\geq \sum_{i=1}^{pn}\binom{n}{i} (q-1)^i(1-p)^n \left(\frac{p}{(1-p)(q-1)}\right)^{pn} & \text{$\frac{p}{(1-p)(q-1)}\leq 1$ as $p\leq 1-\frac{1}{q}$} \\
          &= \sum_{i=1}^{pn}\binom{n}{i} (q-1)^i(1-p)^n \left(\frac{p}{(1-p)(q-1)}\right)^{pn} \\
          &= \sum_{i=1}^{pn}\binom{n}{i} (q-1)^i(1-p)^{n(1-p)} \left(\frac{p}{q-1}\right)^{pn}\\
          &= \sum_{i=1}^{pn}\binom{n}{i} (q-1)^i q^{-H_q(p)n}\\
          &\geq \Vol_q(pn,n)q^{-H_q(p)n}.
    \end{align*}
    (i) follows.
    
    \item Using Stirling's Approximation \ref{stirlingApproximation}, we have
    \begin{align*}
        \binom{n}{pn} &= \frac{n!}{(pn)!(n(1-p))!} \\
                      &> \frac{(n/e)^n}{(pn/e)^{pn} (n(1-p)/e)^{n(1-p)}} \cdot \frac{1}{\sqrt{2\pi p(1-p)n}}\cdot e^{\lambda_1(n)-\lambda_2(pn)-\lambda_2(n(1-p))} \\
                      &= \frac{1}{p^{pn}(1-p)^{n(1-p)}} l(n)
    \end{align*}
    where $$l(n)=\frac{e^{\lambda_1(n)-\lambda_2(pn)-\lambda_2(n(1-p))}}{\sqrt{2\pi p(1-p)n}}.$$ Note that $l(n)=q^{-o(n)}$.
    
    Now, we have
    \begin{align*}
        \Vol_q(pn,n)&= \sum_{i=1}^{pn}\binom{n}{i}(q-1)^i \\
                    &\geq \binom{n}{pn}(q-1)^{pn} \\
                    &> \frac{(q-1)^{pn}}{p^{pn}(1-p)^{n(1-p)}} l(n) \\
                    &= q^{H_q(p)} q^{-o(n)}.
    \end{align*}
    This proves the required result.
\end{enumerate}
\end{proof}

\subsection{The Hamming Bound and the Singleton Bound}

Recall the Hamming Bound \ref{hammingBound} which put a bound on the dimension $k$ in terms of $n,q$ and $d$:

$$\frac{k}{n}\leq 1 - \frac{\log_q\Vol_q\left(\left\lfloor\frac{d-1}{2}\right\rfloor, n\right)}{n}.$$

From \ref{volumeBound1}, we have
$$\Vol_q\left(\left\lfloor\frac{d-1}{2}\right\rfloor, n\right) \leq q^{H_q\left(\frac{\delta}{2}\right)n-o(n)}.$$

Putting everything in terms of rate and relative distance,

$$R\leq 1-H_q\left(\frac{\delta}{2}\right)+o(1)$$

\begin{theorem}[Singleton Bound]
    For valid $n,q,d$, we have
    $$A_q(n,d)\leq q^{n-d+1}.$$
\end{theorem}
\begin{proof}
    Let $C$ be an $(n,A_q(n,d),d)_q$-code. Let $C'$ be the code of length $(n-d+1)$ code obtained by deleting the first $d-1$ letters of each codeword of $C$. Since the minimum distance of $C$ is $d$, the words obtained after deleting the first $d-1$ letters of distinct codewords of $C$ must also be distinct. This implies that $|C'| =|C| =A_q(n,d)$. As $| C'| \leq q^{n-d+1}$, the result follows.
\end{proof}

The asymptotic version of the singleton bound gives that
$$\frac{k}{n}\leq 1-\frac{d}{n}+\frac{1}{n}.$$
Alternatively,
$$R\leq 1-\delta+o(1).$$

\subsection{The Gilbert-Varshamov Bound}

\begin{theorem}[Gilbert-Varshamov Bound]
    For valid $n,q,d$, we have
    $$A_q(n,d)\geq \frac{q^n}{\sum_{i=1}^{d-1}\binom{n}{i}(q-1)^i}$$
\end{theorem}

\begin{proof}
    Let $C$ be a $(n,A_q(n,d),d)_q$-code. Then for all $x\in\Sigma^n$, there exists $c_x\in C$ such that $d(x,c_x)< d$.
    
    This gives
    
    $$\left| \bigcup_{c\in C}B(c,d-1)\right| =q^n.$$
    
    If the above equality does not hold, then there exists some $v\in \Sigma^n\setminus C$ such that $d(c,v)\geq d$ for all $c\in C$, which contradicts the maximality of $C$.
    
    We now have
    \begin{align*}
        q^n &= \left| \bigcup_{c\in C}B(c,d-1)\right|  \\
            &\leq \sum_{c\in C}\left| B(c,d-1)\right|  \\
            &= A_q(n,d) |B(c,d-1)|  \\
            &= A_q(n,d)\Vol_q(d-1,n)
    \end{align*}
    
    Substituting the value of $\Vol_q(d-1,n)$ proves the required result.
\end{proof}

In terms of rate and relative distance, we have $\Vol_q(d-1,n)\leq q^{H_q(\delta)n}$ by \ref{volumeBound1}.

\vspace{2mm}
The asymptotic version of the Gilbert-Varshamov bound gives that for every $0<\delta\leq 1-\frac{1}{q}$ there exists a code of rate $R$ and relative distance $\delta$ such that $$R\geq 1-H_q(\delta).$$

%SHOW LINEAR CODE CONSTRUCTION?
%ex 4.9

\subsection{The Plotkin Bound}

\begin{lemma}[Mapping Lemma]
\label{mappingLemma}
    Let $C\subseteq[q]^n$. Then there exists a function $f:C\to \mathbb{R}^{nq}$ such that
    \begin{enumerate}[(i)]
        \item for every $\textbf{c}\in C$, $\norm{f(c)}=1$.
        \item for every $\textbf{c}_1\neq \textbf{c}_2$ in $C$,
        $$f(\textbf{c}_1)\cdot f(\textbf{c}_2)=1-\left(\frac{q}{q-1}\right)\left(\frac{d(\textbf{c}_1,\textbf{c}_2)}{n}\right)$$
    \end{enumerate}
\end{lemma}

\begin{proof}
    Define $\varphi:[q]\to\mathbb{R}^q$ by
    $$\varphi(i)=\left(\frac{1}{q},\frac{1}{q},\cdots,\frac{1-q}{q},\cdots,\frac{1}{q}\right) \text{ for each $i\in[q]$}.$$
    
    Note that for any $i\neq j$ in $[q]$,
    $$\norm{\varphi(i)}^2=\frac{q-1}{q}
    \text{ and }
    \varphi(i)\cdot\varphi(j)=-\frac{1}{q}.$$
    Define the required function $f$ as follows. For each $\textbf{c}=(c_1,c_2,\ldots,c_n)\in [q]^n$,
    $$f(\textbf{c})=\sqrt{\frac{q}{n(q-1)}}(\varphi(c_1),\varphi(c_2),\ldots,\varphi(c_n))$$
    (Identify this vector in $(\mathbb{R}^q)^n$ to the corresponding one in $R^{nq}$) It may be verified by the reader that this $f$ satisfies both conditions mentioned in the question.
\end{proof}

\begin{theorem}[Plotkin Bound]
    Let $C$ be an $(n,M,d)_q$-code. Then
    \begin{enumerate}[(i)]
        \item If $d= n\left(1-\frac{1}{q}\right)$, $M\leq 2qn$.
        \item If $d> n\left(1-\frac{1}{q}\right)$, then $M\leq \frac{qd}{qd-(q-1)n}$.
    \end{enumerate}
\end{theorem}
\begin{proof}
    Let $C=\{c_1,c_2,\ldots,c_M\}$. Let $f$ be the function mentioned in the mapping lemma \ref{mappingLemma}. For $i\neq j$ in $[M]$,
    \begin{align*}
        f(c_i)\cdot f(c_j) &= 1 - \left(\frac{q}{q-1}\right)\left(\frac{d(c_i,c_j)}{n}\right) \\
                           &\leq 1 - \frac{qd}{(q-1)n}.
    \end{align*}
    
    \begin{enumerate}[(i)]
        \item If $d=n(1-\frac{1}{q})$, then $f(i)\cdot f(j)\leq 0$ for all $i\neq j$, and the required result follows by the first part of \ref{geometricLemma}.
        
        \item If $d>n(1-\frac{1}{q})$, then we have
        $$f(c_i)\cdot f(c_j) \leq -\left(\frac{qd-(q-1)n}{(q-1)n}\right) \leq 0.$$
        The result then follows by the second part of \ref{geometricLemma}.
    \end{enumerate}
\end{proof}

We now present the following bound, which is an improvement on part (i) of the Plotkin bound in the binary case.

\begin{theorem}
    Let $C$ be a binary $(n,M,\frac{n}{2})_2$-code. Then $M\leq 2n$.
\end{theorem}
\begin{proof}
    Let $C=\{\textbf{c}_1,\textbf{c}_2,\ldots,\textbf{c}_M\}$ and $\textbf{c}_i=(c_{i,1},c_{i,2},\ldots,c_{i,n})$ fo each $i$. Consider the map $f:C\to\mathbb{R}^n$ given by
    $$f(\textbf{c}_i)=((-1)^{c_{i,1}}, (-1)^{c_{i,2}},\ldots,(-1)^{c_{i,n}}) \text{ for each $i$.}$$
    For any valid $i\neq j$,
    $$f(\textbf{c}_i)\cdot f(\textbf{c}_j)=n-2d(\textbf{c}_i,\textbf{c}_j)\leq 0.$$
    The result follows on using \ref{geometricLemma} on the $f(\textbf{c}_i)$'s.
\end{proof}

\subsection{The Griesmer Bound}

\begin{lemma}
\label{griesmerPrelim}
    If there exists an $[n,k,d]_q$-code, then there also exists an $[n-d,k-1,d'\geq\lceil\frac{d}{q}\rceil]_q$-code.
\end{lemma}
\begin{proof}
    Let $C$ be an $[n,k,d]_q$-code. Let $G$ be a generator matrix of $C$ such that the first row vector of $G$ is of the form $\textbf{v}=(1,1,\ldots,1,0,0,\ldots,0)$ where all $\alpha_i$s are non-zero (We may assume this by considering an equivalent code). Write $G$ as follows.
    $$G=
    \begin{pmatrix}
    1 & \cdots & 1 & 0 & \cdots & 0 \\
    * & * & * & & G' & \\
    \end{pmatrix}
    $$
    where $G'$ is a $(k-1)\times (n-d)$ matrix. Consider the code $C'$ generated by $G'$. $C'$ clearly has length $n-d$ and dimension $k-1$. Let $d'$ be the length of $C'$. Let $\textbf{u}\in C'$ such that $\wt(\textbf{u})=d'$. Then there exists some $\textbf{w}=(w_1,w_2,\ldots,w_d)\in\mathbb{F}_q^d$ such that $(\textbf{w}\mid \textbf{u})\in C$, where $(\textbf{w}\mid \textbf{u})$ represents the concatenation of $w$ and $u$.
    
    By the Pigeonhole Principle, there exists $\alpha\in\mathbb{F}_q$ such that at least $\lceil\frac{d}{q}\rceil$ of $w_1,w_2,\ldots,w_d$ are equal to $\alpha$.
    
    Since $(\textbf{w}\mid \textbf{u})-\alpha \textbf{v}\in C$, we have
    \begin{align*}
        d &\leq \wt((\textbf{w}\mid \textbf{u})-\alpha \textbf{v}) \\
          &= \wt((\textbf{w} - (\alpha,\alpha,\ldots,\alpha))\mid \textbf{u}) \\
          &= \wt(\textbf{w}- (\alpha,\alpha,\ldots,\alpha)) + \wt(\textbf{u}) \\
          &\leq \left(d-\left\lceil\frac{d}{q}\right\rceil\right)+d'
    \end{align*}
    This gives $d'\geq \left\lceil\dfrac{d}{q}\right\rceil$, which proves the result.
\end{proof}

\begin{theorem}[Griesmer Bound]
    For any $[n,k,d]_q$-code,
    $$n\geq\sum_{i=0}^{k-1}\left\lceil\frac{d}{q^i}\right\rceil.$$
\end{theorem}
\begin{proof}
    For a given $k$ and $d$, we denote by $N_{k,d}$ the minimum value of $n$ for which there exists an $[n,k,d]$-code. We shall prove the result by induction on $k$. The base case $k=0$ is clear.
    
    Let the result be true for $k=k_0-1$ and let $C$ be an $[N_{k_0,d},k_0,d]$-code. Then by \ref{griesmerPrelim}, there exists an $[N_{k_0,d}-d, k_0-1, d'\geq \lceil\frac{d}{q}\rceil]$-code. By the induction, this gives
    \begin{align*}
        N_{k_0,d}-d &\geq \sum_{i=0}^{k-2}\left\lceil\frac{\lceil\frac{d}{q}\rceil}{q^i}\right\rceil \\
                    &\geq \sum_{i=0}^{k-2}\left\lceil\frac{d}{q^{i+1}}\right\rceil \\
    \end{align*}
    Thus,
    $$N_{k_0,d}\geq \sum_{i=0}^{k-1}\left\lceil\frac{d}{q^i}\right\rceil$$
    and the result is proved.
\end{proof}

\clearpage