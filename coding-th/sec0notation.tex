\section{Notation}

$\mathbb{N}$ represents the set $\{1,2,3,\ldots\}$.

$\mathbb{Z}$ represents the set of integers $\{\ldots,-2,-1,0,1,2,\ldots\}$.

$\mathbb{R}$ represents the set of real numbers.

\vspace{2mm}
For $n\in\mathbb{N}$, $[n]$ represents the set $\{1,2,\ldots,n\}$.

\begin{definition}
    An \textit{alphabet} is a finite non-empty set. Elements of an alphabet are typically called \textit{letters} or \textit{symbols}.
\end{definition}

An alphabet is usually denoted by $\Sigma$. We typically use $q$ to denote $|\Sigma|$.

For $n\in\mathbb{N}$, $ \Sigma^n$ represents the set of length $n$ strings of $\Sigma$, that is, the set $\{a_1a_2a_3\cdots a_n\mid a_i\in\Sigma\text{ for all }i\in[n]\}$. We also often represent an element of $\Sigma^n$ as a row vector.

For a set $\Omega$, we denote the power set of $\Omega$ by $2^\Omega$.

\begin{definition}
    A \textit{permutation} of a set $S=\{x_1,x_2,\ldots,x_n\}$ is a bijection from $S$ to itself.  We denote a permutation $f$ of $S$ by
    $$\begin{pmatrix}x_1 & x_2 & \cdots & x_n \\ \downarrow & \downarrow & & \downarrow \\ f(x_1) & f(x_2) & \cdots & f(x_n)
    \end{pmatrix}$$
\end{definition}

Unless mentioned otherwise, assume that $\log=\log_2$.

We assume that the reader is familiar with $o, O, \omega,$ and $\Omega$ notation used to describe the asymptotic behaviour of functions.

\clearpage