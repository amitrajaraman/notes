\section{Introduction}

\begin{fdef}[Convex set]
	A set $C \subseteq \R^d$ is said to be \emph{convex} if for any $p,q \in C$, the line segment $pq$ is also contained in $C$. That is, for any $\lambda \in [0,1]$, $\lambda p + (1-\lambda) q \in C$.\\
\end{fdef}

\begin{fdef}
	The convex hull $\Hull(S)$ of a set $S \subseteq \R^d$ is the smallest convex set containing $S$. More concretely,
	\[ \Hull(S) = \bigcap_{\substack{\text{convex $X$} \\ S \subseteq X}} X. \]
\end{fdef}

Note that the above definition requires us to show that an arbitrary union of convex sets is convex -- this is trivial. How do we \emph{algorithmically} find the convex hull of some finite set of points, say in $\R^2$?

\begin{flem}
	For a set of points $P \subseteq \R^2$, $\Hull(P)$ is a convex polygon containing $P$ whose vertices are in $P$.
\end{flem}
% give proof

Now, we would like an algorithm that given a set of points $\{p_1,\ldots,p_n\}$, outputs the vertices of $\Hull(P)$ in clockwise order. Note that $(p,q)$ is an edge (in the clockwise order) iff all vertices in $P$ are either on $\vv{pq}$ or to its right. This yields an algorithm, since we can check for all $n(n-1)$ ordered pairs of points whether all vertices are to the edges right. Overall, this is an $O(n^3)$ time algorithm.\\

Let us try to do better. What if we order the points from left to right, compute the ``upper'' and ``lower'' hulls, which are the 