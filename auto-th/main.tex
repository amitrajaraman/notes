\documentclass{article}

\title{Automata and Languages}
\author{Amit Rajaraman}
\date{March 2020}

\usepackage[utf8]{inputenc}
\usepackage{amsmath}
\usepackage{amssymb}
\usepackage{amsthm}
\usepackage{amsfonts}
\usepackage{enumerate}
\usepackage[margin=1in]{geometry}
\usepackage[colorlinks]{hyperref}
\usepackage{tikz}
\usepackage[framemethod=tikz]{mdframed}
\usepackage{framed}
\usepackage{xcolor}
\usepackage{graphicx}
\usetikzlibrary{automata, positioning, arrows, matrix}
\tikzset{->, >=stealth', node distance = 2cm}

\usepackage{titlesec}
\titleformat{\section}[block]{\sffamily\Large\filcenter\bfseries}{\S\thesection.}{0.25cm}{\Large}
\titleformat{\subsection}[block]{\large\bfseries\sffamily}{\thesubsection.}{0.2cm}{\large}

\usepackage{fancyhdr}
\lhead{\sffamily{Automata and Languages}}
\chead{\sffamily{\thepage}}
\rhead{\sffamily{Amit Rajaraman}}
\cfoot{}
\pagestyle{fancy}

\setlength\parindent{0pt}

\renewcommand{\qedsymbol}{$\blacksquare$}

\newcommand{\yields}{\Rightarrow}
\newcommand{\derives}{\overset{*}{\yields}}
\newcommand{\writeNPDA}{(Q,\Sigma,\Gamma,\delta,q_0,Z_0,F)}

\numberwithin{equation}{section}
\theoremstyle{definition}
\newtheorem{theorem}{Theorem}
\newenvironment{ftheo}
  {\colorlet{shadecolor}{orange!15}\begin{shaded}\begin{theorem}}
  {\end{theorem}\end{shaded}}
\newtheorem{lemma}[theorem]{Lemma}
\newenvironment{flem}
  {\colorlet{shadecolor}{orange!15}\begin{shaded}\begin{lemma}}
  {\end{lemma}\end{shaded}}
\newtheorem{corollary}[theorem]{Corollary}
\newenvironment{fcor}
  {\colorlet{shadecolor}{orange!15}\begin{shaded}\begin{corollary}}
  {\end{corollary}\end{shaded}}
\newtheorem{porism}[theorem]{Porism}
\newenvironment{fpor}
  {\colorlet{shadecolor}{orange!15}\begin{shaded}\begin{porism}}
  {\end{porism}\end{shaded}}
\newtheorem{definition}{Definition}
\newenvironment{fdef}
  {\colorlet{shadecolor}{green!7}\begin{shaded}\begin{definition}}
  {\end{definition}\end{shaded}}
\newtheorem*{conjecture*}{Conjecture}
\newenvironment{fcon}
  {\colorlet{shadecolor}{blue!10}\begin{shaded}\begin{conjecture*}}
  {\end{conjecture*}\end{shaded}}
\numberwithin{definition}{section}
\numberwithin{theorem}{section}
\newtheorem{exercise}{Exercise}
\newtheorem*{example}{Example}

\theoremstyle{remark}
\newtheorem*{remark}{Remark}

\numberwithin{exercise}{section}

\mdfdefinestyle{solutionstyle}{%
  % linecolor=gray,linewidth=1pt,%
  % frametitlerule=true,%
  frametitlebackgroundcolor=white,
  % backgroundcolor=  gray!20,  
  bottomline=false, topline=false, rightline=false, leftline=true,
  innerlinewidth=0.7pt, outerlinewidth=0.7pt, middlelinewidth=2pt, middlelinecolor=white, %
  innerleftmargin=6pt,innerbottommargin=0pt,
  skipabove=0pt
}
\mdtheorem[style=solutionstyle]{solution}{Solution}
\numberwithin{solution}{section}

\newcommand{\HRule}[1]{\rule{\linewidth}{#1}}

\begin{document}

\maketitle
\thispagestyle{empty}

\tableofcontents
\clearpage

\section{Introduction}

%%% LECTURE 1

\subsection{Some basic definitions}

	Consider the equation $X^2 + 1 = 0$. Clearly, this equation has no roots over $\R$. Consider the set
	\[ \C = \{ (a,b) : a,b\in\R \} = \R^2, \]
	and define addition and subtraction over $\C$ as
	\begin{align*}
		(a,b) + (c,d) &= (a+c,b+d) \\
		(a,b) \cdot (c,d) &= (ac-bd,ad+bc).
	\end{align*}
	It is easy to show that $(\C,+,\cdot)$ is a field with additive identity $(0,0)$ and multiplicative identity $(1,0)$. Further observe that $\R$ is a subfield of $\C$ -- consider the field homomorphism $\R \to \C$ defined by $a \mapsto (a,0)$.\\
	Now, we denote $\iota = (0,1)$, and write $(a,b)$ as $a+b\iota$.\\

	Observe that the equation $X^2 + 1 = 0$ \emph{does} have roots over $\C$ since it can be written as $(X+\iota)(X-\iota)$. For the sake of completeness, we also note that the multiplicative identity of $a+\iota b$ is
	\[ \frac{1}{a+\iota b} = \frac{a - \iota b}{a^2 + b^2} = \frac{a}{a^2+b^2} - \frac{b}{a^2+b^2}\iota. \]

	When writing $z = a + b\iota$ where $a,b\in\R$, we write $a = \Re z$ (the real part of $z$) and $b = \Im z$ (the imaginary part of $z$). We also define the absolute value $|z| = (a^2 + b^2)^{1/2}$ of $z$, and the \emph{conjugate} $\overline{z} = a - \iota b$ of $z$. We clearly have
	\begin{align*}
		z\overline{z} &= |z|^2 \\
		\Re z &= \frac{z+\overline{z}}{2} \\
		\Im z &= \frac{z-\overline{z}}{2\iota}.
	\end{align*}
	It is easy to check that
	\[ \overline{z+w} = \overline{z} + \overline{w} \text{ and } \overline{z\cdot w} = \overline{z}\cdot \overline{w}. \]
	We also have
	\begin{align*}
		\left| \frac{z}{w} \right| &= \frac{|z|}{|w|} \\
		|\overline{z}| = |z|.
	\end{align*}

	\begin{exercise}
		Check that the set
		\[ M = \begin{pmatrix} \alpha & \beta \\ -\beta & \alpha \end{pmatrix} : \alpha,\beta\in\R \]
		with matrix addition and multiplication is a field isomorphic to $\C$.
	\end{exercise}

	To close out the tedious part of things, we have
	\begin{align}
		|z+w|^2 &= |z|^2 + |w|^2 + 2 \Re(z\overline{w}) \nonumber \\
		|z+w| &\le |z| + |w| \label{eqn: triangle inequality}
	\end{align}

	\Cref{eqn: triangle inequality} is referred to as the \emph{triangle inequality}.

\subsection{Polar representations and roots}

	Consider $z = x + \iota y \in \C$. We may then define
	\[ x = r\cos\theta \;\;\;\;\; y = r\sin\theta, \]
	where $|z| = r$ and the angle $\theta$ is called the \emph{argument} of $z$ as is denoted $\theta = \arg z$. We typically restrict $\theta$ to $(-\pi,\pi]$.\\
	We denote $\cis\theta = \cos\theta + \iota\sin\theta$. Therefore, we have
	\[ z = |z| \cis(\arg z). \]
	Observe that rather conveniently,
	\[ \cis\theta_1 \cdot \cis\theta_2 = \cis(\theta_1 + \theta_2). \]
	Therefore, inductively,
	\[ z_1 z_2 \cdots z_n = \left(\prod_{i} |z_i|\right) \cis\left(\sum_i \arg z_i \right). \]
	In particular,
	\[ z^n = r^n \cis(n\theta) \]
	for any $n > 0$. If $z \ne 0$ (equivalently, $r\ne0$), the above holds for all $n \in \Z$.\\
	In the case where $r=1$, we have
	\begin{equation}
		\label{eqn: de moivre's}
		(\cos\theta + \iota\sin\theta)^n = \cos(n\theta) + \iota\sin(n\theta)
	\end{equation}
	\Cref{eqn: de moivre's} is referred to as \emph{de Moivre's Formula}.\\

	Let us consider the equation $z^n = a$. This equation has $n$ roots of the form
	\[ z = |a|^{1/n} \cis\left( \frac{2k\pi + \arg z}{n} \right) \]
	for $k = 0,1,\ldots,n-1$.

	A \emph{line} in the complex plane is a set of the form
	\[ L = \{ z = a+tb : t \in \R \}, \]
	for some fixed $a,b\in\C$, where $b$ is a \emph{directional} vector whose absolute value may be assumed to be $1$. Since $b \ne 0$, we equivalently have
	\[ L = \{ z : \Im\left( \frac{z-a}{b} \right) = 0 \}. \]
	We can also define the half-planes
	\begin{align*}
		H_a &= \{ z : \Im\left( \frac{z-a}{b} \right) > 0 \} \\
		K_a &= \{ z : \Im\left( \frac{z-a}{b} \right) < 0 \}.
	\end{align*}
	Note that $H_a = a + H_0$, where the addition is Minkowski addition:
	\[ H_a = \{ a + z : z \in H_0 \}. \]

\subsection{The extended plane}
	
	Define $\C_\infty = \C \cup \{\infty\}$ and let $S = \{(x_1,x_2,x_3) : x_1^2 + x_2^2 + x_3^2 = 1\}$ be the unit sphere in $\R^3$. We shall show a bijection from $\C_\infty$ to $S$.\\
	Let $N = (0,0,1)$ be the `north pole' of $S$, and orient $\C$ (as $\R^2$) in the horizontal plane in a manner such that $\C$ cuts $S$ along the equator. For $z = x + \iota y \in \C$, let us define the corresponding point $Z = (x_1,x_2,x_3) \in S$. We shall draw a line connecting $z$ to $N$, and let $Z$ be the point of intersection (other than $N$) of this line with $S$. Finally, we shall map $\infty$ to $N$.\\
	Let us define this more explicitly. The line through $N$ and $z$ is
	\[ L = \{ tN + (1-t)z : t \in \R \}. \]
	Then, letting $z = (x,y,0)$, we have
	\[ t^2 + (1-t)^2 |z|^2 = 1. \]
	So,
	\[ |z|^2 = \frac{1-t^2}{(1-t)^2} = \frac{1+t}{1-t} \]
	and
	\[ t = \frac{|z|^2-1}{|z|^2+1}. \]
	Therefore, we map $z$ to
	\[ Z = \left( \frac{2 \Re z}{|z|^2+1} , \frac{2 \Im z}{|z|^2+1} , \frac{|z|^2-1}{|z|^2+1}. \right) \in S. \]
	Based on this, we can define a distance metric between points in $\C_\infty$. For $z,z' \in \C_\infty$ mapping to $Z,Z' \in S$, we let $d(z,z')$ be the Euclidean distance between $Z,Z'$ in $\R^3$. More explicitly,
	\begin{align*}
		d(z,z')^2 &= (x_1 - x_1')^2 + (x_2 - x_2')^2 + (x_3 - x_3')^2 \\
			&= 2 - 2(x_1x_1' + x_2x_2' + x_3x_3') \\
			&= \frac{2|z-z'|}{\left((|z|^2+1) (|z'|^2+1)\right)^{1/2}}
	\end{align*}
	when $z,z'\in\C$ and if $z' = \infty$ (so $Z' = (0,0,1)$), we have
	\[ d(z,z') = \frac{4}{|z|^2 + 1} \]

	This correspondence between points of $S$ and $\C_\infty$ is called the \emph{stereographic projection}. 

	\begin{exercise}
		If $P$ is a plane in $\R^3$ and $\Lambda = P \cap S$ is a circle on $S$, show that the projection of $\Lambda$ on $\C$ under the stereographic projection is a circle as well (possibly a circle of infinite radius, namely a line).
	\end{exercise}

\subsection{Power series}

	In this section, we begin discussing convergence of series in $\C$ and related properties.

	\begin{fdef}
		If $a_n \in \C$ for every $n \ge 0$, the series $\sum_{n=0}^\infty a_n$ is said to \emph{converge} to $z$ iff for all $\epsilon>0$, there exists $N \in \N$ such that
		\[ \left| \sum_{n=0}^m a_n - z  \right| < \epsilon \]
		for all $m \ge N$.\\
		The series $\sum_{n=0}^\infty a_n$ is said to converge \emph{absolutely} if $\sum_{n=0}^\infty |a_n|$ converges.
	\end{fdef}

	\begin{theorem}
		$\C$ is complete. That is, every Cauchy sequence in $\C$ is convergent.
	\end{theorem}
	\begin{proof}
		Suppose $\{x_n + \iota y_n\}$ is a Cauchy sequence in $\C$, where $x_n, y_n \in \R$ for each $n$. We then have the existence of $N \in \N$ such that for all $m,k > N$, $|(x_m - x_k) + \iota(y_m - y_k)| < \epsilon$. Consequently, $|x_m - x_k| < \epsilon$ and $|y_m - y_k| < \epsilon$. However, since $\R$ is complete, this implies that $(x_n)$ and $(y_n)$ are convergent, completing the proof. 
	\end{proof}	

	\begin{theorem}
		If $\sum a_n$ converges absolutely, $\sum a_n$ converges.
	\end{theorem}
	\begin{proof}
		Let $\epsilon > 0$, $z_n = \sum_{i=0}^n a_i$, and $S_n = \sum_{i=0}^n |a_i|$. Because $\C$ is complete, it suffices to show that $(z_n)$ is Cauchy. \\
		Since $\sum |a_n|$ is convergent, there exists $N \in \N$ such that $|S_m - S_k| < \epsilon$ for all $m,k > N$. Supposing $m > k$, we have
		\[ S_m - S_k = \sum_{i=k+1}^m |a_i|. \]
		So,
		\begin{align*}
			|z_m - z_k| &= \left| \sum_{i=k+1}^m a_i \right| \\
				&\le \sum_{i=k+1}^m |a_i| < \epsilon,
		\end{align*}
		completing the proof.
	\end{proof}

	\begin{exercise}
		\label{ex: conv of summ zn}
		Show that $\sum_{n=0}^\infty z^n$ converges iff $|z| < 1$.
	\end{exercise}

	\begin{ftheo}
		For a given power series $\sum_{n=0}^\infty a_n (z-a)^n$, define the number $R$ ($0\le R\le \infty$) by
		\[ \frac{1}{R} = \limsup_{n\to\infty} |a_n|^{1/n}. \]
		Then,
		\begin{itemize}
			\item[(a)] If $|z-a| < R$, the series converges absolutely.
			\item[(b)] If $|z-a| > R$, the terms of the series become unbounded and the series diverges.
			\item[(b)] If $0 < r < R$, the series converges uniformly on the set $\{ z : |z-a| \le r \}$.
		\end{itemize}
	\end{ftheo}

	This $R$ is referred to as the \emph{radius of convergence} of the power series.

	\begin{proof}
		\phantom{easter}
		\begin{itemize}
			\item[(a)]
			We assume without loss of generality that $a = 0$. If $|z| < R$, there exists $r$ with $|z| < r < R$. By the definition of $R$, for all $\epsilon>0$, there exists $N \in \N$ such that
			\[ \frac{1}{R} - \epsilon < \sup_{k \ge n} |a_k|^{1/k} < \frac{1}{R} + \epsilon \]
			for all $n > N$. If we take $\epsilon = 1/r - 1/R$, it follows that $|a_n|^{1/n} < 1/r$ for all $n > N$. That is, for all $n > N$, $|a_n| < 1/r^n$ and so 
			\[ |a_n z^n| < \left(\frac{|z|}{r}\right)^n. \]
			Therefore, $\sum_{n=N}^\infty a_n z^n$ is dominated by $\sum_{n=N}^\infty (|z|/r)^n$. Now however, we can just use the result of \Cref{ex: conv of summ zn} to conclude absolute convergence since $|z|/r < 1$.
			
			\item[(b)]
			Let $|z| > R$ and choose $r$ with $|z| > r > R$. For $\epsilon > 0$, there exists $N \in \N$ such that
			\[ \frac{1}{R} - \epsilon < \sup_{k \ge n} |a_k|^{1/k} \text{ for all $n > N$}. \]
			Choosing $\epsilon = 1/R - 1/r$,
			\[ |a_n|^{1/n} > 1/r \]
			for infinitely many $n \in \N$. It follows that $|a_n z^n| > (|z|/r)^n$ for infinitely many $n \in \N$. Since $|z|/r > 1$, these terms become unbounded and therefore the series diverges.

			\item[(c)]
			Now, suppose $r<R$ and choose $\rho$ such that $r<\rho<R$. Similar to the argument in (a), we get that
			\[ |a_n| < \frac{1}{\rho^n} \text{ for all $n\ge N$.} \]
			If $|z| \le r$, $|a_n z^n| \le (r/\rho)^n$ and $r/\rho < 1$. The Weierstrass $M$-test then gives that the power series converges uniformly on $\{z : |z| \le r\}$. \qedhere
		\end{itemize}
	\end{proof}

	It should be noted that we cannot conclude anything when $|z-a|=R$.

	\begin{theorem}
		If $\sum a_n (z-a)^n$ is a power series with radius of convergence $R$, then if it exists,
		\[ \lim_{n\to\infty} \left| \frac{a_n}{a_{n+1}} \right| = R. \] 
	\end{theorem}
	\begin{proof}
		Again, assume that $a = 0$ and let $\alpha = \lim |a_n / a_{n+1}|$, which we assume exists. Suppose that $|z| < \alpha$ and take $r \in \R$ such that $|z| < r < \alpha$. For all $\epsilon > 0$, there exists $N \in \N$ such that for $n \ge N$,
		\[ \alpha - \epsilon < \left|\frac{a_n}{a_{n+1}}\right| < \alpha + \epsilon. \]
		Taking $\epsilon = \alpha - r$, $|a_n / a_{n+1}| > r$ for all $n \ge N$. Let $B = |a_N| r^N$. Then,
		\[ a_{N+1} r^{N+1} = |a_{N+1}| r \cdot r^N < |a_N| r^N = B. \]
		Similarly, we get that $|a_n| r^n < B$ for all $n \ge N$. Therefore,
		\[ |a_n z^n| < B \left( \frac{|z|}{r} \right)^n \]
		for all $n \ge N$. Thus, the sequence converges absolutely since $|z| < r$.\\
		Since $r < \alpha$ was arbitrary, this implies that $\alpha \le R$.\\

		On the other hand, if $|z| > \alpha$, take $r \in \R$ such that $|z| > r > \alpha$. Taking $\epsilon = r - \alpha$, we get $N \in \N$ such that
		\[ \left|\frac{a_n}{a_{n+1}}\right| < r \]
		for all $n \ge N$. Letting $B = |a_N| r^N$ again, we once more obtain that $|a_n| r^n > B$ for all $n \ge N$. This gives that
		\[ |a_n z^n| > B \left( \frac{|z|}{r} \right)^n  \]
		for all $n \ge N$, and since $|z| > r$, the sequence diverges (we may assume that $B \ne 0$ by making $N$ larger if required to ensure that $a_N \ne 0$ -- if this is not possible, the problem is trivial since it means that $(a_n)$ is eventually $0$). Since the choice of $r$ was arbitrary, this implies that $R \le \alpha$, completing the proof.
	\end{proof}

	Now, consider the series
	\[ \sum_{n=0}^\infty \frac{z^n}{n!}. \]
	The radius of convergence of this series is $\infty$. So, it converges for any complex number $z$, and convergence is uniform on every compact subset of $\C$.\\
	The above defines a function $\exp:\C\to\C$.\\
	We also denote $e^z = \exp(z)$.

	\begin{fdef}[Differentiability]
		If $G$ is an open set in $\C$ and $f : G \to \C$, then $f$ is said to be \emph{differentiable} at a point $a \in G$ if the limit
		\[ \lim_{h\to 0} \frac{f(a+h) - f(a)}{h} \]
		exists. If it exists, the value of this limit is denoted $f'(a)$ and is called the \emph{derivative} of $f$ at $a$.
	\end{fdef}

	If $f$ is differentiable at each point of $G$, we say that $f$ is differentiable on $G$. Note that if $f$ is differentiable on $G$, then $f' : G \to \C$ is a function. If $f'$ is continuous, $f$ is said to be \emph{continuously differentiable}.

	\begin{theorem}
		If $f : G \to \C$ is differentiable at a point $a \in G$, $f$ is continuous at $a$.
	\end{theorem}
	\begin{proof}
		The proof of this is direct:
		\begin{align*}
			\lim_{z \to a} |f(z) - f(a)| &= \left( \lim_{z \to a} \frac{|f(z)-f(a)|}{|z-a|} \right) \cdot \lim_{z\to a} |z-a| \\
				&= f'(a) \cdot 0 = 0. \qedhere
		\end{align*}
	\end{proof}

	\begin{fdef}
		A function $f : G \to \C$ is said to be \emph{analytic} if $f$ is continuously differentiable on $G$.
	\end{fdef}

	Let $f,g$ be analytic on $G$ and $\Omega$ respectively, and suppose that $f(G) \subseteq \Omega$. Then, $g \circ f$ is analytic on $G$ and
	\[ (g \circ f)'(z) = g'(f(z)) \cdot f'(z) \]
	for all $z \in G$. This is called the \emph{chain rule}.\\

	We shall show later that if $f$ is differentiable then its derivative is continuous, and so $f$ is analytic.

	\begin{ftheo}
		Let $f(z) = \sum_{n=0}^\infty a_n (z-a)^n$ have radius of convergence $R > 0$. Then
		\begin{itemize}
			\item[(a)] For each $k \ge 1$, the series
			\[ \sum_{n=k}^\infty n(n-1)\cdots(n-k+1) a_n (z-a)^{n-k} \]
			has radius of convergence $R$.

			\item[(b)] The function $f$ is infinitely differentiable on $B(a,R)$ (the open ball of radius $R$ centered at $a$), and further, $f^{(k)}(z)$ is given by the series in (a) for all $k \ge 1$ and $|z-a| < R$.

			\item[(c)] For $n \ge 0$, $a_n = \frac{1}{n!} f^{(n)}(a)$. 
		\end{itemize}
	\end{ftheo}

	\begin{proof}
		Assume that $a = 0$.
		\begin{itemize}
			\item[(a)] Note that it suffices to prove the result for $k = 1$ (Why?). To show this, it is enough to show that
			\[ \limsup_{n\to\infty} |a_n|^{1/n} = \limsup_{n\to\infty} |na_n|^{1/(n-1)} \]
			First, it is not difficult to show that $\lim_{n\to\infty} n^{1/(n-1)} = 1$. It may be shown that for any sequences $(c_n),(d_n)$ in $\R$ where $c_n \ge 0$, if $\lim c_n = c$ and $\limsup d_n = d$, then $\limsup c_n d_n = cd$. Therefore, we are done if we show that $\limsup_{n\to\infty} |a_n|^{1/n} = \limsup_{n\to\infty} |a_n|^{1/(n-1)}$.
			\[ \sum_{n=0}^\infty a_n z^n = a_0 + z\sum_{n=0}^\infty a_{n+1} z^n. \]
			Let $R'$  be the radius of convergence of $\sum_{n=0}^\infty a_{n+1} z^n$. We want to show that $R' = R$. \\
			If $|z| < R'$, then 
			\[ \sum |a_n z^n| \le |a_0| + |z| \sum_{n=0}^\infty |a_{n+1} z^n| < \infty, \]
			so $R' \le R$. On the other hand, if $|z| < R$ and $z \ne 0$,
			\[ \sum |a_{n+1} z^n| < \frac{1}{|z|} \left(\sum |a_n z^n| + |a_0|\right) < \infty, \]
			so $R \le R'$ and we are done.

			\item[(b)] Once again, it suffices to prove the result for $k = 0$. For $|z| < R$ and $g(z) = \sum_{n=1}^\infty n a_n z^{n-1}$,
			\[ s_n(z) = \sum_{k=0}^n a_k z^k \text{ and } R_n(z) = \sum_{k=n+1}^\infty a_k z^k, \]
			fix a point $w \in B(0,R)$ and $r$ such that $|w| < r < R$. We wish to show that $f'(w)$ exists and is equal to $g(w)$. Let $\delta > 0$ be arbitrary with $\overline{B(w,\delta)} \subseteq B(0,r)$. Letting $z \in B(w,\delta)$, we have
			\[ \frac{f(z)-f(w)}{z-w} - g(w) = \frac{s_n(z)-s_n(w)}{z-w} - s_n'(w) + s_n'(w) - g(w) + \frac{R_n(z) - R_n(w)}{z-w}. \]
			We have
			\[ |z^k - w^k| = |z-w||z^{k-1} + z^{k-2}w + \cdots + w^{k-1}| \le |z-w| kr^{k-1}. \]
			Therefore,
			\[ \left| \frac{R_n(z) - R_n(w)}{z-w} \right| = \left| \sum_{k=n+1}^\infty a_k \frac{z^k - w^k}{z-w} \right| \le \sum_{k=n+1}^\infty |a_k| k r^{k-1}. \]
			Since $r < R$, $\sum_{k=1}^\infty |a_k| k r^{k-1}$ converges and so for any $\epsilon > 0$, there exists $N_1 \in N$ such that for $n \ge N_1$,
			\[ \left| \frac{R_n(z) - R_n(w)}{z-w} \right| < \epsilon/3. \]
			Since $\lim s_n'(w) = g(w)$, there exists $N_2 \in \N$ such that
			\[ |s_n'(w) - g(w)| < \epsilon/3 \]
			for $n \ge N_2$. Choose $n \ge \max(N_1, N_2)$. Then, there exists $\delta > 0$ such that whenever $0 < |z-w| < \delta$,
			\[ \left| \frac{s_n(z) - s_n(w)}{z-w} - s_n'(w) \right| < \epsilon/3. \]
			Putting all these together, we get the desideratum.

			\item[(c)] This is straightforward using the explicit expression for $f^{(k)}(a)$. \qedhere
		\end{itemize}
	\end{proof}

	If the series $f(z) = \sum_{n=0}^\infty a_n (z-a)^n$ has radius of convergence $R > 0$, then $f$ is analytic on $B(a,R)$. Therefore, $\exp$ is analytic on $\C$.\\
	%%% LECTURE 4
	Further, letting $g = \exp$,
	\[ g'(z) = \sum_{n=1}^\infty \frac{n}{n!} z^{n-1} = \sum_{n=1}^\infty \frac{1}{(n-1)!} z^{n-1} = g(z). \]

	Define the functions $\cos$ and $\sin$ using power series as
	\begin{align*}
		\cos z &= 1 - \frac{z^2}{2!} + \frac{z^4}{4!} + \cdots + (-1)^k \frac{z^{2k}}{(2k)!} + \cdots \\
		\sin z &= z - \frac{z^3}{3!} + \frac{z^5}{5!} + \cdots + (-1)^k \frac{z^{2k+1}}{(2k+1)!} + \cdots
	\end{align*}
	Note that
	\[ \cos z = \frac{e^{\iota z} + e^{-\iota z}}{2} \text{ and } \sin z = \frac{e^{\iota z} - e^{-\iota z}}{2\iota}. \]
	Therefore,
	\[ e^{\iota z} = \cos z + \iota \sin z. \]
	In particular, if $z = \theta \in \R$,
	\[ e^{\iota \theta} = \cos \theta + \iota \sin \theta. \]
	It is direct to show next that $\cos^2 z + \sin^2 z = 1$ for $z \in \C$.

	\begin{definition}
		A function $f$ is said to be \emph{periodic} with period $c$ if $f(z) = f(z+c)$ for all $z\in\C$.
	\end{definition}

	$e^{z}$ is periodic with period $2\pi\iota$.\\

	Similar to $\cos$ and $\sin$, one can define the function $\log$ as
	\[ \log (1 + z) = z - \frac{z^2}{2} + \frac{z^3}{3} - \frac{z^4}{4} + \cdots. \]
	$\log z$ is defined only when $|z-1| < 1$. Further note that we cannot define $\log$ as the inverse of $\exp$ (as we do over the reals) since $\exp$ is not injective here.\\
	We would like to define $\log$ such that $w = \exp z$ when $z = \log w$. Since $\exp$ is non-zero, also suppose that $w \ne 0$. If $z = x + \iota y$, then $|w| = e^x$ and $\arg w = y + 2\pi k\iota$ for some $k\in\Z$. Therefore, the solution set for $e^z = w$ is
	\[ \{ \log w + \iota (\arg w + 2\pi k) : k \in \Z \}. \]

	\begin{fdef}
		If $G$ is an open connected set in $\C$ and $f : G \to \C$ is a continuous function such that $z = \exp(f(z))$ for all $z \in G$, then $f$ is a \emph{branch of the logarithm}.
	\end{fdef}

	\begin{lemma}
		If $G \subseteq \C$ is open and connected and $f$ is a branch of the logarithm on $G$, then the totality of the branches of $\log z$ are the functions $f(z) + 2\pi k\iota$ for $k \in \Z$.
	\end{lemma}
	\begin{proof}
		If $g(z) = f(z) + 2\pi k\iota$ for some $k \in \Z$, then $\exp(g(z)) = \exp(f(z)) = z$, so $g$ is also a branch of the logarithm.\\
		On the other hand, suppose that $g$ is a branch of the logarithm. For $z\in G$, $\exp(f(z)) = \exp(g(z)) = z$, so $g(z) = f(z) + 2\pi k\iota$. However, note that this $k$ depends on $z$. We must show that the same $k$ works for all $z$. Indeed, $h(z) = (g(z) - f(z))/2\pi\iota$ is continuous on $G$ and $h(G) \subseteq \Z$, so the required follows.
	\end{proof}


	Now, let $G = \C \setminus \R_{\le 0}$. Clearly, $G$ is connected and each $z \in G$ can be uniquely denoted by $|z| e^{\iota\theta}$, where $-\pi < \theta < \pi$. For $\theta$ in this range, define
	\[ f(r e^{\iota\theta}) = \log r + \iota\theta. \]
	This is a branch of the logarithm on $G$, and is commonly referred to as the \emph{principal branch}.

	\begin{theorem}
		Let $G,\Omega$ be open subsets of $\C$. Suppose that $f : G \to \C$ and $g : \Omega \to \C$ are continuous such that $g(f(z)) = z$ for all $z \in G$. If $G$ is differentiable and $g'(z) \ne 0$, $f$ is differentiable and
		\[ f'(z) = \frac{1}{g'(f(z))}. \]
		If $g$ is analytic, so is $f$.
	\end{theorem}
	\begin{proof}
		Fix $a \in G$ and let $h \in \C \setminus \{0\}$ with $a+h \in G$. Since $g(f(a)) = a \ne a+h = g(f(a+h))$, $f(a) \ne f(a+h)$.
		Also,
		\[ 1 = \frac{g(f(a+h)) - g(f(a))}{h} = \frac{g(f(a+h)) - g(f(a))}{f(a+h)-f(a)} \cdot {f(a+h)-f(a)}{h}. \]
		Take the limit of either side as $h \to 0$. The first fraction is equal to $g'(f(a))$ since $\lim_{h\to 0} (f(a+h) - f(a)) = 0$, and therefore $\lim_{h\to0} (f(a+h) - f(a))/h = f'(a)$ exists, and $1 = g'(f(a)) \cdot f'(a)$. The required follows.\\
		If $g$ is analytic, then $g'$ is continuous so $f$ is analytic.
	\end{proof}

	\begin{corollary}
		Any branch of the logarithm function is analytic and has derivative $z \mapsto 1/z$.
	\end{corollary}

	Given a branch of the logarithm $f$ on an open connected set $G$ and fixed $b \in \C$, define $g(z) = \exp(bf(z))$. If $b \in \Z$, $g(z) = z^b$. In general, this defines a branch of $z^b$ ($b\in\C$) for any open connected set on which there is a branch of $\log z$.\\
	If we write $z^b$ as a function, it is implicitly understood that the $f$ in $\exp(bf(z))$ is the principal branch of the logarithm. Since $\log$ is analytic, so is $z \mapsto z^b$.



	%%% LECTURE OF 19-01-2022 (MISSED)

\subsection{Cauchy-Riemann Equations}

	Let $f:G\to\C$ be analytic and let
	\[ u(x,y) = \Re(f(x+\iota y)), v(x,y) = \Im(f(x+\iota y)) \]
	for $x+\iota y \in G$. Let us evaluate the limit
	\[ f'(z) = \lim_{h\to 0} \frac{f(z+h)-f(z)}{h}. \]
	in two different ways.\\
	First, if we let $h\to 0$ through real values, we get
	\[ f'(z) = \frac{\partial u}{\partial x}(x,y) + \iota \frac{\partial v}{\partial x}(x,y). \]
	Along the imaginary axis, we get
	\[ f'(z) = -\iota \frac{\partial u}{\partial y}(x,y) + \frac{\partial v}{\partial y}(x,y). \]
	Therefore,
	\[ \frac{\partial u}{\partial x} = \frac{\partial v}{\partial y} \text{ and } \frac{\partial u}{\partial y} = - \frac{\partial v}{\partial x}. \]

	Supposing that $u$ and $v$ have continuous second derivative (we shall later show that they are infinitely differentiable), we have that
	\[ \frac{\partial^2 u}{\partial x^2} = \frac{\partial^2 v}{\partial x \partial y} \text{ and } \frac{\partial^2 u}{\partial y^2} = - \frac{\partial^2 v}{\partial y \partial x}. \]
	Therefore, since the second derivatives are continuous,
	\begin{equation}
		\label{eqn-harmonic}
		\frac{\partial^2 u}{\partial x^2} + \frac{\partial^2 u}{\partial y^2} = 0.
	\end{equation}
	A function $u$ with continuous second partial derivatives satisfying \Cref{eqn-harmonic} is said to be \emph{harmonic}. Similarly, $v$ is also harmonic.

	\begin{ftheo}
		Let $u$, $v$ be real-valued functions defined on an open connected set (a \emph{region}) $G$ and suppose that they have continuous second partial derivatives. Then, $f:G\to\C$ defined by $f(z) = u(z) + \iota v(z)$ is analytic iff $u$ and $v$ satisfy the Cauchy-Riemann equations.
	\end{ftheo}
	\begin{proof}
		We have already shown the forward direction.\\
		For the other direction, let $z = x + \iota y \in G$ and $B(z,r) \subseteq G$. Let $h = s + \iota t \in B(0,r)$. Our goal is to show that for all $\epsilon > 0$, there exists $\delta > 0$ such that
		\[ \left| \frac{f(z+h) - f(z) - f'(z)h}{h} \right| < \epsilon \]
		for all $h \in B(0,\delta)$ for some $f'(z) \in \C$.
		Note that
		\[ u(x+s,y+t) - u(x,y) = \left( u(x+s,y+t) - u(x,y+t) \right) + \left( u(x,y+t) - u(x,y) \right). \]
		Now, for fixed $t \in (-r,r)$, $s \mapsto u(x+s,y+t)$ is a differentiable function on $(-r,r)$. We apply the mean value theorem to conclude that there exist $s_1,t_1 \in (-r,r)$ for each $s+\iota t \in B(0,r)$ such that $|s_1| < |s|$, $|t_1| < |t|$, and
		\begin{align*}
			u(x+s,y+t) - u(x,y+t) &= u_x(x+s_1,y+t)s \\
			u(x,y+t) - u(x,y) &= u_y(x,y+t_1)t.
		\end{align*}
		Now, let
		\[ \varphi(s,t) = \left( u(x+s,y+t) - u(x,y) \right) - \left( u_x(x,y)s + u_y(x,y)t \right). \]
		We get that
		\[ \varphi(s,t) = \left(s u_x(x+s_1,y+t) - s u_x(x,y)\right) + \left(t u_y(x,y+t_1) - t u_y(x,y)\right). \]
		So,
		\[ \frac{\varphi(s,t)}{s+\iota t} = \frac{s}{s+\iota t} \left(u_x(x+s_1,y+t) - u_x(x,y)\right) + \frac{t}{s+\iota t} \left(u_y(x,y+t_1) - u_y(x,y)\right) \]
		and on taking the limit of both sides as $s+\iota t \to 0$, we can use the fact that $|s| \le |s+\iota t|$, $|t| \le |s+\iota t|$, $|s_1| < |s|$, $|t_1| < t$, and the continuity of $u_x$, $u_y$, to conclude that
		\[ \lim_{s+\iota t \to 0} \frac{\varphi(s,t)}{s+\iota t} = 0. \]
		Therefore,
		\[ u(x+s,y+t) - u(x,y) = u_x(x,y)s + u_y(x,y) t + \varphi(s,t). \]
		We get a similar equation for $v$ as well, with a function $\psi$ (in place of $\varphi$). Combining the two,
		\begin{align*}
			\frac{f(z+s+\iota t) - f(z)}{s+\iota t} &= \frac{u(x+s,y+t)-u(x,y)}{s+\iota t} + \iota \frac{v(x+s,y+t) - v(x,y)}{s+\iota t} \\
				&= \frac{s u_x(x,y) + t u_y(x,y) + \varphi(s,t) + \iota \left( s v_x(x,y) + t v_y(x,y) + \psi(s,t) \right)}{s+\iota t} \\
				&= \frac{u_x(x,y) (s+\iota t) + \iota v_x(x,y) (s+\iota t) + \varphi(s,t) + \iota \psi(s,t)}{s+\iota t},
		\end{align*}
		where we used Cauchy-Riemann equations in the final step and thus,
		\[ \lim_{s+\iota t \to 0} \frac{f(z+s+\iota t) - f(z)}{s+\iota t} = u_x(x,y) + \iota v_x(x,y), \]
		completing the proof. Since $u_x$ and $v_x$ are continuous, $f'$ is continuous and $f$ is analytic.
	\end{proof}

	A next question is: given some $u$ such that
	\[ \frac{\partial^2 u}{\partial x^2} + \frac{\partial^2 u}{\partial y^2}, \]
	when does there exist harmonic $v$ such that $u + \iota v$ is analytic? Such a $v$ is referred to as a \emph{harmonic conjugate} of $u$.

	It turns out that the answer is not always. Indeed, $u(x,y) = \log((x^2+y^2)^{1/2})$ on $\C\setminus\{0\}$, despite being harmonic, does not have a harmonic conjugate.

	\begin{ftheo}
		Let $G$ be either the entirety of $\C$ or some open disk. If $u : G \to \R$ is a harmonic function, then $u$ has a harmonic conjugate.
	\end{ftheo}
	\begin{proof}
		Let $G = B(0,R)$ for some $0 < R \le \infty$ and let $u : G \to \R$ be analytic. Define
		\[ v(x,y) = \int_0^y u_x(x,t) \dif t + \varphi(x) \]
		so that $u_x = v_y$. We shall determine $\varphi$ such that $v_x = - u_y$. Differentiating with respect to $x$, we get
		\begin{align*}
			v_x(x,y) &= \int_0^y u_{xx}(x,t) \dif t + \varphi'(x) \\
				&= - \int_0^y u_{yy}(x,t) \dif t + \varphi'(x) \\
				&= - u_y(x,y) + u_y(x,0) + \varphi'(x).
		\end{align*}
		Therefore, $\varphi'(x) = - u_y(x,0)$, and the function
		\[ v(x,y) = \int_0^y u_x(x,t) \dif t - \int_0^x u_y(s,0) \dif s \]
		is a harmonic conjugate of $u$.
	\end{proof}

	The above proof requires that the entire segments $[(0,0),(x,0)]$ $[(x,0),(x,y)]$ are contained in $G$, which is true when we are on a disk.

\subsection{Transformations}

	Consider the two hyperbolas defined by
	\begin{align*}
		x^2-y^2 &= c \\
		2xy &= d,
	\end{align*}
	where $c,d \ne 0$.\\
	This gives
	\[ y = \pm \sqrt{ \frac{-c \pm \sqrt{d^2+c^2}}{2} }. \]
	Consider the functions
	\begin{align*}
		u(x,y) &= x^2 - y^2 \\
		v(x,y) &= 2xy.
	\end{align*}
	The two hyperbolas above are mapped by this $f = u + \iota v$ to the straight lines $u=c$ and $v=d$.\\

	\begin{fdef}
		A \emph{path} in a region $G \subseteq \C$ is a continuous function $\gamma:[a,b]\to G$ for some interval $[a,b]$ in $\R$. If $\gamma'(t)$ exissts for each $t \in [a,b]$ and $\gamma':[a,b]\to\C$ is continuous, then $\gamma$ is said to be \emph{smooth}. $\gamma$ is said to be \emph{piecewise smooth} if there is a partition $a=t_0 < t_1 < \cdots t_{n-1} < t_n = b$ of $[a,b]$ such that $\gamma$ is smooth on each subinterval $[t_{i-1},t_i]$ for $1\le i\le n$.\\
		For a path $\gamma : [a,b] \to \C$, $[a,b]$ is sometimes referred to as the \emph{trace} of $\gamma$ and denoted $\{\gamma\}$.
	\end{fdef}

	By the existence of $\gamma'$, we mean that the two-sided limit
	\[ \lim_{h\to 0} \frac{\gamma(t+h)-\gamma(t)}{h} \]
	exists for $t \in (a,b)$ and the right and left sided limits exist for $t = a,b$ respectively. This is equivalent to saying that $\Re \gamma$ and $\Im \gamma$ have derivatives.

	Suppose $\gamma : [a,b] \to G$ is a smooth path and for some $t_0 \in (a,b)$, $\gamma'(t_0) \ne 0$. Then, $\gamma$ has a \emph{tangent line} at the point $z_0 = \gamma(t_0)$. This line goes through the point $z_0$ in the direction of the vector $\gamma'(t_0)$, that is, the slope of the line is $\tan(\arg \gamma'(t_0))$.\\

	If $\gamma_1$ and $\gamma_2$ are two smooth paths with $\gamma_1(t_1) = \gamma_2(t_2) = z_0$ and $\gamma_1'(t_1),\gamma_2'(t_2) \ne 0$, then define the \emph{angle} between the paths $\gamma_1,\gamma_2$ at $z_0$ to be $\arg(\gamma_2'(t_2)) - \arg(\gamma_1'(t_1))$.\\

	Suppose $\gamma$ is a smooth path in $G$ and $f:G\to\C$ is analytic. Then, $\sigma = f \circ \gamma$ is also a smooth path and $\sigma'(t) = f'(\gamma(t)) \cdot \gamma'(t)$. Further, if $z_0$ is a fixed point of $f$ with $\gamma(t_0) = z_0$,
	\[ \arg(\sigma'(t_0)) - \arg(\gamma'(t_0)) = \arg(f'(z_0)). \]
	Let $\gamma_1,\gamma_2$ be smooth paths with $\gamma_1(t_1) = \gamma_2(t_2) = z_0$ with non-zero derivatives at $t_1,t_2$ respectively, and let $\sigma_1 = f \circ \gamma_1, \sigma_2 = f \circ \gamma_2$. Further suppose that the two paths $\gamma_1,\gamma_2$ are not tangent to each other at $z_0$. Then,
	\[ \arg(\gamma_2'(t_2)) - \arg(\gamma_1'(t_1)) = \arg(\sigma_2'(t_2)) - \arg(\sigma_1'(t_1)). \]
	This says that the angle between two paths are preserved after applying an analytic function to both. A function $f$ satisfying this is said to have the \emph{angle-preserving property}.

	\begin{fdef}
		A function $f : G \to \C$ which has the angle-preserving property and also has 
		\[ \lim_{z\to a} \left|\frac{f(z)-f(a)}{z-a}\right| \]
		existing is called a \emph{conformal map}.
	\end{fdef}

	It turns out that a function $f$ is a conformal map if and only if it is analytic and $f'(z) \ne 0$ for all $z$ (How?).

	\begin{fdef}
		A mapping of the form
		\[ S(z) = \frac{az+b}{cz+d} \]
		is called a \emph{linear fractional transformation}. If we further have that $ad-bc \ne 0$, then $S(z)$ is called a \emph{M\"{o}bius transformation}.
	\end{fdef}

	We have
	\[ S'(z) = \frac{ad-bc}{(cz+d)^2}. \]

	If $w = S(z)$, it is relatively simple to show that
	\[ z = S^{-1}(w) = \frac{dw-b}{-cw+a}. \]
	Therefore, the inverse of a M\"{o}bius transformation is a M\"{o}bius transformation. The composition of two M\"{o}bius transformations is a M\"{o}bius transformation as well.\\
	Also observe that the coefficiencts $a,b,c,d$ for a given M\"{o}bius transformation are not unique since we can multiply them by a constant. We may also extend $S$ to $\C_\infty$ with $S(\infty) = a/c$ and $S(-d/c) = \infty$.\\

	$S(z) = z+a$ is called a \emph{translation}, $S(z) = az$ with $a\ne 0$ is called a \emph{dilation}, $S(z) = e^{\iota\theta} z$ is called a \emph{rotation}, and $S(z) = 1/z$ is called the \emph{inversion}. We shall see later that any M\"{o}bius transformation is a composition of these five types of transformations.

	What are the fixed points of a M\"{o}bius transformation $S$? $S(z) = z$ gives
	\[ cz^2 + (a-d)z + b = 0. \]
	Therefore, a M\"{o}bius transformation has at most two fixed points unless $S(z) = z$ for all $z \in \C_\infty$.\\

	Let $a,b,c\in\C_\infty$ be distinct with $S(a) = \alpha$, $S(b) = \beta$, $S(c) = \gamma$. Let $T$ be another M\"{o}bius transformation with $T(a) = \alpha$, $T(b) = \beta$, $T(c) = \gamma$. Then $T^{-1} \circ S$ has three (distinct) fixed points, and therefore $S = T$.\\
	Therefore, any M\"{o}bius transformation is uniquely determined by its value at any three distinct points.\\

	Let $z_2,z_3,z_4 \in \C_\infty$ be distinct. Define $S:\C_\infty\to\C_\infty$ by
	\[
	S(z) =
	\begin{cases}
		\frac{(z-z_3)/(z-z_4)}{(z_2-z_3)/(z_2-z_4)}, & z_2,z_3,z_4 \in \C, \\
		\frac{z_2-z_4}{z-z_4}, & z_3 = \infty, \\
		\frac{z-z_3}{z_2-z_3}, & z_4 = \infty. \\
	\end{cases}
	\]
	In any case, $S(z_2) = 1$, $S(z_3) = 0$, $S(z_4) = \infty$, and $S$ is the only transformation having this property.

	\begin{definition}
		If $z_1 \in \C_\infty$, then $(z_1,z_2,z_3,z_4)$ is referred to as the {\emph{cross-ratio}} of $z_1,z_2,z_3,z_4$ and is the image of $z_1$ under the M\"{o}bius transformation described above, which is the unique M\"{o}bius transformation taking $z_2$ to $1$, $z_3$ to $0$, and $z_4$ to $\infty$.
	\end{definition}

	For example, $(z_2,z_2,z_3,z_4) = 1$ and $(z,1,0,\infty) = z$.\\
	If $M$ is any M\"{o}bius transformation with $M(w_2) = 1$, $M(w_3) = 0$, $M(w_4) = \infty$, then $M(z) = (z,w_2,w_3,w_4)$ for all $z \in \C_\infty$.

	\begin{theorem}
		\label{thm: cross-product mobius equal}
		If $z_2,z_3,z_4$ are distinct points and $T$ is any M\"{o}bius transformation, then
		\[ (z_1,z_2,z_3,z_4) = (Tz_1,Tz_2,Tz_3,Tz_4). \]
	\end{theorem}
	\begin{proof}
		Let $S(z) = (z,z_2,z_3,z_4)$. If $M = ST^{-1}$, then
		\[ M(T(z_2)) = 1, \quad M(T(z_3)) = 0, \quad M(T(z_4)) = \infty. \]
		Therefore, $M = (z,Tz_2,Tz_3,Tz_4)$. That is,
		\[ ST^{-1}z = (z,Tz_2,Tz_3,Tz_4) \]
		for all $z \in \C_\infty$. Setting $z = Tz_1$ yields the required.
	\end{proof}

	\begin{lemma}
		If $\{z_2,z_3,z_4\},\{w_2,w_3,w_4\} \subseteq \C_\infty$, then there exists a unique M\"{o}bius transformation $S$ with $Sz_i = w_i$ for each $i$.
	\end{lemma}
	
	We omit the proof of the above.

	\begin{lemma}
		\label{thm: mobius real to circle}
		Let $\{z_1,z_2,z_3,z_4\} \subseteq \C_\infty$. Then, $(z_1,z_2,z_3,z_4)$ is real iff the four points lie on a circle.
	\end{lemma}
	\begin{proof}
		Define $S:\C_\infty\to\C_\infty$ by $Sz = (z,z_2,z_3,z_4)$. We are done if we show that $S^{-1}(\R_\infty)$ is a circle (since a circle is uniquely determined by three distinct points on it).\\
		Let $S^{-1}(z) = (az+b)/(cz+d)$.\\

		First, let us show that $S^{-1}(\R_\infty) \subseteq \Gamma$ for a circle $\Gamma$ in $\C_\infty$. Let $w \in S^{-1}(\R_\infty)$. Then, $Sw = \overline{Sw}$ so
		\[ \frac{aw+b}{cw+d} = \frac{\overline{a}\overline{w} + \overline{b}}{\overline{c}\overline{w} + \overline{d}}. \]
		This gives that
		\begin{equation}
			\tag{$*$}
			\label{eqn: mobius circle real}
			(a\overline{c} - \overline{a}c)|w|^2 + (a\overline{d} - \overline{b}c)w + (b\overline{c} - d\overline{a})\overline{w} + (b\overline{d} - \overline{b}d) = 0.
		\end{equation}
		If $a\overline{c}$ is real, we get that
		\[ \Im\left((a\overline{d}-\overline{b}c)w + b\overline{d}\right) = 0, \]
		which is a circle through $\infty$ (a line).\\
		If on the other hand $a\overline{c}$ is not real, then (\ref{eqn: mobius circle real}) becomes
		\[ 2\iota\underbrace{\Im(a\overline{c})}_{\alpha \ne 0}|w|^2 + (a\overline{d} - b\overline{c})w + (b\overline{c} - \overline{a}d)\overline{w} + (b\overline{d} - \overline{b}d) = 0. \]
		Dividing by $2\iota\alpha$,
		\[ |w|^2 + \frac{(a\overline{d} - b\overline{c})w}{2\iota\alpha} + \frac{(b\overline{c} - \overline{a}d)\overline{w}}{2\iota\alpha} + \frac{(b\overline{d} - \overline{b}d)}{2\iota\alpha} = 0. \]
		Since $\alpha$ is real,
		\[ \overline{\frac{(b\overline{c} - \overline{a}d)\overline{w}}{2\iota\alpha}} = \frac{(a\overline{d} - b\overline{c})w}{2\iota\alpha} \]
		and
		\[ \frac{(b\overline{d} - \overline{b}d)}{2\iota\alpha} \]
		is real. This gives
		\[ |w|^2 + \overline{\gamma}w + \gamma\overline{w} - \delta = 0 \]
		for some $\gamma \in \C, \delta \in \R$. This is equivalent to $|w+\gamma| = (|\gamma|^2 + \delta)^{1/2}$, which is the equation of a circle\footnote{it may be checked that $|\gamma|^2 + \delta$ is a positive real by substituting their values.}.\\

		Letting $T = S^{-1}$ and $\Gamma$ be the circle obtained in the previous part of the proof, we must now show that $T(\R_\infty) = \Gamma$. Since $\R_\infty$ is connected and compact and $T$ is a homeomorphism, $T(\R_\infty)$ is a closed arc, say $\Gamma_1$, of $\Gamma$. If $\Gamma_1 \ne \Gamma$, let $z_1,z_2$ be the endpoints of this arc. If $T(\infty) = z_3$ which is distinct from $z_1,z_2$, then $\R_\infty \setminus \{\infty\}$ is connected but $\Gamma_1 \setminus \{z_1\}$ is disconnected, which is a contradiction. So, suppose $T(\infty) = z_1$. Then, $\R_\infty \setminus \{\infty,T^{-1}(z_2)\}$ is disconnected but $\Gamma_1 \setminus \{z_1,z_2\}$ is connected, yielding a contradiction once more and completing the proof.
	\end{proof}

	Next, we give a more general version of the above.
	
	\begin{ftheo}
		A M\"{o}bius transformation takes circles to circles.
	\end{ftheo}

	Note that \Cref{thm: mobius real to circle} follows from this since $\R_\infty$ is a circle (of infinite radius) in $\C_\infty$.

	\begin{proof}
		Let $\Gamma$ be a circle in $\C_\infty$ and $S$ a M\"{o}bius transformation. Let $z_2,z_3,z_4$ be three distinct points on $\Gamma$, and set $w_j = Sz_j$ for each $j$. We claim that $S(\Gamma)$ is the circle $\Gamma'$ determined by $w_2,w_3,w_4$. Indeed,
		\[ (z,z_2,z_3,z_4) = (Sz,w_2,w_3,w_4) \]
		for any $z$, and if $z \in \Gamma$, the LHS is real by \Cref{thm: mobius real to circle}, and using the same theorem on the RHS completes the proof.
	\end{proof}

	\begin{fdef}
		Let $\Gamma$ be a circle through $z_2,z_3,z_4$. The points $z,z^* \in \C_\infty$ are said to be \emph{symmetric} with respect to $\Gamma$ if
		\[ (z^*,z_2,z_3,z_4) = \overline{(z,z_2,z_3,z_4)}. \]
	\end{fdef}

	\begin{remark}
		The above definition only depends on $\Gamma$, not the choice of $z_2,z_3,z_4$.
	\end{remark}

	%%% LECTURE 8

	Observe that $z$ is symmetric with respect to itself with respect to $\Gamma$ if and only if $z \in \Gamma$. Indeed, it implies that $(z,z_2,z_3,z_4)$ is real, which by \Cref{thm: mobius real to circle} implies that $z \in \Gamma$.\\
	
	What does it mean for $z,z^*$ to be symmetric?\\
	If $\Gamma$ is a straight line, $z,z^*$ are symmetric with respect to $\Gamma$ iff their perpendicular bisector is equal to $\Gamma$. That is, the line joining $z,z^*$ is perpendicular to $\Gamma$ and they are the same distance from $\Gamma$ (but on opposite sides). Indeed, choosing $z_4 = \infty$, we get that
	\[ \frac{z^* - z_3}{z_2 - z_3} = \frac{\overline{z} - \overline{z_3}}{\overline{z_2} - \overline{z_3}}, \]
	so
	\[ |z - z_3| = |z^* - z_3| \]
	for all $z_3 \in \Gamma$.\\
	Now, suppose that $\Gamma = \{ z : |z-a| = R \}$ for some $0 < R < \infty$. We extensively use \Cref{thm: cross-product mobius equal} and the five types of M\"{o}bius translations in the following sequence of equations. Then,
	\begin{align*}
		(z^*,z_2,z_3,z_4) &= \overline{(z,z_2,z_3,z_4)} \\
			&= \overline{(z-a,z_2-a,z_3-a,z_4-a)} \\
			&= \left(\overline{z}-\overline{a},\frac{R^2}{z_2 - a},\frac{R^2}{z_3 - a},\frac{R^2}{z_4 - a}\right) \\
			&= \left(\frac{R^2}{\overline{z}-\overline{a}},z_2-a,z_3-a,z_4-a\right) \\
			&= \left( \frac{R^2}{\overline{z} - \overline{a}} + a , z_2, z_3, z_4 \right).
	\end{align*}

	Therefore, $z^* = a + \frac{R^2}{\overline{z}-\overline{a}}$, that is,
	\[ (z^* - a) (\overline{z} - \overline{a}) = R^2. \]
	Since
	\[ \frac{z^* - a}{z - a} = \frac{R^2}{|z-a|^2} > 0 \]
	is real, it follows that $z^*$ is on the ray $\{a + t(z-a) : 0 < t < \infty\}$. We also have that
	\[ |z^* - a| |z - a| = R^2, \]
	so one can easily obtain $z^*$ from $z$ or vice-versa. 

	\begin{lemma}[Symmetry Principle]
		If a M\"{o}bius transformation takes a circle $\Gamma_1$ to the circle $\Gamma_2$, then any pair of points symmetric with respect to $\Gamma_1$ is mapped to a pair of points symmetric with respect to $\Gamma_2$.
	\end{lemma}
	\begin{proof}
		The proof of this is near-direct.
		\begin{align*}
			(Tz,Tz_2,Tz_3,Tz_4) &= (z^*,z_2,z_3,z_4) \\
				&= \overline{(z,z_2,z_3,z_4)} \\
				&= \overline{(Tz,Tz_2,Tz_3,Tz_4)}. \qedhere
		\end{align*}
	\end{proof}


	\begin{fdef}
		If $\Gamma$ is a circle, then an \emph{orientation} for $\Gamma$ is an ordered triple $(z_1,z_2,z_3)$ of points in $\Gamma$.
	\end{fdef}
	An orientation is used to represent a ``direction'' of the circle, where we ``go'' from $z_1$ to $z_2$ to $z_3$.\\

	Let $\Gamma = \R$ and $z_1,z_2,z_3 \in \R$. Also put $Tz = (z,z_1,z_2,z_3)$. Since $T(\R_\infty) = \R_\infty$, $a,b,c,d$ can be chosen to be reals. Then,
	\begin{align*}
		Tz &= \frac{az+b}{cz+d} \\
			&= \frac{az+b}{|cz+d|^2} (c\overline{z}+d) \\
			&= \frac{1}{|cz+d|^2} \left( ac |z|^2 + bd + bc\overline{z} + adz \right).
	\end{align*}
	So,
	\[ \Im(z,z_1,z_2,z_3) = \frac{ad-bc}{|cz+d|^2} \Im z \]
	and thus, $\{ z : \Im (z,z_1,z_2,z_3) \}$ is either the upper or lower half-plane depending on whether $ad-bc$ is positive or negative. Note that $ad-bc$ is the determinant of $\begin{pmatrix}a & b \\ c & d\end{pmatrix}$.\\

	Let $\Gamma$ be an arbitrary circle and suppose that $z_1,z_2,z_3 \in \Gamma$. Then, for any M\"{o}bius transformation $S$,
	\begin{align*}
		\{ z : \Im(z,z_1,z_2,z_3) > 0 \} &= \{ z : \Im(Sz,Sz_1,Sz_2,Sz_3) > 0 \} \\
			&= S^{-1} \{ z : \Im(z,Sz_1,Sz_2,Sz_3) > 0 \}.
	\end{align*}
	So, if $S$ is chosen to map $\Gamma$ to $\R_\infty$, then the above set is equal to $S^{-1}$ of either the upper or lower halfspace.\\

	\begin{definition}
		If $z_1,z_2,z_3$ is an orientation of $\Gamma$, we denote the \emph{right side} and \emph{left side} of $\Gamma$ (with respect to $(z_1,z_2,z_3)$) to be
		\[ \{ z : \Im(z,z_1,z_2,z_3) > 0 \} \text{ and } \{z : \Im(z,z_1,z_2,z_3) < 0\} \]
		respectively.
	\end{definition}

	\begin{theorem}[Orientation Principle]
		Let $\Gamma_1,\Gamma_2$ be circles in $\C_\infty$ such that $T\Gamma_1 = \Gamma_2$ for some M\"{o}bius transformation $T$. Let $(z_1,z_2,z_3)$ be an orientation of $\Gamma_1$. Then, $T$ takes the right side (resp. left side) of $\Gamma_1$ with respect to the orientation $(z_1,z_2,z_3)$ to the right side (resp. left side) of $\Gamma_2$ with respect to the orientation $(Tz_1,Tz_2,Tz_3)$.
	\end{theorem}
	The proof of the above is left as an exercise to the reader.\\

	Since $(z,1,0,\infty) = z$ by definition, the right side of $\R_\infty$ with respect to the orientation $(1,0,\infty)$ is the upper half-plane.\\

	\begin{exercise}
		Find an analytic function $f : G \to \C$ where $G = \{ z : \Re z > 0 \}$, such that $f(G) = \{ z : |z| < 1 \}$.
	\end{exercise}

	Similar to the above exercise, one may show that
	\[ g(z) = \frac{e^z - 1}{e^z + 1} \]
	maps the infinite strip $\{ z : |\Im z| < \pi/2 \}$ to the open unit disk $D$.

\section{Integration}

\subsection{Basic definitions}

	\subsubsection{Integrals of real functions}

		First, let us recall the definition of the Riemann integral\footnote{technically the Darboux integral?} of functions on $\R$.

		\begin{fdef}[Riemann Integral]
			Let $[a,b]$ be a given interval. A \emph{partition} $\mathcal{P}$ of $[a,b]$ is a finite set of points $x_0,x_1,\ldots,x_n$ where
			\[ a = x_0 \le x_1 \le \cdots \le x_{n-1} \le x_n = b.  \]
			We also write $\Delta x_i = x_i - x_{i-1}$ for $i = 1,2,\ldots,n$.\\
			For a bounded real function $f$ on $[a,b]$ and each partition $\mathcal{P}$ of $[a,b]$, we set
			\[ M_i = \sup_{x_{i-1} \le x \le x_i} f(x), \qquad m_i = \inf_{x_{i-1} \le x \le x_i} f(x). \]
			Further, set
			\[ U(\mathcal{P},f) = \sum_{i=1}^{n} M_i \Delta x_i, \qquad L(\mathcal{P},f) = \sum_{i=1}^n m_i \Delta x_i \]
			as the upper and lower Riemann sum respectively,
			and finally,
			\[ \overline{\int_a^b} f \dif x = \inf_{\mathcal{P}} U(\mathcal{P},f), \qquad \underline{\int_a^b} f \dif x = \sup_{\mathcal{P}} L(\mathcal{P},f) \]
			as the upper and lower Riemann integrals of $f$.\\
		\end{fdef}

		Next, we define the slightly more general Riemann-Stieltjes integral. Note that this is the same as the usual Riemann integral when $\alpha$ is the identity function.

		\begin{fdef}[Riemann-Stieltjes Integral]
			Let $\alpha : [a,b] \to \R$ be a monotonically increasing function on $[a,b]$. Corresponding to each partition $\mathcal{P}$ of $[a,b]$, write $\Delta \alpha_i = \alpha(x_i) - \alpha(x_{i-1})$. Clearly, $\Delta \alpha_i \ge 0$ for each $i$.\\
			For any real function $f$ which is bounded on $[a,b]$, we put
			\[ U(\mathcal{P},f,\alpha) = \sum_{i=1}^n M_i \Delta \alpha_i, \qquad L(\mathcal{P},f,\alpha) = \sum_{i=1}^n m_i \Delta \alpha_i, \]
			where $M_i,m_i$ are defined as in the definition of the Riemann integral. We then define
			\[ \overline{\int_a^b} f \dif \alpha = \inf_{\mathcal{P}} U(\mathcal{P},f,\alpha), \qquad \underline{\int_a^b} f \dif \alpha = \sup_{\mathcal{P}} L(\mathcal{P},f,\alpha). \]
			If these two are equal, we say that $f$ is \emph{Riemann-Stieltjes integrable} with respect to $\alpha$ on $[a,b]$ and denote the common value as $\int_a^b f \dif \alpha$.
		\end{fdef}

		We also remark that
		\[ \int_a^b f \dif \alpha = \lim_{\max \Delta \alpha_k \to 0} \sum_{k=1}^n f(\tau_k) \Delta \alpha_k, \]
		where $x_{k-1} \le \tau_k \le x_k$ for each $k$.\\
		More generally, we define the \emph{mesh} of $\mathcal{P}$ with respect to $\alpha$ as
		\[ \norm{\mathcal{P}} = \max\{ \Delta\alpha_i : 1 \le i \le n \}. \]
		So for all $\epsilon > 0$, there exists $\delta > 0$ such that for any partition $\mathcal{P}$ of $[a,b]$ with $\norm{P} < \delta$, then
		\[ \left| \sum_{k=1}^n f(\tau_k) \Delta\alpha_k - \int_a^b f \dif \alpha \right| < \epsilon \]
		for any choice of points $x_{k-1} \le \tau_k \le x_k$.

	\subsubsection{Riemann-Stieltjes integrals of complex-valued functions}

		\begin{fdef}
			A function $\gamma : [a,b] \to \C$ for $[a,b] \subseteq \R$ is said to be of \emph{bounded variation} if there exists $M > 0$ such that for any partition $\mathcal{P} = \{ a = t_0 < t_1 < \cdots < t_{m-1} < t_m = b \}$ of $[a,b]$,
			\[ v(\gamma;\mathcal{P}) = \sum_{k=1}^m |\gamma(t_k) - \gamma(t_{k-1})| \le M. \]
			The \emph{total variation} $V(\gamma)$ of $\gamma$ is defined by
			\[ V(\gamma) = \sup \{ v(\gamma;\mathcal{P}) : \mathcal{P}\text{ is a partition of }[a,b] \}. \]
			Clearly, $V(\gamma) \le M < \infty$.
		\end{fdef}

		\begin{lemma}
			Let $\gamma : [a,b] \to \C$ be of bounded variation. Then,
			\begin{enumerate}
				\item If $\mathcal{P},\mathcal{Q}$ are partitions of $[a,b]$ with $\mathcal{P} \subseteq \mathcal{Q}$, then $v(\gamma;\mathcal{P}) \le v(\gamma;\mathcal{Q})$.
				\item If $\sigma : [a,b] \to \C$ is also of bounded variation and $\alpha,\beta\in\C$, then $\alpha\gamma + \beta\sigma$ is of bounded variation and
				\[ V(\alpha\gamma + \beta\sigma) \le |\alpha| V(\gamma) + |\beta| V(\sigma). \]
			\end{enumerate}
		\end{lemma}
		We omit the proof of the above, which is direct on using the triangle inequality on the definition of $v(\gamma;\mathcal{P})$.

		\begin{lemma}
			If $\gamma : [a,b] \to \C$ is piecewise smooth, $\gamma$ is of bounded variation and
			\[ V(\gamma) = \int_a^b |\gamma'(t)| \dif t. \]
		\end{lemma}
		\begin{proof}
			It suffices to show the required in the case where $\gamma$ is smooth, since in general we can consider the refinement of any partition that splits along the pieces along which $\gamma$ is smooth.\\
			The right hand side is well-defined since $\gamma'$ is continuous. Let $\mathcal{P} = \{ a = t_0 < t_1 < \cdots < t_{m-1} < t_m = b \}$. By definition,
			\begin{align*}
				v(\gamma,\mathcal{P}) &= \sum_{k=1}^m |\gamma(t_k) - \gamma(t_{k-1})| \\
					&= \sum_{k=1}^m \left| \int_{t_{k-1}}^{t_k} \gamma'(t) \dif t \right| \\
					&\le \sum_{k=1}^m \int_{t_{k-1}}^{t_k} |\gamma'(t)| \dif t = \int_a^b |\gamma'(t)| \dif t.
			\end{align*}
			Therefore, $V(\gamma) \le \int_a^b |\gamma'(t)| \dif t$, so $\gamma$ is of bounded variation.\\
			Since $\gamma'$ is continuous, it is uniformly continuous. So, if $\epsilon > 0$, we may choose $\delta_1 > 0$ such that
			\[ |s-t| < \delta_1 \implies |\gamma'(s) - \gamma'(t)| < \epsilon. \]
			Also, let $\delta_2 > 0$ such that if $\norm{P} < \delta_2$, then
			\[ \left| \int_a^b |\gamma'(t)| \dif t - \sum_{k=1}^m |\gamma'(\tau_k)| (t_k - t_{k-1}) \right| < \epsilon, \]
			where $\tau_k$ is any point in $[t_{k-1},t_k]$. Therefore,
			\begin{align*}
				\int_a^b |\gamma'(t)| \dif t &\le \epsilon + \sum_{k=1}^{m} |\gamma'(t_k)| (t_k - t_{k-1}) \\
					&= \epsilon + \sum_{k=1}^m \left| \int_{t_{k-1}}^{t_k} \gamma'(\tau_k) \dif t \right| \\
					&\le \epsilon + \sum_{k=1}^m \left| \int_{t_{k-1}}^{t_k} (\gamma'(\tau_k) - \gamma'(t)) \dif t \right| + \sum_{k=1}^m \left| \int_{t_{k-1}}^{t_k} \gamma'(t) \dif t \right|.
			\end{align*}
			If $\norm{P} < \delta = \min(\delta_1,\delta_2)$, then $|\gamma'(\tau_k) - \gamma'(t)| < \epsilon$ for all $t \in [t_{k-1},t_k]$ and
			\begin{align*}
				\int_a^b |\gamma'(t) \dif t| &\le \epsilon + \epsilon(b-a) + \sum_{k=1}^{m} |\gamma(t_k) - \gamma(t_{k-1})| \\
					&= \epsilon(1 + b-a) + V(\gamma;P) \le \epsilon(1 + b-a) + V(\gamma),
			\end{align*}
			so we are done since $1+b-a > 0$ is finite and $\epsilon$ can be made arbitrarily small.
		\end{proof}

		\begin{ftheo}
			Let $\gamma : [a,b] \to \C$ be of bounded variation and suppose that $f : [a,b] \to \C$ is continuous. Then, there exists a (unique) complex number $\mathcal{I}$ such that for every $\epsilon > 0$, there exists $\delta > 0$ such that when $\mathcal{P} = \{ t_0 < t_1 < \cdots < t_m \}$ is a partition of $[a,b]$ with $\norm{P} = \max_{1 \le k \le m} (t_k - t_{k-1}) < \delta$,
			\[ \left| \mathcal{I} - \sum_{k=1}^m f(\tau_k) (\gamma(t_k) - \gamma(t_{k-1})) \right| < \epsilon \]
			for any choice of points $\tau_k$ with $t_{k-1} \le \tau_k \le t_k$.\\
			This $\mathcal{I}$ is called the integral of $f$ with respect to $\gamma$ over $[a,b]$ and is denoted by
			\[ \mathcal{I} = \int_a^b f \dif \gamma = \int_a^b f(t) \dif\gamma(t). \]
		\end{ftheo}
		\begin{proof}
			First of all, note that it suffices to consider the case where $\gamma$ is real-valued, since we can write $\gamma = \gamma_1 + \iota \gamma_2$, where $\gamma_1,\gamma_2$ are real-valued, to get two integrals $\mathcal{I}_1,\mathcal{I}_2$ (for $\gamma_1,\gamma_2$ respectively), and finally use the triangle inequality to get $\mathcal{I} = \mathcal{I}_1 + \iota\mathcal{I}_2$.\\
			Since $f$ is continuous, it is uniformly continuous. We can (inductively) find positive numbers $\delta_1 > \delta_2 > \cdots $ such that if $|s-t| < \delta_m$, $|f(s) - f(t)| < 1/m$. For each $M \ge 1$, let $\mathcal{P}_m$ be the collection of all partitions $P$ of $[a,b]$ with $\norm{P} \le \delta_m$, so $\mathcal{P}_1 \supseteq \mathcal{P}_2 \supseteq \cdots \supseteq \mathcal{P}_m \supseteq \cdots$. Finally, define $F_m$ to be the closure of the set
			\[ \left\{ \sum_{k=1}^n f(\tau_k) (\gamma(t_k) - \gamma(t_{k-1})) : P \in \mathcal{P}_m \text{ and } t_{k-1} \le \tau_k \le t_k \right\}. \]
			Because $\mathcal{P}_1 \supseteq \mathcal{P}_2 \supseteq \cdots$, it follows trivially that
			\[ F_1 \supseteq F_2 \supseteq \cdots. \]
			We claim that
			\begin{align}
				\diam F_m &\le \frac{2}{m} V(\gamma). \label{eqn: diam-bound}
			\end{align}
			If we do this, then Cantor's Theorem (since $\C$ is complete) implies that there is precisely one complex number $\mathcal{I}$ such that $\mathcal{I} \in F_m$ for all $m \ge 1$. Then, for any $\epsilon > 0$, we may let $m > (2/\epsilon)V(\gamma)$ so $\epsilon > (2/m) V(\gamma) \ge \diam F_m$. Since $\mathcal{I} \in F_m$, $F_m \subseteq B(\mathcal{I},\epsilon)$. Therefore, $\delta = \delta_m$ gets the job done.\\
			So, we must show that
			\[ \diam \left\{ f(\tau_k) \left( \gamma(t_k) - \gamma(t_{k-1}) \right) : P \in \mathcal{P}_m \text{ and } t_{k-1} \le \tau_k \le t_k \right\} \le \frac{2}{m} V(\gamma). \]
			To do this, if $P = \{ t_0 < \cdots < t_n \}$ is a partition, denote by $S(P)$ a sum of the form $\sum f(\tau_k) \left( \gamma(t_k)  - \gamma(t_{k-1})\right)$ where $t_{k-1} \le \tau_k \le t_k$ for each $k$. Fixing $m \ge 1$, let $P \in \mathcal{P}_m$. If $P \subseteq Q$ (so $Q \in \mathcal{P}_m$ as well), then
			\[ |S(P) - S(Q)| < \frac{1}{m} V(\gamma). \]
			We only show this in the case where $Q$ is obtained from $P$ by adding a single extra partition point (the general case follows similarly). Let $Q = \{ t_0 < t_1 < \cdots < t_{p-1} < t^* < t_p < \cdots t_n \}$. If $t_{p-1} \le \sigma \le t^*$ and $t^* \le \sigma' \le t_p$. Then,
			\[ S(Q) = \sum_{k \ne p} f(\sigma_k) ( \gamma(t_k) - \gamma(t_{k-1}) ) + f(\sigma) \left( \gamma(t^*) - \gamma(t_{p-1}) \right) + f(\sigma') \left( \gamma(t_p) - \gamma(t^*) \right). \]
			Then, using the definition of $\delta_m$,
			\begin{align*}
				\left| S(P) - S(Q) \right| &= \left| \sum_{k\ne p} \left( f(\tau_k) - f(\sigma_k) \right) \left( \gamma(t_k) - \gamma(t_{k-1}) \right) \right. \\
				&\qquad\left. + f(\tau_p) (\gamma(t_p) - \gamma(t_{p-1})) - f(\sigma)(\gamma(t^*) - \gamma(t_{p-1})) - f(\sigma')( \gamma(t_p) - \gamma(t^*) ) \right| \\
				&\le \frac{1}{m} \sum_{k \ne p} |\gamma(t_k) - \gamma(t_{k-1}| + \left| \left( f(\tau_p) - f(\sigma) \right) \left( \gamma(t^*) - \gamma(t_{p-1}) \right) + \left( f(\tau_p) - f(\sigma') \right) \left( \gamma(t_p) - \gamma(t^*) \right) \right| \\
				&\le \frac{1}{m} \sum_{k \ne p} \left| \gamma(t_k) - \gamma(t_{k-1}) \right| + \frac{1}{m} \left| \gamma(t^*) - \gamma(t_{p-1}) \right| + \frac{1}{m} \left| \gamma(t_p) - \gamma(t^*) \right| \\
				&\le \frac{1}{m} V(\gamma).
			\end{align*}

			Next, let $P,R$ be any two partitions in $\mathcal{P}_m$, and $Q = P \cup R$ a partition that contains $P$ and $R$. Using the first part,
			\[ |S(P) - S(Q)| \le |S(P) - S(Q)| + |S(Q) - S(R)| \le \frac{2}{m} V(\gamma). \]
			It follows that the diameter of the set of interest is at most $(2/m) V(\gamma)$, completing the proof.
		\end{proof}

		\begin{ftheo}
			Let $f,g$ be continuous functions on $[a,b]$ and let $\gamma,\sigma$ be functions of bounded variation on $[a,b]$. Then for any scalars $\alpha,\beta$,
			\begin{align*}
				\int_a^b (\alpha f + \beta g) \dif \gamma &= \alpha \int_a^b f \dif \gamma + \beta \int_a^b g \dif \gamma \\
				\int_a^b f \dif( \alpha\gamma + \beta\sigma ) &= \alpha\int_a^b f \dif \gamma + \beta\int_a^b f \dif \sigma.
			\end{align*}
		\end{ftheo}

		% \begin{proof}
		% 	It follows near-directly that $\int_a^b \alpha f \dif \gamma = \alpha \int_a^b f \dif \gamma$ (just use $|\alpha|\epsilon$ instead of $\epsilon$ in the definition). So, to show \Cref{eqn: scalar-addition-closure of fn int}, it suffices to show it for $\alpha = \beta = 1$. To do this, use $\epsilon/2$ in the definitions of integrability, and take $\delta = \min\{\delta_1,\delta_2\}$ (where $\delta_1,\delta_2$ are the values obtained from the integrability of $f,g$).\\
		% 	Similarly, $\int_a^b f \dif (\alpha \gamma) = \alpha \int_a^b f \dif \gamma$ and $\int_a^b f \dif(\gamma+\sigma) = \int_a^b f \dif \gamma + \int_a^b f \dif \sigma$.
		% \end{proof}

		\begin{prop}
			\label{lemma: can split integral}
			Let $\gamma : [a,b] \to \C$ be of bounded variation and let $f : [a,b] \to \C$ be continuous. If $a = t_0 < t_1 < \cdots < t_{n-1} < t_n = b$, then
			\[ \int_a^b f \dif \gamma = \sum_{k=1}^n \int_{t_{k-1}}^{t_k} f \dif \gamma. \]
		\end{prop}

		We omit the proofs of the above.

		\begin{ftheo}
			If $\gamma$ is piecewise smooth and $f : [a,b] \to \C$ is continuous, then $\int_a^b f \dif \gamma = \int_a^b f(t) \gamma'(t) \dif t$.
		\end{ftheo}
		\begin{proof}
			It suffices to consider the case where $\gamma$ is smooth by \Cref{lemma: can split integral}. Also, by looking at the real and imaginary parts of $\gamma$ separately, it suffices to consider the case where $\gamma$ is real-valued on $[a,b]$. Let $\epsilon > 0$ and choose $\delta > 0$ such that if $P = \{ a = t_0 < t_1 < \cdots < t_n = b \}$ has $\norm{P} < \delta$, then
			\[ \left| \int_a^b f \dif \gamma - \sum_{k=1}^n f(\tau_k) (\gamma(t_k) - \gamma(t_{k-1})) \right| < \epsilon/2 \]
			and
			\[ \left| \int_a^b f(t) \gamma'(t) \dif t - \sum_{k=1}^n f(\tau_k) \gamma'(\tau_k) (t_k - t_{k-1}) \right| < \epsilon/2 \]
			for any $t_{k-1} \le \tau_k \le t_k$ for each $k$.\\
			Applying the mean value theorem on $\gamma$ (this requires that $\gamma$ be real-valued), one gets that there exists $\tau_k \in [t_{k-1},t_k]$ for each $k$ such that
			\[ \gamma'(\tau_k) = \frac{\gamma(t_k) - \gamma(t_{k-1})}{t_k - t_{k-1}}. \]
			Using these $\tau_k$ specifically, 
			\[ \left| \int_a^b f \dif \gamma - \sum_{k=1}^n f(\tau_k) \gamma'(\tau_k) (t_k - t_{k-1}) \right| < \epsilon/2, \]
			so
			\[ \left| \int_a^b f \dif \gamma - \int_a^b f(t) \gamma'(t) \dif t \right| < \epsilon, \]
			completing the proof.
		\end{proof}

\subsection{Integrals On Curves}

	\begin{fdef}
		$\gamma:[a,b]\to\C$ is called a \emph{rectifiable path} if it is continuous and of bounded variation. Note that if $\gamma$ is piecewise smooth, then it is rectifiable and its length is
		\[ \int_a^b |\gamma'(t)| \dif t = V(\gamma). \]
	\end{fdef}

	\begin{fdef}
		If $\gamma : [a,b] \to \C$ is a rectifiable path and $f$ is a function continuous on $\{\gamma\}$, then the \emph{(line) integral} of $f$ along $\gamma$ is
		\[ \int_a^b f(\gamma(t)) \dif \gamma(t). \]
		This line integral is also denoted as
		\[ \int_\gamma f = \int_\gamma f(z) \dif z. \]
	\end{fdef}

	For example, if $\gamma:[0,2\pi] \to \C$ as $\gamma(t) = e^{\iota t}$,
	\[ \int_\gamma \frac{1}{z} \dif z = \int_0^{2\pi} e^{-\iota t} (\iota e^{\iota t}) \dif t = 2 \pi \iota. \]
	and
	\[ \int_\gamma z^m \dif z = \int_0^{2\pi} e^{\iota m t} (\iota e^{\iota t}) \dif t = \iota \int_0^{2\pi} \cos((m+1)t) \dif t - \int_0^{2\pi} \sin((m+1)t) \dif t = 0. \]

	\begin{theorem}
		If $\gamma : [a,b] \to \C$ is a rectifiable path and $\varphi : [c,d] \to [a,b]$ is a continuous non-decreasing function with $\varphi(c) = a, \varphi(d) = b$, then for any function $f$ continuous on $\gamma$,
		\[ \int_\gamma f = \int_{\gamma \circ \varphi} f. \]
	\end{theorem}

	\begin{remark}
		The above uses the fact that $\gamma \circ \varphi$ is also rectifiable (Why is this true?).
	\end{remark}

	\begin{proof}
		Let $\epsilon > 0$ and choose $\delta_1 > 0$ such that for a partition $\{ s_0 < s_1 < \cdots  < s_n \}$ of $[c,d]$ with $(s_{k} - s_{k-1}) < \delta_1$ and any $s_{k-1} \le \sigma_k \le s_k$,
		\[ \left| \int_{\gamma \circ \varphi} f - \sum_{k=1}^{n} f((\gamma \circ \varphi)(s_k)) - f((\gamma \circ \varphi)(s_{k-1})) \right| < \epsilon/2. \]
		Similarly, choose $\delta_2 > 0$ such that for a partition $\{ t_0 < t_1 < \cdots < t_m \}$ of $[a,b]$ with $(t_k - t_{k-1}) < \delta_2$ and $t_{k-1} \le \tau_k \le t_k$,
		\[ \left| \int_{\gamma} f - \sum_{k=1}^{m} f(\gamma(t_k)) - f(\gamma(t_{k-1})) \right| < \epsilon/2. \]
		Since $\varphi$ is uniformly continuous on $[c,d]$, there exists $\delta > 0$ less than $\delta_1$ such that $|\varphi(s) - \varphi(t)| < \delta_2$ whenever $|s-t| < \delta$. So, if $\{ s_0 < s_1 < \cdots < s_n \}$ is a partition of $[c,d]$ with $(s_k - s_{k-1}) < \delta < \delta_1$ and $t_k = \varphi(s_k)$, then $\{ t_0 < t_1 < \cdots < t_n \}$ is a partition of $[a,b]$ with $(t_k - t_{k-1}) < \delta_2$. If $s_{k-1} \le \sigma_k \le s_k$ and $\tau_k = \varphi(\sigma_k)$, then we can use the two earlier inequalities to conclude that
		\[ \left| \int_\gamma f - \int_{\gamma \circ \varphi} f \right| < \epsilon, \]
		completing the proof.
	\end{proof}

	\begin{definition}
		Let $\gamma : [a,b] \to \C$ be a rectifiable path, and for $a \le t \le b$, set $|\gamma|(t) = V(\gamma;[a,t])$. That is,
		\[ |\gamma|(t) = \sup\left\{ \sum_{k=1}^{n} |\gamma(t_k) - \gamma(t_{k-1})| : \{ t_0 < t_1 < \cdots < t_n \}\text{ is a partition of }[a,t] \right\}. \]
		Clearly, $|\gamma|$ is increasing on $[a,b]$ and of bounded variation. In fact, $V(|\gamma|;[a,b]) = |\gamma|(b) - |\gamma|(a)$. If $f$ is continuous on $[a,b]$, define
		\[ \int f |{\dif} {z}| = \int_a^b f(\gamma(t)) \dif |\gamma|(t). \]
	\end{definition}

	\begin{ftheo}
		Let $\gamma : [a,b] \to \C$ be a rectifiable curve and suppose that $f$ is a function continuous on $\{\gamma\}$. Then,
		\begin{equation}
			\label{eqn: 2.2}
			\int_\gamma f = - \int_{-\gamma} f
		\end{equation}
		where $(-\gamma)(t) = \gamma(a+b-t)$,
		\begin{equation}
			\label{eqn: 2.3}
			\left| \int_\gamma f \right| \le \int_\gamma |f| |{\dif} {z}| \le V(\gamma) \sup\{ |f(z)| : z \in \{\gamma\} \},
		\end{equation}
		and for $c \in \C$,
		\begin{equation}
			\label{eqn: 2.4}
			\int_\gamma f(z) \dif z = \int_{\gamma + c} f(z-c) \dif z.
		\end{equation}
	\end{ftheo}
	\begin{proof}
		\Cref{eqn: 2.2,eqn: 2.4} follow near-directly from the definition, so we prove only \Cref{eqn: 2.3}. Let $\epsilon > 0$. Then, there exists $\delta > 0$ such that if $P = \{ t_0 < t_1 < \cdots t_n \}$ is a partition of $[a,b]$ with $\norm{P} < \delta$, then
		\[ \left| \left| \int_\gamma f(z) \dif z \right| - \left| \sum_{k=1}^n f(\gamma(\tau_k)) (\gamma(t_k) - \gamma(t_{k-1})) \right| \right| \le \left| \int_\gamma f(z) \dif z - \sum_{k=1}^n f(\gamma(\tau_k)) (\gamma(t_k) - \gamma(t_{k-1})) \right| < \epsilon/2 \]
		for any $t_{k-1} \le \tau_k \le t_k$.
		That is,
		\begin{align*}
			\left| \int_\gamma f(z) \dif z \right| &< \left| \sum_{k=1}^n f(\gamma(\tau_k)) (\gamma(t_k) - \gamma(t_{k-1})) \right| + \epsilon/2 \\
			&\le  \sum_{k=1}^n \left| f(\gamma(\tau_k)) \right| \left|\gamma(t_k) - \gamma(t_{k-1}) \right| + \epsilon/2.
		\end{align*}
		We may also assume that for this same $\delta$,
		\[ \sum_{k=1}^n |f(\gamma(t_k))| (|\gamma|(t_k) - |\gamma|(t_{k-1})) < \int_\gamma |f(z)| |{\dif} {z}| + \epsilon/2 . \]
		Recall that $|\gamma|(t)$ is an increasing function. So,
		\[ |\gamma|(t_k) - |\gamma|(t_{k-1}) \ge |\gamma(t_k) - \gamma(t_{k-1})| \]
		Therefore,
		\begin{align*}
			\left| \int_\gamma f(z) \dif z \right| &< \sum_{k=1}^n |f(\gamma(\tau_k))| \left( |\gamma|(t_k) - |\gamma|(t_{k-1}) \right) + \epsilon/2 \\
				&< \int_\gamma |f(z)| |{\dif} {z}| + \epsilon.
		\end{align*}
		It follows that
		\[ \left| \int_\gamma f(z) \dif z \right| \le \int_\gamma |f(z)| |{\dif} {z}|. \]
		To conclude the proof, note that
		\[ \int_\gamma |{\dif} {z}| = |\gamma|(b) - |\gamma|(a) = |\gamma|(b) = V(\gamma), \]
		so
		\[ \int_\gamma |f(z)| |{\dif} {z}| \le V(\gamma) \sup_{z \in \{\gamma\}} |f(z)|. \]
	\end{proof}

	\begin{flem}
		\label{lemma: polygonal}
		If $G$ is an open set in $\C$, $\gamma : [a,b] \to G$ is a rectifiable path, and $f : G \to \C$ is continuous, then for every $\epsilon > 0$ there exists a polygonal path $\Gamma$ in $G$ such that $\Gamma(a) = \gamma(a)$, $\Gamma(b) = \gamma(b)$, and
		\[ \left| \int_\gamma f - \int_\Gamma f \right| < \epsilon \]
	\end{flem}
	\begin{proof}
		We prove the result in the case where $G$ is an open disk. In the general case where $G$ need not be a disk, since $\{\gamma\}$ is compact, there exists a number $r$ with $0 < r < d(\{\gamma\},\partial G)$. Choose $\delta > 0$ such that $|\gamma(s) - \gamma(t)| < r$ when $|s-t| < \delta$. The idea is that we shall take several smaller disks and stitch together the polygonal paths on each of these sections.\\
		If $P = \{ t_0 < t_1 < \cdots < t_n \}$ is a partition of $[a,b]$ with $\norm{P} < \delta$, then $|\gamma(t_k) - \gamma(t_{k-1})| < r$ for $t_{k-1} \le t \le t_k$. That is, if $\gamma_k : [t_{k-1}, t_k] \to G$ is defined by $\gamma_k(t) = \gamma(t)$, then $\{\gamma_k\} \subseteq B(\gamma(t_{k-1}),r)$ for $1 \le k \le n$. Getting a polygonal path $\Gamma_k$ for each $k$ such that
		\[ \left| \int_{\gamma_k} f - \int_{\Gamma_k} f \right| < \epsilon/n, \]
		defining $\Gamma(t) = \Gamma_k(t)$ on $[t_{k-1},t_k]$ does the job.
		\\

		Now, let us prove the result in the disk case.\\
		Because $\{\gamma\}$ is a compact set, $d = d(\{\gamma\},\partial G) > 0$. It follows that if $G = B(c,r)$, then $\{\gamma\} \subseteq B(c,\rho)$ where $\rho = r - d/2$.\\
		Now, observe that $f$ is uniformly continuous on $\overline{B}(c,\rho) \subseteq G$. Thus, we may assume without loss of generality that $f$ is uniformly continuous on $G$. Now, choose $\delta > 0$ such that if $|z-w| < \delta$, then $|f(z) - f(w)| < \epsilon$. If $\gamma : [a,b] \to G$, then $\gamma$ is uniformly continuous so there is a partition $P = \{t_0 < t_1 < \cdots < t_n\}$ of $[a,b]$ such that if $t_{k-1} \le s,t \le t_k$, $|\gamma(s) - \gamma(t)| < \delta$, and such that for $t_{k-1} \le \tau_k \le t_k$,
		\[ \left| \int_\gamma f - \sum_{k=1}^n f(\gamma(\tau_k)) (\gamma(t_k) - \gamma(t_{k-1})) \right| < \epsilon. \]
		Now, define $\Gamma : [a,b] \to G$ by
		\[ \Gamma(t) = \frac{(t_k - t) \gamma(t_{k-1}) + (t - t_{k-1}) \gamma(t_k)}{t_k - t_{k-1}} \]
		if $t_{k-1} \le t \le t_k$. This is the polygonal path we shall consider. Indeed,
		\begin{align*}
			\Gamma(t) - \gamma(\tau_k) &= \frac{t_k - t}{t_k - t_{k-1}} (\gamma(t_{k-1}) - \gamma(\tau_k)) + \frac{t - t_{k-1}}{t_k - t_{k-1}} (\gamma(t_k) - \gamma(\tau_k)),
		\end{align*}
		so
		\begin{align*}
			|\Gamma(t) - \gamma(\tau_k)| &\le \left| \frac{t_k - t}{t_k - t_{k-1}} \right| \left| \gamma(t_{k-1}) - \gamma(\tau_k) \right| + \left| \frac{t - t_{k-1}}{t_k - t_{k-1}} \right| \left| \gamma(t_k) - \gamma(\tau_k) \right| \\
				&\le \left| \gamma(t_{k-1}) - \gamma(\tau_k) \right| + \left| \gamma(t_k) - \gamma(\tau_k) \right| < 2\delta.
		\end{align*}
		Thus,
		\begin{align*}
			\int_\Gamma f &= \int_a^b f(\Gamma(t)) \Gamma'(t) \dif t \\
				&= \sum_{k-1}^n \frac{\gamma(t_k) - \gamma(t_{k-1})}{t_k - t_{k-1}} \int_{t_{k-1}}^{t_k} f(\Gamma(t)) \dif t
		\end{align*}
		and
		\begin{align*}
			\left| \int_\gamma f - \int_\Gamma f \right| &= \left| \int_\gamma f - \sum_{k=1}^n f(\gamma(\tau_k)) (\gamma(t_k) - \gamma(t_{k-1})) \right| + \left| \sum_{k=1}^n f(\gamma(\tau_k)) (\gamma(t_k) - \gamma(t_{k-1})) - \int_\Gamma f \right| \\
				&\le \epsilon + \left| \sum_{k=1}^n f(\gamma(\tau_k)) (\gamma(t_k) - \gamma(t_{k-1})) - \int_\Gamma f \right| \\
				&\le \epsilon + \sum_{k=1}^n \frac{|\gamma(t_k) - \gamma(t_{k-1})|}{t_k - t_{k-1}} \int_{t_{k-1}}^{t_k} |f(\Gamma(t)) - f(\gamma(\tau_k))| \dif t \\
				&\le \epsilon + \epsilon \sum_{k=1}^n |\gamma(t_k) - \gamma(t_{k-1})| \\
				&\le \epsilon (1 + V(\gamma)),
		\end{align*}
		which can be made arbitrarily small, thus completing the proof.
	\end{proof}

	The following can be thought of as an analogue of the Fundamental Theorem of Calculus for complex functions.

	\begin{ftheo}
		\label{theo: ftc-like}
		Let $G$ be open in $\C$ and $\gamma$ be a rectifiable path in $G$ with initial and end points $\alpha,\beta$ respectively. If $f : G \to \C$ is a continuous function with a primitive $F : G \to \C$ ($F$ is differentiable and $F' = f$), then
		\[ \int_\gamma f = F(\beta) - F(\alpha). \]
	\end{ftheo}
	\begin{proof}
		When $\gamma : [a,b] \to \C$ is piecewise smooth,
		\begin{align*}
			\int_\gamma f &= \int_a^b f(\gamma(t)) \gamma'(t) \dif t \\
				&= \int_a^b F'(\gamma(t)) \gamma'(t) \dif t \\
				&= \int_a^b (F \circ \gamma)'(t) \dif t \\
				&= (F \circ \gamma) (b) - (F \circ \gamma) (a) & \text{(by the Fundamental Theorem of Calculus)} \\
				&= F(\beta) - F(\alpha).
		\end{align*}
		In general, we may use \Cref{lemma: polygonal}. For $\epsilon > 0$, let $\Gamma$ be a polygonal path of the described form. Since $\Gamma$ is piecewise smooth, $\int_\Gamma f = F(\beta) - F(\alpha)$, so
		\[ \left| \int_\gamma f - (F(\beta) - F(\alpha)) \right| < \epsilon. \]
		Since $\epsilon$ was chosen arbitrarily, the desideratum follows.
	\end{proof}

	The fundamental theorem of calculus says that each continuous function has a primitive. However, this is not true for functions of complex variables. For example, letting $f(z) = |z|^2$, if $F$ is a primitive of $f$, then $F$ is analytic. So, if $F = U + \iota V$, $x^2 + y^2 = F'(x+\iota y)$. Consequently,
	\begin{align*}
		\dpd{U}{x} &= \dpd{V}{y} = x^2 + y^2 \\
		\dpd{U}{y} &= \dpd{V}{x} = 0.
	\end{align*}
	However, $\pd{U}{y} = 0$ implies that $U(x,y) = u(x)$ for some function $u$, which implies that $u'(x) = x^2 + y^2$, a contradiction.

\subsection{Power series representation of analytic functions}

	Recall the following result which we had used in the proof of \Cref{theo: open disk harmonic conjugate}.

	\begin{ftheo}
		Let $\varphi : [a,b] \times [c,d] \to \C$ be a continuous function and defined $g : [c,d] \to \C$ by
		\[ g(t) = \int_a^b \varphi(s,t) \dif s. \]
		Then $g$ is continuous. Moreover, if $\pd{\varphi}{t}$ exists and is a continuous function on $[a,b] \times [c,d]$, then $g$ is continuously differentiable and
		\[ g'(t) = \int_a^b \dpd{\varphi}{t}(s,t) \dif s. \]
	\end{ftheo}

	This is referred to as the Leibniz rule.\\

	For example, this may be used to prove that if $|z| < 1$,
	\[ \int_0^{2\pi} \frac{e^{\iota s}}{e^{\iota s} - z} = 2 \pi. \]
	To do so, let $\varphi(s,t) = e^{\iota s} / (e^{\iota s} - tz)$ for $0 \le t \le 1$ and $0 \le s \le 2\pi$. Observe that $\varphi$ is continuously differentiable since $|z| < 1$. Thus,
	\[ g(t) = \int_0^{2\pi} \varphi(s,t) \dif s \]
	is continuously differentiable. Since $\varphi(s,0) = 1$, $g(0) = 2\pi$. Now,
	\begin{align*}
		g'(t) &= \int_0^{2\pi} \dpd{\varphi}{t}(s,t) \dif s \\
			&= \int_0^{2\pi} \frac{ze^{\iota s}}{(e^{\iota s} - tz)^2} \dif s.
	\end{align*}
	For fixed $t$, $\Phi(s) = z\iota/(e^{\iota s} - tz)$ satisfies
	\[ \Phi'(s) = -\frac{\iota z}{(e^{\iota s} - tz)^{2}} \cdot \iota e^{\iota s} = \frac{ze^{\iota s}}{(e^{\iota s} - tz)^2}. \]
	Therefore, $g'(t) = \Phi(2\pi) - \Phi(0) = 0$, so $g$ is a constant and $g(t) = g(0) = 2\pi$ for any $t$, $1$ in particular.

	\begin{ftheo}
		Let $f : G \to \C$ be analytic and suppose that $\overline{B(a,r)} \subseteq G$ for some $r > 0$. If $\gamma(t) = a + re^{\iota t}$ for $ 0 \le t \le 2\pi$, then
		\[ f(z) = \frac{1}{2\pi\iota} \int_\gamma \frac{f(w)}{w - z} \dif w \]
		for $|z-a| < r$.
	\end{ftheo}
	\begin{proof}
		Defining $G_1 = \left\{ (z-a)/r : z \in G \right\}$ and $g(z) = f(a + r z)$, it suffices to consider the case where $a = 0$ and $r = 1$.\\
		Fix $z$ with $|z| < 1$. It must be shown that
		\[ f(z) = \int_{2\pi\iota} \int_\gamma \frac{f(w)}{w-z} \dif w = \frac{1}{2\pi} \int_0^{2\pi} \frac{f(e^{\iota s}) e^{\iota s}}{e^{\iota s} - z} \dif s. \]
		That is, we want to show that
		\begin{align*}
			 0 &= \int_0^{2\pi} \frac{f(e^{\iota s}) e^{\iota s}}{e^{\iota s} - z} \dif s - 2 \pi f(z) \\
			 	&= \int_0^{2\pi} \left(\frac{f(e^{\iota s}) e^{\iota s}}{e^{\iota s} - z} - f(z)\right) \dif s.
		\end{align*}
		For this, let
		\[ \varphi(s,t) = \frac{f(z + t(e^{\iota s} - z)) e^{\iota s}}{e^{\iota s} - z} - f(z) \]
		for $0 \le t \le 1$ and $0 \le s \le 2\pi$, and
		\[ g(t) = \int_0^{2\pi} \varphi(s,t) \dif s. \]
		We wish to show that $g(1) = 0$. Observe that
		\[ g(0) = \int_0^{2\pi} \frac{f(z) e^{\iota s}}{e^{\iota s} - z} - f(z) \dif s = f(z) \int_0^{2\pi} \frac{e^{\iota s}}{e^{\iota s} - z} \dif s - 2\pi f(z) = 0. \]
		Also,
		\begin{align*}
			g'(t) &= \int_0^{2\pi} \dpd{\varphi}{t}(s,t) \dif s \\
				&= \int_0^{2\pi} \frac{e^{\iota s}}{e^{\iota s} - z} f'(z + t(e^{\iota s} - z)) (e^{\iota s} - z) \dif s \\
				&= \int_0^{2\pi} e^{\iota s} f'(z + t(e^{\iota s - z})) \dif s \\
				&= \frac{1}{t} f(z + t(e^{\iota s} - z)) \Biggr|_{s=0}^{s=2\pi} \\
				&= 0,
		\end{align*}
		completing the proof.
	\end{proof}

	If $|z-a| < r$ and $w$ is such that $|w-a| = r$, then
	\[ \frac{1}{w-z} = \frac{1}{w-a} \cdot \frac{1}{1 - \frac{z-a}{w-a}} = \frac{1}{w-a} \sum_{i=0}^{\infty} \left( \frac{z-a}{w-a} \right)^i. \]
	since $|z-a| < |w-a|$.\\
	Now, multiplying by $f(w)/2\pi\iota$ and integrating around the circle $\gamma$ defined by $|w-a|=r$, we get that
	\[ f(z) = \int_\gamma \frac{f(w)}{2\pi\iota} \sum_{i=0}^\infty \frac{(z-a)^{i}}{(w-a)^{i+1}} \dif w. \]
	But how do we simplify the right hand side? We do not know (\emph{yet}) that the integral and summation may be switched. So, let us get to showing this.

	\begin{lemma}
		Let $\gamma$ be a rectifiable curve in $\C$ and suppose that $F_n$ and $F$ are continuous functions on $\{\gamma\}$. If $(F_n)$ uniformly converges to $F$ on $\{\gamma\}$, then
		\[ \int_\gamma = \lim_{n\to\infty} \int_\gamma F_n. \]
	\end{lemma}
	\begin{proof}
		Let $\epsilon > 0$ and let $N \in \N$ such that
		\[ |F_n(w) - F(w)| < \frac{\epsilon}{V(\gamma)} \]
		for $n \ge N$. This implies that
		\[ \left| \int_\gamma F_n - \int_\gamma F \right| \le V(\gamma) \sup_{w} |F_n(w) - F(w)| \le \epsilon \]
		for $n \ge N$, completing the proof.
	\end{proof}

	\begin{ftheo}
		Let $f$ be analytic on $B(a,R)$. Then,
		\[ f(z) = \sum_{n=0}^\infty a_n(z-a)^n \]
		for all $|z-a| < R$, where $a_n = f^{(n)}(a)/n!$ and this series has radius of convergence at least $R$.
	\end{ftheo}
	\begin{proof}
		Let $0 < r < R$ such that $\overline{B(a,r)} \subseteq B(a,R)$. Let $\gamma(t) = a + re^{\iota t}$ ($0\le t\le 2\pi$). Since $|z-a| < r$, if $M = \max\{|f(w)| : |w-a| = r\}$,
		\[ \frac{|f(w)||z-a|^n}{|w-a|^{n+1}} \le \frac{M}{r} \left( \frac{|z-a|}{r} \right)^n. \]
		Since $|z-a| < r$,
		\[ \sum_{n=0}^{\infty} f(w) \frac{(z-a)^n}{(w-a)^{n+1}} \]
		converges uniformly for $w$ on $\{\gamma\}$. By the discussion before the previous lemma together with the lemma itself,
		\begin{equation}
			\label{eqn: unsimplified taylor}
			\tag{$*$}
			f(z) = \sum_{n=0}^{\infty} \left(\frac{1}{2\pi\iota} \int_\gamma \frac{f(w)}{(w-a)^{n+1}}\right) (z-a)^n.
		\end{equation}
		Since
		\[ a_n = \frac{1}{2\pi\iota} \int_\gamma \frac{f(w)}{(w-a)^{n+1}}. \]
		is independent of $z$, \eqref{eqn: unsimplified taylor} converges for $|z-a| < R$. However, we now know from \Cref{theo: power series rad conv}(c) that $a_n = f^{(n)}(a)/n!$, completing the proof.
	\end{proof}

	\begin{corollary}
		If $f$ is analytic,
		\[ f^{(n)}(a) = \frac{1}{2\pi\iota} \int_\gamma \frac{f(w)}{(w-a)^{n+1}} \dif w \]
		where $\gamma = a + re^{\iota t}$ and $r < R$, the radius of convergence of the series.
	\end{corollary}

	\begin{corollary}
		If $f:G\to\C$ is analytic, then $f$ is infinitely differentiable.
	\end{corollary}
	Indeed, this follows directly from the fact that
	\[ f^{(n)}(a) = \frac{n!}{2\pi\iota} \int_\gamma \frac{f(w)}{(w-a)^{n+1}} \dif w \]
	where $\gamma(t) = a+re^{\iota t}$ for $0 \le t \le 2\pi$.\\

	\begin{corollary}[Cauchy's Estimate]
		\label{theo: cauchys estimate}
		Let $f$ be analytic on $B(a,R)$ and suppose $|f(z)| \le M$ for all $z \in B(a,R)$. Then
		\[ |f^{(n)}(a)| \le \frac{n!M}{R^n}. \]
	\end{corollary}
	Indeed, the above applies with $r < R$ so we get that
	\[ |f^{(n)}(a)| \le \frac{n!}{2\pi} \int_\gamma \frac{|f(w)|}{|w-a|^{n+1}} |{\dif} {w}| \le \frac{n!}{2\pi} \cdot \frac{M}{r^{n+1}} \cdot 2\pi r = \frac{n!M}{r^n}. \]
	Since $r < R$ is arbitrary, we may let $r \to R^{-}$.

	\begin{prop}
		Let $f$ be analytic on the disk $B(a,R)$ and suppose that $\gamma$ is a closed rectifiable curve in $B(a,R)$. Then $\int_\gamma f = 0$.
	\end{prop}
	\begin{proof}
		Due to \Cref{theo: ftc-like}, it suffices to show that $f$ has a primitive. We know that
		\[ f(z) = \sum_{n=0}^{\infty} a_n (z-a)^n \]
		for $|z-a| < R$, where $a_n = f^{(n)}(a)/n!$. Consider the function
		\[ F(z) = (z-a) \sum_{n=0}^{\infty} \frac{a_n}{n+1} (z-a)^{n}. \]
		Since $\lim_{n\to\infty} (n+1)^{1/n} = 1$, this power series has the same radius of convergence as $\sum a_n (z-a)^n$. Therefore, $F$ is defined on $B(a,R)$. Moreover, $F'(z) = f(z)$ for $|z-a| < R$ by \Cref{theo: power series rad conv}(b), completing the proof.
	\end{proof}

	\begin{fdef}
		An \emph{entire} function is a function which is defined and analyitc on the whole complex plane $\C$.
	\end{fdef}

	\begin{prop}
		If $f$ is entire, then it has a power series expansion with infinite radius of convergence.
	\end{prop}
	Therefore, entire functions may be considered as polynomials of ``infinite degree''. Polynomials of finite non-zero degree are typically unbounded. These two insights lead to the following result.

	\begin{ftheo}[Liouville's Theorem]
		\label{liouvilles theorem}
		If $f$ is a bounded entire function, then $f$ is constant.
	\end{ftheo}
	\begin{proof}
		Suppose that $|f(z)| \le M$ for all $z \in \C$. We shall show that $f'(z) = 0$ for all $z \in \C$. By \nameref{theo: cauchys estimate}, since $f$ is analytic on any disk $B(z,R)$, $|f'(z)| \le M/R$. However, $R$ is arbitrary so $f'(z) = 0$ for any $z \in \C$.
	\end{proof}

	\begin{ftheo}[Fundamental Theorem of Algebra]
		If $p$ is a non-constant polynomial with coefficients in $\C$, then there exists $a \in \C$ with $p(a) = 0$.
	\end{ftheo}
	\begin{proof}
		Suppose $p(z) \ne 0$ for all $z \in \C$. Consider the entire function $f(z) = 1/p(z)$. This function is then bounded as $p(z)$ goes to $\infty$ as $z$ goes to infinity. By \nameref{liouvilles theorem}, $f$ (and thus $p$) is constant, which is a contradiction.
	\end{proof}

	Due to the above, $\C$ is an algebraically closed field.

	\begin{corollary}
		If $p(z)$ is a polynomial and its roots are $(p_j)$ with multiplicity $k_j$ (for $1\le j\le m$), then $p(z) = C (z-a_1)^{k_1} (z-a_2)^{k_2} \cdots (z-a_m)^{k_m}$ for some constant $C$, where $\sum k_j$ is the degree of $p$.
	\end{corollary}

	It is not too difficult to show that if $p(z)$ is a non-constant polynomial, then $p$ is a surjective analytic function on $\C$. However, we know that the map  $z \mapsto e^z$ is an entire function but there is no $b \in C$ such that $e^b = 0$. So, power series (``polynomials of infinite degree'') cannot be thought of in the same way as ordinary polynomials (of finite degree). However, we shall see later that given a non-constant entire function $f$, there exists at most one $a \in \C$ that is not in the image of $f$. This is referred to as Little Picard's Theorem.

	\begin{ftheo}
		Let $G$ be a connected open set and $f : G \to \C$ be analytic. Then, the following are equivalent statements.
		\begin{enumerate}[label=(\alph*)]
			\item $f$ is identically zero.
			\item There exists $a \in \C$ such that for all $n \ge 0$, $f^{(n)}(a) = 0$.
			\item $\{ z \in G : f(z) = 0 \}$ has a limit point in $G$.
		\end{enumerate}
	\end{ftheo}
	\begin{proof}
		Clearly, (a) implies (b) and (c).\\
		Next, let us show that (c) implies (b). Let $a \in G$ be a limit point of the zero set of $f$. Let $R > 0$ such that $B(a,R) \subseteq G$. Since $a$ is a limit point of $z$ and $f$ is continuous, $f(a) = 0$. Let $n \ge 1$ such that $f^{(k)}(a) = 0$ for $k < n$ and $f^{(n)}(a) \ne 0$. Expanding $f$ as a power series about $a$ gives that
		\[ f(z) = \sum_{k=n}^\infty a_k (z-a)^{k} \]
		for $|z-a| < R$ and $a_n \ne 0$. Let
		\[ g(z) = \sum_{k=n}^\infty a_k (z-a)^{k-n}. \]
		Since $g$ is continuous in $B(a,R)$ and $g(a) \ne 0$, let $r<R$ such that $g(z) \ne 0$ when $|z-a| < r$. Since $a$ is a limit point of $z$, there exists $b$ with $f(b) = 0$ and $0 < |a-b| < r$. This gives $0 = (b-a)^n g(b)$, so $g(b) = 0$, a contradiction. Therefore, no such $n$ can be found and (b) is true.\\
		Finally, let us show that (b) implies (a). Let
		\[ A = \{ z \in G : f^{(n)}(z) = 0 \text{ for all } n \ge 0 \}. \]
		By the definition of (b), $A \ne \ emptyset$. We shall show that $A$ is both open and closed in $G$, and by the connectedness of $G$ is follows that $A$ is the entirety of $G$. Showing that $A$ is closed is direct -- if $z \in \overline{A}$ and $(z_k)$ a sequence such that $z_k \to z$, then since each $f^{(k)}$ is continuous, $f^{(n)}(z) = \lim f^{(n)}(z_k) = 0$ for all $n \ge 0$, and so $z \in A$. On the other hand, if $a \in A$, we can write $f(z) = \sum_{n = 0}^\infty \frac{f^{(n)}(a)}{n!} (z-a)^n = 0$ on $B(a,R)$ (for some $R > 0$), so $B(a,R) \subseteq A$ and $A$ is open, completing the proof.
	\end{proof}

	\begin{corollary}
		If $f,g$ are analytic on a region $G$, then $f \equiv g$ iff $\{ z \in G : f(z) = g(z) \}$ has a limit point in $G$.
	\end{corollary}

	\begin{corollary}
		\label{cor: finite multiplicity}
		If $f$ is non-trivial and analytic on an open connected set $G$, then each zero of $f$ has finite multiplicity. More explicitly, for each $a \in G$ with $f(a) = 0$, there is an integer $n \ge 1$ and an analytic function $g : G \to \C$ such that $g(a) \ne 0$ and $f(z) = (z-a)^n g(z)$ for all $z \in G$.
	\end{corollary}
	\begin{proof}
		It is clear that there exists a largest $n \ge 1$ such that $f^{(k)}(a) = 0$ for all $k \le n-1$.
	\end{proof}

	\begin{corollary}
		If $f : G \to \C$ is non-trivial and analytic, and $a \in G$ with $f(a) = 0$, then there exists $R > 0$ such that $B(a,R) \subseteq G$, and $f(z) \ne 0$ for all $0 < |z-a| < R$. 
	\end{corollary}
	The above follows from the fact that the zeros of $f$ are isolated.

	\begin{ftheo}[Maximum Modulus Theorem]
		\label{theo: maximum modulus theorem}
		If $G$ is a region and $f:G\to\C$ is an analytic function such that there is a point $a\in G$ with $|f(a)| \ge |f(z)|$ for all $z \in G$, then $f$ is constant.
	\end{ftheo}
	That is, if $|f|$ attains its maximum, $f$ is constant.
	\begin{proof}
		Let $\overline{B(a,r)} \subseteq G$ and $\gamma(t) = a + re^{\iota t}$ for $0 \le t \le 2\pi$. Then,
		\begin{align*}
			f(a) &= \frac{1}{2\pi\iota} \int_\gamma \frac{f(w)}{w-a} \dif w \\
				&= \frac{1}{2\pi} \int_0^{2\pi} f(a + re^{\iota t}) \dif t.
		\end{align*}
		Therefore,
		\[ |f(a)| \le \frac{1}{2\pi} \int_0^{2\pi} |f(a+re^{\iota t})| \dif t \le |f(a)|. \]
		Therefore,
		\[ 0 = \int_0^{2\pi} \left( |f(a)| - |f(a+re^{\iota t})| \right) \dif t. \]
		Since the integrand is continuous is non-negative, $|f(a)| = |f(a+re^{\iota t})|$ for all $t \in [0,2\pi]$. If $f(a) = 0$, we are clearly done. Otherwise, since $r$ was arbitrary, $f$ maps any disk $B(a,R)$ to the circle $|z| = |f(a)|$. It may then be shown using the Cauchy-Riemann equations that $f$ is constant on $B(a,R)$ and is equal to $f(a)$ for all $|z-a| < R$. Therefore, $f(z) = f(a)$ for all $z \in G$ since the zeros of $f-f(a)$ are not isolated.
	\end{proof}

\subsection{Integrals along closed curves}

	Recall that
	\[ \int_\gamma \frac{1}{z-a} \dif z  = 2 \pi \iota n \]
	if $\gamma(t) = a + e^{\iota n t}$ for $t \in [0,2\pi]$. However, this property is not peculiar to the path $\gamma$, as shown by the following result.

	\begin{ftheo}
		If $\gamma : [0,1] \to \C$ is a closed rectifiable curve and $a \not\in \{\gamma\}$, then
		\[ \frac{1}{2\pi\iota} \int_\gamma \frac{1}{z-a} \dif z \]
		is an integer.
	\end{ftheo}
	\begin{proof}
		Using \Cref{lemma: polygonal}, we may assume that $\gamma$ is piecewise smooth (Why?). \\
		Let us assume that $\gamma$ is smooth.
		Define $g : [0,1] \to \C$ by
		\[ g(t) = \int_0^t \frac{\gamma'(s)}{\gamma(s) - a} \dif s. \]
		Then, $g(0) = 0$ and $g(1) = \int_\gamma 1/(z-a) \dif z$. We also have that
		\[ g'(t) = \frac{\gamma'(t)}{\gamma(t) - a} \]
		for $0 \le t \le 1$. This gives that
		\[ \od{}{t} \left(e^{-g(t)}(\gamma(t) - a)\right) = e^{-g(t)} \gamma'(t) - g'(t) e^{-g(t)} (\gamma(t)-a) = 0. \]
		Therefore,
		\[ e^{-g(0)} (\gamma(0) - a) = e^{-g(1)} (\gamma(1) - a). \]
		Because $\gamma(0) = \gamma(1)$ (the curve is closed) and $g(0) = 0$, $g(1) = 2\pi\iota n$ for some integer $n$.
		In the case where $\gamma$ is piecewise-smooth, we can define $g$ by integrating over each of the smooth intervals and the result follows near-identically.
	\end{proof}

	\begin{fdef}
		If $\gamma$ is a closed rectifiable curve in $\C$ then for $a \not\in \{\gamma\}$,
		\[ n(\gamma;a) = \frac{1}{2\pi\iota} \int_\gamma \frac{1}{z-a} \dif z \]
		is called the \emph{index} of $\gamma$ with respect to the point $a$. It is also sometimes referred to as the \emph{winding number} of $\gamma$ around $a$.
	\end{fdef}
	Recall the definition of $(-\gamma)$ from \eqref{eqn: 2.2}, also denoted $\gamma^{-1}$. If $\gamma$ and $\sigma$ are curves on $[0,1]$ with $\gamma(1) = \sigma(0)$, $\gamma+\sigma$ is the curve
	\[ (\gamma+\sigma)(t) = \begin{cases} \gamma(2t), & 0 \le t \le 1/2, \\ \sigma(2t-1), & 1/2 \le t \le 1. \end{cases} \]

	\begin{prop}
		If $\sigma,\gamma$ are closed rectifiable curves with the same initial (and final) points, then
		\begin{equation}
			n(\gamma;a) = -n(-\gamma;a)
		\end{equation}
		for all $a \not\in \{\gamma\}$ and
		\begin{equation}
			n(\gamma+\sigma;a) = n(\gamma;a) + n(\sigma;a)
		\end{equation}
		for all $a \not\in \{\sigma\} \cup \{\gamma\}$.
	\end{prop}
	We omit the proof of the above.\\

	The reason for $n(\cdot;\cdot)$ being called the winding number is clear from what happens in the case of a circle. For $a + e^{2\pi\iota n t}$, then $n(\gamma;a) = n$ is the number of times this curve ``winds'' or ``wraps'' around $a$. In fact, if $|b-a| < 1$, $n(\gamma;b) = n$ and if $|b-a| > 1$, $n(\gamma;b) = 0$.\\

	Recall that the components of a set are its maximal connected subsets.

	\begin{ftheo}
		\label{theo: winding number constant on components}
		Let $\gamma$ be a closed rectifiable curve in $\C$. Then $n(\gamma;a)$ is constant for $a$ belonging to a component of $G = \C \setminus \{\gamma\}$. Also, $n(\gamma;a) = 0$ for $a$ belonging to the unbounded component of $G$.
	\end{ftheo}
	\begin{remark}
		Since $\{\gamma\}$ is compact, the connected set $\{ z : |z| > R \} \subseteq G$ for sufficiently large $R$, so $\gamma$ has precisely one unbounded component.
	\end{remark}
	\begin{proof}
		Define $f : G \to \C$ by $f(a) = n(\gamma;a)$.
		If we manage to show that $f$ is continuous on $G$, we are done since the image of this map is a subset of the integers and each component is connected by definition, so $f$ is constant on each component.\\
		Recall that components of $G$ are open. Fix $a \in G$ and let $r = d(a , \{\gamma\}) > 0$. If $|a-b| < \delta \le r/2$ (we shall fix $\delta$ more precisely later), then
		\begin{align*}
			|f(a) - f(b)| &= \frac{1}{2\pi} \left| \int_\gamma \left( \frac{1}{z-a} - \frac{1}{z-b} \right) \dif z \right| \\
				&\le \frac{|a-b|}{2\pi} \int_\gamma \frac{1}{|z-a||z-b|} |{\dif} {z}|.
		\end{align*}
		By definition, $|z-a| \ge r$ for any $a \in \{\gamma\}$ and $|z-b| \ge |z-a| - |a-b| \ge r/2$. So,
		\begin{align*}
			|f(a) - f(b)| &\le \frac{|a-b|}{2\pi} \int_\gamma \frac{2}{r^2} |{\dif} {z}| \\
				&\le \frac{\delta}{\pi r^2} V(\gamma).
		\end{align*}
		For a given $\epsilon > 0$, setting $\delta = \min\{r/2, \epsilon\pi r^2 / V(\gamma)\}$ does the job, completing the first part of the proof.\\

		It remains to show that $\lim_{a\to\infty}f(a) = 0$ (Why does this imply the required?). Let $U$ be the unbounded component of $G$. For a given $R > 0$, let $a \in U$ such that $d(a;\gamma) > R$. Then,
		\[ |f(a)| = \frac{1}{2\pi} \int_\gamma \left|\frac{1}{z-a}\right| |{\dif} {z}| \le \frac{1}{2\pi R} \int_\gamma |{\dif} {z}| = \frac{V(\gamma)}{2\pi R}. \]
		$R$ can be made arbitrarily large (as $a\to\infty$), so we are done.
	\end{proof}

	Now, one would expect to see that for a ``nice'' $f$ defined on a nice region $G$, for closed rectifiable paths $\gamma$, $\int_\gamma f$ is zero. Indeed, this is evidenced by how we saw that $n(\gamma;a)$ is zero on the unbounded component of $\C\setminus\{\gamma\}$. Even before that, we had seen that $\int_\gamma f = 0$ if $f : G \to \C$ is analytic, $\gamma$ is a closed rectifiable curve, and $G = B(a,R)$.\\
	It turns out that the last of the above statements is true for a more general class of regions, not just disks. It is not true on any region however, since the winding number of a path about a point can be nonzero. It turns out that this winding number situation is the only real problematic case, and we shall see in \nameref{cauchys theorem v1} that this ``general class of regions'' is the set of regions without any ``hole''.\\
	On the other hand, one may ask the question: for a fixed domain $G$ and $f$ analytic on $G$, for what $\gamma$ inside $G$ is $\int_\gamma f = 0$?

	\begin{flem}
		\label{lemma 2.31: Fm}
		Let $\gamma$ be a rectifiable curve and suppose $\varphi : \{\gamma\} \to \C$ is continuous. Then, for each $m \ge 1$, defining
		\[ F_m(z) = \int_\gamma \frac{\varphi(w)}{(w-z)^m}, \]
		$F_m$ is analytic on $\C\setminus\{\gamma\}$ and $F_m'(z) = m F_{m+1}(z)$.
	\end{flem}
	Note that this matches the power series expansion for a general function we had got earlier, where $a_n$, which is related to the $n$th derivative of $f$ at that point, was evaluated as an integral of the above form.
	\begin{proof}
		Let us first show that $F_m$ is continuous for each $m$. Let $a \in \C\setminus\{\gamma\}$. We have
		\begin{align}
			F_m(z) - F_m(a) &= \int_\gamma \varphi(w) \left( \frac{1}{(w-z)^m} - \frac{1}{(w-a)^m} \right) \dif w \nonumber \\
				&= \int_\gamma \left( \frac{1}{w-z} - \frac{1}{w-a} \right) \sum_{k=1}^{m} \frac{1}{(w-z)^{m-k} (w-a)^{k-1}} \dif w \nonumber \\
				&= \int_\gamma (z-a) \sum_{k=1}^{m} \frac{1}{(w-z)^{m-k+1} (w-a)^{k}} \dif w \label{eqn: 2.7}
		\end{align}
		So,
		\[ |F_m(z) - F_m(a)| \le \int_\gamma |\varphi(w)| |z-a| \sum_{k=1}^m \frac{1}{|w-z|^{m+1-k}|w-a|^{k}} |{\dif} {w}| \]
		Since $\varphi$ is continous on $\{\gamma\}$ and $\{\gamma\}$ is compact, there exists $M > 0$ such that $|\varphi(w)| \le M$ for all $w \in \{\gamma\}$. Because $a\not\in\{\gamma\}$, $r = d(a,\{\gamma\}) > 0$. Let $\delta \le r/2$. Then, for $z \in \C\setminus\{\gamma\}$ with $|z-a| < \delta$, we have that $|w-z| \ge r$ and $|w-a| \ge |w-z| - |z-a| \ge r/2$. So,
		\begin{align*}
			|F_m(z) - F_m(a)| &\le \int_\gamma |\varphi(w)| |z-a| \sum_{k=1}^m \frac{1}{|w-z|^{m+1-k}|w-a|^{k}} |{\dif} {w}| \\
				&= M\delta \int_\gamma \sum_{k=1}^m \frac{1}{|w-z|^{m+1-k}|w-a|^{k}} |{\dif} {w}| \\
				&\le M\delta \int_\gamma \sum_{k=1}^m \frac{1}{(r/2)^{m+1}} |{\dif}{w}| \\
				&= \delta \cdot Mm \left(\frac{2}{r}\right)^{m+1} V(\gamma).
		\end{align*}
		Taking $\delta$ appropriately small, we are done with the first part of the proof.\\
		Now, let us show the differentiability of $F_m$. Rewriting \eqref{eqn: 2.7},
		\[ \frac{F_m(z) - F_m(a)}{z-a} = \sum_{k=1}^m \int_\gamma \frac{\varphi(w)(w-a)^{-k}}{(w-z)^{m+1-k}} \dif w. \]
		The limit of this as $z\to a$ is clearly well-defined, so $F_m$ is differentiable. Because $a\not\in\gamma$ by definition, each of the $m$ integrands above is a continuous function of $w$.\\
		Therefore,
		\begin{align*}
			\lim_{z\to a} \frac{F_m(z) - F_m(a)}{z-a} &= \sum_{k=1}^m \int_\gamma \frac{\varphi(w)(w-a)^{-k}}{(w-a)^{m+1-k}} \dif w \\
				&= \sum_{k=1}^m \int_\gamma \frac{\varphi(w)}{(w-a)^{m+1}} \dif w = m F_{m+1}(a). \qedhere
		\end{align*}
	\end{proof}

	\begin{definition}
		If $\gamma$ is a closed rectifiable curve and $G$ is a region, we say that $\gamma$ is \emph{homologous} to $0$ on $G$ and write $\gamma \approx 0$ in $G$ if $n(\gamma;w) = 0$ for all $w \in \C\setminus G$.
	\end{definition}

	\begin{ftheo}[Cauchy's Integral Formula, version 1]
		\label{cauchy integral formula v1}
		Let $G$ be an open subset of $\C$ and $f : G \to \C$ be analytic. If $\gamma$ is a closed rectifiable curve in $G$ such that $n(\gamma;w) = 0$ for $w \in \C\setminus G$, then for $a \in G\setminus\{\gamma\}$,
		\[ \frac{1}{2\pi\iota} \int_\gamma \frac{f(z)}{z-a} \dif z = n(\gamma;a) f(a). \]
		In particular, if $n(\gamma;a) = 0$, the integral on the left is zero.
	\end{ftheo}
	\begin{proof}
		Define $\varphi : G \times G \to \C$ as
		\[ \varphi(z,w) =
			\begin{cases}
				\frac{f(z) - f(w)}{z-w}, & z\ne w \\
				f'(z), & z=w.
			\end{cases}
		\]
		Observe that if we show that $\int_\gamma \varphi(z,w) \dif z = 0$, then
		\[ f(z) \int_\gamma \frac{1}{w-z} \dif w = \int_\gamma \frac{f(w)}{w-z} \dif z, \]
		which implies the required since the left-hand side is just $2\pi\iota n(\gamma;z) f(z)$.\\
		It is not too difficult to show that $\varphi$ is continuous $G\times G$ (this uses the continuity of $f'$!).\\
		Fix some $w \in G$. We shall first show that $\psi_w$ that maps $z \mapsto \varphi(z,w)$ is analytic on $G$. First, let us check at $a \ne w$. We have
		\begin{align*}
		 	\lim_{h\to 0} \frac{\varphi(a+h,w) - \varphi(a,w)}{h} &= \lim_{h\to 0} \frac{1}{h} \left( \frac{f(a+h) - f(w)}{a+h-w} - \frac{f(a) - f(w)}{a-w} \right) \\
		 		&= \lim_{h \to 0} \frac{1}{h} \left( \frac{(a-w) (f(a+h) - f(w)) - (a-w) f(a) - hf(a) + (a+h-w)f(w)}{(a+h+w)(a-w)} \right) \\
		 		&= \lim_{h \to 0} \frac{1}{h} \left( \frac{(a-w)(f(a+h)-f(a)) - h(f(a)-f(w))}{(a+h-w)(a-w)} \right) \\
		 	\psi_w'(a) &= \frac{f'(a)}{a-w} - \frac{f(a)-f(w)}{(a-w)^2}.
		\end{align*} 
		Since $f$ is analytic, $\psi_w$ is analytic on $G \setminus \{w\}$.\\
		For $a=w$ on the other hand,
		\begin{align}
			\lim_{h\to 0} \frac{\varphi(w+h,w) - \varphi(w,w)}{h} &= \lim_{h\to 0} \frac{1}{h} \left( \frac{f(w+h) - f(w)}{h} - f'(w) \right) \nonumber \\
				&= \lim_{h\to 0} \frac{f(w+h) - f(w) - hf'(w)}{h^2} \label{eqn: 2.8} \\
			\psi_w'(w) &= \frac{1}{2} f''(w), \nonumber
		\end{align}
		where the final step is direct on using the fact that $f$ has a power series expansion on some $B(w,r)$ for small $r$.\\
		Checking that $\psi_w$ is analytic at $G$ is not too difficult on using the power series expansion of $f$ about $w$ (we in fact get a limit similar to \eqref{eqn: 2.8}).\\
		So, we now have that $\psi_w$ is analytic. Define
		\[ H = \{ w \in \C : n(\gamma;w) = 0 \}. \]
		By \Cref{theo: winding number constant on components}, $H$ is open. Moreover, $G \cup H = \C$. Define $g:\C\to\C$ by
		\[
			g(z) =
			\begin{cases}
				\int_\gamma \psi_w(z) \dif w, & z \in G \\
				\int_\gamma \frac{f(w)}{w-z} \dif w, & z \in H.
			\end{cases}
		\]
		We shall show that $g$ is bounded and entire, and thus constant. If we then show that $\lim_{z\to 0} g(z) = 0$ (this involves only the second part of the definition of $g$), we have $g(z) = 0$ on $G$ as well, which is exactly what we want.\\
		Boundedness of the first part is straightforward as $G$ may be assumed to be bounded. For the second part,
		\[ \int_\gamma \frac{|f(w)|}{|w-z|} |{\dif}{w}| \le M \int_\gamma \frac{1}{|w-z|} |{\dif}{w}|, \]
		where $M$ is the supremum of $f$. However, the integral is clearly bounded, and the integrand (so the integral) may even be made infinitely small as $z \to \infty$. If we show now that $g$ is entire, then $g$ is zero everywhere on $\C$ and we are home.
	\end{proof}

	\begin{ftheo}[Cauchy's Integral Formula, version 2]
		\label{cauchys integral formula v2}
		Let $G$ be an open subset of $\C$ and $f : G \to \C$ be analytic. If $\gamma_1,\ldots,\gamma_m$ are closed rectifiable curves in $G$ such that $\sum_k n(\gamma_k;w) = 0$ for $w \in \C\setminus G$, then for $a \in G\setminus\bigcup_k\{\gamma_k\}$,
		\[ \sum_{k=1}^m \frac{1}{2\pi\iota} \int_{\gamma_i} \frac{f(z)}{z-a} \dif z = f(a) \sum_{k=1}^m n(\gamma_k;a). \]
	\end{ftheo}
	The idea of the proof is very similar to that of \nameref{cauchy integral formula v1}, with the only difference being that we define
	\[ H = \{ z \in \C : \sum_k n(\gamma_k;z) = 0 \} \]
	and
	\[
		g(z) = 
		\begin{cases}
			\sum_{k=1}^m \int_{\gamma_k} \frac{f(w)}{w-z} \dif w, & z \in H, \\
			\sum_{k=1}^m \varphi(z,w) \dif w, & z \in G.
		\end{cases}
	\]

	% \begin{fcor}[Cauchy's Theorem, version 1]
	\begin{fcor}[Cauchy's Theorem]
		\label{cauchys theorem v1}
		Let $G$ be an open subset of $\C$ and $f : G \to \C$ be analytic. If $(\gamma_k))_{k=1}^m$ are closed rectifiable curves in $G$ such that $\sum_k n(\gamma_k;w) = 0$ for $w \in \C \setminus G$, then
		\[ \sum_{k=1}^m \int_{\gamma_k} f = 0. \]
	\end{fcor}

	The above follows directly from \nameref{cauchys integral formula v2} on setting $g(z) = f(z) (z-a)$ for some $a \in G \setminus \bigcup_k \{\gamma_k\}$. Indeed, such an $a$ exists since $\bigcup_k \{\gamma_k\}$ is a finite union of compact sets so is closed and bounded, but $G$ is open (if it is closed, it must be $\C$, which is not bounded).\\
	In fact, we may even prove \Cref{cauchys integral formula v2} from \Cref{cauchys theorem v1} by using it on the analytic function
	\[
		g(z) =
		\begin{cases}
			\frac{f(z)-f(a)}{z-a}, & z \ne a, \\
			f'(a), & z = a.
		\end{cases}
	\]

	Going back to \Cref{lemma 2.31: Fm}, we have that
	\[ F(z) = n(\gamma;z) = \frac{1}{2\pi\iota} \int_\gamma \frac{f(w)}{w-z} \dif w \]
	for $z \in G \setminus \{\gamma\}$. The result there says that
	\[ F^{(m)}(a) = m! \frac{1}{2\pi\iota} \int_\gamma \frac{f(w)}{(w-a)^{m+1}} \dif w. \]
	Further,
	\[ F^{(m)}(a) = n(\gamma;a) f^{(m)}(a) \]
	since $n(\gamma,\cdot)$ is constant on components.

	\begin{theorem}
		Let $G$ be an open subset of $\C$ and $f : G \to \C$ be analytic. If $(\gamma_k))_{k=1}^m$ are closed rectifiable curves in $G$ such that $\sum_k n(\gamma_k;w) = 0$ for $w \in \C \setminus G$, then for $a \in G \setminus \bigcup \{\gamma_k\}$ and $r \ge 1$,
		\[ f^{(r)}(a) \sum_{k=1}^m n(\gamma_k;a) = r! \sum_{k=1}^m \frac{1}{2\pi\iota} \int_{\gamma_k} \frac{f(w)}{(w-a)^{r+1}} \dif w. \]
	\end{theorem}

	\begin{ftheo}[Morera's Theorem]
		\label{morera's theorem}
		Let $G$ be a region and $f : G \to \C$ be continuous such that for any triangular path $T$ in $G$, $\int_T f = 0$. Then $f$ is analytic.
	\end{ftheo}
	Above, a triangular path is a closed polygonal curve that consists of three ``edges''. That is, it looks like a triangle.
	\begin{proof}
		It is enough to show that $f$ is analytic on each open disk contained inside $G$, so assume wlog that $G$ is an open disk $B(a,R)$.
		We are done if we find a primitive $F$ of $f$. Indeed, this would mean that $F$, and thus $f$, is analytic.\\
		For $z \in G$, define
		\[ F(z) = \int_{[a,z]} f, \]
		where $[a,z]$ is the segment joining $a$ and $z$. More concretely, $[a,z]$ is the curve given by
		\[ \gamma(t) = a + t(z-a) \]
		for $t \in [0,1]$.\\
		Fix some $z_0 \in G$. We shall show that $F'(z_0) = f(z_0)$ For any $z \in G$,
		\[ F(z) = \int_{[a,z_0]} f + \int_{[z_0,z]} f = F(z_0) + \int_{[z_0,z]} f . \]
		Then,
		\begin{align*}
			\frac{F(z) - F(z_0)}{z-z_0} &= \frac{1}{z-z_0} \int_{[z_0,z]} f \\
				&= f(z_0) + \frac{1}{z-z_0} \int_{z_0,z} f(w) - f(z_0) \dif w.
		\end{align*}
		Now, fixing $\epsilon > 0$, use the continuity of $f$ to get $\delta > 0$ such that if $|z_0 - w| < \delta$, then $|f(z_0) - f(w)| < \epsilon$. Then, when $|z_0 - z| < \delta$,
		\begin{align*}
			\left| \frac{F(z) - F(z_0)}{z-z_0} - f(z_0) \right| \le \frac{1}{|z-z_0|} \int_{[z_0,z]} |f(w) - f(z_0)| |{\dif}{w}| \\
				&\le \frac{1}{|z-z_0|} \int_{[z_0,z]} \epsilon |{\dif}{w}| \\
				&= \epsilon,
		\end{align*}
		completing the proof.
	\end{proof}

	Recall \Cref{cor: finite multiplicity}. Similarly, if $a_1,\ldots,a_k$ are the zeroes of $f$ (repeated according to multiplicity), then
	\[ f(z) = (z-a_1)(z-a_2)\cdots(z-a_k) g(z), \]
	where $g(a_i) \ne 0$ for all $i$. Therefore,
	\[ \frac{f'(z)}{f(z)} = \frac{1}{z-a_1} + \frac{1}{z-a_2} + \cdots + \frac{1}{z-a_k} + \frac{g'(z)}{g(z)}. \]

	\begin{prop}
		Let $G$ be a region and $f$ an analytic function on $G$ with finitely many zeroes $a_1,\ldots,a_k$ (repeated according to multiplicity). If $\gamma$ is a closed rectifiable curve in $G$ that does not pass through any $a_j$ and $\gamma \approx 0$ on $G$, then
		\[ \frac{1}{2\pi\iota} \int_\gamma \frac{f'(z)}{f(z)} \dif z = \sum_j n(\gamma;a_j). \]
	\end{prop}

	We do not prove the above since it follows directly from the prior discussion and \Cref{cauchys theorem v1}.

	\begin{corollary}
		Let $G$ be a region and $f$ an analytic function on $G$ with finitely many points $a_1,\ldots,a_k$ with $f(a_i) = \alpha$. If $\gamma$ is a closed rectifiable curve in $G$ that does not pass through any $a_j$ and $\gamma \approx 0$ on $G$, then
		\[ \frac{1}{2\pi\iota} \int_\gamma \frac{f'(z)}{f(z)-\alpha} \dif z = \sum_j n(\gamma;a_j). \]
	\end{corollary}

	The above follows directly on applying the previous proposition to $f-\alpha$.\\

	Recall that if $\gamma:[0,1] \to G$ is rectifiable and $f:G\to\C$ is analytic, then $\sigma = (f\circ\gamma)$ is rectifiable. \\
	Suppose we further have that $\gamma$ is closed and smooth and $\gamma \approx 0$ in $G$. Let $\alpha \in \C \setminus \{\sigma\}$. Then,
	\[ n(\sigma;\alpha) = \frac{1}{2\pi\iota} \int_\sigma \frac{1}{w-\alpha} \dif w = \frac{1}{2\pi\iota} \int_0^1 \frac{f'(\gamma(t))\gamma'(t)}{f(\gamma(t)) - \alpha} \dif t = \frac{1}{2\pi\iota} \int_\gamma \frac{f'(z)}{f(z)-\alpha}, \]
	which we just evaluated above. As might be expected, this is in fact true for any closed rectifiable $\gamma$.
\end{document}