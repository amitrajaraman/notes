\section{Fundamental Results in Extremal Graph Theory}

	% In this section, we describe some fundamental, yet extremely powerful, results in extremal graph theory.

	\subsection{The Erd\H{o}s-Stone-Simonovits Theorem}

		\subsubsection{Motivation}

			% Next, we look at $\exx(n;C_k)$.\\
			% Recall that we have already seen that $\exx(n;C_4) \le c n^{3/2}$ for some universal constant $c$.\\
			Using \href{https://proofwiki.org/wiki/Graph_is_Bipartite_iff_No_Odd_Cycles}{the folklore result} that a graph is bipartite iff it has no odd cycle, we have
			\[ \exx(n;C_{2k+1}) \ge \left\lfloor \frac{n^2}{4} \right\rfloor. \]
			\begin{remark}
				It in fact turns out that for $n\gg 0$, $\exx(n;C_{2k+1}) = \lfloor n^2/4 \rfloor$, as we shall see later..
			\end{remark}

			Is there any more general relationship between forbidden subgraphs and $r$-partite graphs?

			\begin{fdef}[Chromatic Number]
				\label{def: chromatic number}
				Given a graph $G$, its \textbf{chromatic number} is given by
				\[ \chi(G) = \min\{r : G\text{ is $r$-partite}\}. \]
			\end{fdef}

			Alternatively, we can define the above using the following.

			\begin{fdef}
				\label{def: r-coloring}
				Given a graph $G=(V,E)$, an \textbf{$r$-coloring} of $G$ is a function $f : V \to [r]$ such that for any $uv\in E$, $f(u) \ne f(v)$.\\
				The chromatic number of a graph is the least $r$ such that it is $r$-colorable.
			\end{fdef}

			Suppose $H$ is an arbitrary graph such that $\chi(H) = r+1$. Then, no $r$-partite graph contains $H$. As a result,
			\[ \exx(n;H) > t_r(n). \]
			Our earlier observation on $\exx(n;C_{2k+1})$ is then just a consequence of the fact that $C_{2k+1}$ is $3$-colorable.

			Let us give a more concrete example. Suppose we want to find $\exx(n;\mathsf{Petersen})$, where $\mathsf{Petersen}$ is the \href{https://en.wikipedia.org/wiki/Petersen_graph}{Petersen graph}.

			% diagram?

			It may be checked that $\mathsf{Petersen}$ has chromatic number $3$. So,
			\[ \exx(n;\mathsf{Petersen}) > t_2(n) = \left\lfloor \frac{n^2}{4} \right\rfloor. \]

			Can we do better?\\
			This is answered by the Erd\H{o}s-Stone-Simonovits Theorem.

		\subsubsection{The result}

			\begin{ftheo}[Erd\H{o}s-Stone-Simonovits Theorem, Version 1]
				\label{theo: erdos-stone v1}
				Let $r \ge 1$ and $0 < \varepsilon < 1/2$. Then, for $n\gg 0$ and any graph $G_n$, if
				\begin{equation}
					\label{eqn: erdos-stone v1 condition}
					\delta(G_n) \ge \left(1 - \frac{1}{r} + \varepsilon\right)n,
				\end{equation}
				there exist pairwise disjoint subsets $V_1,\ldots,V_{r+1}$ of $V$ such that for each $i$,
				\[ |V_i| = t \ge \frac{\varepsilon \log n}{2^{r-1}(r-1)!} \]
				and the complete $(r+1)$-partite graph on these subsets is contained in $G_n$.
			\end{ftheo}

			\begin{ftheo}[Erd\H{o}s-Stone-Simonovits Theorem, Version 2]
				\label{theo: erdos-stone v2}
				Let $r \ge 1$ and $0 < \varepsilon < 1/2$. Then, for $n\gg 0$ and any graph $G_n$, if
				\begin{equation}
					\label{eqn: erdos-stone v2 condition}
					e(G_n) \ge \left(1 - \frac{1}{r} + \varepsilon\right)\binom{n}{2},
				\end{equation}
				there exist pairwise disjoint subsets $V_1,\ldots,V_{r+1}$ of $V$ such that for each $i$,
				\[ |V_i| = t \ge \frac{\varepsilon \log n}{2^{r+1}(r-1)!} \]
				and the complete $(r+1)$-partite graph on these subsets is contained in $G_n$.
			\end{ftheo}

			Observe the difference in the exponent of $2$ in the denominator of $t$ in the two versions.

			For example, if $n \ge e^{32/\varepsilon}$ and \Cref{eqn: erdos-stone v2 condition} is satisfied, then $T_{3}(12)$, and thus $\mathsf{Petersen}$, is a subgraph of $G_n$.\\
			Therefore, (replacing $\varepsilon$ with $2\varepsilon$)
			\[ \exx(n;\mathsf{Petersen}) \le \left(\frac{1}{2} + 2\varepsilon\right)\binom{n}{2} \le \left(\frac{1}{4} + \varepsilon\right)n^2. \]
			for all $\varepsilon > 0$.

			\begin{fcor}
				\label{theo: ESS exx square tight coloring}
				If $\chi(H) = r+1$, then for any $\varepsilon > 0$, for $n\gg 0$,
				\[ \left(1 - \frac{1}{r}\right)\frac{n^2}{2} \le \exx(n;H) \le \left(1 - \frac{1}{r} + \varepsilon\right)n^2 - \mathcal{O}(n). \]
			\end{fcor}

			This implies that if $\chi(H) = r+1$, then for $n\gg 0$
			\[ \left( 1 - \frac{1}{r} \right)\frac{n^2}{2} \le \exx(n;H) \le \left( 1 - \frac{1}{r} + o(1) \right)\frac{n^2}{2} \]

			% \begin{fdef}
			% 	A graph $G$ is said to be \textbf{$c$-dense} (for $0<c<1$), if $e(G_n) \ge cn^2$.
			% \end{fdef}

			\begin{proof}[Proof of \nameref{theo: erdos-stone v1}]
				We first show the result for $r=1$. Let $G_n$ be a graph such that $\delta(G_n) \ge \varepsilon n$. We want a ``large'' $t$ such that $K_{t,t}$ is isomorphic to a subgraph $G_n$.\\
				Suppose that $G_n$ is $K_{t,t}$-free (we shall fix $t$ later).
				Now, we have that
				\[ t \binom{n}{t} \stackrel{(1)}{>} |\{(v,T) : v\in V, T \subseteq \Gamma(v), |T|=t\}| = \sum_{v\in V} \binom{d(v)}{t} \stackrel{(2)}{\ge} n \binom{\varepsilon n}{t}, \]
				where $(2)$ follows from the hypothesis on $\delta(G_n)$ and $(1)$ follows due to our assumption that $G_n$ is $K_{t,t}$-free.\\
				Therefore,
				\[ n\binom{\varepsilon n}{t} < t \binom{n}{t}. \]
				So,
				\begin{align*}
					1 &< \frac{t}{n} \cdot \frac{n(n-1)\cdot (n-t+1)}{\varepsilon n(\varepsilon n-1)\cdot (\varepsilon n-t+1)} \\
					&< \frac{t}{n} \cdot \left(\frac{n}{\varepsilon n-t+1}\right)^t \\
					&= \frac{t}{n} \frac{1}{\varepsilon^t} \left(\frac{1}{1 - \frac{t-1}{\varepsilon n}}\right)^t.
				\end{align*}
				We desire a large $t$ such that the above results in a contradiction.
				Let us look at the last term. Suppose we want
				\[ \left(1 - \frac{t-1}{\varepsilon n}\right)^t \to 1. \]
				Equivalently,
				\[ t \log\left(1 - \frac{t-1}{\varepsilon n}\right) \to 0 \]
				\[ t \left(\frac{t-1}{\varepsilon n} + \frac{1}{2}\left(\frac{t-1}{2n}\right)^2 + \frac{1}{3} \left(\frac{t-1}{\varepsilon n}\right)^3 + \cdots\right) \to 0. \]
				% \begin{align*}
				% 	t \log\left(1 - \frac{t-1}{\varepsilon n}\right) &\to 0 \\
				% 	t \left(\frac{t-1}{\varepsilon n} + \frac{1}{2}\left(\frac{t-1}{2n}\right)^2 + \frac{1}{3} \left(\frac{t-1}{\varepsilon n}\right)^3\right) &\to 0.
				% \end{align*}
				If $\varepsilon n > t^4$, the above holds (Why?).\\
				That is, if $t < (\varepsilon n)^{1/4}$, then for sufficiently large $n, t$,
				\[ \left(1 - \frac{t-1}{\varepsilon n}\right)^{-t} < 2 \]
				and
				\[ 1 < \frac{t}{n} \cdot \frac{1}{\varepsilon^t} \left(1 - \frac{t-1}{\varepsilon n}\right)^{-t} < \frac{2t}{\varepsilon^t n}. \]
				If the above is $\le 1$, we shall arrive at a contradiction. That is, if
				\[ \log(2t) + t \log\left(\frac{1}{\varepsilon}\right) < \log n, \]
				then we arrive at a contradiction. In particular,
				\[ t = \lceil \varepsilon \log n \rceil \]
				is a suitable choice (Why?), completing the proof of the theorem for $r=1$.\\

				Now, let us prove the general case by performing induction on $r$. Let $G_n$ be a graph with
				\[ \delta(G_n) \ge \left(1 - \frac{1}{r} + \varepsilon\right)n. \]
				Now,
				\[ 1 - \frac{1}{r} + \varepsilon > 1 - \frac{1}{r-1} + \frac{1}{r(r-1)}. \]
				Let $\varepsilon' = 1 / r(r-1)$. By induction, there are $V_1',\ldots,V_r'$ such that 
				\begin{equation}
					\label{eqn: ESS expression for t'}
					|V_i'| \ge t' = \frac{\log n}{r(r-1) 2^{r-2} (r-2)!} = \frac{\log n}{2^{r-2} r!}
				\end{equation}
				for each $i$ and the complete $r$-partite graph on these sets is a subgraph of $G_n$. Let $K = \bigcup V_i'$.\\
				We obviously have $\varepsilon < 1/r$, since the claim is vacuously true otherwise.\\
				We shall find $V_i \subseteq V_i'$ for each $1 \le i \le r$ and some $V_{r+1} \subseteq V \setminus K$ such that the complete bipartite graph on $(V_1,\ldots,V_{r+1})$ is the required subgraph of $G_n$.\\
				Now, define
				\[ U = \left\{x \in V\setminus K : d(x,K) \ge \left(1 - \frac{1}{r} + \lambda\right)|K| \right\} \]
				for some $\lambda$ we shall fix later. We shall bound $e(K,V\setminus K)$ in two different ways. From the perspective of $K$,
				\[ e(K,V\setminus K) \ge |K| \left(\left(1 - \frac{1}{r} + \varepsilon\right)n - |K|\right). \]
				From the perspective of $V\setminus K$,
				\begin{align*}
					e(K, V\setminus K) &= e(V\setminus (K\sqcup U), K) + e(U,K) \\
					&\le \left(n - |U| - |K|\right) \left(1 - \frac{1}{r} + \lambda\right)|K| + |U||K|.
				\end{align*}
				Putting the two together,
				\begin{align*}
					|K| \left(\left(1 - \frac{1}{r} + \varepsilon\right)n - |K|\right) &\le \left(n - |U| - |K|\right) \left(1 - \frac{1}{r} + \lambda\right)|K| + |U||K| \\
					\left(1 - \frac{1}{r} + \varepsilon\right)n - |K| &\le \left(n - |U| - |K|\right) \left(1 - \frac{1}{r} + \lambda\right) + |U|.
				\end{align*}
				Set $\lambda = \varepsilon/2$. We then have
				\[ \frac{r}{2}\varepsilon n \le \left(1 - \frac{r\varepsilon}{2}\right)(|U| + |K|). \]
				% ??????? Am I not getting (1 - r \varepsilon/2)(|U|+|K|).
				So,
				\[ |U| \ge \frac{r\varepsilon}{2 - r\varepsilon} n - |K| \stackrel{(*)}{\ge} \frac{\varepsilon}{1-\varepsilon}n - rt', \]
				where $(*)$ follows from the fact that $r\varepsilon/(2-r\varepsilon)$ is increasing in $r$ and $r\ge 2$.\\
				Since $t'$, and thus $rt'$, is of the order of $\log n$, the first term in the expression dominates for sufficiently large $n$. So, for $n\gg 0$, $|U| \ge \varepsilon n$.\\

				Now,
				\begin{align*}
					\left(1 - \frac{1}{r} + \frac{\varepsilon}{2} \right)|K| &\ge \left(1 - \frac{1}{r} + \frac{\varepsilon}{2} \right) rt' \\
					&\ge (r-1)t' + \frac{\varepsilon r}{2} t'.	
				\end{align*}
				This implies that each $u \in U$ has at least $(\varepsilon r/2)t'$ neighbours in each $V_i'$.\\
				We can now use a pigeonhole argument to choose a subset of $U$ whose vertices are all adjacent to some common set of vertices in each $V_i'$. To do so, consider
				\[ |\{(u, W_1,\ldots,W_r) : W_i \subseteq V_i', |W_i| = (\varepsilon r/2)t',\text{ and $u$ is adjacent to all the vertices of each $W_i$}\}|. \]
				By our earlier observation, this must be at least $|U| \ge \varepsilon n$.\\
				On the other hand, it is at most the number of ways of choosing the $W_i$, which is $\binom{t'}{(\varepsilon r/2)t'}^r$.\\
				In particular, using a pigeonhole argument, there exist $V_1,\ldots,V_r$ and a $V_{r+1}\subseteq U$ such that $V_i \subseteq V_i'$ for each $1\le i\le r$ and for all $u \in V_{r+1}$, $(u,V_1,\ldots,V_r)$ is in the set whose cardinality we just considered, and $|V_{r+1}| \ge \varepsilon n / \binom{t'}{(\varepsilon r/2)t'}^r$. Let us now bound this expression.
				\begin{align*}
					|V_{r+1}| &\ge \dfrac{\varepsilon n}{\dbinom{t'}{(\varepsilon r/2)t'}^r} \\
					&\ge \frac{\varepsilon n}{ (2e/\varepsilon r)^{t'\varepsilon r^2/2}} & \text{(see \href{https://math.stackexchange.com/questions/132625/}{here} for the bound used)} \\
					&\ge \varepsilon n\cdot\left(\frac{\varepsilon}{e}\right)^{t'\varepsilon r^2/2} & (\text{since $r \ge 2$, $2/r \le 1$}).
				\end{align*}
				Setting
				\[ t = \frac{\varepsilon\log n}{2^{r-1}(r-1)!}, \]
				we see that $t = \varepsilon rt'/2$.\\
				Keeping in mind that the bound we want is $|V_{r+1}| \ge t$,
				\begin{align*}
					\log \left(n \left(\frac{\varepsilon}{e}\right)^{t' \varepsilon r^2/2}\right) &\ge \log n + \frac{\varepsilon r^2 t'}{2} \log\left(\frac{\varepsilon}{e}\right) \\
					&\ge t' \left( 2^{r-2} r! - \log\left(\frac{e}{\varepsilon}\right) \cdot \frac{\varepsilon r^2}{2} \right) & \text{(using the expression for $t'$ in \eqref{eqn: ESS expression for t'})} \\
					&\ge t' \left( 2^{r-2} r! - \log(2e) \cdot \frac{r^2}{4} \right) & \text{($-\varepsilon \log(e/\varepsilon)$ is decreasing in $\varepsilon$)} \\
					% &\ge e \log(t') \left(2^{r-2} r! - \log(2e) \cdot \frac{r^2}{4}\right) & \text{$x/\log x \ge e$ for $x>1$} \\
					&\ge \log(rt'/2).
				\end{align*}
				Therefore, $V_{r+1} \ge t$. Since $|V_i| = \varepsilon rt'/2 = t$ by definition, the proof is complete.
			\end{proof}

			Whew. Let us now give a simple corollary of the above result.

			\begin{fpor}
				Suppose $H_1,\ldots,H_m$ are graphs. Then, for any $\varepsilon > 0$ and $n \gg 0$,
				\[ \exx(n;H_1,\ldots,H_m) \le \left(1 - \frac{1}{r} + \varepsilon\right)\frac{n^2}{2} - \mathcal{O}(n^2), \]
				where $r+1=\max\{\chi(H_1),\ldots,\chi(H_m)\}$.
			\end{fpor}


			Before we move onto the proof of the second version of the Erd\H{o}s-Stone-Simonovits Theorem, we give a lemma that will assist in its proof.

			\begin{flem}
				For $n\gg 0$, if $e(G_n) \ge (c+\varepsilon)\binom{n}{2}$ for some $c > 0$ and $\varepsilon > 0$, then there exists $H\subseteq G_n$ such that
				\begin{enumerate}
					\item $|H| \ge \sqrt{\varepsilon} n$.
					\item $\delta(H) \ge c|H|$.
				\end{enumerate}
			\end{flem}

			\begin{proof}
				Assume that $G_n$ itself does not satisfy the conclusions. Then, there exists some $x_n \in V(G_n)$ such that $d(x_n) < cn$.\\
				Let $H_{n-1} = G \setminus \{x_n\}$. If $H_{n-1}$ fails the second conclusion of the theorem, there exists $x_{n-1} \in V(H_{n-1})$ such that $d(x_{n-1}) < c (n-1)$.\\
				Repeating the above, we get a sequence of graphs $G_n = H_n \supsetneq H_{n-1} \supsetneq \cdots \supsetneq H_{\ell}$, where $V(H_{n-r}) \setminus V(H_{n-r-1}) = \{x_{n-r}\}$ for each $r$ and $\ell \ge \sqrt{\varepsilon} n$. Further, for each $i$, $d_{H_i}(x_i) < c i$ for $i = \ell,\ldots,n$.\\
				Now,
				\begin{align*}
					e(H_\ell) &> e(H_n) - \left(cn + c(n-1) + \cdots + c(\ell+1)\right) \\
					&= (c+\varepsilon)\binom{n}{2} - c \left(\frac{n(n+1)}{2} - \frac{\ell(\ell+1)}{2}\right) \\
					&= (c+\varepsilon)\binom{n}{2} - c \left(\binom{n}{2} + n - \binom{\ell+1}{2}\right) \\
					&= \varepsilon\binom{n}{2} - cn + c \binom{\ell+1}{2}.
				\end{align*}
				Now, the final expression must be at most $\binom{\ell}{2}$ (since $H_\ell$ has $\ell$ vertices).\\
				As a result, it would suffice to show that the above expression on taking $\ell = \lfloor \varepsilon n \rfloor$ is greater than $\binom{\ell}{2}$ for sufficiently large $n$ (this implies that the sequence must stop before reaching this $\ell$ due to one of the graphs satisfying the conclusions). Indeed, this is seen to be true as
				\begin{align*}
					\varepsilon\binom{n}{2} - cn + c\binom{\ell+1}{2} &\ge \varepsilon\binom{n}{2} + c\left(\frac{(\lfloor\sqrt{\varepsilon}n\rfloor+1)\lfloor\sqrt{\varepsilon}n\rfloor}{2} - n\right) \\
					&\ge \varepsilon\binom{n}{2} & (\text{for sufficiently large $n$}) \\
					&\ge \binom{\ell}{2},
				\end{align*}
				completing the proof.
			\end{proof}

			% In particular, taking $c = 1 - \frac{1}{r} + \frac{\delta}{2}$ reduces v2 to v1.

			% \begin{proof}[Proof of \nameref{theo: erdos-stone v2}]
			% 	We shall reduce this Theorem to an instance of \nameref{theo: erdos-stone v1}. Let $G_n$
			% \end{proof}


			% \begin{flem}
			% 	Let $d$ be the average degree of $G$. Then $G$ contains a (non-empty) subgraph $H$ with $\delta(H) \ge d/2$.
			% \end{flem}

	\subsection{An Introduction to Random Graphs}

		\begin{fdef}[Erd\H{o}s-R\'{e}nyi Model]
			Fix $0\le p\le 1$. The \textbf{Erd\H{o}s-R\'{e}nyi random graph model}, denoted $G_{n,p}$ is the random variable which is a graph with vertex set $[n]$, such that for each $\{i,j\} \in \binom{[n]}{2}$, $\{i,j\}$ is an edge with probability $p$, independently across distinct pairs.
		\end{fdef}

		% That is, $G_{n,p}$ is a distribution on all graphs with vertex set $[n]$. 
		So, for any graph $H$ on vertex set $[n]$,
		\[ \Pr(G_{n,p} = H) = p^{e(H)} (1-p)^{\binom{n}{2} - e(H)}. \]

		\begin{remark}
			$G_{n,1/2}$ is the uniform distribution on the set of graphs on $[n]$.
		\end{remark}

		A recurring theme in probability theory is that random objects tend to behave very nicely given a large number of samples (along the lines of the laws of large numbers and the central limit theorem).

		\subsubsection{A motivating extremal problem (bounding \texorpdfstring{$\exx(n;C_{2k}$)}{ex(n;C2k)})}

			To understand why random graphs are important, let us look at $\exx(n;C_{2k})$.
			% We have already seen that $\exx(n;C_{2k+1}) \ge \lfloor n^2/4 \rfloor$. Is it true that $\exx(n;C_{2k+1}) = \lfloor n^2 / 4 \rfloor$ for $n \gg 0$?\\
			% While \Cref{theo: ESS exx square tight coloring} does say that $\exx(n;C_{2k+1}) \le (1/4 + o(1)) n^2$, it \emph{could} still be as large as $n^2/4 + n^{1.99}$, say.\\
			% It turns out, however, that our claim happens to be true, as we shall see later.\\

			% Before moving to this however, let us look at $\exx(n;C_{2k})$, and at the very least get a lower bound on this quantity.

			\begin{ftheo}
				\label{theo: C_2k free}
				For any $k$, there exists a constant $c$ such that for $n \gg 0$, there is a $C_k$-free $G_n$ with
				\[ e(G_n) \geq c \cdot n^{1 + 1/(k-1)}. \]
			\end{ftheo}

			The above result does not yield anything useful for $k$ odd.

			\begin{proof}
				Consider $G_{n,p}$ for some $p$ we shall fix later. Let $N(G)$ be the number of copies of $C_k$ in a given graph $G$.\\
				Given a cycle $(v_1,\ldots,v_k)$, observe that the sequences $(v_2,v_3,\ldots,v_k, v_1)$ and $(v_k,v_{k-1},\ldots,v_1)$ determine the same cycle. That is, performing cyclic shifts of a sequence of vertices or reversing their order around gives the same cycle.\\
				Let $\mathcal{C}$ be the set of all these cycles.\footnote{This can be made more formal by taking all length $k$ sequences of $[n]$ consisting of distinct elements and considering the equivalence classes formed by the equivalence relation defined on the previous line.} By our observation,
				\[ |\mathcal{C}| = \frac{n(n-1)\cdots(n-k+1)}{2k} = \frac{n!}{2k\cdot k!}. \]
				Now,
				\begin{align*}
					\expec[N(G_{n,p})] &= \expec\left[ \sum_{(v_1,\ldots,v_k) \in \mathcal{C}} \indic_{v_1v_2, v_2v_3, \ldots, v_{k-1}v_k, v_kv_1\text{ are edges}} \right] \\ 
					&= \sum_{(v_1,\ldots,v_k) \in \mathcal{C}} \Pr\left[ \indic_{v_1v_2, v_2v_3, \ldots, v_{k-1}v_k, v_kv_1\text{ are edges}} \right] & \text{(linearity of expectation)} \\
					&= \sum_{(v_1,\ldots,v_k) \in \mathcal{C}} p^k \\
					&= \frac{n!}{2k \cdot k!} p^k.
				\end{align*}
				This quantity is obviously less than $(np)^k / 2k$. If $p$, and thus the expectation is small, we expect to not see many $C_k$s. On the other hand,
				\begin{align*}
					\expec[e(G_{n,p})] &= \expec \left[ \sum_{\{i,j\} \in \binom{[n]}{2}} \indic_{ij\text{ is an edge}} \right] \\
					&= \binom{n}{2} p.
				\end{align*}
				Given a graph, if we delete an (arbitrary) edge from each copy of $C_k$ in it, we will be left with no cycles. That is, given any graph $G$, there is a graph on the same vertex set with $e(G) - N(G)$ edges that is $C_k$-free. Inspired by this, by the linearity of expectation,
				\[ \expec[e(G_{n,p}) - N(G_{n,p})] \ge \binom{n}{2} p - \frac{(np)^k}{2k}. \]
				If we set $p = \left(\frac{k}{2}\right)^{1/(k-1)} n^{-1 + 1/(k-1)}$, then the above quantity is at least $n(n-1)p/4 \ge c \cdot n^{1 + 1/(k-1)}$ for an appropriate constant $c$ and $n\gg 0$, completing the proof.
			\end{proof}

		\subsubsection{Digression: A coloring result of Erd\H{o}s}

			The question we consider in this section is:

			\begin{quote}
				Are there $C_3$-free graphs with large chromatic number?
			\end{quote}

			More generally,

			\begin{quote}
				Are there graphs with large chromatic number and large girth?
			\end{quote}

			For example, if the girth of a graph is $7$, then there cannot be adjacent vertices $v,w$ such that $\Gamma(v) \cap \Gamma(u)$ and $\Gamma(w) \cap \Gamma(u)$ are non-empty for some $u$ distinct from $v,w$. As a result, we can draw a ``$2$-step tree'' rooted at any $u$, which has $\Gamma(u)$ at the first level and the neighbours (other than $u$) of vertices of $\Gamma(u)$ at the second.\\
			This seems to suggest some level of sparseness in the graph, due to which there are not too many edges and as a result, the chromatic number is low. However, it turns out that this intuition is not true, as proved by Erd\H{o}s in \cite{erdos_1959_girth_chromaticnum}.

			\begin{ftheo}
				There exist graphs with arbitrarily large girth and chromatic number. That is, given a $g,k \ge 3$, there exists a graph $G_n$ such that $\girth(G) > g$ and $\chi(G) > k$.
			\end{ftheo}

			\begin{proof}
				Consider $G_{n,p}$ for some $p$ we fix later. Further assume that $np \ge 1$.\\% (if $p < 1/n$, the graph is extremely sparse).\\
				Given a graph $G$, let $N_i(G)$ (for $3 \le i \le g$) be the number of cycles of size $i$ in $G$. As we saw in the proof of \Cref{theo: C_2k free},
				\[ \expec[N_i(G_{n,p})] = \frac{n!}{2i \cdot i!} p^i < \frac{(np)^i}{6}. \]
				Let $N = \sum_{i=3}^g N_i$. Then
				\begin{align*}
					\expec\left[N(G_{n,p})\right] &= \expec\left[\sum_{i=3}^g N_i(G_{n,p})\right] \\
					&< \frac{(np)^3}{6} \left(\frac{(np)^{g-2} - 1}{np - 1}\right) \\
					&< \frac{(np)^g}{3}. & \text{(since $np\ge 1$)}
				\end{align*}
				Using \href{https://en.wikipedia.org/wiki/Markov%27s_inequality}{Markov's inequality},
				\begin{equation}
					\label{eqn: 2.5}
					\Pr\left[ N(G_{n,p}) > \frac{2}{3} (np)^g \right] < \frac{1}{2} \\
				\end{equation}
				This takes care of the girth (we want the above probability to be small).
				On the other hand, we need to make the chromatic number large. Towards this, observe that $\chi(G) \ge n/\alpha(G)$ (Why?). % Try thinking of a colouring as a splitting of the graph into independent sets).
				We have
				\begin{align*}
					\Pr\left[\alpha(G_{n,p}) \ge r\right] &= \Pr\left[ \bigcup_{X\subseteq [n]: |X|=r} \{X\text{ is independent}\} \right] \\
					&\le \sum_{X \subseteq [n] : |X|=r} \Pr[X\text{ is independent}] & \\% (\text{by the \href{https://en.wikipedia.org/wiki/Boole%27s_inequality}{union bound}}) \\
					&= \sum_{X \subseteq [n] : |X|=r} (1-p)^{\binom{r}{2}} \\
					&\le \binom{n}{r} \cdot e^{-pr(r-1)/2} \\
					&\le \left(\frac{en}{r}\right)^r \cdot e^{-pr^2/3} \\
					&= \left(\frac{e^{1-pr/3}n}{r}\right)^r.
				\end{align*}
				We shall choose $r$ and $p$ such that with positive probability, $N(G_{n,p}) \le 2(np)^g/3$ \emph{and} $\alpha(G_{n,p}) < r$. This implies the existence of a graph $G_n$ such that both of the above hold.\\
				We cannot use the tactic of removing edges we did in the earlier proof since that might increase $\alpha$. Deleting vertices on the other hand works, since this can increase neither $\alpha$ nor $N$.\\
				If we delete a single vertex from each cycle involved in $N$, the resulting graph will have girth greater than $g$. That is, given a graph $G_n$, there exists a graph with at least $n - N(G_n)$ vertices that has girth greater than $g$. Denote this corresponding graph as $G_n'$.\\

				Set $p = n^{1/(g+1) - 1}$. In this case, $2(np)^g/3 < n/2$ for $n\gg 0$ and using \eqref{eqn: 2.5},
				\begin{equation}
					\label{eqn: 2.6}
					\Pr\left[ n - N(G_{n,p}) > \frac{n}{2} \right] \ge \frac{1}{2}.
				\end{equation}
				% On the other hand, we want $\Pr[\alpha(G_{n,p}) \ge r]$ to be small. This is accomplished if
				% \begin{align*}
				% 	en &\ll e^{pr/3} \\
				% 	\log n &\ll \frac{pr}{3}.
				% \end{align*}
				Set $r = 4 \log n / p = 4 n^{1 - 1/(g+1)} \log n$. For these values of $p$ and $r$,
				\begin{equation}
					\label{eqn: 2.7}
					\Pr\left[\alpha(G_{n,p}) \ge r\right] \xrightarrow{n\to\infty} 0.
				\end{equation}

				Using \eqref{eqn: 2.6}, our construction of $G'$, and \eqref{eqn: 2.7}, it is true with with positive probability that
				\[ |G_{n,p}'| \ge \frac{n}{2}, \girth(G_{n,p}') > g, \]
				and
				\[ \chi(G_{n,p}') \ge \frac{n}{\alpha(G_{n,p}')} \ge \frac{n}{r} = \frac{n^{1/(g+1)}}{4\log n} \xrightarrow{n\to\infty} \infty. \]
				Therefore, for $g,k \ge 3$, there exists $n\gg 0$ and graph $G$ on $n$ vertices such that $\girth(G) > g$ and $\chi(G) > k$. 
			\end{proof}

	\subsection{Szemer\'{e}di's Regularity Lemma}

		The second of our powerful results in extremal graph theory (after the Erd\H{o}s-Stone-Simonovits Theorem) is Szemer\'{e}di's Regularity Lemma, which says that any sufficiently large graph behaves in some way like a random graph.

		\subsubsection{Motivation}

			First, let us give a bound from probability theory that will be useful.

			\begin{flem}[Chernoff Bound]
				\label{chernoff bound}
				Suppose $X \sim B(n,p)$, the \href{https://en.wikipedia.org/wiki/Binomial_distribution}{binomial distribution} with parameters $n,p$. Then for any $t \ge 0$,
				\[ \Pr\left[ X - \expec[X] \ge t \right] \le \exp\left(-\frac{t^2}{2\left(\expec[X] + t/3\right)}\right) \]
				and
				\[ \Pr\left[ X - \expec[X] \le -t \right] \le \exp\left(-\frac{t^2}{2\expec[X]}\right). \]
			\end{flem}

			Fix disjoint $A, B \subseteq [n]$ and let $|A|=a$, $|B|=b$. Then given a graph $G$ on $[n]$,
			\[ e(A,B) = \sum_{x \in A, y \in B} \indic_{xy \in E(G)}. \]
			Fix $0<p<1$. Then if $G \sim G_{n,p}$,
			\[ e(A,B) \sim B\left( ab, p \right). \]
			By the \nameref{chernoff bound}, for some fixed constant $c$,
			\[ \Pr\left[ |e(A,B) - pab| > c \left(b\sqrt{pa\log\left(\frac{2n}{b}\right)}\right) \right] \xrightarrow{n\to\infty} 0. \]

			In particular, if $a = b = \alpha n$ for some $0 < \alpha < 1/3$, then with high probability,
			\[ |e(A,B) - pab| = \mathcal{O}\left(b\sqrt{ap\log\left(\frac{2n}{b}\right)}\right) \]
			for any sets $A, B$ of sizes $a$ and $b$ respectively.

			This seems to say that the actual number of edges between two sets of the given size does not deviate very much from the expected number of edges between the two sets. The expression on the right is of the order of $\mathcal{O}(n\sqrt{n})$, which is asymptotically less than the expectation $pab = \mathcal{O}(n^2)$.

		\subsubsection{The Result}

			The regularity lemma gives a qualitative version of the above observation. Before we move to the actual result, let us provide some notation.

			\begin{fdef}[Density]
				Given a graph $G = (V,E)$ and $U,W \subseteq V$, the \textbf{density} $d(U,W)$ is equal to $e(U,W) / |U||W|$.
			\end{fdef}

			\begin{remark}
				Here, $e(U,W)$ is $\{(u,w) \in U \times W : \{u,w\} \in E\}$. If $U$ and $W$ are disjoint, this is the same as our earlier definition of $e(\cdot,\cdot)$. If they are not disjoint however, edges within the intersection are counted \emph{twice} in our current definition.\\
				This does not matter all that much since we usually apply the regularity lemma on disjoint sets.
			\end{remark}

			\begin{fdef}[$\varepsilon$-regular pair]
				Suppose $0 < \varepsilon < 1$. A pair of subsets $(U,W)$ is said to be \textbf{$\varepsilon$-regular} if for any $A \subseteq U$, $B\subseteq W$ with $|A| \ge \varepsilon |U|$ and $|B| \ge \varepsilon |W|$, we have
				\[ |d(A,B) - d(U,W)| \le \varepsilon \]
			\end{fdef}

			This corresponds to some sort of uniform behaviour throughout the sets, where subsets behave similarly to their parent sets in terms of density. If $(U,W)$ is $\varepsilon$-regular, all sufficiently large subsets of $U,W$ have roughly the same edge density as $(U,W)$.

			\begin{ftheo}[Szemer\'{e}di's Regularity Lemma]
				\label{theo: szemeredi's regularity lemma}
				Given $0 < \varepsilon < 1$, there exists $M$ such that for $n\gg 0$, any graph $G_n$ admits a vertex partition $\mathcal{P} = (V_0, V_1, \ldots, V_k)$, where
				\begin{itemize}
					\item $k \le M$,
					\item $|V_0| \le \varepsilon n$ ($V_0$ is known as an ``exceptional set''),
					\item all the $V_i$ for $1\le i\le k$ are of equal size, and
					\item the number of $\varepsilon$-\emph{ir}regular pairs $(V_i,V_j)$ ($1 \le i,j\le k$) is at most $\varepsilon k^2$.
				\end{itemize}
			\end{ftheo}
			Such a partition where the number of $\varepsilon$-regular pairs is at most $\varepsilon k^2$ is often referred to as an \textbf{$\varepsilon$-regular partition}.\\
			% Qualitiatively, this means that any graph can be split into a bunch of pieces such that the edges between them are almost random, and the non-random part is small.\\

			We present the proof of the above, that uses an ``energy increment'' argument, over a series of lemmas.\\

			\begin{fdef}[Energy]
				Given a graph $G_n$ with vertex set $V$, for disjoint $U,W \subseteq V$, define the \textbf{energy} of the pair $(U,W)$ by
				\[ q(U,W) = \frac{|U||W|}{n^2} d^2(U,W). \]
				If $\mathcal{U} = \{U_1,\ldots,U_m\}$ and $\mathcal{W} = \{W_1,\ldots,W_\ell\}$ are partitions of $U$ and $W$ respectively, then the energy of the pair $(\mathcal{U},\mathcal{W})$ is
				\[ q(\mathcal{U},\mathcal{W}) = \sum_{\substack{1\le i\le m \\ 1 \le j \le \ell}} q(U_i,W_j). \]
				For a partition $\mathcal{P}$ of $V$, let
				\[ q(\mathcal{P}) = q(\mathcal{P},\mathcal{P}) = \sum_{U,W\in\mathcal{P}} q(U,W) . \]
			\end{fdef}

			\begin{lemma}
				\label{lemma: szemeredi lemma energy increases on partitioning}
				If $U,W$ are disjoint subsets of $V$ and $\mathcal{U},\mathcal{W}$ are partitions of $U,W$ respectively, then
				\[ q(\mathcal{U},\mathcal{W}) \ge q(U,W). \]
			\end{lemma}

			\begin{proof}
				Independently pick $u,w$ uniformly randomly from $V$. Define the random variable
				\[ Z(u,w) =
				\begin{cases}
					d(U', W'), & u \in U' \in \mathcal{U}, w \in W' \in \mathcal{W}, \\
					0, & \text{otherwise.}
				\end{cases}
				\]
				Then,
				\begin{align*}
					\expec[Z]^2 &= \left(\sum_{\substack{U' \in \mathcal{U} \\ W' \in \mathcal{W}}} d(U',W') \left(\frac{|U'||W'|}{|U||W|}\right)\right)^2 \\
						&= \left(\frac{1}{|U||W|} \sum_{\substack{U' \in \mathcal{U} \\ W' \in \mathcal{W}}} e(U',W')\right)^2 \\
						&= d(U,W)^2 \\
						&= \frac{n^2}{|U||W|} q(U,W) \text{ and} \\
					\expec[Z^2] &= \sum_{\substack{U' \in \mathcal{U} \\ W' \in \mathcal{W}}} d^2(U',W') \left(\frac{|U'||W'|}{|U||W|}\right) \\
						&= \frac{n^2}{|U||W|} q(\mathcal{U},\mathcal{W}).
				\end{align*}
				Since variance is always non-negative, $\expec[Z^2] \ge \expec[Z]^2$ and thus,
				\[ q(\mathcal{U},\mathcal{W}) \ge q(U,W). \qedhere \]
			\end{proof}

			Partitioning increases energy! Also observe that since the random variable $Z$ is at most $1$ ($1/2$, in fact), so is the energy.\\
			This also implies that if the partition $\mathcal{Q}$ is a refinement of $\mathcal{P}$, then
			\[ q(\mathcal{Q}) \ge q(\mathcal{P}). \]

			\begin{lemma}
				\label{lemma: szemeredi lemma energy increases for irregular}
				If $(U,W)$ is not $\varepsilon$-regular, there is a partition $\mathcal{U} = \{U_1,U_2\}$ and $\mathcal{W} = \{W_1,W_2\}$ of $U,W$ respectively such that
				\[ q(\mathcal{U},\mathcal{W}) > q(U,W) + \varepsilon^4 \frac{|U||W|}{n^2}. \]
			\end{lemma}
			\begin{proof}
				By the $\varepsilon$-irregularity, there exist $U_1\subseteq U$ and $W_1\subseteq W$ with $|U_1| \ge \varepsilon|U|$, $|W_1| \ge \varepsilon|W|$, and
				\[ |d(U_1,W_1) - d(U,W)| > \varepsilon. \]
				Consider the partitions $\mathcal{U} = \{U_1,U\setminus U_1\}$ and $\mathcal{W} = \{W_1,W\setminus W_1\}$. With the same $Z$ as in the proof of the previous lemma,
				\begin{align*}
					\expec[Z^2] - \expec[Z]^2 &= \expec[(Z - \expec[Z])^2] \\
						&\ge \expec[(Z - \expec[Z])^2 \indic_{u\in U', w\in W'}] \\
						&= \varepsilon^4.
				\end{align*}
				The claim follows.
			\end{proof}

			\begin{flem}
				\label{lemma: final step of szemeredi's regularity lemma}
				Suppose $0 < \varepsilon < 1/4$ and let $\mathcal{P} = (V_0,V_1,\ldots,V_k)$ be a partition of $V$ such that
				\begin{itemize}
				 	\item $\mathcal{P} \setminus \{V_0\}$ is not $\varepsilon$-regular (there are at least $\varepsilon k^2$ irregular pairs),
				 	\item $\mathcal{P} \setminus \{V_0\}$ is equitable\footnotemark, and
				 	\item $|V_0| \le \varepsilon n$.
				\end{itemize}
				Then, there is a refinement $\mathcal{Q} = (V_0',V_1',\ldots,V_\ell')$ of $\mathcal{P}$ such that
				\begin{itemize}
					\item $\ell \le k 4^k$,
					\item $\mathcal{Q} \setminus \{V_0'\}$ is equitable,
					\item $|V_0'| \le |V_0| + n / 2^k$, and
					\item $q(\mathcal{Q}) \ge q(\mathcal{P}) + \varepsilon^5/2$.
				\end{itemize}
			\end{flem}
			\footnotetext{All the blocks of the partition are of equal size.}

			\begin{proof}
				Let $|V_i| = t$ for all $1\le i\le k$.\\
				Suppose $(V_i,V_j)$ is an $\varepsilon$-irregular pair in $\mathcal{P}$ and let $(V_{i,1},V_{i,2})$ and $(V_{j,1},V_{j,2})$ be the partitions of $V_i,V_j$ respectively described in \Cref{lemma: szemeredi lemma energy increases for irregular}. If $\mathcal{Q}_1$ is this particular refinement
				\[ \mathcal{P} \cup \{V_{i,1},V_{i,2},V_{j,1},V_{j,2}\} \setminus \{V_i,V_j\} \]
				of $\mathcal{P}$, then
				\[ q(\mathcal{Q}_1) \ge q(\mathcal{P}) + \varepsilon^4 \frac{|V_i||V_j|}{n^2} \]
				Let us similarly produce refinements corresponding to all irregular pairs of $\mathcal{P}$. Let $\mathcal{Q}'$ be the ``common'' refinement of all these partitions $(\mathcal{Q}_i)$. That is, for each $v \in V$, $v$ is placed in the subset $\bigcap_{W \in \mathcal{Q}_i : v \in W} W$.\\
				We then have
				\begin{align}
					q(\mathcal{Q}') &\ge q(\mathcal{P}) + \varepsilon^4 \cdot \frac{t^2}{n^2} (\varepsilon k^2) \nonumber \\
					&= q(\mathcal{P}) + \varepsilon^5 \frac{(tk)^2}{n^2} \nonumber \\
					&\ge q(\mathcal{P}) + \frac{\varepsilon^5}{2}. & \text{($tk = n - |V_0| \ge (1-\varepsilon)n \ge 3n/4$ and $9/16 \ge 1/2$)} \label{eqn: 2.7}
				\end{align}
				To make the partition $\mathcal{Q}'$ equitable, create the partition $\mathcal{Q}$ using it as follows.\\
				Suppose we partition $V_i$ into $(V_{ij})$, where $V_{ij}$ is formed due to the irregularity of $(V_i,V_j)$. Partition each of these parts of $\mathcal{Q}'$ into sets of size $b\coloneqq \lfloor t/4^k\rfloor$. Whatever residual part cannot be cut out in this manner, we merge with $V_0$.\\
				Since partitioning can only increase energy, this operation will only strengthen \eqref{eqn: 2.7}, if anything.\\

				Since the size of any block of $\mathcal{Q}$ is $b$, the number of blocks is at most $n/b = k 4^k$.\\
				Finally, what is the size of $|V_0'|$?\\
				$\mathcal{Q}'$ has at most $k\cdot 2^{k-1}$ parts (there are at most $(k-1)$ $2$-part partitions of each block of $\mathcal{P}$, and together they give a partition of size at most $2^{k-1}$). So,\\
				\begin{align*}
					|V_0'| &\le |V_0| + |\mathcal{Q}'| b \\
					&\le |V_0| + k 2^{k-1} \frac{t}{4^k} \\
					&\le |V_0| + \frac{n}{2^k},
				\end{align*}
				as desired.
			\end{proof}

			Observe that with the above lemma, \nameref{theo: szemeredi's regularity lemma} follows without too much difficulty.\\
			Begin with a partition $\mathcal{P}_0$ with $k_0$ parts, where $2^{k_0} \ge 2/\varepsilon$.\\
			Given $\mathcal{P}_k$, let $\mathcal{P}_{k+1}$ be the partition defined by \Cref{lemma: final step of szemeredi's regularity lemma}. Since the energy of any partition is bounded above by $1$, this process must terminate after at most $2/\varepsilon^5$ steps.\\
			Further, since the size of the refined partition is bounded between quantities dependent solely on the old partition, the size of the final partition after termination of the above process is at most some quantity dependent only on $\varepsilon$ (this quantity might be massive\footnote{Unbelievably so. For example, $\varepsilon=1/8$ gives a bound of the order of $4\uparrow\uparrow 2^{15}$ (using \href{https://en.wikipedia.org/wiki/Knuths_up-arrow_notation}{Knuth's up-arrow notation}). It further turns out that such a massive tetration-type bound is necessary, as proved in \cite{Gowers1997}.}, but that is besides the point).\\

	\subsection{Some Corollaries of Szemer\'{e}di's Regularity Lemma}

		In this section, we cover several interesting corollaries of \nameref{theo: szemeredi's regularity lemma}.\\

		Before we begin however, how do we process the regularity lemma? We typically begin with a ``cleaning'' of the graph as follows.
		\begin{itemize}
			\item Given $\varepsilon > 0$, get a partition as described in the lemma.
			\item First, delete all edges between irregular pairs. This loses at most $(\varepsilon k^2) t^2 < \varepsilon n^2$ edges.
			\item Delete all edges between ``sparse'' pairs, where we say that a pair $(V_i,V_j)$ is sparse if $d(V_i,V_j) < \varepsilon$ (say). This loses at most $\binom{k}{2} (\varepsilon t^2) < (\varepsilon/2)n^2$ edges.
			\item Delete all edges inside the exceptional part $V_0$. This loses at most $(\varepsilon^2/2)n^2$ edges. 
		\end{itemize}
		All of these deletions cost at most $2\varepsilon n^2$ edges. In the remaining graph, \emph{all} pairs $(V_i,V_j)$ for $1 \le i,j\le k$ are $\varepsilon$-regular. Further, if $e(V_i,V_j) > 0$ (for $i\ne j$), then $d(V_i,V_j) \ge \varepsilon$.\\

		\subsubsection{The graph counting and removal lemmas}

			\begin{ftheo}[Triangle Counting Lemma]
				\label{theo: triangle counting lemma}
				Suppose $V_1,V_2,V_3$ forms a pairwise $\varepsilon$-regular partition of the vertex set of graph $G_n$, and that $d(V_i,V_j) \ge d$ (for some $d\ge 2\varepsilon$). Then, there are at least 
				\[ \varepsilon^2(1-2\varepsilon)(d-\varepsilon)|V_1||V_2||V_3| \]
				triangles $xyz$ with $x\in V_1$, $y\in V_2$, $z\in V_3$. 
			\end{ftheo}
			\begin{proof}
				Let
				\[ X_1 = \{x\in V_1 : d(x,V_2) \le \varepsilon |V_2|\}. \]
				We claim that $|X_1| < \varepsilon|V_1|$. Suppose otherwise. Then $e(X_1,V_2) / |X_1||V_2| < \varepsilon$ and further, $\varepsilon$-regularity implies that $d(X_1,V_2) \ge (d-\varepsilon)$, which leads to a contradiction.\\
				Similarly, $X_2 \coloneqq \{x\in V_1 : d(x,V_3) \le \varepsilon|V_3|\}$ has size less than $\varepsilon|V_1|$ too.\\
				For all $x\in V_1 \setminus (X_1 \cup X_2)$, $d(x,V_2) \ge \varepsilon|V_2|$ and $d(x,V_3) \ge \varepsilon|V_3|$. Fix such an $x$ and let $X_i' = \Gamma(x)\cap V_i$ (for $i=2,3$).\\
				By the $\varepsilon$-regularity of $(V_2,V_3)$, $d(V_2,V_3) - d(X_2',X_3') \le \varepsilon$ so $d(X_2',X_3') \ge d-\varepsilon$. As a result,
				\[ e(X_2',X_3') \ge (d-\varepsilon)|X_2'||X_3'| \ge (d-\varepsilon)\varepsilon^2|V_2||V_3|. \]
				Therefore, the number of triangles of the desired form is at least
				\[ \underbrace{(1-2\varepsilon)|V_1|}_{\substack{\text{the number of} \\ \text{such $x$}}}\cdot \underbrace{(d-\varepsilon)\varepsilon^2|V_2||V_3|}_{\substack{\text{the number of triangles} \\ \text{corresponding to each $x$}}}, \]
				completing the proof.
			\end{proof}

			Along similar lines is the following result.

			\begin{ftheo}[Triangle Removal Lemma]
				\label{theo: triangle removal lemma}
				Given $\varepsilon>0$, there exists $\delta$ (depending only on $\varepsilon$) such that for $n\gg 0$, any graph $G_n$ with at most $\delta n^3$ triangles can be made triangle-free by deleting at most $\varepsilon n^2$ edges.
			\end{ftheo}
			\begin{proof}
				Start with an $(\varepsilon/4)$ regular partition of $G_n$ using Szemer\'{e}di's Regularity Lemma and the cleaning process from earlier. Delete all edges within each of the $V_i$. This costs at most $k (t^2/2) < (kt)^2/2k \le \varepsilon n^2/4$ edges.\\
				So, we have lost $\le \varepsilon n^2$ edges in all. If there is a triangle remaining in the graph, it must come from a triple $(V_i,V_j,V_k)$ with all three pairs being $(\varepsilon/4)$-regular and density at least $\varepsilon/2$. We can then use the triangle counting lemma to conclude that there are at least $t^3 (\varepsilon/4)^3 (1 - \varepsilon/2) > n^3 (\varepsilon/8)^3 / M(\varepsilon)^3$. Letting $\delta$ to be the $1/6$ this quantity, we are done.
			\end{proof}

			Let $(V_1,V_2)$, $(V_2,V_3)$, and $(V_3,V_1)$ all be $\varepsilon$-regular pairs with densities $d_{12}, d_{23}, d_{13}>2\varepsilon$. A random graph on $V_1\sqcup V_2\sqcup V_3$ with edge probabilities $d_{ij}$ between $V_i$ and $V_j$ has an expected number of triangles of $|V_1||V_2||V_3| d_{12}d_{23}d_{31}$.\\
			Ideally, we would have a result that the number of triangles is indeed to this quantity (so the graph behaves almost randomly, in a sense similar to that in Szemer\'{e}di's Regularity Lemma).

			\begin{ftheo}[Graph Counting Lemma]
				\label{theo: graph counting lemma}
				Let $G$ be a graph on $n$ vertices and $H$ a graph on $[k]$. Let $V_1,\ldots,V_k\subseteq V(G)$ such that $(V_i,V_j)$ is $\varepsilon$-regular whenever $ij\in E(H)$. Then,
				\[ \left|\{(v_1,\ldots,v_k) : v_i \in V_i \text{ and } \{v_1,\ldots,v_k\} \text{ form a copy of $H$ in $G$}\}\right| \]
				is within $\varepsilon e(H)|V_1||V_2|\cdots|V_k|$ of
				\[ \prod_{i=1}^k |V_i| \prod_{ij\in E(H)} d(V_i,V_j),  \]
				assuming that $\prod_{ij\in E(H)} d(V_i,V_j) > \varepsilon e(H)$.
			\end{ftheo}

			\begin{proof}
				We shall prove this by inducting on $e(H)$. If $e(H)=0$, the result is trivial.\\
				Let us rephrase the problem probabilistically. Pick $v_i\in V_i$ independently and uniformly. Then, we wish to prove that
				\begin{equation}
					\label{eqn: 2.8}
					\left|\Pr\left[ v_iv_j\in E(G) \text{ for all } ij\in E(H) \right] - \prod_{ij\in E(H)} d(V_i,V_j)\right| \le \varepsilon e(H).
				\end{equation}
				Suppose $\{1,2\}\in E(H)$. It suffices to show that
				\begin{equation}
					\label{eqn: 2.9}
					\left| \Pr\left[ v_iv_j \in E(G) \text{ for all }ij\in E(H) \right] - d(V_1,V_2) \Pr\left[ v_iv_j\in E(G) \text{ for all }E(H)\setminus\{1,2\} \right] \right| \le \varepsilon.
				\end{equation}
				Indeed, by induction,
				\begin{equation}
					\label{eqn: 2.10}
					\left| \Pr\left[ v_iv_j\in E(G) \text{ for all }E(H)\setminus\{1,2\} \right] - \prod_{\substack{ij\in H \\ ij\neq\{1,2\}}} d(V_i,V_j) \right| < \varepsilon(e(H) - 1),
				\end{equation}
				and \eqref{eqn: 2.8} follows from \eqref{eqn: 2.9} and \eqref{eqn: 2.10} on using the \href{https://en.wikipedia.org/wiki/Boole%27s_inequality}{union bound}. We shall prove that \eqref{eqn: 2.9} holds when we condition on the choices $v_i$ for $i>2$, and thus generally. Let
				\[ A_j = \{ v_j \in V_j : \{v_j,v_i\} \in E(G) \text{ when } \{j,i\} \}\in E(H). \]
				for $j = 1,2$. \Cref{eqn: 2.9} is then equivalent to
				\begin{equation}
					\label{eqn: 2.11}
					\left| \frac{e(A_1,A_2)}{|V_1||V_2|} - d(V_1,V_2) \frac{|A_1||A_2|}{|V_1||V_2|} \right| \le \varepsilon.
				\end{equation}
				We claim that \eqref{eqn: 2.11} holds for all $A_1,A_2$. If $|A_1| \ge \varepsilon|V_1|$ and $|A_2|\ge\varepsilon|V_2|$, then $\varepsilon$-regularity implies that $|d(A_1,A_2) - d(V_1,V_2)| \le \varepsilon$, that is,
				\[ \left| \frac{e(A_1,A_2)}{|A_1||A_2|} - d(V_1,V_2) \right| \le \varepsilon. \]
				So,
				\[ \frac{|V_1||V_2|}{|A_1||A_2|} \left| \frac{e(A_1,A_2)}{|V_1||V_2|} - d(V_1,V_2) \frac{|A_1||A_2|}{|V_1||V_2|} d(V_1,V_2) \right| \le \varepsilon, \]
				and \eqref{eqn: 2.11} easily follows. If $|A_1| < \varepsilon|V_1|$, then the above follows immediately anyway, since
				\begin{align*}
					\left| \frac{e(A_1,A_2)}{|V_1||V_2|} - d(V_1,V_2) \frac{|A_1||A_2|}{|V_1||V_2|} \right| &= \frac{|A_1||A_2|}{|V_1||V_2|} \left| \frac{e(A_1,A_2)}{|A_1||A_2|} - d(V_1,V_2) \right| \\
					&= \frac{|A_1||A_2|}{|V_1||V_2|} \left| d(A_1,A_2) - d(V_1,V_2) \right| \\
					&\le \frac{|A_1|}{|V_1|} < \varepsilon,
				\end{align*}
				so we are done.
			\end{proof}

			The generalized version of the triangle removal lemma is the following.

			\begin{ftheo}[Graph Removal Lemma]
				Given $\varepsilon>0$ and any graph $H$, there exists $\delta$ (depending only on $\varepsilon$) such that for $n\gg 0$, any graph $G_n$ with at most $\delta n^{V(H)}$ subgraphs isomorphic to $H$ can be made $H$-free by deleting at most $\varepsilon n^2$ edges.
			\end{ftheo}

		\subsubsection{Roth's Theorem and Corners}

			Next, we describe Roth's Theorem. The result deals with a conjecture of Erd\H{o}s and Tur\'{a}n:
			\begin{quote}
				Given $\varepsilon > 0$ and $r\in\N$, there exists $N_0$ such that for all $N \ge N_0$, the following holds. If $A \subseteq [N]$ with $|A| \ge \varepsilon N$, then $A$ contains an arithmetic progression of length $r$. 
			\end{quote}

			Roth proved in the early 50s that the conjecture holds for $r=3$. Szemer\'{e}di proved in the early 70s that it holds for all $r$. In fact, Szemer\'{e}di had originally come up with the regularity lemma to prove this result (this is why it is a \emph{lemma}).

			\begin{flem}
				\label{lemma: roth's theorem edge in unique triangle}
				Suppose that every edge of $G_n$ is in exactly one triangle. Then, $e(G_n) = o(n^2)$.
			\end{flem}

			How is this related to Roth's Theorem? Given $N\gg 0$ and $A\subseteq[N]$, suppose $A$ is $3$-AP free. Let $M = 2N+1$ and construct a $3$-partite graph $G$ whose three parts $X$, $Y$, and $Z$ are copies of $\Z/M$.\\
			For $x\in X$, $y\in Y$, and $z\in Z$, keep an edge from $x$ to $y$ iff $y-x \in A$, $y$ to $z$ iff $z-y \in A$, and $z$ to $x$ iff $(z-x)/2 \in A$. The final part is well-defined since $2$ is invertible in $\Z/M$ (Why?). \\
			Observe that if $xyz$ is a triangle with $a = y-x$, $b = z-y$, and $2c = z-x$, then $a+b=2c$, so $a,c,b$ are in AP.\\
			Since $a$, $b$, and $c$ are all in $A$, then by our assumption we must have $a=b=c$, $y-x=z-y=(z-x)/2$, and so $x$, $y$, $z$ must be in AP in $\Z/M$.\\
			This implies that any edge must be in a unique triangle.\\
			By \Cref{lemma: roth's theorem edge in unique triangle}, $e(G) = o((3M)^2) = o(N^2)$. On the other hand, $e(G) = 3M|A|$. So, $3M|A| = o(N^2)$ and $|A| = o(N)$.

			\begin{proof}[Proof of \Cref{lemma: roth's theorem edge in unique triangle}]
				The number of triangles in $G_n$ is exactly $e(G_n)/3 = o(n^3)$. By the \nameref{theo: triangle removal lemma}, $G_n$ can be made triangle-free by removing $o(n^2)$ edges. However, we must remove at least $e(G_n)/3$ edges to make the graph triangle-free, so $e(G_n) = o(n^2)$.
			\end{proof}

			Inspired by this problem, a natural question to ask is: what is the maximum sized $A\subseteq[N]$ that is $3$-AP free?\\
			Picking $A$ greedily gives $O(\sqrt{n})$ elements.\\
			
			It is possible to do better, as shown by \emph{Behrend's construction}.\\
			The basic idea is that on a sphere, the midpoint of any two points does not lie on the sphere.\\
			Consider the set $S = [k]^d\subseteq\R^d$ for some $k,d$. Observe that $\norm{x}^2 \le k^2 d$ for any point in $S$. By the pigeonhole principle, there is a (centered) sphere with at least $k^{d-2}/d$ of these points.\\
			To get a subset $A$ from these points, project $x = (a_1,\ldots,a_d)$ to $\sum_{j=1}^d a_j (2k+1)^{j-1}$, thus converting it to an integer in base $(2k+1)$. Observe that if $x_1 \mapsto m_1$ and $x_2 \mapsto m_2$, then $(x_1 + x_2)/2 \mapsto (m_1 + m_2)/2$. As a result, if $X$ is a subset of $S$ on a single sphere, its projection by this map gives a $3$-AP free subset.\\
			The maximum element by this projection is $k\sum_{i=0}^{d-1} (2k+1)^{i} = ((2k+1)^d - 1)/2$. Set $(2k+1)^d = 2N+1$. By the argument from before, there is a $3$-AP free set of size $k^{d-2}/d$.\\
			This achieves a bound of
			\[ \frac{N}{e^{c\sqrt{\log n}}} \]
			for some constant $c$. This is $\Omega(N^{1-\delta})$ for any (fixed) $\delta>0$!\\

			\begin{definition}[Corner]
				An $A\subseteq[N]^2$ is said to have a \textbf{corner} if the points $(x,y)$, $(x+d,y)$, and $(x,y+d)$ are in $A$ for some $d>0$ and $x,y\in[N]$.
			\end{definition}

			\begin{ftheo}[No Corners Theorem]
				\label{no corners theorem}
				Suppose $A\subseteq[N]^2$ has no corners. Then $|A| = o(N^2)$.
			\end{ftheo}
			\begin{proof}
				First, let us get rid of the $d>0$ clause.\\
				Denote by $A+A$ the set $\{a+b : a,b\in A\} \subseteq [2N]^2$ (the \href{https://en.wikipedia.org/wiki/Minkowski_addition}{Minkowski sum} of the two sets) and by $x+A$ $\{x+a : a\in A\}$.\\
				By the pigeonhole principle, there exists a $z\in[2N]^2$ such that $z=a+b$ for at least $|A|^2/4N^2$ pairs $(a,b)\in A^2$. Consider $A' = A \cap (z-A)$ for such a $z$. Then
				\[ |A'| \ge \frac{|A|^2}{4N^2} \]
				by the definition of $z$. It is not too difficult to see that $A' = (z - A')$. As a result, there is a correspondence between corners with positive $d$ and negative $d$ in $A'$. Further, if $A$ is corner-free, so is $A'$. By the cardinality bound on $A'$, it suffices to show that $|A'|=o(N^2)$.\\
				So, let us drop the $d > 0$ condition on corners and work with a corner-free $A$.\\

				We construct a $3$-partite graph $G$. Let the three parts of the vertex set $V$ be the set $H$ of horizontal lines in $[N]^2$, the set $V$ of vertical lines in $[N]^2$, and the set $D$ of lines with slope $-1$ in $[2N]^2$.\\
				Given $\ell$, $\ell'$ in distinct parts of $V$, let $\ell$ and $\ell'$ be adjacent in $G$ iff $\ell\cap\ell'\in A$. Triangles in this graph correspond to either corners or three lines all passing through a single point (that is in $A$). However, there are no corners in $A$, so each edge is in a unique triangle. The result then follows from \Cref{lemma: roth's theorem edge in unique triangle}.
			\end{proof}


		\subsubsection{A weaker version of the Erd\H{o}s-Stone-Simonovits Theorem}

			Next, we shall prove a slightly weaker version of the \nameref{theo: erdos-stone v2} using \nameref{theo: szemeredi's regularity lemma}.

			\begin{ftheo}[Erd\H{o}s-Stone-Simonovits Theorem, Version 3]
				Fix $t\in\N$ and $\varepsilon>0$. For $n\gg 0$, if $G_n$ is $K_{r+1}(t)$-free,\footnotemark\ then
				\[ e(G_n) \le \left( 1 - \frac{1}{r} \right) \frac{n^2}{2} + \varepsilon n^2. \]
			\end{ftheo}
			\footnotetext{This is the complete $(r+1)$-partite graph with $t$ vertices in each part, that is, $T_{r+1}(t(r+1))$.}

			This is weaker than the original result we saw since here, $t$ is fixed and so cannot grow to the $\log n$ estimate we gave earlier.

			\begin{proof}
				Suppose $e(G_n) > (1 - 1/r) n^2/2 + \varepsilon n^2$ edges.\\
				Let $\delta>0$ which we shall fix later. By Szemer\'{e}di's regularity lemma, we get a partition $(V_0,V_1,\ldots,V_k)$ such that $|V_0| \le \delta n$, $|V_i| = \ell$ (for some $\ell$) for $1\le i\le k$, and at most $\delta k^2$ of the pairs are $\delta$-irregular.\\
				`Clean' the graph by deleting all edges within the $V_i$ (losing $\le \ell^2k/2$ edges), edges adjacent to $V_0$ (losing $\le \delta n^2$ edges), edges in irregular pairs (losing $\le \delta k^2\ell^2$ edges), and edges between pairs of density less than $\delta$ (losing $\le \ell\delta (k^2/2)$ edges).\\
				The number of deleted edges is thus at most $4\delta n^2$. As a result, the graph after purification still has $>(1-1/r)n^2/2 + (\varepsilon - 4\delta)n^2$ edges. \\

				Let $H$ be a graph with vertex set $\{V_i : 1 \le i \le k\}$, and $V_i V_j \in E(H)$ iff $(V_i,V_j)$ is $(\varepsilon/8)$-regular with density at least $\varepsilon/4$.\\
				By the pigeonhole principle, since $e(V_i,V_j) \le \ell^2$
				\[ e(H) \ge \left( 1 - \frac{1}{r} + \varepsilon \right) \frac{n^2}{2\ell^2} \ge \left( 1 - \frac{1}{r} + \varepsilon \right) \frac{k^2}{2} . \]
				By Tur\'{a}n's Theorem, $H$ contains $K_{r+1}$. Say $(V_1,\ldots,V_{r+1})$ form this $K_{r+1}$. All $(V_i,V_j)$ are $(\varepsilon/8)$-regular with density at least $\varepsilon/4$.\\
				We now use the \nameref{theo: graph counting lemma}. Take $F = K_{r+1}(t)$. The number of copies of $F$ in $(V_1,\ldots,V_{r+1})$ is at least
				\[ \prod_{i=1}^{r+1} |V_i|^t \left((\varepsilon/4)^{t^2\binom{r+1}{2}} - \delta t^2 \binom{r+1}{2}\right). \]
				So, if
				\[ \delta \le \frac{(\varepsilon/4)^{t^2\binom{r+1}{2}}}{2t^2\binom{r+1}{2}},  \]
				then there are several copies of $K_{r+1}$ in $G$, completing the proof.
			\end{proof}

		\subsubsection{An application in computer science}

			Now, we give a delightful application from computer science.\\
			Suppose we are given a graph $G_n$ and we wish to check (algorithmically) whether $G$ has any triangles. An easy way to do this is to iterate through all triples and check if any of them form a triangle, which takes $\mathcal{O}(n^3)$ time.\\
			In this context, we give a paradigm due to Shafi Goldwasser and Madhu Sudan. We want to come up with an algorithm that takes a \emph{constant} number of verifications, and still returns a reasonable output. It is clearly impossible to return the answer correctly all the time. The output we desire is such that with high probability,
			\begin{itemize}
				\item if $G_n$ has no triangles, our output is correct.
				\item if $G_n$ has triangles and we output otherwise, then $G_n$ must be ``close'' to triangle-free.
			\end{itemize}
			Such a randomized constant time algorithm is known as a \textbf{property tester}. The question we wish to answer is: which graph properties admit such testers?\\
			For starters, does triangle-freeness admit a property tester? It turns out that it \emph{does}! The algorithm is in fact very simple.

			\begin{itemize}
				\item Pick a random triple $(x,y,z)$ and check if it forms a triangle.
				\item If yes, return ``not triangle-free''.
				\item If no, return ``$\varepsilon$-close to triangle-free'', where at most $\varepsilon n^2$ edges need to be deleted to remove all triangles.
			\end{itemize}

			Why is this a tester?\\
			By the \nameref{theo: triangle removal lemma}, if $G_n$ is $\varepsilon$-close to being triangle-free, then $G_n$ has at most $\delta n^3$ triangles for some $\delta$. That is,
			\[ \Pr[xyz \text{ is not a triangle}] \ge 1 - \frac{\delta n^3}{\binom{n}{3}} \ge 1 - 6\delta, \]
			so our algorithm is indeed a property tester.\\
			We shall return to property testing in more detail later.

		\subsubsection{The Frankl-Pach Theorem}

			\begin{ftheo}[Frankl-Pach Theorem]
				Suppose $s,r\in\N$ and $0<c<1$. Then, for $n\gg 0$,
				\[ \exx(n;K_{r+1},K_{s,\lceil cn\rceil}) \le \left( c^{1/s} \left( 1 - \frac{1}{r} \right)^{1/s} + o(1) \right) n^2. \]
			\end{ftheo}

			Observe that one of the graphs we are forbidding depends on $n$ itself.

			\begin{proof}
				Let $\varepsilon > 0$ and $G$ be a graph on $n$ vertices that is $K_{r+1}$- and $K_{s,\lceil cn\rceil}$-free. Using Szemer\'{e}di's regularity lemma and the cleaning process, obtain a graph $G'$ such that
				\begin{itemize}
					\item $V(G') = V_1 \sqcup \cdots \sqcup V_k$ where $k=O_\delta(1)$,
					\item $|V_i| = \ell$ for all $i$ and $V(G') \ge (1-\delta)n$,
					\item All pairs $(V_i,V_j)$ are $\delta$-regular,
					\item if $e(V_i,V_j) > 0$, $d(V_i,V_j) \ge \eta$, and
					\item $e(G) - e(G') \le \varepsilon n^2$
				\end{itemize}
				for some $\delta,\eta > 0$ depending only on $\varepsilon$. Henceforth, we operate only in $G'$. \\

				Construct a graph $H$ on vertex set $[k]$ with $ij\in E(H)$ iff $d(V_i,V_j)\ge \eta$. By the \nameref{theo: graph counting lemma}, if $H$ contains $K_{r+1}$, so does $G'$ and thus $G$.
				Therefore, $H$ is $K_{r+1}$-free and by Tur\'{a}n's Theorem, $e(H) \le (1-1/r)k^2/2$.\\
				Since $n\gg 0$ and $k=O_\delta(1)$, if $K_{s,\lceil cn\rceil}$ appears in $G$, one of the $V_i$s must have at least $s$ vertices for $n\gg 0$ (by a simple pigeonhole argument).\\
				So, let us bound $e(G)$ given that $K_{s,\lceil cn\rceil}$ is forbidden and the size $s$ part of the bipartite graph does not come from a single $V_i$.\\
				Similar to what we did in the \nameref{theo: kovari sos turan}, let us count pairs $(x,S)$ where $|S|=s$, $S\subseteq V_i$ for some $i$, and $x$ is adjacent to every $y\in S$. We call such a structure an \emph{$(i,s)$-claw}. We have that
					\[ \text{number of $(i,s)$-claws } = \sum_{x\in V'} \sum_{i=1}^{k} \binom{|N(x)\cap V_i|}{s}. \]
				On the other hand, by our assumption regarding $K_{s,\lceil cn\rceil}$-freeness,
					\[ \text{number of $(i,s)$-claws } \le ( \lceil cn\rceil - 1 ) \sum_{i=1}^k \binom{|V_i|}{s} \le cn k \binom{\ell}{s}. \]
				Therefore,
					\[ \sum_{x\in V'}\sum_{i=1}^{k} \binom{|N(x)\cap V_i|}{s} \le cn k \binom{\ell}{s}. \]
				Define $\mathcal{P} = \{ (x,i) : |N(x)\cap V_i| > 0 \}$. Applying Jensen's inequality in a manner similar to that in proof of the K\H{o}v\'{a}ri-S\'{o}s-Tur\'{a}n Theorem,
					\begin{align*}
						\sum_{(x,i)\in\mathcal{P}} \binom{|N(x)\cap V_i|}{s} &\ge |\mathcal{P}| \binom{(1/|\mathcal{P}|) \sum_{(x,i)\in\mathcal{P}} | N(x) \cap V_i| }{s} \\
						&= |\mathcal{P}| \binom{2e(G')/|\mathcal{P}|}{s} \\
						&= |\mathcal{P}| \binom{u}{s},
					\end{align*}
				where $u = 2e(G')/|\mathcal{P}|$. Then,
					\begin{align*}
						cn k \frac{\ell^s}{s!}  &\ge cn k \binom{\ell}{s} \\
						&\ge \sum_{x\in V'}\sum_{i=1}^{k} \binom{|N(x)\cap V_i|}{s} \\
						&= \sum_{(x,i) \in \mathcal{P}} \binom{|N(x)\cap V_i|}{s} \\
						&\ge |\mathcal{P}| \binom{u}{s} \\
						&\ge |\mathcal{P}| \frac{(u-s+1)^s}{s!}.
					\end{align*}
				Simplifying,
					\begin{equation}
						\label{eqn: 2.12}
						e(G') \le \frac{u|\mathcal{P}|}{2} \le \frac{1}{2} \left((cn k)^{1/s} \ell | \mathcal{P} |^{1 - 1/s}  + |\mathcal{P}| (s-1)\right).
					\end{equation}
				To bound $|\mathcal{P}|$ from above, observe that
					\[ |\mathcal{P}| \le e(H) \cdot 2 \ell \]
				since each edge in $H$ gives at most $2\ell$ choices for $x$ (if the edge is $ij$, $x$ is in $V_i$ or $V_j$). By $K_{r+1}$-freeness, $e(H) \le (1 - 1/r) k^2/2$ so
					\[ |\mathcal{P}| \le \left( 1 - \frac{1}{r} \right) k^2 \ell. \]
				Substituting the above back in \eqref{eqn: 2.12},
					\begin{align*}
						e(G') &\le \frac{1}{2} \left((cn k)^{1/s} \ell | \mathcal{P} |^{1 - 1/s}  + |\mathcal{P}| (s-1)\right) \\
							&\le \frac{1}{2} \left( (cn)^{1/s} \left( 1 - \frac{1}{r} \right)^{1/s} (k\ell)^{2-1/s} + \left( 1 - \frac{1}{r} \right) k^2\ell (s-1) \right) \\
							&\le \frac{1}{2} \left( c^{1/s} \left( 1 - \frac{1}{r} \right)^{1/s} n^2 + \left( 1 - \frac{1}{r} \right) k (s-1) n \right) \\
							&= \left( c^{1/s} \left( 1 - \frac{1}{r} \right)^{1/s} + o(1) \right) \frac{n^2}{2}.
					\end{align*}
				The result follows since $e(G) - e(G') \le \varepsilon n^2$.
			\end{proof}

			It turns out that the bound given by the Frankl-Pach Theorem is asymptotically tight!

		\subsubsection{A look at \texorpdfstring{$\exx(n;C_{2k+1})$}{ex(n;C 2k+1)}}

			As we saw back in \Cref{theo: ESS exx square tight coloring},
			\[ \frac{n^2}{4} \le \exx(n;C_{2k+1}) \le \left( \frac{1}{4} + o(1) \right) n^2. \]

			Recall that if $e(G_n) \ge n^2/4 - t$ and $G_n$ is $K_3$-free, then $G_n$ can be made bipartite by deleting atmost $t$ edges.

			\begin{flem}
				\label{lemma: exx n C 2k+1 stability}
				Given $\varepsilon > 0$, there exists $\delta > 0$ such that the following holds for $n\gg 0$. If $G_n$ is $C_{2k+1}$-free and has $\ge (1/4 - \delta)n^2$ edges, it can be made bipartite by deleting at most $\varepsilon n^2$ edges.
			\end{flem}
			\begin{proof}
				First, we shall try to make $\delta(G_n)$ `large'. If $G_n$ has a vertex of degree $<(1/2 - 2\sqrt{\delta})n$, delete this vertex to get a graph $G_{n-1}$. Perform this process repeatedly to finally arrive at some $G_\ell$ such that
				\begin{align*}
					e(G_\ell) &> \left(\frac{1}{4} - \delta\right)n^2  - \left( \frac{1}{2} - 2\sqrt{\delta} \right) \left( n + (n-1) + \cdots + (\ell+1) \right) \\
					&= \left(\frac{1}{4} - \delta\right)n^2  - \left( \frac{1}{2} - 2\sqrt{\delta} \right) \left( \binom{n+1}{2} - \binom{\ell+1}{2} \right) \\
					&= \left( \frac{1}{4} - \delta \right)n^2 - \left( \frac{1}{4} - \sqrt{\delta} \right)\left( n^2 + n - \ell^2 - \ell \right) \\
					&\ge \left( \frac{1}{4} - \delta \right) \ell^2 + (\sqrt{\delta}-\delta) n^2 - \left( \frac{1}{4} - \delta \right)n.
				\end{align*}
				If the quantity above is at least $\left( 1/4 + \eta \right) \ell^2$ for some constant $\eta > 0$, we have
				\[ \ell < \sqrt{ \frac{\sqrt{\delta}-\delta}{\eta+\delta} } n. \]
				This implies that for sufficiently large $\ell$ (by virtue of sufficiently large $n$), $G_\ell$ has $C_{2k+1}$ which yields a contradiction.\\
				Therefore, setting $\eta$ appropriately, this process must terminate with $\ell \ge (1-4\sqrt{\delta}) n$.\\
				Therefore, suppose $G' = (V',E')$ is a subgraph of $G_n$ such that $|V'| = n' \ge (1-4\sqrt{\delta}) n$, $\delta(G') \ge (1/2 - 2\sqrt{\delta}) n'$, and $G'$ is $C_{2k+1}$-free. We have $e(G') \ge (1/4 - \sqrt{\delta})n'^2$.\\
				Recall that $\exx(n;C_{2k})$ is $o(n^2)$. As a result, it follows that if $n\gg 0$, $G'$ contains a copy of $C_{2k}$, say $v_1 v_2 \cdots v_{2k} v_1$. Letting $W = V' \setminus \{v_1,\ldots,v_k\}$, we must have that $A_1 = N(v_1) \cap W$ and $A_2 = N(v_2) \cap W$ are disjoint. Since $G'$ is $C_{2k+1}$-free, the subgraph induced on $A_1$ (or $A_2$) is $P_{2k}$-free. By \Cref{theo: exx n P k+1}, $e(A_1)$ (or $e(A_2)$) is at most $kn'$. To summarize,
				\begin{itemize}
					\item $|A_1|, |A_2| \ge (1/2 - 2\sqrt{\delta})n' - 2k \ge (1/2 - \sqrt{\delta})n'$.
					\item $e(A_1), e(A_2) \le kn'$.
					\item $A_1 \cap A_2 = \emptyset$.
				\end{itemize}
				Delete all edges that are not between $A_1$ and $A_2$ (namely those within the $A_i$s and those that touch at least one vertex not in $A_1 \cup A_2$). This deletes at most
				\[ \underbrace{2kn'}_{e(A_i)} + \underbrace{2\sqrt{\delta} n'^2}_{|V' \setminus (A_1\cup A_2)| \ge 2\sqrt{\delta}n'}. \]
				The resulting graph is clearly bipartite. Thus, from $G_n$, we have deleted at most, say, $5\sqrt{\delta}n^2$ edges, so setting $\delta$ less than say $\varepsilon^2/100$ works.
			\end{proof}

			\begin{ftheo}
				For $n\gg 0$,
				\[ \exx(n;C_{2k+1}) = \left\lfloor \frac{n^2}{4} \right\rfloor. \]
			\end{ftheo}
			\begin{proof}
				Let $G_n$ be $C_{2k+1}$-free and have the maximum possible number of edges. Since the complete bipartite graph is $C_{2k+1}$-free, $e(G_n) \ge \lfloor n^2/4 \rfloor$.\\
				Let $\varepsilon > 0$. For starters, let us assume that $\delta(G_n) \ge (1/2 - 2\sqrt{\varepsilon})n$. By \Cref{lemma: exx n C 2k+1 stability}, $G_n$ can be made bipartite by deleting at most $\varepsilon n^2$ edges (due to the assumption on $\delta(G_n)$ and the previous proof). Let $(A,B)$ be the parts of such a bipartite graph. Further assume that $e(A,B)$ is the maximum possible (we choose that bipartite subgraph with the most edges). We claim that $|A|$ and $|B|$ are at least $(1/2 - \varepsilon)n$. Indeed, otherwise, since $|A|+|B|=n$,
				\[ e(A,B) \le |A||B| < \left( \frac{1}{2} - \sqrt{\varepsilon} \right) \left( \frac{1}{2} + \varepsilon \right) n^2 = \left( \frac{1}{4} - \varepsilon \right)n^2. \]
				In this case, $e(G_n) \le e(A,B) + \varepsilon n^2 < \lfloor n^2/4\rfloor$, proving the claim.\\

				Since $e(A,B)$ is the maximum possible, it follows that for any $a\in A$, $d(a,B) \le d(a,A)$ (otherwise we can switch $a$ from $A$ to $B$).\\
				Next, we claim that no vertex $a$ of $A$ is such that $d(a,A) \ge 2\sqrt{\varepsilon} n$. Suppose otherwise. We then have $d(a,B) \ge 2\sqrt{\varepsilon} n$ as well. Let $A_1 = A \cap \Gamma(a)$ and $B_1 = B \cap \Gamma(a)$. Consider the bipartite subgraph on $(A_1,B_1)$. This subgraph is $P_{2k}$-free. By \Cref{theo: exx n P k+1},
				\[ e(A_1, B_1) \le 2kn. \]
				In particular, the number of missing edges between $A_1$ and $B_1$ is at least $4\varepsilon n^2 - 2kn > \varepsilon n^2 \ge e(A) + e(B)$ for $n\gg 0$. As a result,
				\begin{align*}
					e(G_n) &= e(A,B) + e(A) + e(B) \\
						&\le e(A,B) + (\text{number of missing pairs between $A_1$ and $B_1$}) \\
						&< |A||B| \le n^2/4,
				\end{align*}
				which gives a contradiction.\\

				Let $a,a'\in A$ be adjacent. We have
				\begin{align*}
					|(\Gamma(a)\cap B) \cap (\Gamma(a')\cap B)| &\ge d(a,B) + d(a',B) - |B| \\
						&> d(a) - 2\sqrt{\varepsilon}n + d(a') - 2\sqrt{\varepsilon}n - \left( \frac{1}{2} + \sqrt{\varepsilon} \right) n \\
						&\ge 2\delta(G_n) - 4\sqrt{\varepsilon}n - \left( \frac{1}{2} + \sqrt{\varepsilon} \right) n \\
						&\ge \left( \frac{1}{2} - 9\sqrt{\varepsilon} \right) n.
				\end{align*}
				Let $B' = \Gamma(a) \cap \Gamma(a') \cap B$ and $A' = A \setminus \{a,a'\}$ and consider the bipartite subgraph between these two parts. Since the previous subgraph was $C_{2k+1}$-free, there cannot be a $P_{2k-1}$ between two vertices in $B'$ (or $A'$). It follows that there does not exist a path of length $2k$ between \emph{any} two vertices of $A'\cup B'$. So,
				\[ e(A',B') \le 2kn. \]
				Estimating the number of edges again,
				\begin{align*}
				 	e(G_n) &\le e(A',B') + e(A\setminus A', V) + e(B\setminus B', V) \\
				 		&= 2kn + \underbrace{2n}_{|A\setminus A'| = 2} + \underbrace{10\sqrt{\varepsilon}n}_{\substack{|B|\le (1/2+\sqrt{\varepsilon})n \\ |B'| \ge (1/2-9\sqrt{\varepsilon})n}} \\
				 		&< \frac{n^2}{4}\text{ for $n\gg 0$},
				\end{align*}
				giving a contradiction again.\\

				Therefore, it now suffices to show why we can make our initial assumption that $\delta(G_n) \ge (1/2 - 2\sqrt{\varepsilon})n$. As in the proof of the previous lemma, repeatedly delete any vertex that has degree less than $(1/2 - 2\sqrt{\varepsilon})n$. We must stop at some $G_\ell$ with $\ell > (1-4\sqrt{\varepsilon})n$. Further, $G_\ell$ must have greater than $\ell^2/4$ edges and is $C_{2k+1}$-free, so we arrive at a contradiction anyway due to our proof taking the assumption at the beginning.
			\end{proof}

	\subsection{Pseudorandomness}

		Szemer\'{e}di's regularity lemma, in layman terms, says that any dense graph consists `mostly' of a `small' number of `random-like' bipartite graphs. The meaning of the first two words within the quotes should be clear; they just say that $|V_0|$ is small and that $k$ is bounded. The question then is: what does it mean to say that a graph is random-like? We have a single graph, not a distribution.

		\subsubsection{Notions of randomness}

			Let us first list a couple of features that feel like something a random-like graph should have. Suppose that $e(G_n) has (p + o(1)) \binom{n}{2}$ edges, where $p$ is a fixed constant.
			\begin{enumerate}
				\item $\mathsf{Disc}$ (\emph{Discrepancy}). For all $X,Y\subseteq V(G)$, 
				\[ \left| e(X,Y) - p|X||Y| \right| = o(n^2). \]
				
				\item $\mathsf{Disc}'$. For all $X\subseteq V(G)$,
				\[ \left| e(X) - p \binom{|X|}{2} \right| = o(n^2). \]
				
				\item $\mathsf{Count}$. For a fixed $H$, the number of labelled copies of $H$ in $G$ is $(1+o(1))p^{e(H)} n^{|V(H)|}$ (Why choose this number?).
				
				\item $\mathsf{Codegree}$. We have
				\[ \sum_{x,y \in V} |d(x,y) - p^2 n| = o(n^3). \]

				\item $\mathsf{Eigen}$. If $\lambda_1 \ge \cdots \ge \lambda_n$ are the eigenvalues of the adjacency matrix of $G_n$, then
				\[ \lambda_1 = pn(1+o(1)) \]
				and
				\[ \max_{i > 1} |\lambda_i| = \max\{|\lambda_2|,|\lambda_n|\} = o(n). \]
			\end{enumerate}
			Observe that there are already asymptotics at play here since $o(\cdot)$ is not very meaningful when we are dealing with a single graph.\\

			While the reader is likely familiar with what an adjacency matrix is, we define it here for the sake of completeness.
			\begin{definition}[Adjacency Matrix]
				Given a graph $G=(V,E)$, the \emph{adjacency matrix} $A$ of $G$ is a matrix whose rows and columns are indexed by $V$, and for $v,w\in V$
				\[
				A_{vw} =
				\begin{cases}
					1, & vw\in E, \\
					0, & vw\not\in E.
				\end{cases}
				\]
				The eigenvalues of $A$ are sometimes referred to as the eigenvalues of $G$.
			\end{definition}

			While the first four properties above might make sense based on our discussion thus far, the last seems to come out of nowhere.

			To motivate $\mathsf{Eigen}$, we give the following result.

			\begin{ftheo}[Expander Mixing Lemma]
				\label{theo: expander mixing lemma}
				Suppose $G=(V,E)$ is a $d$-regular graph on $n$ vertices and $\lambda = \max\{|\lambda_2|,|\lambda_n|\}$, where $\lambda_1 \ge \cdots \ge \lambda_n$ are the eigenvalues of $G$. Then for any $X,Y\subseteq V$,
				\[ \left| e(X,Y) - \frac{d}{n}|X||Y| \right| \le \lambda\sqrt{|X||Y|}. \]
			\end{ftheo}
			\begin{proof}
				Observe that
				\[ e(X,Y) = \indic_X^\top A \indic_Y, \]
				where $\indic_X$ is the `indicator vector' of $X$, a vector indiced by the vertices of $G$, with $1$ at the positions in $X$ and $0$ elsewhere.\\
				Let $A$ be the adjacency matrix of $G$. Since $A$ is real symmetric, we can apply the spectral theorem to write $A = \sum_i \lambda_i v_i v_i^\top$, where $v_i$ is an eigenvector for $\lambda_i$.\\
				Observe that $d$ is an eigenvalue of $A$ with corresponding eigenvector $\mathbf{1}$, the all $1$s vector. In fact, $d$ is the largest eigenvalue -- if $\lambda_1$ is the largest eigenvalue and $x$ is an eigenvector with $x_v = \max_u |x_u|$,
				\[ |\lambda_1 x_v| = |(Ax)_v| = \left|\sum_{u \in \Gamma(v)} x_u \right| \le d(v) |x_v| \le \Delta(G) |x_v|. \]
				Letting $J$  be the all $1$s matrix of appropriate size,
				\begin{align*}
					\left| e(X,Y) - \frac{d}{n}|X||Y| \right| &= \left| \indic_X^\top A \indic_Y - \frac{d}{n} \indic_X J \indic_Y \right| \\
						&= \left| \indic_X^\top \left( A - \frac{d}{n} J \right) \indic_Y \right| \\
						&\le \norm{\indic_X^\top} \norm{A - \frac{d}{n} J} \norm{\indic_Y}.
				\end{align*}
				We claim that $\norm{A - (d/n)J} \le \lambda$. Since $A$ is symmetric with all row sums as $d$, it commutes with $J$, and so they can be simultaneously diagonalized. Further, $n$ is an eigenvalue of $J$ and $d$ is an eigenvalue of $A$ with eigenvector $\mathbf{1}$.
			\end{proof}

			Next, the following result says that all five notions of randomness we came up with are equivalent!

			\begin{ftheo}[Chung-Graham-Wilson Theorem]
				$\mathsf{Disc}$, $\mathsf{Disc}'$, $\mathsf{Count}$, $\mathsf{Codegree}$, and $\mathsf{Eigen}$ are equivalent.
			\end{ftheo}
			\begin{proof}
				To prove this we shall show that all of our 5 are equivalent to the following.\\
				$C_4$-$\mathsf{Count}$. The number of labelled $C_4$s is $(p^4+o(1))n^4$ (this is just $\mathsf{Count}$ for $H = C_4$ specifically).\\
				\begin{itemize}
					\item $\mathsf{Disc}$ to $\mathsf{Disc}'$. Setting $X=Y$ then dividing by $2$ gives the result directly.
					
					\item $\mathsf{Disc}'$ to $\mathsf{Disc}$. Let $X,Y\subseteq V$. We have
					\[ e(X,Y) = e(X \cup Y) + e(X\cap Y) - e(X\setminus Y) - e(Y\setminus X) \]
					Applying $\mathsf{Disc}'$ to each of these $4$ sets gives the results near directly:
					\begin{align*}
						e(X,Y) &= p\binom{|X\cup Y|}{2} + p\binom{|X\cap Y|}{2} - p\binom{|X\setminus Y|}{2} - p\binom{|Y\setminus X|}{2} + o(n^2) \\
						&= p|X||Y| + o(n^2).
					\end{align*}
					
					\item $\mathsf{Disc}$ to $\mathsf{Count}$. This is just the \nameref{theo: graph counting lemma} (we only use the $\mathsf{Disc}$ part of regularity in the proof).

					\item $\mathsf{Count}$ to $C_4$-$\mathsf{Count}$. This is clear from the definition.

					\item $C_4$-$\mathsf{Count}$ to $\mathsf{Codegree}$. Now,
					\begin{align*}
						\sum_{x,y\in V} d(x,y) &= \sum_{x\in V} d(x)^2 \\
							&\ge \frac{1}{n} \left(\sum_{x\in V} d(x)\right)^2 \\
							&= \frac{4e(G)^2}{n} = (1+o(1))p^2n^3.
					\end{align*}
					Also note that
					\[ \sum_{x,y\in V} d(x,y)^2 = \text{number of labelled $C_4$s} + o(n^4) = (1+o(1)) p^4 n^4. \]
					So,
					\begin{align*}
						\sum_{x,y\in V} |d(x,y) - p^2 n| &\le n \left( \sum_{x,y\in V} (d(x,y) - p^2n)^2 \right)^{1/2} \\
							&= n \left( \sum_{x,y\in V} d(x,y)^2 - 2 p^2n \sum_{x,y\in V} d(x,y) + p^4 n^4 \right)^{1/2} \\
							&\le n\left( (1+o(1))p^4n^4 - 2p^2n (1+o(1))p^2n^3 + p^4n^2 \right)^{1/2} \\
							&= n (o(1)n^4)^{1/2} \\
							&= o(n^3).
					\end{align*}

					\item $\mathsf{Codegree}$ to $\mathsf{Disc}$. We have
					\begin{align*}
						\sum_{x\in V} |d(x) - pn| &\le \sqrt{n} \left( \sum_{x\in V} (d(x)-pn)^2 \right)^{1/2} \\
							&= \sqrt{n} \left( \sum_{x\in V} d^2(x) - 2pn \sum_{x\in V}d(x) + p^2n^3 \right) \\
							&= \sqrt{n} \left( \sum_{x,y\in V} d(x,y) - 2p^2n^3 + p^2n^3 o(n) \right) \\
							&= \sqrt{n} \left( \sum_{x,y \in V} \left(d(x,y) - p^2n\right) \right) \\
							&\le \sqrt{n} \left( \sum_{x,y \in V} \left|d(x,y) - p^2n\right| \right) = o(n^2).
					\end{align*}
					For $X,Y\subseteq V$,
					\begin{align*}
						\left| e(X,Y) - p|X||Y| \right| &\le \sum_{x\in X} \left| d(x,y) - p|Y| \right| \\
						&\le \sqrt{|X|} \left( \sum_{x\in X} (d(x,Y) - p|Y|)^2 \right)^{1/2} \\
						&\le \sqrt{|X|} \left( \sum_{v\in V} (d(v,Y) - p|Y|)^2 \right)^{1/2} \\
						&= \sqrt{|X|} \left( \sum_{v\in V} d(v,Y)^2 - 2p|Y|\sum_{v\in V} d(v,Y) + p^2 |Y|^2 n \right)^{1/2} \\
						&= \sqrt{|X|} \left( \sum_{y,y'\in Y} d(y,y')^2 - 2p|Y|\sum_{y\in Y} d(y) + p^2 |Y|^2 n \right)^{1/2} \\
						&= \sqrt{|X|} 
					\end{align*}

					\item $\mathsf{Eigen}$ to $C_4$-$\mathsf{Count}$. Note that the number of labelled $C_4$s in $G$ equals the number of closed walks of length $4$ minus $O(n^3)$. Now, observe that given two vertices $v,w$, the number of walks from $v$ to $w$ of length $4$ is just $(A^4)_{vw}$ (Why?). In particular, the number of closed walks from $v$ to $v$ is $(A^4)_{vv}$. Recalling that the trace of a matrix is the sum of its eigenvalues,
					\begin{align*}
						\text{number of $C_4$s} &= \Tr(A^4) - O(n^3) \\
							&\le \lambda_1^4 + \sum_{i>1} \lambda_i^4 \\
							&\le (1+o(1)) (pn)^4 + \left(\max_{i>1} \lambda_i^2\right) \sum_{i=1}^n \lambda_i^2 \\
							&= (1+o(1)) (pn)^4 + o(n^2) \cdot 2e(G) \\
							&= (1+o(1)) (pn)^4 + o(n^4) = (1+o(1)) p^4 n^4.
					\end{align*}

					\item $C_4$-$\mathsf{Count}$ to $\mathsf{Eigen}$. The maximum eigenvalue of $A$ is just
					\[ \lambda_1 = \sup_{\norm{x}_2 = 1} x^\top A x. \]
					Taking $x = (1/\sqrt{n})\mathbf{1}$, we have
					\[ \lambda_1 \ge \frac{2e(G)}{n} = (p - o(1))n. \]
					For the remaining eigenvalues,
					\begin{align*}
						\max_{i\ne 1} |\lambda_i|^4 &\le \Tr(A^4) - \lambda_1^4 \\
							&\le \underbrace{(p^4 + o(1))n^4}_{\text{from }C_4-\mathsf{Count}} - (p - o(1))^4 n^4 \\
							&= o(n^4),
					\end{align*}
					completing the proof.
				\end{itemize}
			\end{proof}

			\begin{remark}
				\begin{itemize}
					\item The proof requires $p = \Omega(1)$, in particular to show that $\mathsf{Disc}$ implies $\mathsf{Count}$.
					\item $\mathsf{Count}$ does not hold in sparse regimes. In fact, Noga Alon in the 90s gave a graph with $p = \Omega(n^{-1/3})$ which is $\mathsf{Eigen}$-pseudorandom, but triangle-free.
					% It is known that a graph with $p = o(n^{-1/2})$ is almost surely triangle-free and a graph with $p = \Omega(n^{1/2})$ almost surely has triangles.
				\end{itemize}
			\end{remark}

		\subsubsection{Strongly regular graphs}

			Do we have any `known' families of pseudorandom graphs?

			\begin{fdef}
				Suppose $q$ is a prime power such that $q \equiv 1 \pmod{4}$. Consider the \textbf{Paley graph} $P_q$ with vertex set $V(P_q) = \F_q$ and $xy$ is an edge iff $x-y$ is a quadratic residue\footnotemark.
			\end{fdef}
			\footnotetext{$x-y = a^2$ for some $a\in\F_q^\times$.}

			Let us see a couple of properties of the Paley graph.

			\begin{fdef}
				A graph $G$ is said to be an \textbf{$(n,d,\lambda)$-graph} if it is a regular graph of degree $d$ over $n$ vertices with $\max_{i\ne 1} |\lambda_i| \le \lambda$, where $\lambda_1 \ge \cdots \ge \lambda_n$ are the eigenvalues of the graph.
			\end{fdef}

			\begin{fdef}
				A graph $G=(V,E)$ is a \textbf{$(n,d,\lambda,\mu)$-strongly regular graph} if
				\begin{itemize}
					\item $|V|=n$,
					\item $G$ is regular with degree $d$,
					\item Any two adjacent vertices have exactly $\lambda$ common neighbbours, and
					\item Any two non-adjacent vertices have exactly $\mu$ common neighbours.
				\end{itemize}
				If a graph is $(n,d,\lambda,\mu)$-strongly regular for some $n,d,\lambda,\mu$, it is just said to be strongly regular.
			\end{fdef}

			For example, a complete graph on $n$ vertices is $(n,n-1,n-1,0)$-strongly regular and $C_5$ is $(5,2,0,1)$-strongly regular.\\
			Slightly more complicatedly, the Petersen graph is $(10,3,0,1)$-strongly regular.\\

			Observe that a graph is $(n,d,\lambda,\mu)$-strongly regular iff its adjacency matrix $A$ is such that
			\[
			(A^2)_{vw} =
			\begin{cases}
				d, & i=j, \\
				\lambda, & v,w\text{ are adjacent}, \\
				\mu, & v,w\text{ are not adjacent}.
			\end{cases}
			\]
			More succinctly,
			\[ A^2 = d I + \lambda A + \mu(J - I - A). \]
			We see that
			\begin{align*}
				A^3 &= d A + \lambda A^2 + \mu d J - \mu(A+A^2) \\
				0 &= A^3 + (\mu-\lambda)A^2 - d A - \mu d J.
			\end{align*}
			The graph is strongly regular iff its adjacency matrix satisfies a cubic with integer coefficients and leading coefficient $1$!\\
			This further implies that a strongly regular graph has only $3$ eigenvalues, with one of them being $d$.\\

			Returning to the Paley graph, it is not too difficult to check that it is $( q , (q-1)/2 , (q-5)/4 , (q-1)/4 )$-strongly regular. Consequently, it is a $\left( q , (q-1)/2, (\sqrt{q}+1)/2 \right)$-graph.\\
			Observe that any strongly regular graph has eigenvalues (other than $\lambda_1$) bounded in a similar fashion.\\

		\subsubsection{The sparse regime}

			Let us restrict ourselves to $(n,d,\lambda)$-graphs. A popular family of $(n,d,\lambda)$-graphs comes from the following.

			\begin{fdef}
				Suppose $\Gamma$ is a finite group and $S\subseteq\Gamma\setminus\{1\}$ is such that $S = S^{-1}$. We define the \textbf{Cayley graph} $\Cay(\Gamma,S)$ as that on vertex set $\Gamma$ with $gh \in E(G)$ iff $g^{-1} h \in S$.
			\end{fdef}

			It is easy to see that $\Cay(\Gamma,S)$ is $|S|$-regular. In fact, $P_q$ is just a Cayley graph with $\Gamma = \F_q$ (under addition) and $S$ is the set of quadratic residues in $\F_q$ other than $1$.

			\begin{ftheo}[Conlon-Zhao, 2017]
				For Cayley graphs, $\mathsf{Disc}$ and $\mathsf{Eigen}$ are equivalent.
			\end{ftheo}
			\begin{proof}
				To show this, we require a result of Grothendieck's:
				\begin{quote}
					There exists an absolute constant $K > 0$ such that for all real matrices $A = (a_{ij})$ and a Hilbert space $\mathcal{H}$,
					\[ \sup_{\substack{x_i,y_j \in \mathcal{H} \\ \norm{x},\norm{y} = 1}} \sum_{ij} a_{ij} \langle x_i , y_j \rangle \le K \sup_{\{u,v\}\in\{-1,1\}^n} \sum_{i,j} a_{ij} u_i v_j. \]
				\end{quote}
				The best possible $K$ is unknown, but we do know that $K < 1.79$ is possible. We skip the proof of this.\\

				First, note that $\mathsf{Eigen}$ implies $\mathsf{Disc}$ with the same constant for both. That is, if
				\[  \max_{i\ne 1} |\lambda_i| \le \varepsilon n.  \]
				then
				\[ \left| e(X,Y) - p|X||Y| \right| \le \varepsilon n^2 \],
				by the \nameref{theo: expander mixing lemma}.
				Suppose that $\mathsf{Disc}$ holds with a constant of $\varepsilon$ and let $A$ be the adjacency matrix of $G = \Cay(\Gamma,S)$. Let $x,y\in\{-1,1\}^n$. Denote $x^+$ and $x^-$ the positive and negative parts of $x$ respectively. That is, $x = x^+ - x^-$. Letting $X^+ \subseteq V$ be the support of $x^+$ (the subset of vertices with non-zero components),
				\[ (x^\pm)^\top \left( A - \frac{d}{n} J \right) y^\pm = e(X^\pm,Y^\pm) - \frac{d}{n} |X^\pm| |Y^\pm|, \]
				where the sign over the $X$s ($Y$s) are the same as that over $x$ ($y$). Therefore, using $\mathsf{Disc}$,
				\[ x^\top \left( A - \frac{d}{n}J \right)y \le 4(\varepsilon d)n. \]
				Since
				\[ \max_{i \ne 1} |\lambda_i| = \sup_{\norm{x} = \norm{y} = 1} x^\top \left(A - \frac{d}{n}J\right) y, \]
				it suffices to show that the quantity on the right is `small'.
			\end{proof}