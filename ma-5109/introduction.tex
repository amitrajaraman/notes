
\section{Introduction}

	\subsection{Basic Definitions}

		\begin{fdef}
			A (simple undirected) \textbf{graph} $G$ is an ordered pair $(V,E)$ where $V$ is a finite set called the \emph{vertex set} and $E$, called the \emph{edge set}, is a subset of $\binom{V}{2}$, where $\binom{S}{k}$ represents the set of all $k$-element subsets of $S$.
		\end{fdef}

		We typically represent graphs pictorially, showing vertices as dots and edges as arcs joining the vertices present in the corresponding subset.

		A few important graphs are:
		\begin{itemize}
			\item the \emph{null graph} with vertex set $V$, where $E = \emptyset$.
			\item the \emph{complete graph} $K_n$, where $V = [n]$ and $E = \binom{[n]}{2}$.
			\item the \emph{complete bipartite graph} $K_{m,n}$, where $V = A \cup B$ with $|A|=m$, $|B|=n$, and $A,B$ are disjoint, and $E = \{\{a,b\} : a \in A, b \in B\}$.
			\item the \emph{path graph} $P_{n+1}$ of length $n$, where $V = [n+1]$ and $E = \{\{m,m+1\} : m \in [n]\}$.
			\item the \emph{cycle} of length $n$, where $V = [n]$ and $E = \{\{l,m\} : l,m\in [n] , (m - l) \equiv 1 \pmod{n}\}$.
		\end{itemize}

		Now, consider the graph $G$ with vertex set $[4]$ and edge set $\{\{1,3\},\{3,2\},\{2,4\}\}$. This graph appears to be the same as the path graph of length $3$, but how do we make this correspondence more concrete?\\
		Relabeling vertices doesn't create a ``new'' graph.

		\begin{fdef}[Graph Isomorphism]
			Two graphs $G = (V,E)$ and $G' = (V',E')$ are said to be \textbf{isomorphic} and we write $G \simeq G'$ if there exists a bijection $f : V \to V'$ such that there is an edge between two vertices $u$ and $v$ in $G$ if and only if there is an edge between $f(u)$ and $f(v)$ in $G'$.
		\end{fdef}

		If two graphs are isomorphic, they are identical for our purposes (we only care about graphs up to isomorphism).

		We now give a few more definitions that are useful.

		\begin{fdef}[Subgraph]
			Given a graph $G = (V,E)$, a \textbf{subgraph} $H = (V',E')$ is a graph such that $V' \subseteq V$ and $E' \subseteq E$. Given $V' \subseteq V$, the subgraph \emph{induced} by $V'$ on $G$ is that with vertex set $V'$ and edge set $\binom{V'}{2} \cap E$.
		\end{fdef}

		\begin{fdef}[$r$-partite Graph]
			A graph $G = (V,E)$ is said to be \textbf{$r$-partite} if there exists a partition $V_1, V_2, \ldots, V_r$ of $V$ such that for any edge $e = uv \in E$, $u$ and $v$ are in distinct $V_i$. That is, there are no edges within any of the $V_i$.\\
			In particular, a $2$-partite graph is said to be \textbf{bipartite}.
		\end{fdef}

		\begin{fdef}[Independent Set]
			Given a graph $G=(V,E)$, $I \subseteq V$ is said to be \textbf{independent} if no two vertices of $I$ are adjacent (the subgraph induced by $I$ is null).\\
			$\alpha(G)$, the \emph{independence number} of $G$, denotes the size of the largest independent set in $G$.
		\end{fdef}

		\begin{fdef}[Clique]
			Given a graph $G=(V,E)$, $K \subseteq V$ is said to be a \textbf{clique} if any two vertices of $K$ are adjacent (the subgraph induced by $I$ is complete).
			$\omega(G)$, the \emph{clique number} of $G$, denotes the size of the largest clique in $G$.
		\end{fdef}

		\begin{fdef}[Complement Graph]
			Given a graph $G=(V,E)$, the \textbf{complement graph} of $G$ is $\bar{G} = (V,\binom{V}{2} \setminus E)$.
		\end{fdef}

		Observe that $S \subseteq V$ is independent in $G$ if and only if $S$ is a clique in $\bar{G}$. In particular, $\alpha(G) = \omega(\bar{G})$.

		\begin{fdef}[Connectedness]
			A graph $G$ is said to be \textbf{connected} if for any pair of vertices $u, v$, there is a sequence $u=v_0,v_1,\ldots,v_r = v$ for some $r$ such that $v_{i-1} v_i$ is an edge for each $i\in[r]$.
		\end{fdef}

	\subsection{\texorpdfstring{$K_{r+1}$}{K r+1}-free graphs}

		Extremal graph theory is motivated by the following simple problem:

		\begin{quote}
			At most how many edges can a graph $G_n$ have if it contains no triangles?
		\end{quote}

		More precisely, what is
		\[ \max_{\substack{\text{no subgraph of $G_n$} \\ \text{is isomorphic to $K_3$}}} e(G_n)? \]

		Clearly, this number is well-defined since a graph on $n$ vertices cannot have more than $\binom{n}{2}$ vertices.

		A simple observation is that any complete bipartite graph has no triangles: if there were a triangle, then two vertices would be in the same ``part'', which contradicts the existence of edges only between the two parts.\\
		As a consequence, for any $1 \le m \le n$, it is possible to construct $m \times (n-m)$ edges (with this bound being attained for $K_{m,n-m}$). In particular, it is possible to construct a graph with $\lfloor n^2 / 4 \rfloor$ edges.

		\begin{ftheo}[Mantel's Theorem]
		\label{Mantel's Theorem}
			If $G_n$ has no triangle, then
			\[ e(G_n) \le \left\lfloor \frac{n^2}{4} \right\rfloor. \]
			Further, equality is attained iff $G_n \simeq K_{\lfloor n/2\rfloor , \lceil n/2 \rceil}$.
		\end{ftheo}
		\begin{proof}
			Suppose $G_n$ has no triangles. Saying that $G_n$ has no triangles is equivalent to saying that for distinct adjacent $u,v$, $N(u) \cap N(v) = \emptyset$.\\
			So, $d(u) + d(v) \le n$. Therefore,
			\begin{align*}
				ne(G_n) &\stackrel{(1)}{\ge} \sum_{uv \in E} d(u) + d(v) \\
					&= \sum_{uv \in E} |N(u) \cup N(v)| \\
					&= \left| (e, w) : e = uv \in E, w \in N(u) \cup N(v) \right| \\
					&= \sum_{u\in V} |\{ (e , w) : w \in N(u) , e = uv \in E \}| \\
					&= \sum_{u \in V} |\{ (v,w) : v,w \in N(u) \}| \\
					&\stackrel{(2)}{=} \sum_{u \in V} d(u)^2 \\
					&\stackrel{(3)}{\ge} \frac{1}{n} \left( \sum_{u \in V} d(u) \right)^2 \\
					&\stackrel{(4)}{=} \frac{4 e(G_n)^2}{n},
			\end{align*}
			where $(2)$ follows from the changing the main thing being summed over to $u$, the ``middle'' vertex in the $L$-like structure, $(3)$ follows from the \href{https://en.wikipedia.org/wiki/Cauchy-Schwarz_inequality}{Cauchy-Schwarz inequality}, and $(3)$ follows from the \href{https://en.wikipedia.org/wiki/Handshaking_lemma}{handshaking lemma}.\\

			What happens when equality is attained? Let us look at the case where $n$ is even.\\
			$(1)$ is only tight when $d(u) + d(v) = n$ for all edges $uv$ and $(3)$ is only tight when $d(u)$ is a constant (independent of $u$). This implies that $d(u) = \frac{n}{2}$ for every $u\in V$. Now, if $uv$ is an edge, $N(u) \cap N(v) = \emptyset$ implies that $N(u) \cup N(v) = V$, and so $G_n = K_{\frac{n}{2}, \frac{n}{2}}$.\\
			The case where $n$ is odd is analyzed similarly, with slight nuances in $(3)$ since exact equality is not attained.
		\end{proof}
		
		While the above is one of the early results in extremal graph theory, the subject was only really born due to Tur\'{a}n in the following result.

		\begin{ftheo}[Tur\'{a}n's Thoerem]
			If $G_n$ has no $K_{r+1}$ ($r \ge 2$), then $e(G_n) \le t_r(n)$, with equality attained iff $G \simeq T_r(n)$.
		\end{ftheo}

		The version for $r=2$ is just a triangle-free graph and is the same as \nameref{Mantel's Theorem}. In the proof of this, we split the vertex set into two parts and dumped all the edges between these parts.\\
		If we want to avoid $K_4$ ($r=3$), then perhaps we could split the vertex set into three parts and dump all the edges between these parts.\\
		In general, we want to partition $V$ of size $n$ into $r$ ``almost equal'' parts and set only those edges between vertices in distinct parts -- such a graph is known as the \textbf{Tur\'{a}n graph} $T_r(n)$ and the number of edges $e(T_r(n))$ is the \textbf{Tur\'{a}n number} $t_r(n)$.\\
		In particular, when $r \mid n$,
		\[ t_r(n) = \binom{r}{2} \left(\frac{n}{r}\right)^2 = \frac{n^2}{2} \left(1 - \frac{1}{r}\right). \]


		Here, we give three proofs of Tur\'{a}n's Theorem.

		\begin{proof}[Proof of Tur\'{a}n's Theorem]
			We perform strong induction on $n+r$. We have already proved the result for $r=2$.\\
			Suppose $e(G_n) \ge t_r(n)$ and $G_n$ is $K_{r+1}$-free, where $r > 2$. We wish to prove that $G \simeq T_r(n)$.\\
			Since $t_r(n) > t_{r-1}(n)$\footnote{To show this, use the fact that any $(r-1)$-partite graph can be thought of as $r$-partite graph with one of the pieces being empty, and then that the Tur\'{a}n graph has the most edges among $r$-partite graphs.}, the inductive hypothesis implies that $G$ has a copy $K \subseteq V$ of $K_r$. Observe that for $v \not\in K$, $d(v,K) \le r-1$ -- otherwise, there would be a copy of $K_{r+1}$ in $G$.\\
			As a result, $e(V \setminus K, K) \le (r-1)(n-r)$. By the induction hypothesis, $e(V\setminus K, V\setminus K) \le t_r(n-r)$. Therefore,
			\[ t_r(n) \le e(G_n) \le t_r(n-r) + (r-1)(n-r) + \binom{r}{2}. \]
			However, as can be checked manually, $t_r(n-r) + (r-1)(n-r) + \binom{r}{2} = t_r(n)$!\\
			It follows that equality holds everywhere -- $e(G_n) = t_r(n)$, $e(V \setminus K) = t_r(n-r)$, and $d(v,K) = r-1$ for all $v \in V \setminus K$.\\
			This graph is then isomorphic to $T_r(n)$ -- for each $v \in V \setminus K$, we can put the vertex in $K$ that is not adjacent to $v$ in the same bucket as $v$. Then, the only edges are those between distinct buckets (Why?), so $G_n \simeq T_r(n)$.
		\end{proof}

		\begin{proof}[Erd\H{o}s' Proof of Tur\'{a}n's Theorem]
			Erd\H{o}s proves a slightly more general claim: given a $K_{r+1}$-free graph $G_n$, there exists an $r$-partite graph $H$ on $V$ such that $d_G(v) \le d_H(v)$ for all $v \in V$.\\
			It is then a simple task to check that among the $r$-partite graphs on $n$ vertices, the Tur\'{a}n graph $T_r(n)$ has the most edges.\\

			To prove our claim, we perform induction on $r$.\\
			The claim is trivial for the base case $r=1$.\\
			Now, suppose the claim holds for values less than $r$. Let $v_0 \in V$ such that $d_G(v_0) = \max_{v \in V} d_G(v)$ (the vertex of maximum degree in $G$) and $W = N(z)$. Since $G$ is $K_{r+1}$-free, $W$ is $K_r$-free. Inductively, there is an $(r-1)$-partite graph $H'$ on $W$ such that for all $v \in W$, $d_{H'}(v) \ge d_W(v)$.\\
			Let $U = V \setminus W$. For each $u \in U$, remove all its edges in $G$ and set its new neighbour set as $W$.\\
			Our desired graph $H$ is that with these edges along with those in $H'$ and the edges from $v_0$ to $W$. That is, the $r$th part is $U \cup \{v_0\}$ and the remaining $(r-1)$ parts are those formed by $H'$. The graph is clearly $r$-partite by definition. What about the degree inequality?
			\begin{itemize}
				\item $d_G(v_0) = d_H(v_0)$ trivially.
				\item For $u \in U$, $d_H(u) = d_G(v_0) \ge d_G(u)$.
				\item For $w \in W$,
				\[ d_H(w) = |U| + 1 + d_{H'}(w) \ge |U| + 1 + d_W(w) \ge d_U(w) + 1 + d_W(w) = d_G(w). \qedhere \]
			\end{itemize}

			(Why does equality imply that the graph is isomorphic to $T_r(n)$?)
		\end{proof}

		\begin{ftheo}[Tur\'{a}n's Theorem, reformulation]
			If $d = e(G_n) / n$ is the average degree of the vertices of $G_n$, then $G_n$ has an independent set of size at least $n/(d+1)$.
		\end{ftheo}

		\begin{proof}
			Why is this equivalent to Tur\'{a}n's Theorem?\\
			If $G_n$ has no $K_{r+1}$, then $\alpha(\bar{G}) \le r$. If $\bar{G_n}$ has average degree $d$, the above result would imply that $r \ge n/(d+1)$, that is, $d \le (n/r) - 1$. The total number of edges in $G_n$ is then
			\[ \binom{n}{2} - \frac{nd}{2} \le \binom{n}{2} - \frac{n}{2} \left(\frac{n}{r} - 1\right) = \frac{n^2}{2} \left(1 - \frac{1}{r}\right), \]
			which gives Tur\'{a}n's bound!\\

			Let us now get to the proof of the above reformulation. First, consider the following algorithm to come up with \emph{some} independent set in $G$:
			\begin{enumerate}
				\item Order $V$ to get $\{v_1,\ldots,v_n\}$ and initialize $S = \emptyset$.
				\item Add $v_1$ to $S$.
				\item Having processed vertices $v_1$ through $v_i$, add $v_{i+1}$ to $S$ iff there is no vertex in $S$ that is adjacent to $v_{i+1}$.
			\end{enumerate}
			It is clear that this always produces an independent set, but the size of the independent set depends on the ordering we choose at the beginning.\\
			For a given ordering $\sigma$, denote by $\mathcal{A}(\sigma)$ the independent set produced by the algorithm.\\
			How do we choose a ``good'' ordering?\\
			Enter the probabilistic method. Define the random variable $\pi$ to be uniformly random on the set of all permutations of $V$. Then,
			\begin{align*}
				\expec[ | \mathcal{A}(\pi) | ] &= \expec\left[ \sum_{v \in V} \indic_{v \in \mathcal{A}(\pi)} \right] \\
				&= \sum_{v \in V} \expec\left[ \indic_{v \in \mathcal{A}(\pi)} \right] \\
				&= \sum_{v \in V} \Pr\left[ v \in \mathcal{A}(\pi) \right].
			\end{align*}
			Fix some $v \in V$ and permutation $\sigma$. What is the probability that $v \in \mathcal{A}(\sigma)$?\\
			If at the time of processing $v$ for the ordering $\sigma$, $N(v) \cap S \ne \emptyset$, then $v$ is not picked. In particular, if $v$ is the first element of $N(v) \cup \{v\}$ in the ordering $\sigma$, then it is definitely chosen by the algorithm. The probability of this occurring is $\frac{1}{d(v) + 1}$.
			So,
			\begin{align*}
				\expec[ |\mathcal{A}(\pi)| ] &= \sum_{v \in V} \Pr\left[ \indic_{v \in \mathcal{A}(\pi)} \right] \\
				&\ge \sum_{v \in V} \frac{1}{d(v) + 1} \\
				&\stackrel{(*)}{\ge} \frac{n^2}{\sum_{v \in V} (d(v) + 1)} = \frac{n}{d+1},
			\end{align*}
			% where $(*)$ follows from \href{https://en.wikipedia.org/wiki/Cauchy%E2%80%93Schwarz_inequality#Titu's_lemma_-_Positive_real_numbers}{Titu's Lemma}.\\
			where $(*)$ follows from the \href{https://en.wikipedia.org/wiki/HM-GM-AM-QM_inequalities}{AM-HM inequality}.\\
			Since the expectation of $|\mathcal{A}(\pi)|$ is at least $n/(d+1)$, there must exist some permutation $\sigma$ such that $|\mathcal{A}(\sigma)| \ge n/(d+1)$, proving the result.
		\end{proof}

	\subsection{The Zarankiewicz Problem}

		Tur\'{a}n's Theorem is the primary result that birthed Extremal Graph Theory. To generalize the problem studied in the previous section, define the following.

		\begin{fdef}[Extremal Function]
			Given a graph $H$, define the \textbf{extremal function}
			\begin{equation}
				\label{eqn: def extremal function}
				\exx(n;H) = \max_{\substack{\text{no subgraph of }G_n \\\text{is isomorphic to }H}} e(G_n)
			\end{equation}
			as the maximum number of edges in a graph on $n$ vertices without $H$ as a subgraph.
		\end{fdef}

		With this notation, Tur\'{a}n's Theorem then says that $\exx(n;K_{r+1}) = t_r(n)$, with the corresponding maximum in \eqref{eqn: def extremal function} being attained iff $G_n \simeq T_r(n)$.

		\begin{fdef}[Zarankiewicz Function]
			Fix $s, t \in \N$ with $t \ge s \ge 2$ and $m,n\in\N$.
			The \textbf{Zarankiewicz function} $z(m,n;s,t)$ is the maximum number of edges in a bipartite graph $G = (A \sqcup B, E)$ such that
			\begin{itemize}
				\item the two components $A$ and $B$ of the graph are of cardinality $m$ and $n$ respectively\footnotemark, and
				\item there exist no $S \subseteq A$, $T \subseteq B$ with $|S|=s$, $|T|=t$, and all edges between $S$ and $T$ present in $E$.\footnotemark
			\end{itemize}
		\end{fdef}
		\footnotetext[1]{by ``components'' of the bipartite graph we mean that for any edge $uv$ in the graph, $u\in A$ and $v\in B$ or $u \in B$ and $v\in A$.}
		\footnotetext[2]{we make the tacit assumption that $s \le m$ and $t \le n$.}

		For ease of writing, we refer to the above described criterion as the \emph{Zarankiewicz condition}.

		That is, we ``forbid'' the subgraph $K_{s,t}$ with the components of size $s$ and $t$ on the side of $A$ and $B$ respectively.\\

		The \textbf{Zarankiewicz problem} asks for a closed form representation of $z(m,n;s,t)$. Failing this, for fixed $t$, it asks for a tight asymptotic bound on $z(n,n;t,t)$ as $n$ grows large.\\
		Perhaps surprisingly, this problem remains unsolved! (as of the time of writing these notes)

		\begin{ftheo}[K\H{o}v\'{a}ri-S\'{o}s-Tur\'{a}n Theorem]
			\label{theo: kovari sos turan}
			For $t\ge s\ge 2$ and $m,n\in\N$,
			$$z(m,n;s,t) \le (s-1)^{1/t} (n-t+1)m^{1-1/t} + (t-1)m.$$
		\end{ftheo}
		\begin{proof}
			Let $G$ be bipartite with vertex set $A \sqcup B$ and satisfy the Zarankiewicz condition.\\
			Fix $a \in A$. By definition, $|N(a)| = d(a)$. Now, $\binom{d(a)}{t}$ is the number of $t$-element subsets of $N(a)$. We may assume that for all $a \in A$, $d(a) \ge t-1$. Indeed, otherwise, we may add arbitrary edges to $a$ to make its degree $t-1$; any vertex from $A$ in a subgraph isomorphic to $K_{s,t}$ must have at least degree $t$ so $a$ cannot be part of it.

			We have that
			\begin{align*}
				\sum_{x \in A} \binom{d(x)}{t} &= \left| \{ (x,T) : x \in A, T \subseteq N(x), |T|=t \} \right| \\
				&= \sum_{\substack{T \subseteq B \\ |T| = t}} |\{x \in A : T \subseteq N(x)\}|.
			\end{align*}
			If we fix a $T$, then the number of such $x$ for that $T$ is at most $s-1$, due to our assumption. As a result,
			\[ \sum_{x \in A} \binom{d(x)}{t} \le \binom{n}{t} (s-1). \]
			Now, observe that the function
			\[ f(x) = \frac{x(x-1)\cdots(x-t+1)}{t!} \]
			is convex on $[t-1,\infty)$.
			Using \href{https://en.wikipedia.org/wiki/Jensen%27s_inequality#Finite_form}{Jensen's inequality} together with our assumption that $d(x) \ge t-1$ for all $x\in A$,
			\begin{align}
				\binom{n}{t} (s-1) &\ge \sum_{x\in A} \binom{d(x)}{t} \nonumber \\
				 &\ge m \binom{\frac{1}{m}\sum_{x\in A} d(x)}{t} \nonumber
				  \\
				 &= m \binom{e(G)/m}{t}. \label{eqn: kovari sos turan intermediate step}
			\end{align}
			Let $d = e(G)/m$, the average degree of the vertices in $A$. Simplifying the above expression,
			\[ \frac{s-1}{m} \ge \frac{d(d-1)\cdots (d-t+1)}{n(n-1)\cdots(n-t+1)} \ge \left(\frac{d-t+1}{n-t+1}\right)^t, \]
			where the second inequality follows from the fact that $d \le n$. Therefore,
			\[ e(G) \le m \left( \left(\frac{s-1}{m}\right)^{1/t} (n-t+1) + (t-1) \right) = (s-1)^{1/t} (n-t+1) m^{1 - 1/t} + (t-1)m, \]
			completing the proof.
		\end{proof}

		Next, let us look at some consequences of the above bound.

		\subsubsection{The Zarankiewicz problem and the extremal function for complete bipartite graphs}

			We get a bound on $\exx(n;K_{s,t})$. We claim that for $n\in\N$ and $s,t\ge 2$,
			\begin{equation}
				\label{eqn: exx n Kst and z n n s t}
				\exx(n;K_{s,t}) \le \frac{1}{2} z(n,n;s,t).
			\end{equation}
			Indeed, if $G_n = (V,E)$ has no $K_{s,t}$, make a bipartite graph $G'$ that has vertex set $V \times \{0,1\}$ such that $uv$ is an edge in $G$ iff $\{(u,0),(v,1)\}$ is an edge in $G'$.\\
			$G'$ satisfies the Zarankiewicz condition. If there do exist $S \subseteq V\times\{0\}$ and $T' \subseteq V \times \{1\}$ such that all $S$-$T$ edges are in $G$, then $\pi_1(S) \cap \pi_1(T') = \emptyset$ (otherwise, a vertex would be adjacent to itself in $G$, which is clearly false). This implies that $K_{s,t} \subseteq G$, which is a contradiction.\\
			Since $e(G') = 2e(G)$, the claim follows.

		\subsubsection{The case where \texorpdfstring{$s=t=2$}{s=t=2}}

			When $s=t=2$, we get
			\[ z(m,n;2,2) \le (n-1)m^{1/2} + m \text{ and } z(n,n;2,2) \le (n-1)n^{1/2} + n. \]
			Therefore,
			\[ \exx(n;K_{2,2}) \le \frac{1}{2} (n + (n-1)\sqrt{n}). \]
			Note that $K_{2,2}\simeq C_4$. Therefore, a square-free graph on $n$ vertices has $\mathcal{O}(n^{3/2})$ edges.\\

			In fact, this bound is tight! We give an algebraic construction of a suitable graph with no $K_{2,2}$, known as the \emph{Levi graph for the projective plane}.\\
			Let $q$ be a prime power and consider the $3$-dimensional vector space $\mathcal{V} = \mathbb{F}_q^3$ (over $\mathbb{F}_q$). Let $\mathcal{P}$ and $\mathcal{L}$ be the set of all $1$- and $2$-dimensional subspaces of $\mathcal{V}$ respectively.\\
			Define the graph $G = ( \mathcal{P} \sqcup \mathcal{L} , E )$ as follows. For $x \in \mathcal{P}$ and $L \in \mathcal{L}$, let $x$ be adjacent to $L$ in $G$ iff $x \subseteq L$.\\
			We claim that $G$ has no $K_{2,2}$. Suppose otherwise and let $x_1,x_2\in\mathcal{P}$ and $L_1,L_2\in\mathcal{L}$ such that the $x_i$ are adjacent to the $L_j$. If $x_1 = \langle u \rangle$ and $x_2 = \langle v \rangle$, then $u$ and $v$ are linearly independent, which implies that $L_i = \langle u,v\rangle$. This contradicts the fact that the $L_i$ are distinct!\\
			What are the cardinalities of $\mathcal{P}$ and $\mathcal{L}$?
			\begin{itemize}
				\item To get a $1$-dimensional subspace, we pick a non-zero $u$ and consider $\langle u\rangle$. In $\mathcal{V}$, there are $q^3 - 1$ non-zero $u$. We must now divide by $q-1$ to account for the fact that linearly dependent vectors generate the same $1$-dimensional subspace. Any non-zero $u$ has precisely $q-1$ non-zero multiples. Therefore,
				\[ |\mathcal{P}| = \frac{q^3 - 1}{q - 1} = q^2 + q + 1. \]

				\item It turns out that the number of $2$-dimensional subspace is equal to the number of $1$-dimensional subspaces (more generally, the number of $d$-dimensional subspaces is equal to the number of $(n-d)$-dimensional subspaces of $\mathbb{F}_q^n$), so
				\[ |\mathcal{L}| = q^2 + q + 1. \]
			\end{itemize}
			How many edges does $G$ have?\\
			Fix $x = \langle u \rangle \in \mathcal{P}$. We wish to determine how many $L\in\mathcal{L}$ are adjacent to $x$ in $G$. Such an $L$ can be created by choosing $v \not\in x$ and letting $L = \langle u,v\rangle$.\\The number of choices of $v$ is $q^3 - q$, but each such subspace is repeated for $q^2 - q$ choices of $v$ since the cardinality of $\langle u,v\rangle$ is $q^2$. Therefore,
			\[ d(x) = \frac{q^3 - q}{q^2 - q} = q + 1. \]
			The total number of edges is
			\[ |\mathcal{P}| (q + 1) = (q+1)(q^2+q+1) = q^3 + 2q^2 + 2q + 1, \]
			which is $\mathcal{O}(|\mathcal{P}|^{3/2})$!
			Therefore, our $\mathcal{O}(n^{3/2})$ bound is tight.\\
			In fact, the Levi graph is optimal in the case where $n = 2(q^2 + q + 1)$, as seen in \Cref{theo: levi graph optimality}.

			\begin{corollary}
				\label{theo: levi graph optimality}
				For $n\in\N$,
				\begin{equation}
					\label{eqn: z n n 2 2}
					z(n,n;2,2) \le \frac{n (1 + \sqrt{4n-3})}{2}.
				\end{equation}
				Consequently,
				\begin{equation}
					\label{eqn: exx n C4}
					\exx(n;C_4) \le \frac{1}{4} n (1 + \sqrt{4n-3}).
				\end{equation}
			\end{corollary}
			\begin{proof}
				\Cref{eqn: exx n C4} clearly follows from \Cref{eqn: z n n 2 2,eqn: exx n Kst and z n n s t}, so it suffices to show the first equation. \Cref{eqn: kovari sos turan intermediate step} in the proof of the \nameref{theo: kovari sos turan} for the case where $s=t=2$ just says that
				\[ \binom{n}{2} \ge n \binom{d}{2}, \]
				where $d = e(G)/m$. That is, $d^2 - d - (n-1) \le 0$. Then,
				\[ d \le \frac{1 + \sqrt{1 + 4(n-1)}}{2} = \frac{1 + \sqrt{4n-3}}{2}, \]
				which is exactly the bound we want. This bound is tight in the case where $n = 2(q^2+q+1)$, as seen in the Levi graph.
			\end{proof}

			Before we move on to the next section, let us build a tiny bit of intuition for why the construction of the Levi graph works.\\
			The projective plane is chosen to ensure that any two distinct points determine a unique line (which holds even in the non-projective plane setting), and any two distinct lines determine a unique point. This corresponds to the absence of $K_{2,2}$ -- if it \emph{was} present as a subgraph, then there would be two lines (points) that determine two points (lines), which cannot happen!

		\subsubsection{The case where \texorpdfstring{$s=t=3$}{s=t=3}}

			We next look at $\exx(n;K_{3,3})$.\\

			The \nameref{theo: kovari sos turan} applied here gives
			\[ z(n,n;3,3) \le (2)^{1/3} (n-2)n^{2/3} + 2n, \]
			which is $\mathcal{O}(n^{5/3})$.

			Similar to the Levi graph, we construct an (algebraic) extremal example. Let $p$ be a prime and fix some $r\in\mathbb{F}_p$.
			Consider the graph $G$ that has vertex set $\mathcal{V} = \mathbb{F}_p^3$ where $(x,y,z)$ is adjacent to $(u,v,w)$ iff 
			\begin{equation}
				\label{eqn: exx n n 3 3}
				(x-u)^2 + (y-v)^2 + (z-w)^2 = r.
			\end{equation}
			Before moving on to explaining why this works, let us try to impart some intuition. \eqref{eqn: exx n n 3 3} resembles the equation of a sphere in $\R^3$. If we take three spheres of the same radius, the points of intersection of all three must lie on two circles, corresponding to the circles of intersection of two pairs of spheres. Since any two circles meet at atmost two points, the absence of $K_{3,3}$ follows.\\
			It may be shown that even over $\F_p$, two ``spheres'' intersect on a ``circle'' and any two circles meet at at most $2$ points (Check this!).\\
			So, if we have a $K_{3,3}$ in the described graph, we have three spheres (centered at each of the three points) that intersect at three points, which is not possible.\\

			It remains to count the number of edges in this graph. To do so, let us estimate the degree of $(0,0,0)$, since all vertices have the same degree (Why?). That is, we want to determine
			\[ |\{(x,y,z)\in\mathbb{F}_p^3 : x^2 + y^2 + z^2 = r\}|. \]
			Letting $z$ be arbitrary, we want to find
			\[ N(\xi) = |\{(x,y)\in\mathbb{F}_p^2 : x^2 + y^2 = \xi\}| \]
			for any arbitrary $\xi \in \mathbb{F}_p$.

			\begin{fdef}[Legendre Symbol]
				For $a \in \mathbb{F}_p$, define the \textbf{Legendre symbol}
				\[
					\left(\frac{a}{p}\right) =
					\begin{cases}
						1, & \text{$a\in\mathbb{F}_p^\times$ is a square}, \\
						-1, & \text{$a\in\mathbb{F}_p^\times$ is not a square}, \\
						0, & a=0.
					\end{cases}
				\]
			\end{fdef}

			With the above, it is not too difficult to see that
			\[ N(\xi) = \sum_{(a,b): a + b = \xi} \left(1 + \left(\frac{a}{p}\right) \right) \left(1 + \left(\frac{b}{p}\right)\right). \]

			Let us now compute the above quantity.
			\begin{itemize}
				\item First of all,
				\[ \sum_{(a,b): a + b = \xi} 1 = p. \]

				\item Observe that
				\[ \sum_{a \in \mathbb{F}_p} \left(\frac{a}{p}\right) = 0. \]
				Indeed, the number of squares and non-squares in $\F_p^\times$ is the same. As a result,
				\[ \sum_{(a,b): a + b = \xi} \left(\frac{a}{p}\right) = \sum_{a\in\mathbb{F}_p} \left(\frac{a}{p}\right) = 0. \]
			\end{itemize}

			Therefore,
			\begin{equation}
				\label{eqn: N xi eliminating middle terms}
				N(\xi) = p + \sum_{(a,b) : a + b = \xi} \left(\frac{a}{p}\right) \left(\frac{b}{p}\right).
			\end{equation}

			Notice that the map $\F_p^\times \to \{-1,1\}$ given by $x \mapsto \left(\frac{x}{p}\right)$ is a group homomorphism.
			\clearpage

			\begin{lemma}
				If $\xi\ne 0$, $N(\xi) = N(1)$.
			\end{lemma}

			\begin{proof}
				Using \eqref{eqn: N xi eliminating middle terms} and letting $a' = a/\xi$ and $b'=b/\xi$,
				\begin{align*}
					N(\xi) - p &= \sum_{a+b=\xi} \left(\frac{a}{p}\right) \left(\frac{b}{p}\right) \\
					&= \sum_{a'+b'=1} \left(\frac{a'\xi}{p}\right) \left(\frac{b'\xi}{p}\right) \\
					&= \sum_{a'+b'=1} \left(\frac{a'}{p}\right) \left(\frac{b'}{p}\right) \left(\frac{\xi^2}{p}\right) \\
					&= \sum_{a'+b'=1} \left(\frac{a'}{p}\right) \left(\frac{b'}{p}\right) = N(1) - p. \qedhere
				\end{align*}
			\end{proof}

			So, we have
			\begin{align*}
				(p-1)(N(1) - p) &= \sum_{\xi\in\F_p^\times} \sum_{(a,b) : a+b=\xi} \left(\frac{a}{p}\right)\left(\frac{b}{p}\right) \\
				&= \sum_{\xi\in\F_p^\times} \sum_{a\in\F_p} \left(\frac{a}{p}\right) \left(\frac{\xi-a}{p}\right) \\
				&= \sum_{a\in\F_p} \left( \left(\frac{a}{p}\right) \sum_{\xi\in\F_p^\times} \left(\frac{\xi-a}{p}\right) \right) \\
				&= \sum_{a\in\F_p} \left(\frac{a}{p}\right) \left(0 - \left(\frac{-a}{p}\right)\right) \\
				&= -\left(\frac{-1}{p}\right) \sum_{a\in\F_p} \left(\frac{a^2}{p}\right) & \text{(using the group homomorphism property)} \\
				&= -(p-1)\left(\frac{-1}{p}\right)
			\end{align*}

			and therefore, $N(1) = p - \left(\frac{-1}{p}\right)$.\\
			If we choose $r$ to be a non-square, then $r-z^2 \neq 0$ for any $z\in\F_p$ and $N(r-z^2) = p - \left(\frac{-1}{p}\right)$. In this case, the degree of $(0,0,0)$ is $\Theta(p^2)$. The size $n$ of the vertex set $\mathcal{V}$ is $p^3$, so the number of edges is $\Theta(p^5)$ which is $\Theta(n^{5/3})$ and thus, the bound given by the \nameref{theo: kovari sos turan} is (asymptotically) tight.


			For $s \le t$, the \nameref{theo: kovari sos turan} gives
			\[ \exx(n;K_{s,t}) < C_{s,t} n^{2 - 1/s} \]
			for some $c_{s,t}$ depending only on $s$ and $t$. However, it is not known if the above bound is tight for most cases.

			Lower bounds of the form
			\[ \exx(n;K_{s,t}) \ge c_{s,t} n^{2 - 1/t} \]
			are known only if $t \ge (s-1)! + 1$. See \cite{ref-1} for more details.

	\subsection{\texorpdfstring{$P_{k+1}$}{P k+1}-free graphs}

		Next, we study $\exx(n;P_{k+1})$.\\

		First of all, note that $K_k$ is $P_{k+1}$-free. So, if we split the $n$ vertices up into blocks of $k$ vertices and add all edges within each block, the resulting graph will be $P_{k+1}$-free as well. That is, $G_n$ is a disjoint union of $K_k$s (and possibly one $K_r$ for $r < k$). For this particular graph $G_n$,
		\[ e(G_n) \le \frac{(k-1)n}{2}. \]

		It turns out that we cannot do better than this.

		\begin{ftheo}
			\label{theo: exx n P k+1}
			$\exx(n;P_{k+1}) \le (k-1)n/2$ with equality if and only if $G_n$ is a disjoint union of $K_k$s.
		\end{ftheo}

		The above theorem just says that equality is not attained for connected graphs without $P_{k+1}$.

		To prove this, we prove the following (seemingly) more general statement.

		\begin{flem}
			\label{theo: 1.8}
			Let $G_n = (V,E)$ be connected and suppose $d(v) \ge k$ for all $v\in V$. If $n\ge 2k$, $G_n$ contains a path of length $2k$. Otherwise, $G_n$ contains an $n$-cycle.
		\end{flem}
		\begin{proof}
			We prove the result for the case where $n\ge 2k$ first. Consider the longest path $u = v_1, v_2, \ldots, v_r = v$ in $G_n$ and let $U = \{v_i : i \in [r]\}$. We must show that $r \ge 2k$. Suppose otherwise and let $r < 2k$.\\
			First of all, we must have $\Gamma(u) \subseteq U$ -- otherwise, the path can be extended by adding another vertex from $\Gamma(u) \setminus U$. Similarly, $\Gamma(v) \subseteq U$.\\
			Next, $v_1$ and $v_r$ cannot be adjacent. If they are, then we can obtain a longer path by cycling through and choosing some edge from a vertex in $U$ to one outside of $U$ (such a vertex must exist since the graph is connected and has at least $2k$ vertices).
			More generally, if there exists $i$ such that $v_1, v_{i+1}$ and $v_i, v_r$ are adjacent, then we arrive at a contradiction (for the same reason).\\
			Let $S = \{i : v_iv_r \in E\}$ and $T = \{i : v_1v_{i+1} \in E\}$. By the above observation, $S \cap T = \emptyset$. However, by our first observation, $|S| = d(v_r) \ge k$ and $|T| = d(v_1) \ge k$. Therefore, $r \ge |S \cup T| \ge 2k$.\\
			The result for $n < 2k$ is shown using nearly the same proof.
		\end{proof}

		\begin{proof}[Proof of \Cref{theo: exx n P k+1}]
			We perform strong induction on $n$. We may assume that $n > k$. Suppose $G_n$ has no $P_{k+1}$. If $G_n$ is \emph{not} connected, then it consists of a disjoint union of connected subgraphs. We may then apply the inductive hypothesis to each of these smaller pieces.\\
			So, let $G_n$ be connected. By \Cref{theo: 1.8}, there is some vertex $v$ such that $d(v) \le (k-1)/2$ (otherwise, there must be a path of length $k$). Additionally, observe that $G_n$ does not have any subgraph isomorphic to $K_k$ -- using the connectedness assumption gives a path of length $k$ otherwise.\\
			The graph $G \setminus \{x\}$\footnote{This is the subgraph induced by $G$ on the vertex set $V\setminus\{x\}$} has no $P_{k+1}$ and further, it has no $K_k$ either. Therefore, $G \setminus \{x\}$ is \emph{not} extremal and $e(G \setminus \{x\}) < (k-1)(n-1)/2$. Therefore,
			\begin{align*}
				e(G) &= d(x) + e(G \setminus x) \\
					&< \frac{k-1}{2} + \frac{(k-1)(n-1)}{2} = \frac{(k-1)n}{2}. \qedhere
			\end{align*}
		\end{proof}

\clearpage