\section{Ramsey Theory}

The namesake of Ramsey theory is the British logician Frank Plumpton Ramsey. An infinite form of one of the results was published in a paper on mathematical logic, and was later rediscovered by Erd\H{o}s and Szekeres.

\subsection{Introduction and the Erd\H{o}s-Szkeres Theorem}

	To begin, we give the following folklore proposition.
	\begin{quote}
		Among any 6 people, eithere there are $3$ who are mutual acquaintances of each other or $3$ who are mutual non-acquaintances of each other (being an acquaintance is a symmetric relation).
	\end{quote}

	This was in fact also discovered by a Hungarian sociologist later. The proof is very simple and just boils down to showing that a graph on $6$ vertices has either a size $3$ clique or a size $3$ independent set.\\
	Equivalently, if we colour each edge of $K_6$ blue or red, there is a monochromatic triangle.\\
	This can be proved as follows. Pick any vertex $v$. By the pigeonhole principle, three of its neighbours $u_1,u_2,u_3$ are such that $vu_1,vu_2,vu_3$ are of the same colour, say red. We are done if we show that one of the edges $u_iu_j$ is red as well. If not, $u_1u_2u_3$ is a monochromatic blue triangle, so we are done.\\
	It is also not difficult to see that $6$ is tight (there exists a graph on $5$ vertices without a monochromatic triangle).\\

	This leads to the following more general question.
	\begin{quote}
		Suppose $s,t\in\N$ at least $2$. What is the minimum $N$ (if one exists) such that if each edge of $K_N$ are coloured red or blue, there is either a red $K_s$ or a blue $K_t$.
	\end{quote}


	\begin{ftheo}
		\label{theo: ramsey number finite}
		Given $s,t\in\N$ at least $2$, there exists a quantity $R(s,t) \in \N$ such that for all $n \ge R(s,t)$, any red-blue colouring of $E(K_n)$ admits a red $K_s$ or a blue $K_t$.
	\end{ftheo}

	It is obvious that $R(s,t) = R(t,s)$.

	\begin{proof}
		We prove this by induction on $(s,t)$.\\
		If $s=2$, it is easy to see that $R(2,t) = t$ (similarly, $R(s,2) = s$).\\
		Let $v$ be an arbitrary vertex. Since $v$ has degree $n-1$, either it has $R_1$ red neighbours or $R_2$ blue neighbours for any $R_1,R_2$ such that $R_1 + R_2 = n$ (we shall fix $R_1$ and $R_2$ later). Suppose $x$ has $R_1$ red neighbours. If $R_1 \ge R(s-1,t)$, then we are done by induction. Similarly, if $R_2 \ge R(s,t-1)$, we are done. So, set $R = R(s,t-1) + R(s-1,t)$.
	\end{proof}

	\begin{fcor}
		\label{cor: trivial ramsey bound}
		We have
		\[ R(s,t) \le \binom{s+t-2}{s-1}. \]
	\end{fcor}
	\begin{proof}
		Letting $N(s,t) = \binom{s+t-2}{s-1}$, note that $R(2,t) = N(2,t)$. It may be checked that $N(s,t) = N(s-1,t) + N(s,t-1)$, and we are done by the preceding theorem.
	\end{proof}
	
	More generally, we have the following definition.

	\begin{fdef}
		For integers $s_1,\ldots,s_r$, the \textbf{Ramsey number} $R(s_1,s_2,\ldots,s_r)$ is the minimum $n$ such that for every $r$-colouring of $E(K_n)$ using $[r]$, there is an $i$-monochromatic clique of size $s_i$ for some $i$.
	\end{fdef}

	We have the following.

	\begin{ftheo}
		For $r\in\N$ $s_1,\ldots,s_r \in \N$, $R(s_1,\ldots,s_r)$ is well-defined and moreover,
		\[ R(s_1,\ldots,s_r) \le R(s_1-1,s_2,\ldots,s_r) + R(s_1,s_2-1,s_3,\ldots,s_r) + \cdots + R(s_1,\ldots,s_{r-1},s_r-1). \]
	\end{ftheo}
	The proof of the above is a straightforward generalization of the induction in \Cref{theo: ramsey number finite}.

	Another generalization is that to a \emph{hyper}graph, in which edges are formed not by pairs of vertices, but instead sets of vertices. The complete $r$-hypergraph on $n$ vertices $K_n^r$ is that where the edge set is $\binom{[n]}{r}$. Then, if we colour each edge of this graph red or blue
	\[ R^{(r)}(s,t) = \min\{n : \text{there is a red $K_s^r$ or a blue $K_t^r$} \}. \]
	Of course, we can get a generalization combining both the hypergraph aspect and the multicolour aspect.

	\begin{ftheo}
		$R^{(r)}(s,t)$ is well-defined and further,
		\[ R^{(r)}(s,t) \le R^{(r-1)}(R^{(r)}(s-1,t), R^{(r)}(s,t-1)) + 1. \]
	\end{ftheo}
	The proof of the above is not too difficult.
	% \emph{Sketch of proof when $r=3$}. Given a vertex $v$, any colouring of $K_n^3$ induces a colouring on $\binom{[n]\setminus\{x\}}{2}$ -- an edge $e$ is coloured the same as the colour of $e \cup \{x\}$.\\
	% So, if $|[n]\setminus\{x\}| \ge R(m,k)$ for some $m,k$, then there is either a red $K_m$ or blue $K_k$. if $m \ge R^3(s-1,t)$, say, then if there is a red $K_m$ in the graph, there is either
	% \begin{itemize}
	% 	\item $S$ of size $s-1$ such that all the $3$-tuples of $S$ are red, or
	% 	\item $T\subseteq X$ of size $t$ with all the $3$-tuples blue.
	% \end{itemize}
	% So, either $T$ or $S \cup \{x\}$ does the job.

	Consider the diagonal case where $s=t$. We showed above that $R(3,3) = 6$. It may also be shown that $R(4,4) = 18$. Beeyond, this we do not know any diagonal Ramsey numbers. By \Cref{cor: trivial ramsey bound},
	\[ R(s,s) \le \binom{2s-2}{s-1}. \]
	Recall Stirling's formula
	\[ n! \sim \sqrt{2\pi n} \left( \frac{n}{e} \right)^n. \]
	It may be checked that
	\[ \binom{2s-2}{s-1} \sim c \cdot \frac{4^n}{\sqrt{n}} \]
	for some constant $c>0$.\\
	Finding the exact Ramsey number is incredibly difficult. In the words of Erd\H{o}s,
	\begin{quote}
		Suppose aliens invade the earth and threaten to obliterate it in a year's time unless human beings can find the Ramsey number for red five and blue five. We could marshal the world's best minds and fastest computers, and within a year we could probably calculate the value. If the aliens demanded the Ramsey number for red six and blue six, however, we would have no choice but to launch a preemptive attack.
	\end{quote}

	This begs two questions:
	\begin{itemize}
	 	\item Is there a better upper bound for $R(s,s)$ than this exponential one? 
	 	\item What about a lower bound on $R(s,s)$?
	 \end{itemize}

	 The second of the above questions asks us to give a red-blue colouring of $E(K_n)$ such that there is no monochromatic $K_s$.\\
	 The best known bound for a long time was quadratic, and Tur\'{a}n believed that the Ramsey number itself might be quadratic.\\

	 Blowing this bound out of the water however, Erd\H{o}s gave the first (documented) example of the probabilistic method.

	 \begin{ftheo}
	 	We have
	 	\[ R(s,s) > 2^{s/2}. \]
	 \end{ftheo}
	 \begin{proof}
	 	Fix some $n$ and colour each edge of $K_n$ blue or red with probability $1/2$ each. For a fixed $S \subseteq V(K_n)$ of size $s$,
	 	\[ \Pr[S\text{ is monochromatic}] = \frac{2}{2^{\binom{s}{2}}}. \]
	 	So,
	 	\begin{align*}
	 		\Pr[\text{there is some monochromatic $K_s$}] &= \Pr\left[ \bigcup_{|S|=s} \{S\text{ is monochromatic}\} \right] \\
	 			&\le \binom{n}{s} \cdot \frac{2}{2^{\binom{s}{2}}}.
	 	\end{align*}
	 	So, if
	 	\[ \frac{\binom{n}{s}}{2^{\binom{s}{2}}} < \frac{1}{2}, \]
	 	$n$ is a lower bound for $R(s,s)$. Simplifying the above, we have
	 	\begin{align*}
	 		\frac{\binom{n}{s}}{2^{\binom{s}{2}}} &\le \frac{n^s}{s! 2^{s(s-1)/2}} \\
	 			&\le \left(\frac{n}{2^{(s-1)/2}}\right)^s \cdot \frac{1}{s!}.
	 	\end{align*}
	 	If $n = 2^{s/2}$, the above is $(\sqrt{2})^s / s! < 1/2$ if $s \ge 3$, completing the proof. 
	 \end{proof}

	 Further note that as $s$ grows, since $(\sqrt{2})^s = o(s!)$, if $n = 2^{s/2}$, the probability of there not existing a monochromatic $K_s$ in a random colouring of $E(K_n)$ goes to $0$.\\
	 So, we now have
	 \[ 2^{s/2} < R(s,s) < 4^s. \]

	 For a very long time, $4^s$ was the best known upper bound. In 2008, David Conlon showed in \cite{conlonRamsey} that
	 \[ R(s+1,s+1) \le \binom{2s}{s} \cdot \frac{1}{s^{O(\log s / \log \log s)}}. \]
	 In 2020, Ashwin Sah improved this in \cite{sahRamsey} to
	 \[ R(s+1,s+1) \le \binom{2s}{s} \cdot \frac{1}{s^{O(\log s)}}. \]


	 There is also a geometric motivation for Ramsey Theory.\\
	 A configuration of points on the plane is said to be in general position if no three of them are collinear. Esther Klein (later Esther Szekeres) noticed the following.
	 \begin{enumerate}
	 	\item Given any $5$ points in general position, $4$ of them are the vertices of a convex quadrilateral.
	 	\item Given any $9$ points in general position, $5$ of them are the vertices of a convex pentagon.
	 \end{enumerate}
	 This leads to the following question.\\
	 \begin{quote}
	 	Given $n\in\N$, is there a (finite) $N(n)$ such that any $N(n)$ points in general position have the vertices of a convex $n$-gon?
	 \end{quote}

	 \begin{ftheo}[Erd\H{o}s-Szekeres Theorem]
	 	Given any $n$, there exists a finite $\ES(n)$ such that any $\ES(n)$ vertices in general position admit the vertices of a convex $n$-gon. Furthermore,
	 	\[ \ES(n) \le \binom{2n-4}{n-2} + 1. \]
	 \end{ftheo}

	 The reader might notice that the above is quite similar to the earlier Ramsey number bound!\\

	 \begin{proof}[Proof 1 showing finiteness in the Erd\H{o}s-Szekeres Theorem, due to M. Tarsi]
	 	This proof hinges on the observation that if a (finite) set $X$ of points is such that every $4$ points of $X$ form a convex quadrilateral, then all the points of $X$ form a convex polygon. We do not prove this.\\
	 	Now, we claim that $N(n) = R^{(4)}(n,5)$ works as an upper bound to $\ES(n)$. Let $X$ be our set of $N$ points (for $N > N(n)$) in general position. Colour each $4$-tuple of the points of the set $X$ red or blue depending on whether or not the points form a convex quadrilateral. Since $R^{(4)}(n,5)$ is finite, either there is a set $Y\subseteq X$ of $n$ points such that all the $4$-tuples of $Y$ are red, or there is a size $5$ set $Z\subseteq X$ such that all $4$-tuples of $Z$ are blue.\\
	 	In the former case, the vertices of $Y$ form a convex $n$-gon by the observation in the first paragraph. Further, the latter case cannot occur, due to the observation of Klein's we had given earlier.
	 \end{proof}

	 This bound is \emph{terrible}.

	 \begin{proof}[Proof 2 due to Erd\H{o}s-Szekeres]
	 	Suppose that the points in general position are $p_i = (x_i,y_i)$ for $1\le i\le N$ and also that $x_1 < x_2 < \ldots < x_N$ (the second may be assumed by rotating the plane appropriately).\\
	 	Call a set $C = \{p_{i_1},\ldots,p_{i_k}\}$ a \textbf{$k$-cup} if $i_1 < \cdots < i_k$ and the slope of the segments $p_{i_j} p_{i_{j+1}}$ is non-decreasing. Similarly, call a set $C = \{p_{i_1},\ldots,p_{i_k}\}$ a \textbf{$k$-cap} if $i_1 < \cdots < i_k$ and the slope of the segments $p_{i_j} p_{i_{j+1}}$ is non-increasing.\\
	 	Clearly, a $k$-cup or $k$-cap form a convex $k$-gon.\\

	 	The main result proved by Erd\H{o}s-Szekeres is that for $N > \phi(k,\ell)$, any set of the assumed form above admits either a $k$-cup or an $\ell$-cap, where
	 	\[ \phi(k,\ell) = \binom{k+\ell-4}{k-2}. \]
	 	We prove this by induction on $k+\ell$. If $k=2$ or $\ell=2$, the result is trivial. Let $X$ be a set of size $\phi(k,\ell)+1$, and suppose that $X$ contains neither a $k$-cup nor an $\ell$-cap. Let $L$ be the set of last points of $(k-1)$-cups. In particular, $X \setminus L$ has neither $(k-1)$-cups nor $\ell$-caps. Consequently, $|X\setminus L| \le \phi(k-1,\ell)$, so
	 	\[ L \ge 1 + \phi(k,\ell) - \phi(k-1,\ell) = \phi(k,\ell-1) + 1. \]
	 	Therefore, $L$ contains either a $k$-cup or an $(\ell-1)$-cap. In the former case, we are done. So, suppose that $L$ contains an $(\ell-1)$-cap $\{p_{i_1},\ldots,p_{i_{\ell-1}}\}$. $p_{i_1}$ is the last point of a $(k-1)$-cup, so let the previous point in this cup be $q$.\\
	 	If the slope of $qp_{i_1}$ is greater than the slope of $p_{i_1}p_{i_2}$, then $qp_{i_1}p_{i_2}\cdots p_{i_{\ell-1}}$ forms an $\ell$-cap. Otherwise, the earlier $(k-1)$-cup together with $p_{i_1}$ forms a $k$-cup, completing the proof.
	 \end{proof}

 	The bound of $\phi(k,\ell) + 1$ is tight for the $k$-cup $\ell$-cap problem. That is, there are point configurations of size $\phi(k,\ell)$ points that have neither $k$-cups nor $\ell$-caps.\\
 	This is proved by induction on $k+\ell$. If $k=2$ or $\ell=2$, then $\phi(k,\ell) = 1$. Our counterexample set $S$ will be of the form
 	\[ S = \{ (i,y_i) : 1 \le  i \le \phi(k,\ell) \}. \]
 	If $k=2$ or $\ell=2$, this is $S = \{ (1,0) \}$.\\
 	Let $Y$ and $Z$ be the counterexample sets for $\phi(k-1,\ell)$ and $\phi(k,\ell-1)$ respectively. Let $Y^{(\varepsilon)} = \{ (i,\varepsilon y_i) : 1 \le i \le \phi(k-1,\ell) \}$ and $Z^{(\varepsilon)} = \{ (\phi(k-1,\ell) + i,y + \varepsilon y_i) : 1 \le i \le \phi(k,\ell-1) \}$. Pick $\varepsilon > 0$ small enough and $y$ large enough that any line through two points of $Y{(\varepsilon)}$ lies below the set $Z^{(\varepsilon)}$ and any line through two points of $Z^{(\varepsilon)}$ lies above the set $Y^{(\varepsilon)}$.\\
 	$S = Y^{(\varepsilon)} \cup Z^{(\varepsilon)}$ provides a counterexample. For example, if $S$ has a $k$-cup, there must be at least two points from $Z^{(\varepsilon)}$ (since there is no $(k-1)$ cup in $Y^{(\varepsilon)}$). However, the line joining these two points is above $Y^\{(\varepsilon)\}$ so cannot be part of a $k$-cup.
		 
\subsection{Infinite Ramsey Theory}

	In this section, we present an infinite version of Ramsey's Theorem. This was the original result given by Ramsey.\\

	\begin{ftheo}
		Suppose $V$ is an infinite set. Let $c : V^{r} \to [k]$. Then, there exists an infinite set $A \subseteq V$ such that for all $(v_1,\ldots,v_r) \in A^{r}$, $c(v_1,\ldots,v_r)$ is the same. That is, $A$ is monochromatic.
	\end{ftheo}

	If $V$ is finite, $r=2$, $c=2$, and $c(v_1,v_2) = c(v_2,v_1)$, we are back to the usual Ramsey's Theorem.

	\begin{proof}
		We perform induction on $r$. The result is trivial for $r=1$ by the pigeonhole principle. Let $A_0 = V$ and $x_1 \in A_0$. Consider the induced $k$-colouring $c^*$ on $(A_0 \setminus \{x_1\})^{r-1}$, where
		\[ c^*(v_1,\ldots,v_{r-1}) = c(x_1,v_1,\ldots,v_{r-1}). \]
		By induction, there exists an infinite $A_1 \subseteq A_0$ such that every $(r-1)$-tuple of $A_1$ gets the same colour $\alpha_1$.\\
		Pick $x_2 \in A_1$, and consider the induced colouring of $(r-1)$ tuples of $A_1 \setminus \{x_2\}$ to get $A_2$ such that all $(r-1)$-tuples of $A_2$ get the colour $\alpha_2$.\\
		This leads to a sequence $A_0 \supseteq A_1 \supseteq A_2 \supseteq \cdots$, where for any $(r-1)$-tuple $(v_1,\ldots,v_{r-1})$ from $A_j$, $(x_j,v_1,\ldots,v_{r-1})$ is coloured $\alpha_j$. However, $\alpha_i \in [k]$, so some colour $\alpha$ occurs infinitely many times in the $(\alpha_i)$ Let $i_1 < i_2 < \cdots$ with $\alpha_{i_j} = \alpha$ for all $j$. \\
		Then, the set $\{x_{i_1}, x_{i_2}, \ldots\}$ does the job.
	\end{proof}

	The infinite version of Ramsey's theorem implies the finite one. For this reduction, we prove the following.

	\begin{ftheo}
		Suppose $\mathcal{H} = (V,E)$ is a hypergraph and $V$ is an arbitrary set with all edge sizes finite. Let $k \ge 2$ be an integer and suppose that for all finite $W \subseteq V$, there is a colouring $\chi_W : W \to [k]$ such that no edge $e \subseteq W$ is monochromatic (under $\chi_W$). Then, there exists $\chi : V \to [k]$ such that no edge of $\mathcal{H}$ is monochromatic under $\chi$.
	\end{ftheo}
	\begin{proof}[Proof 1]
		$k$-colorings are precisely the elements of $[k]^V$. Give $[k]$ the discrete topology (all sets are open/closed), and give $[k]^V$ the product topology. By Tychonoff's Theorem, $[k]^V$ is compact (since $[k]$ is compact). For finite $W \subseteq V$, let
		\[ \mathcal{H}_W = \{ f \in [k]^V : \text{no $e\subseteq W$ is monochromatic under $f$} \}. \]
		Observe that $\mathcal{H}_W \ne \emptyset$ (since $\chi_W \in \mathcal{H}_W$) and further, $\mathcal{H}_W$ is closed in $[k]^V$ (since $W$ is finite). If $W_1, W_2, \ldots, W_\ell$ are finite, then
		\[ \mathcal{H}_{W_1} \cap \mathcal{H}_{W_2} \cap \cdots \cap \mathcal{H}_{W_\ell} \supseteq \mathcal{H}_{W_1 \cup \cdots \cup W_\ell}. \]
		Since the $W_i$ are finite, so is $W_1 \cup \cdots \cup W_\ell$. It follows that $\mathcal{H}_{W_1} \cap \cdots \cap \mathcal{H}_{W_\ell}$ is non-empty.\\
		In particular, the collection $\{\mathcal{H}_W\}_{W\text{ finite}}$ has the closed intersection property. This implies that $\bigcap_{|W| < \infty} \mathcal{H}_W$ is non-empty. Any $\chi$ belonging to this set does the job, completing the proof.
	\end{proof}

	Let us give another proof in a specific case.

	\begin{proof}[Proof 2 when $V = \N$]
		Suppose that for each $n\in\N$, we have $\chi_n : [n] \to [k]$ such that no edge $e \subseteq [n]$ is monochromatic with respect to $\chi_n$. We shall define $\chi^* : \N \to [k]$ such that no edge is monochromatic with respect to $\chi^*$.\\
		Let $A_0 = \N$. There exists some infinite set $A_1 \subseteq A_0$ such that $\chi_n(1) = \alpha_1$ for all $n \in A_1$. Set $\chi^*(1) = \alpha_1$. In general, there exists some infinite $A_{j} \subseteq A_{j-1}$ such that $\chi_n(j) = \alpha_j$ for all $n \in A_j$. Set $\chi^*(j) = \alpha_j$. Consider an edge $e = \{i_1,\ldots,i_r\}$ and suppose $i_1 < i_2 < \cdots < i_r$. Since $\chi^*$ agrees with $\chi_n$ on $[i_r]$ for some $n \ge i_r$ and $\chi_j$ has no monochromatic edges, $e$ is not monochromatic under $\chi^*$ either. 
	\end{proof}

	The contrapositive of this result implies the finite Ramsey result.\\

	Ramsey's proof shows that given an infinite set $A$ and $\binom{A}{r}$ is $[k]$-colored, there is an infinite (countable!) subset $B \subseteq A$ such that $\binom{B}{r}$ is monochromatic. Can we guarantee that $|B| = |A|$? It turns out that we cannot, due to a proof by Sierpinski in 1932.\\
	Let $A = \R$ and $k=r=2$. It is known (assuming the axiom of choice) that any non-empty set has a well-order $<^*$ (every non-empty subset has a least element).\\
	Colour $xy$ red if $(x<y \iff x<^* y)$ and blue otherwise. We claim that under this colouring, any monochromatic infinite set is countable. The colouring is such that for a red set $A$, the two orders $<$ and $<^*$ agree on $A$. That is, $<$ is a well-order on $A$. So, let $a_0$ be the least element of $A$, and in general, $a_{i+1}$ be the least element of $A \setminus \{a_0,a_1,\ldots,a_i\}$. Pick a rational $q_i \in (a_i,a_{i+1})$. Since the rationals are countable, a red monochromatic set is countable. If $A$ is blue, then $<^*$ and $>$ agree on $A$ so the same reasoning applies, thus proving that any monochromatic set is countable. 