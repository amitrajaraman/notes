\documentclass{article}
\usepackage[T1]{fontenc}
\usepackage[utf8]{inputenc}
\usepackage{ma-5109}

\begin{document}

\thispagestyle{empty}

\titleBC
\tableofcontents
\clearpage

\setcounter{section}{-1}

\section{Notation and Prerequisites}
	
	Given $n \in \N$, $[n]$ denotes the set $\{1,\ldots,n\}$ and $[n]_0$ denotes the set $[n] \cup \{0\}$.\\
	$S(n,k)$, a Stirling number of the second kind, is the number of partitions of $[n]$ into exactly $k$ parts. $s(n,k)$, a Stirling number of the first kind, is the number of permutations of $[n]$ with exactly $k$ cycles.\\
	
\clearpage
\section{Introduction}

\begin{exercise}
	Recall that the number of $k$-subsets of $[n]$ is $\binom{n}{k}$. Given a $k$-subset $S = \{x_1,\ldots,x_k\}$ of $[n]$, we write $S_< = \{x_1,\ldots,x_k\}_<$ to denote that $x_1<x_2<\cdots<x_k$. Determine the number of $k$-subsets $\{x_1,\ldots,x_k\}_<$ of $[n]$ such that $x_i \cong i \mod 2$.
\end{exercise}
For example, for $n=6$ and $k=3$, we have the subsets $\{1,4,5\},\{1,2,3\},\{1,2,5\},\{3,4,5\}$.


Broadly, there are three types of ``answers'': a formula, a recurrence, and a generating function. A great example of the second and third is the following.\\
% Euler's Theorem
$p(n)$, the number of number partitions of $n$, is given by the generating function
\[ \sum_{n \ge 0} p(n) x^n = \prod_{i \ge 1} \frac{1}{1-x^i}. \]
Using this, a recursion may be obtained as well.
We do \emph{not} plug in values for $x$ in the above. We merely look at the coefficient of $x^n$ in it. We want the coefficient to be a finite sum for all $n$. If it is an infinite sum, convergence issues may arise.

% www.math.iitb.ac.in/~krishnan/phd-2022/

% The number of derangements of $[n]$ is the integer closest to $n!/e$.
% Bell number B_n (number of set partitions of $[n]$) : \sum_{n \ge 0} B_n x^n / n! = e^{e^x - 1}.

\subsection{Counting in \texorpdfstring{$\mathfrak{S}_n$}{Sn}}

	Recall that $\mathfrak{S}_n$ is generated by transpositions. A transposition $(i,j)$ is a permutation $\sigma$ defined by
	\[ \sigma(k) = \begin{cases} j, & k=i, \\ i, & k=j, \\ k, & \text{otherwise.} \end{cases} \]
	In fact, $\mathfrak{S}_n$ is generated by the set of just ``adjacent transpositions'' $S_i = (i,i+1)$ for $1 \le i < n$.
	We have
	\begin{align*}
		S_i^2 &= \Id \\
		S_i S_{i+1} S_i &= S_{i+1} S_i S_{i+1} \\
		S_i S_j &= S_j S_i \text{ if $|i-j| > 2$.}
	\end{align*}


	\begin{fdef}
		Given a permutation $\pi \in \mathfrak{S}_n$, define the \emph{length} $\ell(\pi)$ of $\pi$ to be the smallest $k$ such that there exist adjacent transpositions $\sigma_1,\cdots,\sigma_k$ such that $\pi = \sigma_1\cdots\sigma_k$.
	\end{fdef}
	% PRESENT PROOF THAT WE CAN WRITE IT IN THIS WAY ON TUESDAY -- 02/08/2022
	% DONE.

	\begin{fprop}
		Consider the \emph{inversion number} $\inv(\pi)$ of a permutation, defined by
		\[ \inv(\pi) = \left|\{ 1 \le i \le j \le n : \pi_i > \pi_j \}\right|. \]
		Then, $\ell(\pi) = \inv(\pi)$.
	\end{fprop}

	\begin{fdef}
		The \emph{sign} of a permutation $\pi$ is defined by $\sign(\pi) = (-1)^{\inv(\pi)}$. Equivalently,
		\[ \sign(\pi) = \frac{\prod_{1 \le i < j \le n} (x_{pi_i} - x_{\pi_j})}{\prod_{1 \le i < j \le n} (x_i - x_j)}. \]
	\end{fdef}

	% LATER, STUDY NUMBER OF ``REDUCED'' PRODUCTS? (NUMBER OF WAYS IN WHICH PI CAN BE WRITTEN AS A PRODUCT OF L(PI) ADJ TRANSPOSITIONS)


	It is straightforward to see that for all $\pi \in \mathfrak{S}_n$, $0 \le \inv(\pi) \le \binom{n}{2}$.
	\begin{fprop}
		\label{prop: inv n}
		Consider $\inv_n(q) = \sum_{\pi \in \mathfrak{S}_n} q^{\inv(\pi)}$. Then,
		\[ \inv_n(q) = \prod_{1 \le m \le n} [n]_q, \]
		where
		\[ [m]_q = \begin{cases} 1 + q + \cdots + q^{m-1}, & m \ge 1, \\ 0, & m = 0. \end{cases} \]
	\end{fprop}
	This quantity $[m]_q$ is called the $q$-analogue of $m$, and similarly, the $q$-analogue of $n!$ is $\prod_{i=1}^n [m]_q$ (this is slightly vague).
	Note in particular that $n! = \inv_n(1)$.
	\begin{proof}
		We prove this by induction. It is easily verified for $n = 2$.\\
		Take $\sigma \in \mathfrak{S}_{n-1}$. There are $n$ ``gaps'' where $n$ can be ``placed'' in $\sigma$ to get a permutation in $\mathfrak{S}_n$. If we place it in the $i$th position from the end (for $0 \le i \le n-1$), the inversion number of the newly obtained permutation is $i$ more than the inversion number of $\sigma$.\\
		As a result,
		\[ \inv_n(q) = \inv_{n-1}(q) + q \inv_{n-1}(q) + q^2 \inv_{n-1}(q) + \cdots + q^{n-1} \inv_{n-1}(q) = [n]_q \inv_{n-1}(q), \]
		where the $q^i \inv_{n-1}(q)$ term corresponds to the case where $n$ is placed in the $i$th position from the end. The required follows by the inductive hypothesis.
	\end{proof}

	\begin{definition}[Descent]
		For $\pi \in \mathfrak{S}_n$, define the \emph{descents} $\DES(\pi) = \{i \in [n-1] : \pi_i > \pi_{i+1}\}$, $\des(\pi) = |\DES(\pi)|$, and $\maj(\pi) = \sum_{i \in \DES(\pi)} i$. 
	\end{definition}
	There are central limit theorems for many of these parameters, which we shall not study.\\
	A permutation $\pi$ has $\des(\pi) + 1$ many ``increasing runs''.

	For example, for the permutation $\pi = (1\mapsto 5,2\mapsto1,3\mapsto2,4\mapsto6,5\mapsto4,6\mapsto3) \in \mathfrak{S}_6$, $\DES(\pi) = \{1,4,5\}$, $\des(\pi) = 3$, and $\maj(\pi) = 10$.

	\begin{fprop}
		\label{prop: maj n}
		The distribution of $\maj(\pi)$ over $\mathfrak{S}_n$ is the same as that of $\inv(\pi)$. Equivalently,
		\[ \maj_n(q) = \sum_{\pi \in \mathfrak{S}_n} q^{\maj(\pi)} = \prod_{m=1}^n [m]_q = \inv_n(q). \] 
	\end{fprop}
	This result took nearly 50 years to prove!
	\begin{proof}
		The strategy is similar to that of \Cref{prop: inv n}. Let $\pi \in \mathfrak{S}_{n-1}$. As before, there are $n$ positions to insert $n$.
		% An insertion increases $\maj$ (compared to $\pi$) only if it is placed in the middle of an increasing run. That is, we must place it  Further, in this case, it increases $\des$ by $1$. What is the sum of $\maj$ increases over all possible insertions? 
		\begin{itemize}
			\item Label the positions of descents of $\sigma$ and the last position from right ot left as $0,1,\ldots,\des(\pi)$.
			\item Label the remaining positions from left to right as $\des(\pi)+1,\ldots,n-1$.
		\end{itemize}
		We claim that inserting $n$ at a position increases $\maj$ by the labelled amount.\\
		If inserted anywhere, all the descent positions starting from there increase by $1$. This explains why the increase is equal to the labelled quantity for positions that are descents, since no new descents are introduced. In the case where we insert it in a position of non-descent, we further introduce a new descent at the position of insertion of $n$, which explains why the increase is equal tot he labelled quantity for positions that are not descents.\\
		The remainder of the proof is identical to that of \Cref{prop: inv n}, since the increases are in bijection with $[n-1]_0$.
	\end{proof}

	\begin{definition}
		A parameter $f : \mathfrak{S}_n \to \R$ of permutations such that
		\[ \sum_{\pi \in \mathfrak{S}_n} q^{f(\pi)} = \prod_{1 \le m \le n} [m]_q \]
		is said to be \emph{Mahonian}.
	\end{definition}

	As we saw in \Cref{prop: inv n,prop: maj n}, both $\inv$ and $\maj$ are Mahonian.

\subsection{Counting spanning trees}

	\begin{problem*}
		Count the number of spanning trees in an arbitrary (finite) graph $G$.
	\end{problem*}

	This was solved by Kirchhoff using the Matrix Tree Theorem.
	
	\begin{ftheo}[Matrix Tree Theorem]
		Consider the \emph{Laplacian} $L = D-A$ of a graph $G$, where $A$ is its adjacency matrix and $D$ is a diagonal matrix with the diagonal entries being the degrees of the vertices. The determinant of any $(n-1)\times(n-1)$ submatrix of $L$ obtained by omitting any arbirary row and column is equal to the number of spanning trees of $G$.
	\end{ftheo}

	% oriented incidence matrix is totally unimodular -- poincare

	In particular, when $G = K_n$, we end up getting the following.

	\begin{ftheo}[Cayley's Theorem]
		\label{theo: cayleys theorem}
		The number of spanningtrees in $K_n$ is $n^{n-2}$.
	\end{ftheo}
	One proof by Pr\"{u}fer gives an explicit bijection between spanning trees and sequences $(v_1,\ldots,v_{n-2})$ of vertices in $G$.\\
	Another proof is of course using the matrix tree theorem, which reduces it to a simple determinant calculation.\\
	Joyal gave another bijection between elements of the form $(T,u,v)$ where $T$ is a spanning tree and $u,v$ are vertices in $G$, and functions from $[n] \to [n]$.\\

	The proof we give uses exponential generating functions. Recall the following result, which we give without proof. Interested readers may consult Corollary 5.1.6 of \cite{ec2} for further details.

	\begin{ftheo}[Exponential Formula]
		\label{ec2: egf sum over partitions}
		Let $\{f_n\}$ be a sequence with exponential generating function
		\[ F(x) = \sum_{n \ge 1} f_n \frac{x^n}{n!}. \]
		Define the sequence $h_n$ by
		\[ h_n = \sum_{\substack{\pi \in \operatorname{SetPartn}([n]) \\ \pi = \{S_1,\ldots,S_k\}}} f_{|S_1|} f_{|S_2|} \cdots f_{|S_k|}  \]
		and $h_0 = 1$, and let
		\[ H(x) = \sum_{n \ge 0} h_n \frac{x^n}{n!}. \]
		Then,
		\[ H(x) = \exp(F(x)) \]
	\end{ftheo}
	Note that the summation of $F$ is for $n \ge 1$, because we may assume that $f_0 = 0$ since $f_0$ does not appear in the expression of any $h_n$.\\

	\begin{fdef}[Compositional inverse]
		Generating functions $F$ and $G$ are said to be \emph{compositional inverses} (of each other) if $F(G(x)) = G(F(x)) = x$.
	\end{fdef}

	Let
	\[ F(x) = \sum_{n \ge 0} f_n x^n \text{ and } G(x) = \sum_{n \ge 0} g_n x^n \]
	be compositional inverses of each other. It is reasonably straightforward to show that $f_0 = g_0 = 0$ and $f_1,g_1 \ne 0$. The first condition implies that the coefficient of any $x^n$ in $F\circ G$ (or $G \circ F$) is finite. 

	\begin{ftheo}[Lagrange Inversion Theorem]
		\label{theo: lagrange inversion}
		Let
		\[ F(x) = \sum_{n \ge 0} f_n x^n \text{ and } G(x) = \sum_{n \ge 0} g_n x^n \]
		be compositional inverses of each other. Then, $ng_n$ is the coefficient of $1/x$ in $(1/F(x))^n$.
	\end{ftheo}
	Equivalently, $ng_n$ is the coefficient of $x^{n-1}$ in $(x/F(x))^n$.
	\begin{proof}
		We have
		\[ x = G(F(x)) = \sum_{i \ge 0} g_i F(x)^i. \]
		Differentiating,
		\[ 1 = \sum_{i \ge 0} g_i i F(x)^{i-1} F'(x). \]
		As a result,
		\[ \left(\frac{1}{F(x)}\right)^n = \sum_{i \ge 0} g_i i F(x)^{i-1-n} F'(x).  \]
		Whenever $i \ne n$, the coefficient of $1/x$ in $F(x)^{i-1-n} F'(x) = \left(F(x)^{i-n}/(i-n)\right)'$ is zero. Indeed, recall that the coefficient of $1/x$ in the derivative of any power series with possibly negative exponents is zero.\\
		As a result, the coefficient of $1/x$ in $(1/F(x))^n$ is equal to the coefficient of $1/x$ in $g_n n F'(x) / F(x)$.
		% , which is the constant term in $n g_n x F'(x) / F(x)$ -- observe that there cannot be a $x^{-m}$ term for any $m \ge 2$.
		We have
		\[ \frac{F'(x)}{F(x)} = \frac{f_1 + 2f_2x + \cdots}{f_1x + f_2x^2 + \cdots}. \]
		The constant term in this is $f_1/f_1 = 1$, and the desideratum follows. 
	\end{proof}

	At long last, let us return to \nameref{theo: cayleys theorem}.

	\begin{proof}[Proof of \nameref{theo: cayleys theorem}]
		Instead of looking at the number $\text{T}_n$ of spanning trees, we shall look at $\text{RT}_n$, the number of \emph{rooted} spanning trees. Clearly, $\text{RT}_n = n \text{T}_n$.\\
		Define $\text{RF}_n$ to be the number of rooted forests on $[n]$ and let
		\begin{align*}
			\operatorname{RF}(x) &= \sum_{n \ge 0} \text{RF}_n \frac{x^n}{n!} \\
			\operatorname{RT}(x) &= \sum_{n \ge 0} \text{RT}_n \frac{x^n}{n!}.
		\end{align*}

		Using \Cref{ec2: egf sum over partitions}, it is not too difficult to see that
		\begin{equation}
			\label{eqn 1}
			\operatorname{RF}(x) = \exp(\operatorname{RT}(x)).
		\end{equation}

		\textbf{Claim} (Polya). $\text{RT}_{n+1} = (n+1) \text{RF}_n$.\\
		Indeed, any rooted tree on $K_{n+1}$ may be obtained from a rooted forest $F$ on $K_n$ by adding a new vertex $v$, adding the edge between each root in $F$ and $v$ to the spanning tree, removing the ``root status'' from all vertices except $v$. $v$ can be labelled in $n+1$ ways, so we are done.\\

		As a result,
		\begin{equation}
			\label{eqn 2}
			\operatorname{RF}(x) = \sum_{n\ge 0} \frac{\operatorname{RT}_{n+1}}{n+1} \cdot \frac{x^n}{n!} = \frac{1}{x} \operatorname{RT}(x).
		\end{equation}
		Combining \Cref{eqn 1,eqn 2},
		\[ \operatorname{RT}(x) = x \exp(\operatorname{RT}(x)). \]
		That is, $\operatorname{RT}$ is the compositional inverse of $x \mapsto xe^{-x}$.
		Now, we use the \nameref{theo: lagrange inversion} to get that $n\text{RT}_n/n!$ is equal to the coefficient of $x^{n-1}$ in $(x/xe^{-x})^n = e^{nx}$, which is $n^{n-1}/(n-1)!$. Therefore, $\text{T}_n = \text{RT}_n/n = n^{n-2}$ and we are done.
	\end{proof}

	% cool fact: matching polynomial of any graph has real roots.

\subsection{Chebyshev polynomials}

	We would like a polynomial $T_n(x)$ such that $T_n(\cos\theta) = \cos(n\theta)$. Why does such a polynomial even exist? Recall that
	\[ (\cos \theta + \iota \sin \theta)^n = \cos n\theta + \iota \sin n\theta. \]
	Since the real part of the left only has even powers of $\sin$, we can convert it to a polynomial of $\cos\theta$s alone.\\
	For example,
	\begin{align*}
		T_0(x) &= 1, \\
		T_1(x) &= x, \\
		T_2(x) &= 2x^2-1.
	\end{align*}

	\begin{fprop}
		\label{prop: recurrence of Tn}
		\[ T_n(x) = \begin{cases} 1, & n=0, \\ x, & n=1, \\ 2xT_{n-1}(x) - T_{n-2}(x), & n \ge 2. \end{cases} \]
	\end{fprop}
	We give the proof of the above in the solution to \Cref{problem: recurrence of Chebyshev of the first kind}.

	\begin{fprop}
		\label{problem: recurrence of Chebyshev of the first kind}
		$T_0(x) = 1$, $T_1(x) = x$, and for $n \ge 2$,
		\[ T_n(x) = 2xT_{n-1}(x) - T_{n-2}(x). \]
	\end{fprop}
	\begin{proof}
		Let $\cos \theta = x$. We have
		\begin{align*}
			T_n(x) = \cos n\theta &= \cos (n-1)\theta \cos \theta - \sin (n-1)\theta \sin \theta \\
				&= x T_{n-1}(x) - ( \sin (n-2)\theta \cos\theta + \cos (n-2)\theta \sin\theta ) \sin\theta \\
				&= x T_{n-1}(x) - T_{n-2}(x) (1-x^2) - x (\sin\theta \sin (n-2)\theta) \\
				&= x T_{n-1}(x) + x^2 T_{n-2}(x) - T_{n-2}(x) - x (\cos\theta \cos (n-2)\theta - \cos (n-1)\theta) \\
				&= 2x T_{n-1}(x) - T_{n-2}(x). \qedhere
		\end{align*}
	\end{proof}


	% % PSHEET 2 EXERCISE 1
	% \begin{ftheo}
	% 	Show that
	% 	\[ T_n(x) = 2xT_{n-1}(x) - T_{n-2}(x). \]
	% 	when $n \ge 2$, and $T_0(x) = 1, T_1(x) = x$.
	% \end{ftheo}

	\begin{fdef}[Chebyshev polynomials]
		The $n$th \emph{Chebyshev polynomial of the first kind} $T_n$ is defined as above.\\
		The $n$th \emph{Chebyshev polynomial of the second kind} $U_n$ is defined by
		\[ U_n(x) = \begin{cases} 1, & n=0, \\ 2x, & n=1, \\ 2xU_{n-1}(x) - U_{n-2}(x), & n \ge 2. \end{cases} \]
	\end{fdef}


	% \begin{exercise}
	% 	Show that
	% 	\begin{enumerate}
	% 		\item \phantom{pain}
	% 		\begin{enumerate}[label=\alph*)]
	% 			\item $T_n(1) = 1$.
	% 			\item $T_n(-1) = (-1)^n$
	% 		\end{enumerate}
	% 		\item \phantom{pain}
	% 		\begin{enumerate}[label=\alph*)]
	% 			\item $U_n(1) = n+1$.
	% 			\item $U_n(-1) = (-1)^n(n+1)$
	% 		\end{enumerate}
	% 		\item $U_n(\iota/2)/\iota^{n}$ is the $(n+1)$th Fibonacci number.
	% 		\item if $m,n \ge 1$,
	% 		\[ T_{m+n}(x) = T_m(x)U_n(x) - T_{m-1}(x)U_{n-1}(x) \]
	% 	\end{enumerate}
	% \end{exercise}

	Consider the number of tilings of a $1\times n$ board $B_n$ using squares ($1\times 1$ pieces) and dimers ($1\times 2$ pieces). It is not too difficult to show that this corresponds to the Fibonacci numbers.\\
	Now, instead consider a \emph{weighted} version of this problem, where we give squares a weight of $2x$ and dimers a weight of $-1$. The weight $\wt(T)$ of a given tiling $T$ is equal to the product of the weights of the pieces used. Then, the Chebyshev polynomial $U_n$ is just the sum of the weights of all tilings of $B_n$!
	\[ U_n(x) = \sum_{\text{tilings $T$ of $B_n$}} \wt(T). \]
	Similar to this, we can get a combinatorial model for $T_n$ as well, with the only difference being that a square piece has weight $x$ if it is at the lefmost $(1,1)$ position.

	Given a tiling $T$, let $S(T)$ and $D(T)$ be the number of squares and dimers in the tiling respectively.
	In general, define
	\[ F_n(s,t) = \sum_{\text{tilings $T$ of $B_n$}} s^{S(T)} t^{D(T)}. \]

	Then,
	\begin{align*}
		F_0(s,t) &= 1, \\
		F_1(s,t) &= s, \\
		F_n(s,t) &= sF_{n-1}(s,t) + tF_{n-2}(s,t).
	\end{align*}

\subsection{More on $q$-analogues}

	Recall the definition of $[n]_q! = \prod_{i=1}^n [i]_q$. Inspired by this, define
	\[ \binom{n}{k}_q = \frac{[n]_q!}{[k]_q[n-k]_q!}. \]
	This is clearly a rational function of $q$. It turns out that this is a polynomial in $q$! For example,
	\[ \binom{5}{2}_q = \frac{[5]_q[4]_q}{[2]_q[1]_q} = 1 + q + 2q^2 + 2q^3 + 2q^4 + q^5 + q^6. \]

	Recall that
	\[ \binom{n}{k} = \binom{n-1}{k-1} + \binom{n-1}{k}. \]

	\begin{exercise}
		Show that
		\[ \binom{n}{q}_q = q^k \binom{n-1}{k}_q + \binom{n-1}{k-1}_q = \binom{n-1}{k}_q + q^{n-k} \binom{n-1}{k-1}_q . \]
	\end{exercise}
	These are called the $q$-Pascal's recurrences.

	\begin{fcor}
		$\binom{n}{k}_q$ is a polynomial. in $q$ with non-negative coefficients.
	\end{fcor}
	It turns out that the coefficients of the polynomial are unimodal and symmetric! We do not prove this, the reader can see \_ for more details.
	% zeinberger, AMM (katyohara?)

	% sperner's theorem(?) : maximal UNIQUE antichain is the binom(n)(n/2) in the middle of the poset

	A natural question to ask then is: what do the coefficients of this polynomial count?\\
	Let $\binom{n}{k}_q = f_{n,k}(q) = \sum_{r \ge 0} a_{n,k}^{(r)} q^r$.
	Can we have
	\[ \binom{n}{k}_q = \sum_{T \in \binom{[n]}{k}} q^{\operatorname{parameter}(T)}? \]
	$a_{n,k}^{(r)}$ then just counts the number of $T$ with the given parameter value.\\
	Recall that $\binom{n}{k}$ is the number of paths from $(0,0)$ to $(n-k,k)$ if only upwards and rightwards movements on the integer lattice $\Z^2$ are allowed. Let $P$ be such a path.\\
	% \prod q^{(x_{i+1}-x_i-1)}
	% number partitions -- ``ferrer's diagram''
	Consider the portion of the box above $P$. This can be viewed as the Ferrer diagram of some number partition $\lambda(P)$. $\lambda(P)$ has at most $k$ parts, and no part is of size more than $n-k$. In fact, \emph{all} such partitions correspond to some path!\\
	What number is $\lambda(P)$ a number partition of? Denote this number as $|\lambda(P)|$. Let $S_{n,k}$ be the set of all paths of the mentioned form. 

	\begin{ftheo}
		\[ \sum_{P \in S_{n,k}} q^{|\lambda(P)|} = \binom{n}{k}_q. \]
	\end{ftheo}
	Surprisingly, the proof of the above is near-straightforward using the $q$-Pascal recurrence -- merely consider two cases depending on whether the first step of the path is right or upwards.

\subsection{Derivative polynomials}

	We begin this section by recalling the following rather interesting result.\\
	Define the \emph{Bell polynomial} $B_{n,k}$ by
	\[ B_{n,k}(x_1,x_2,\ldots,x_{n-k+1}) = \sum \frac{n!}{j_1! (1!)^{j_1} j_2! (2!)^{j_2} \cdots j_{n-k+1}! ((n-k+1)!)^{n-k+1}} \cdot x_1^{j_1} x_2^{j_2} \cdots x_{n-k+1}^{j_{n-k+1}}, \]
	where the summation is taken over all indices $j_1,\ldots,j_{n-k+1}$ of non-negative integers such that
	\begin{align*}
		k &= j_1 + j_2 + \cdots + j_{n-k+1} \text{ and} \\
		n &= j_1 + 2j_2 + 3j_3 + \cdots + (n-k+1)j_{n-k+1}.
	\end{align*}
	This has a natural correspondence to the Stirling numbers of the second kind, with $j_i$ representing the number of partitions of size $i$. In particular, the sum of coefficients of $B_{n,k}$ is $S_{n,k}$.

	\begin{fprop}[Fa\`{a} di Bruno's Formula,\cite{faadibruno}]
		\[ D^n f(g(x)) = \sum f^{(k)}(g(x)) \cdot B_{n,k} (g'(x), g''(x), \ldots, g^{(n-k+1)}(x)). \]
	\end{fprop}

	To illustrate this better, let us look at the first few derivatives explicitly. Dropping the $(x)$ on the right to make the notation more succinct, we have
	\begin{align*}
		Df(g(x)) &= f'(g) g' \\
		D^2f(g(x)) &= f''(g) (g')^2 + f'(g) g'' \\
		D^3f(g(x)) &= f'''(g) (g')^3 + 3 f''(g) g' g'' + f''(g) g'''.
	\end{align*}
	Consider the partitions of $\{1,2,3\}$, given by $1|2|3$, $12|3$, $13|2$, $23|1$, and $123$. The number of partitions of $[n]$ with $n_i$ parts of size $i$ for each $i$ neatly corresponds to the coefficient of $\prod_i (g^{(i)})^{n_i}$!

	% Faa di Bruno's formula
	% Df(g(x)) = f'(g) g'
	% D^2 = f''(g)(g')^2 + f'(g)g''
	% D^3 = f'''(g) (g')^3 + 3 f''(g)(g')(g'') + f''(g)g''' -- 1|2|3 ; 12|3 13|2 12|3 ; 123
	% coefficients relate to Bell numbers!
	% even a multivariate version that relates to multivariate partitions (?)

	Let $y = f(x)$. If $Dy = p(f(x))$ for some polynomial $p$, then $D^ny$ is a polynomial of $f$ as well.\\
	Suppose that $D^ny = p_n(y)$ for some sequence of polynomials $(p_n)$.
	\begin{exercise}
		Show that
		\begin{align*}
			p_0(y) &= y \\
			p_n(y) &= \begin{cases} y & n=0 \\ p_{n-1}(y) \cdot p_1(y) & n \ge 1. \end{cases}
		\end{align*}
	\end{exercise}

	For the remainder of this section, set $y = \tan x$ and $z = \sec x$. Then, $Dy = 1+y^2 = z^2$ and $Dz = yz$. It is not difficult to see that
	\begin{align*}
		D^2y &= 2yz^2 \\
		D^3y &= 4y^2z^2 + 2z^4 \\
		D^4y &= 8y^3z^2 + 16yz^4
	\end{align*}

	\begin{exercise}
		With $y,z$ defined as above, show that
		\begin{enumerate}
			\item $D^ny$ is a homogeneous polynomial in $y,z$ of degree $(n+1)$.
			\item $D^ny$ has only terms with even exponents of $z$.
		\end{enumerate}
	\end{exercise}

	\begin{fcor}
		We can write $D^ny = \sum_{k=0}^{\lfloor (n-1)/2 \rfloor} W_{n,k} z^{2k+2} y^{n-2k-1}$.
	\end{fcor}

	\begin{exercise}
		Prove or disprove that
		\[ \sum W_{n,k} = n!. \]
	\end{exercise}
	Again, we ask the question: is there some parameter on $\pi \in \mathfrak{S}_n$ such that
	\[ W_n(x) = \sum_{k = 0}^{\lfloor (n-1)/2 \rfloor} W_{n,k} x^k = \sum_{\pi \in \mathfrak{S}_n} x^{\operatorname{parameter}(\pi)}? \]

	\begin{fdef}[Peak]
		Given a permutation $\pi \in \mathfrak{S}_n$, we say that $i \in [n] \setminus \{1,n\}$ is a \emph{peak} of $\pi$ if $\pi_i > \pi_{i-1}$ and $\pi_i > \pi_{i+1}$. Denote the set of peaks of $\pi$ by $\Peak(\pi)$, $\pk(\pi) = |\Peak(\pi)|$ the number of peaks.
	\end{fdef}

	\begin{lemma}
		\label{lem: Wnk recurrence}
		Show that
		\[ W_{n,k} = (2k+2) W_{n-1,k} + (n-2k) W_{n-1,k-1}. \]
		for $n,k \ge 1$.
	\end{lemma}
	\begin{proof}
		We have
		\begin{align*}
			D^{n}y &= D \sum_{k=0}^{\lfloor (n-2)/2 \rfloor} W_{n-1,k} z^{2k+2} y^{n-2k-2} \\
				&= \sum_{k=0}^{\lfloor (n-2)/2 \rfloor} (2k+2) W_{n-1,k} z^{2k+1} \cdot zy \cdot y^{n-2k-2} + (n-2k-2) W_{n-1,k} z^{2k+2} y^{n-2k-3} \cdot z^2 \\
				&= \sum_{k=0}^{\lfloor (n-2)/2 \rfloor} (2k+2) W_{n-1,k} z^{2k+2} y^{n-2k-1} + (n-2k-2) z^{2k+4} y^{n-2k-3}.
		\end{align*}
		The required follows.
	\end{proof}

	\begin{ftheo}
		\[ W_n(x) = \sum_{\pi \in \mathfrak{S}_n} x^{\pk(\pi)}. \]
	\end{ftheo}
	\begin{proof}
		Let $Y_n$ be the polynomial on the right, and let $Y_{n,k}$ be its coefficients. It is easily checked that $Y_{n,k}$ and $W_{n,k}$ are equal for $n=0$ or $k=0$. To prove the statement, we shall merely show that $Y_{n,k}$ satisfies the recurrence of \Cref{lem: Wnk recurrence} too.\\
		Similar to what we did in earlier proofs such as those of \Cref{prop: inv n,prop: maj n}, let $\sigma$ be a permutation in $\mathfrak{S}_{n-1}$.\\ We shall use it to get a permutation $\pi \in \mathfrak{S}_n$ by ``inserting'' $n$ at one of the $n$ possible positions. If we insert it at the position of a non-peak of $\sigma$, the number of peaks increases by one. If we insert it before or after the position of a peak, the number of peaks stays the same. Since peaks cannot occur immediately after each other, we can insert it at precisely $2k+2$ positions while ensuring that the number of peaks does not increase (the extra $2$ is for the extreme positions), and so at $n-2k-2$ positions which increases the number of peaks by one. Therefore,
		\[ Y_{n,k} = (2k+2)Y_{n-1,k} + (n-2(k-1)-2)Y_{n-1,k-1} = (2k+2) Y_{n-1,k} + (n-2k)Y_{n-1,k-1}. \]
	\end{proof}

	% \begin{problem}
	% 	\begin{enumerate}
	% 		\item Show that
	% 		\[ W_{n,k} = \_ W_{n-1,k} + \_ W_{n-1,k-1}. \]
	% 	\end{enumerate}
	% \end{problem}

	% z = sec x, Dz = yz.

\subsection{Matching theory}

	\begin{fdef}[Matching]
		Given a graph $G = (V,E)$, a \emph{matching} in $G$ is a collection $M \subseteq E$ of edges such that for any distinct $e_1,e_2 \in M$, $e_1 \cap e_2 = \emptyset$.\\
		The number of $k$-sized matchings is denoted $m_k(G)$. Define the \emph{matching polynomial}
		\[ \Match_G(x) = \sum_{k \ge 0} (-1)^k m_k(G) x^{n-2k} \]
		% \[ \Match_G(x) = \sum_{k \ge 0} m_k(G) x^{k}. \]
	\end{fdef}
	Some books call the above the ``defect'' matching polynomial, taking the actual matching polynomial as $p(x) = \sum_k m_k(G) x^k$. Note that $\Match_G(x) = x^n p(-1/x^2)$. 
	% Some authors also define the polynomial as
	% \[ N_G(x) = \sum_{k \ge 0} (-1)^k m_k(G) x^{n-2k}. \]
	
	Clearly, $N_G(x) = x^n \cdot \Match_G(-1/x^2)$.\\

	There is a very rich literature regarding matching theory. One work that set off a frenzy of results in related areas was \cite{Edmonds1965MaximumMA}, which gave a polynomial-time algorithm to get a maximum weight matching in any graph. It does so by looking at the polytope in $\R^{E}$ that is the convex hull of the indicator functions of all matchings.
	It is worth noting that while there is a polynomial time algorithm to find a maximum weight matching, the problem of determining the number of maximum matchings in a graph is \textsf{\#P}-complete. Consequently, no polynomial time algorithm is known to determine $m_k(G)$ given a graph $G$.\\

	Before moving on, we give some simple lemmas about the matching polynomial.
	\begin{flem}
		\label{lem: matching polynomial basic results}
		\phantom{pain}
		\begin{enumerate}[label=(\alph*)]
			\item If $G$ and $H$ are vertex-disjoint graphs,
			\[ \Match_{G \cup H}(x) = \Match_G(x) \Match_H(x). \]
			\item Given a graph $G$ and vertex $v \in G$,
			\[ \Match_G(x) = x \Match_{G - \{v\}}(x) - \sum_{u : u \leftrightarrow v} \Match_{G - \{u,v\}} (x). \]
			\item Given a graph $G$ and edge $e = \{u,v\} \in G$,
			\[ \Match_G(x) = \Match_{G-e}(x) - \Match_{G-\{u,v\}}(x). \]
		\end{enumerate}
	\end{flem}
	\begin{proof}
		We omit the proof of (a) as it is straightforward.
		\begin{itemize}
			\item[(b)] Let $M$ be a matching of size $k$ on $G$. If $M$ does not have an edge incident on $v$, it is a matching of size $k$ on $G - \{v\}$. Otherwise, there is some edge $e = \{u,v\} \ni M$, and $M \setminus \{e\}$ is a matching on $G - \{u,v\}$. As a result,
			\[ m_k(G) = m_k(G- \{v\}) + \sum_{u : u \leftrightarrow v} m_{k-1}(G-\{u,v\}). \]
			Multiplying with $(-1)^k x^{n-2k}$ and summing over $k$,
			\begin{align*}
				\Match_G(x) &= \sum_k (-1)^k x \cdot x^{(n-1)-2k} m_k(G - \{v\}) - \sum_{u : u \leftrightarrow v} (-1)^{k-1} x^{(n-2)-2(k-1)} m_k(G - \{u,v\}) \\
					&= x\Match_{G-\{v\}} - \sum_{u : u \leftrightarrow v} \Match_{G - \{u,v\}} (x).
			\end{align*}
			\item[(c)] Similar to (b), let $M$ be a matching of size $k$ on $G$. If $M$ does not have $e$, it is a matching of size $k$ on $G - e$. Otherwise, $M \setminus \{e\}$ is a matching on $G - \{u,v\}$. So,
			\[ m_k(G) = m_k(G - e) + m_{k-1}(G - \{u,v\}). \]
			Multiplying with $(-1)^k x^{n-2k}$ and summing over $k$,
			\begin{align*}
				\Match_G(x) &= \sum_k (-1)^k x^{n-2k} m_k(G-e) - (-1)^{k-1} x^{(n-2)-2(k-1)} m_{k-1}(G-\{u,v\}) \\
					&= \Match_{G-e}(x) - \Match_{G-\{u,v\}}(x). \qedhere
			\end{align*}
		\end{itemize}
	\end{proof}


	\begin{fprop}
		\phantom{pain}
		\begin{enumerate}
			\item $m_k(P_n) = \binom{n-k}{k}$.
			\item $m_k(C_n) = \frac{n}{n-k} \binom{n-k}{k}$.
			\item $m_k(K_n) = \binom{n}{2k} \cdot \frac{(2k)!}{2^kk!}$.
			\item $m_k(K_{n,n}) = \binom{n}{k}^2 k!$.
		\end{enumerate}
	\end{fprop}
	\begin{proof}
		\phantom{pain}
		\begin{enumerate}
			\item Collapse every edge in a matching to its left endpoint, and ``mark'' the collapsed vertices. This results in a path with $n-k$ vertices with $k$ marked vertices. This process of marking the vertices using the matching is reversible, and $m_k(G) = \binom{n-k}{k}$.
			\item Fix some edge $e$. $e$ is absent in exactly $(n-k)/n$ of the $k$-matchings of $C_n$. In this case, the remaining matching forms a matching on $C_n - e$, which is isomorphic to $P_n$. Therefore, $(n-k)/n m_k(C_n) = m_k(P_n) = \binom{n-k}{k}$.
			\item A $k$-matching of $K_n$ is obtained by choosing $2k$ vertices (done in $\binom{n}{2k}$) ways, putting the $2k$ vertices in $k$ indistinguishable ``boxes'' by putting $2$ in each (this can be done in $(2k)!/k!2^k$ ways).
			\item A $k$-matching is obtained by choosing $k$ vertices from each side of the bipartite graph (done in $\binom{n}{k}^2$ ways), then assigning each vertex on the left side a vertex on the right that it is joined to in the matching (done in $k!$ ways). \qedhere
		\end{enumerate}
	\end{proof}

	% \begin{exercise}
	% 	Show that
	% 	\begin{enumerate}
	% 		\item $m_k(C_n) = n/(n-k) \binom{n-k}{k}$.
	% 		\item $m_k(K_n) = \binom{n}{2k} \cdot (2k!)/(2^k k!)$.
	% 		\item $m_k(K_{n,n}) = \binom{n}{k} k!$.
	% 	\end{enumerate}
	% \end{exercise}

	% \begin{exercise}
	% 	Given a graph $G$ and edge $e \in G$,
	% 	\[ \Match_G(x) = ? \]
	% \end{exercise}

	\begin{ftheo}
		\label{theo: roots of matching polynomial real}
		Given a graph $G$, all roots of $\Match_G(x)$ are real.
	\end{ftheo}
	The version of the proof of the above we give is due to Godsil \cite{godsil-matching-poly-real-roots}.
	\begin{proof}
		Using \Cref{lem: matching polynomial basic results}(a), we may assume that $G$ is connected.\\
		We first prove the result for the case where $G$ is a tree $T$. To prove this, we shall prove that $\Match_T(x)$ is the characteristic polynomial $\det(xI-A)$ of the adjacency matrix $A$ of $T$(!); the result then follows since $A$ is a real symmetric matrix and thus has real eigenvalues.\\
		Let $xI - A = (b_{ij})$. We have
		\[ \Charpoly(A) = \sum_{\pi \in \mathfrak{S}_n} \sign(\pi) \prod_{i=1}^n b_{i\pi(i)}. \]
		First, we claim that if $\pi \in \mathfrak{S}_n$ has a cycle of length greater than $2$, then the term corresponding to $\pi$ on the right will be zero. In other words, the term is zero if $\pi$ is not an involution. Indeed, this follows immediately since $G$ has no cycles (of length $\ge 3$). As a result, if $(i_1,i_2,\ldots,i_t)$ were a cycle in $\pi$, then there must be some $j$ such that $\{i_j i_{j+1}\}$ is not an edge in $G$ and $b_{i_j \sigma(i_j)} = 0$.\\
		Suppose that some $\pi \in \mathfrak{S}_n$ has $k$ $2$-cycles and $(n-2k)$ fixed points, and also has the term on the right being nonzero. We have $\sign(\pi) = (-1)^{(n- (k + n-2k))} = (-1)^k$. Suppose that the $k$ $2$-cycles are $(i_1,j_1),(i_2,j_2),\ldots,(i_k,j_k)$. We have $b_{i_rj_r} = b_{j_ri_r} = (-1)$ so $b_{i_rj_r}b_{j_ri_r} = 1$, and also that no $i_r$ (or $j_r$) is equal to any other $i_s$ (or $j_s$). That is, the edges constituted by $\{i_r,j_r\}$ form a matching of size $k$! Therefore,
		\[ \Charpoly(A) = \sum_{\substack{\pi \in \mathfrak{S}_n \\ \text{$\pi$ an involution}}} \sign(\pi) \prod_{i=1}^n b_{i\pi(i)} = \sum_{\substack{\pi \in \mathfrak{S}_n \\ \text{$\pi$ an involution}}} (-1)^k x^{n-2k} = \sum_{\text{matchings $M$}} (-1)^{|M|} x^{n-2|M|} = \Match_T(x). \]

		For a general graph, we come up with a tree $T_a(G)$ that depends on a ``starting vertex'' $a \in G$. We then show that the matching polynomial of our graph divides the matching polynomial of the tree.\\
		To define $T_a$, we need paths starting at $a$ in $G$ without repeated vertices. These paths are the vertices of $T_a(G)$. There is an edge between two paths if one is an extension of another by a single vertex -- for example, the paths $abdc$ and $abdce$ would have an edge between them.\\
		% possibly insert diagram of the graph given at http://www.math.iitb.ac.in/~krishnan/phd-2022/matching-poly.pdf
		The heart of the argument is the fact that for any $a\in G$ and $b \in T_a(G)$
		\[ \frac{\Match_{G - a}(x)}{\Match_G(x)} = \frac{\Match_{T_a(G)- b}(x)}{\Match_{T_a(g)}(x)}. \]
	\end{proof}

	% QUIZ 1 ON 1 SEP (THURSDAY)
	
	This result has some consequences.

	\begin{fdef}[Log-concave]
		A sequence $(a_n)_{n\ge0}$ is said to be \emph{log-concave} if $a_n^2 \ge a_{n-1} a_{n+1}$ for all $n$.
	\end{fdef}


	\begin{exercise}
		Show that for a fixed $n$, $\binom{n}{k}$ as $k$ varies is log-concave.
	\end{exercise}

	The stirling numbers of the first and second kind are also log-concave.

	\begin{fprop}
		If $A(x) = \sum_{i=0}^n a_i x^i$ is a polynomial with all real roots, then the sequence of coefficients of $A$ is log-concave.
	\end{fprop}

	\begin{exercise}
		Prove the above.
	\end{exercise}

	The above has an even stronger version.

	\begin{fprop}
		If $A(x) = \sum_{i=0}^n a_i x^i$ is a polynomial with all real roots, then $\left(a_i/\binom{n}{i}\right)_{i \ge 0}$ is log-concave. This is referred to as \emph{ultralogconcavity}.
	\end{fprop}

	\begin{fcor}
		For any graph $G$, $(m_k(G))_{k\ge 0}$ is log-concave. That is, for all $k$, $m_k(G)^2 \ge m_{k-1}(G)m_{k+1}(G)$.
	\end{fcor}

	% log-concavity is important! june huh won the fields medal this year for showing that some sequence is log-concave

	% convex hull of (01) vectors of matchings in R^{|E|} due to edmonds. this is linear program, so we can get a non-linear programming way. this gives a method to get a maximum weight matching!!! (place any weights you want on edges)
	% counting number of matchings is #P-complete though.


\subsection{Colorings}

	\begin{fdef}
		Given a graph $G=(V,E)$, a $k$-\emph{coloring} (sometimes called \emph{proper coloring}) is a function $c:V(G)\to[k]$ such that if $u\Leftrightarrow v$, $c(u) \ne c(v)$.\\
		The \emph{chromatic number} $\chi(G)$ of a graph is the minimum number of colours required to colour it.\\
		We denote by $c_k(G)$ the number of $k$-colorings of $G$. Define the \emph{chromatic polynomial}
		\[ \Chrom_G(x) = \sum_{k \ge 0} c_k(G) x^k. \]
	\end{fdef}
	The chromatic polynomial is not a polynomial, it is a power series -- $c_k(G)$ does not eventually go to zero since the function involved is not required to be surjective.\\

	Determining the chromatic number of a graph is (or rather, determining if there exists a $k$-coloring) is $\mathsf{NP}$-hard. The best known algorithm today outputs a colouring which is a $n/\text{polylog}(n)$-approximation of the minimum coloring.

	\begin{prop}
		A graph $G$ has chromatic number $1$ iff it is an empty graph. It has chromatic number $\le 2$ iff it is bipartite.
	\end{prop}

	It is difficult to approximate a minimum colouring of even $3$-colorable graphs!

	\begin{ftheo}[Four-Color Theorem, \cite{4ct}]
		A planar graph is $4$-colorable.
	\end{ftheo}
	We omit the proof of the above (for obvious reasons).

	If $G$ is the empty graph $\overline{K_n}$, then $c_k(G) = k^n$ and
	\[ \Chrom_G(x) = \sum_{k \ge 1} k^n x^k = \frac{A_n(x)}{(1-x)^{n+1}}. \]

	\begin{fprop}[Deletion-contraction recurrence]
		For any graph $G$, we have for $e \in G$ that
		\[ \Chrom_G(x) = \Chrom_{G-e}(x) - \Chrom_{G\setminus e}(x) \]
	\end{fprop}
	\begin{proof}
		Take any $k$-coloring of $G-e$. If the endpoints of $e$ have the same colour, it corresponds to a $k$-colouring of $G\setminus e$, and if the endpoints have distinct colours, it corresponds to a $k$-coloring of $G$.
	\end{proof}

	This leads to a method to find the chromatic polynomial of any graph. Since we know the chromatic polynomial of the empty graph, we can repeatedly delete and contract edges until from our graph until we get to an empty graph.

\section{Fundamental Results in Extremal Graph Theory}

	% In this section, we describe some fundamental, yet extremely powerful, results in extremal graph theory.

	\subsection{The Erd\H{o}s-Stone-Simonovits Theorem}

		\subsubsection{Motivation}

			% Next, we look at $\exx(n;C_k)$.\\
			% Recall that we have already seen that $\exx(n;C_4) \le c n^{3/2}$ for some universal constant $c$.\\
			Using \href{https://proofwiki.org/wiki/Graph_is_Bipartite_iff_No_Odd_Cycles}{the folklore result} that a graph is bipartite iff it has no odd cycle, we have
			\[ \exx(n;C_{2k+1}) \ge \left\lfloor \frac{n^2}{4} \right\rfloor. \]
			Is there any more general relationship between forbidden subgraphs and $r$-partite graphs?

			\begin{fdef}[Chromatic Number]
				\label{def: chromatic number}
				Given a graph $G$, its \textbf{chromatic number} is given by
				\[ \chi(G) = \min\{r : G\text{ is $r$-partite}\}. \]
			\end{fdef}

			Alternatively, we can define the above using the following.

			\begin{fdef}
				\label{def: r-coloring}
				Given a graph $G=(V,E)$, an \textbf{$r$-coloring} of $G$ is a function $f : V \to [r]$ such that for any $uv\in E$, $f(u) \ne f(v)$.\\
				The chromatic number of a graph is the least $r$ such that it is $r$-colorable.
			\end{fdef}

			Suppose $H$ is an arbitrary graph such that $\chi(H) = r+1$. Then, no $r$-partite graph contains $H$. As a result,
			\[ \exx(n;H) > t_r(n). \]
			Our earlier observation on $\exx(n;C_{2k+1})$ is then just a consequence of the fact that $C_{2k+1}$ is $3$-colorable.

			Let us give a more concrete example. Suppose we want to find $\exx(n;\mathsf{Petersen})$, where $\mathsf{Petersen}$ is the \href{https://en.wikipedia.org/wiki/Petersen_graph}{Petersen graph}.

			% diagram?

			It may be checked that $\mathsf{Petersen}$ has chromatic number $3$. So,
			\[ \exx(n;\mathsf{Petersen}) > t_2(n) = \left\lfloor \frac{n^2}{4} \right\rfloor. \]

			Can we do better?\\
			This is answered by the Erd\H{o}s-Stone-Simonovits Theorem.

		\subsubsection{The Result}

			\begin{ftheo}[Erd\H{o}s-Stone-Simonovits Theorem, Version 1]
				\label{theo: erdos-stone v1}
				Let $r \ge 1$ and $0 < \varepsilon < 1/2$. Then, there exists some $n_0$ such that for any $n \ge n_0$ and graph $G_n$, if
				\begin{equation}
					\label{eqn: erdos-stone v1 condition}
					\delta(G_n) \ge \left(1 - \frac{1}{r} + \varepsilon\right)n,
				\end{equation}
				there exist pairwise disjoint subsets $V_1,\ldots,V_{r+1}$ of $V$ such that for each $i$,
				\[ |V_i| = t \ge \frac{\varepsilon \log n}{2^{r-1}(r-1)!} \]
				and the complete $(r+1)$-partite graph on these subsets is contained in $G_n$.
			\end{ftheo}

			\begin{ftheo}[Erd\H{o}s-Stone-Simonovits Theorem, Version 2]
				\label{theo: erdos-stone v2}
				Let $r \ge 1$ and $0 < \varepsilon < 1/2$. Then, there exists some $n_0$ such that for any $n \ge n_0$ and graph $G_n$, if
				\begin{equation}
					\label{eqn: erdos-stone v2 condition}
					e(G_n) \ge \left(1 - \frac{1}{r} + \varepsilon\right)\binom{n}{2},
				\end{equation}
				there exist pairwise disjoint subsets $V_1,\ldots,V_{r+1}$ of $V$ such that for each $i$,
				\[ |V_i| = t \ge \frac{\varepsilon \log n}{2^{r+1}(r-1)!} \]
				and the complete $(r+1)$-partite graph on these subsets is contained in $G_n$.
			\end{ftheo}

			Observe the difference in the exponent of $2$ in the denominator of $t$ in the two versions.

			For example, if $n \ge e^{32/\varepsilon}$ and \Cref{eqn: erdos-stone v2 condition} is satisfied, then $T_{3}(12)$, and thus $\mathsf{Petersen}$, is a subgraph of $G_n$.\\
			Therefore, (replacing $\varepsilon$ with $2\varepsilon$)
			\[ \exx(n;\mathsf{Petersen}) \le \left(\frac{1}{2} + 2\varepsilon\right)\binom{n}{2} \le \left(\frac{1}{4} + \varepsilon\right)n^2. \]
			for all $\varepsilon > 0$.

			\begin{fcor}
				\label{theo: ESS exx square tight coloring}
				If $\chi(H) = r+1$, then for any $\varepsilon > 0$, there exists $n_0$ such that for all $n\ge n_0$,
				\[ \left(1 - \frac{1}{r}\right)\frac{n^2}{2} < \exx(n;H) < \left(1 - \frac{1}{r} + \varepsilon\right)n^2 - \mathcal{O}(n). \]
			\end{fcor}

			% \begin{fdef}
			% 	A graph $G$ is said to be \textbf{$c$-dense} (for $0<c<1$), if $e(G_n) \ge cn^2$.
			% \end{fdef}

			\begin{proof}[Proof of \nameref{theo: erdos-stone v1}]
				We first show the result for $r=1$. Let $G_n$ be a graph such that $\delta(G_n) \ge \varepsilon n$. We want a ``large'' $t$ such that $K_{t,t}$ is isomorphic to a subgraph $G_n$.\\
				Suppose that $G_n$ is $K_{t,t}$-free (we shall fix $t$ later).
				Now, we have that
				\[ t \binom{n}{t} \stackrel{(1)}{>} |\{(v,T) : v\in V, T \subseteq N(v), |T|=t\}| = \sum_{v\in V} \binom{d(v)}{t} \stackrel{(2)}{\ge} n \binom{\varepsilon n}{t}, \]
				where $(2)$ follows from the hypothesis on $\delta(G_n)$ and $(1)$ follows due to our assumption that $G_n$ is $K_{t,t}$-free.\\
				Therefore,
				\[ n\binom{\varepsilon n}{t} < t \binom{n}{t}. \]
				So,
				\begin{align*}
					1 &< \frac{t}{n} \cdot \frac{n(n-1)\cdot (n-t+1)}{\varepsilon n(\varepsilon n-1)\cdot (\varepsilon n-t+1)} \\
					&< \frac{t}{n} \cdot \left(\frac{n}{\varepsilon n-t+1}\right)^t \\
					&= \frac{t}{n} \frac{1}{\varepsilon^t} \left(\frac{1}{1 - \frac{t-1}{\varepsilon n}}\right)^t.
				\end{align*}
				We desire a large $t$ such that the above results in a contradiction.
				Let us look at the last term. Suppose we want
				\[ \left(1 - \frac{t-1}{\varepsilon n}\right)^t \to 1. \]
				Equivalently,
				\[ t \log\left(1 - \frac{t-1}{\varepsilon n}\right) \to 0 \]
				\[ t \left(\frac{t-1}{\varepsilon n} + \frac{1}{2}\left(\frac{t-1}{2n}\right)^2 + \frac{1}{3} \left(\frac{t-1}{\varepsilon n}\right)^3 + \cdots\right) \to 0. \]
				% \begin{align*}
				% 	t \log\left(1 - \frac{t-1}{\varepsilon n}\right) &\to 0 \\
				% 	t \left(\frac{t-1}{\varepsilon n} + \frac{1}{2}\left(\frac{t-1}{2n}\right)^2 + \frac{1}{3} \left(\frac{t-1}{\varepsilon n}\right)^3\right) &\to 0.
				% \end{align*}
				If $\varepsilon n > t^4$, the above holds (Why?).\\
				That is, if $t < (\varepsilon n)^{1/4}$, then for sufficiently large $n, t$,
				\[ \left(1 - \frac{t-1}{\varepsilon n}\right)^{-t} < 2 \]
				and
				\[ 1 < \frac{t}{n} \cdot \frac{1}{\varepsilon^t} \left(1 - \frac{t-1}{\varepsilon n}\right)^{-t} < \frac{2t}{\varepsilon^t n}. \]
				If the above is $\le 1$, we shall arrive at a contradiction. That is, if
				\[ \log(2t) + t \log\left(\frac{1}{\varepsilon}\right) < \log n, \]
				then we arrive at a contradiction. In particular,
				\[ t = \lceil \varepsilon \log n \rceil \]
				is a suitable choice (Why?), completing the proof of the theorem for $r=1$.\\

				Now, let us prove the general case by performing induction on $r$. Let $G_n$ be a graph with
				\[ \delta(G_n) \ge \left(1 - \frac{1}{r} + \varepsilon\right)n. \]
				Now,
				\[ 1 - \frac{1}{r} + \varepsilon > 1 - \frac{1}{r-1} + \frac{1}{r(r-1)}. \]
				Let $\varepsilon' = 1 / r(r-1)$. By induction, there are $V_1',\ldots,V_r'$ such that 
				\begin{equation}
					\label{eqn: ESS expression for t'}
					|V_i'| \ge t' = \frac{\log n}{r(r-1) 2^{r-2} (r-2)!} = \frac{\log n}{2^{r-2} r!}
				\end{equation}
				for each $i$ and the complete $r$-partite graph on these sets is a subgraph of $G_n$. Let $K = \bigcup V_i'$.\\
				We obviously have $\varepsilon < 1/r$, since the claim is vacuously true otherwise.\\
				We shall find $V_i \subseteq V_i'$ for each $1 \le i \le r$ and some $V_{r+1} \subseteq V \setminus K$ such that the complete bipartite graph on $(V_1,\ldots,V_{r+1})$ is the required subgraph of $G_n$.\\
				Now, define
				\[ U = \left\{x \in V\setminus K : d(x,K) \ge \left(1 - \frac{1}{r} + \lambda\right)|K| \right\} \]
				for some $\lambda$ we shall fix later. We shall bound $e(K,V\setminus K)$ in two different ways. From the perspective of $K$,
				\[ e(K,V\setminus K) \ge |K| \left(\left(1 - \frac{1}{r} + \varepsilon\right)n - |K|\right). \]
				From the perspective of $V\setminus K$,
				\begin{align*}
					e(K, V\setminus K) &= e(V\setminus (K\sqcup U), K) + e(U,K) \\
					&\le \left(n - |U| - |K|\right) \left(1 - \frac{1}{r} + \lambda\right)|K| + |U||K|.
				\end{align*}
				Putting the two together,
				\begin{align*}
					|K| \left(\left(1 - \frac{1}{r} + \varepsilon\right)n - |K|\right) &\le \left(n - |U| - |K|\right) \left(1 - \frac{1}{r} + \lambda\right)|K| + |U||K| \\
					\left(1 - \frac{1}{r} + \varepsilon\right)n - |K| &\le \left(n - |U| - |K|\right) \left(1 - \frac{1}{r} + \lambda\right) + |U|.
				\end{align*}
				Set $\lambda = \varepsilon/2$. We then have
				\[ \frac{r}{2}\varepsilon n \le \left(1 - \frac{r\varepsilon}{2}\right)(|U| + |K|). \]
				% ??????? Am I not getting (1 - r \varepsilon/2)(|U|+|K|).
				So,
				\[ |U| \ge \frac{r\varepsilon}{2 - r\varepsilon} n - |K| \stackrel{(*)}{\ge} \frac{\varepsilon}{1-\varepsilon}n - rt', \]
				where $(*)$ follows from the fact that $r\varepsilon/(2-r\varepsilon)$ is increasing in $r$ and $r\ge 2$.\\
				Since $t'$, and thus $rt'$, is of the order of $\log n$, the first term in the expression dominates for sufficiently large $n$. So, for $n\gg 0$, $|U| \ge \varepsilon n$.\\

				Now,
				\begin{align*}
					\left(1 - \frac{1}{r} + \frac{\varepsilon}{2} \right)|K| &\ge \left(1 - \frac{1}{r} + \frac{\varepsilon}{2} \right) rt' \\
					&\ge (r-1)t' + \frac{\varepsilon r}{2} t'.	
				\end{align*}
				This implies that each $u \in U$ has at least $(\varepsilon r/2)t'$ neighbours in each $V_i'$.\\
				We can now use a pigeonhole argument to choose a subset of $U$ whose vertices are all adjacent to some common set of vertices in each $V_i'$. To do so, consider
				\[ |\{(u, W_1,\ldots,W_r) : W_i \subseteq V_i', |W_i| = (\varepsilon r/2)t',\text{ and $u$ is adjacent to all the vertices of each $W_i$}\}|. \]
				By our earlier observation, this must be at least $|U| \ge \varepsilon n$.\\
				On the other hand, it is at most the number of ways of choosing the $W_i$, which is $\binom{t'}{(\varepsilon r/2)t'}^r$.\\
				In particular, using a pigeonhole argument, there exist $V_1,\ldots,V_r$ and a $V_{r+1}\subseteq U$ such that $V_i \subseteq V_i'$ for each $1\le i\le r$ and for all $u \in V_{r+1}$, $(u,V_1,\ldots,V_r)$ is in the set whose cardinality we just considered, and $|V_{r+1}| \ge \varepsilon n / \binom{t'}{(\varepsilon r/2)t'}^r$. Let us now bound this expression.
				\begin{align*}
					|V_{r+1}| &\ge \dfrac{\varepsilon n}{\dbinom{t'}{(\varepsilon r/2)t'}^r} \\
					&\ge \frac{\varepsilon n}{ (2e/\varepsilon r)^{t'\varepsilon r^2/2}} & \text{(see \href{https://math.stackexchange.com/questions/132625/}{here} for the bound used)} \\
					&\ge \varepsilon n\cdot\left(\frac{\varepsilon}{e}\right)^{t'\varepsilon r^2/2} & (\text{since $r \ge 2$, $2/r \le 1$}).
				\end{align*}
				Setting
				\[ t = \frac{\varepsilon\log n}{2^{r-1}(r-1)!}, \]
				we see that $t = \varepsilon rt'/2$.\\
				Keeping in mind that the bound we want is $|V_{r+1}| \ge t$,
				\begin{align*}
					\log \left(n \left(\frac{\varepsilon}{e}\right)^{t' \varepsilon r^2/2}\right) &\ge \log n + \frac{\varepsilon r^2 t'}{2} \log\left(\frac{\varepsilon}{e}\right) \\
					&\ge t' \left( 2^{r-2} r! - \log\left(\frac{e}{\varepsilon}\right) \cdot \frac{\varepsilon r^2}{2} \right) & \text{(using the expression for $t'$ in \eqref{eqn: ESS expression for t'})} \\
					&\ge t' \left( 2^{r-2} r! - \log(2e) \cdot \frac{r^2}{4} \right) & \text{($-\varepsilon \log(e/\varepsilon)$ is decreasing in $\varepsilon$)} \\
					% &\ge e \log(t') \left(2^{r-2} r! - \log(2e) \cdot \frac{r^2}{4}\right) & \text{$x/\log x \ge e$ for $x>1$} \\
					&\ge \log(rt'/2).
				\end{align*}
				Therefore, $V_{r+1} \ge t$. Since $|V_i| = \varepsilon rt'/2 = t$ by definition, the proof is complete.
			\end{proof}

			Whew. Let us now give a simple corollary of the above result.

			\begin{fpor}
				Suppose $H_1,\ldots,H_m$ are graphs. Then, for any $\varepsilon > 0$ and $n \gg 0$,
				\[ \exx(n;H_1,\ldots,H_m) \le \left(1 - \frac{1}{r} + \varepsilon\right)\frac{n^2}{2} - \mathcal{O}(n^2), \]
				where $r+1=\max\{\chi(H_1),\ldots,\chi(H_m)\}$.
			\end{fpor}


			Before we move onto the proof of the second version of the Erd\H{o}s-Stone-Simonovits Theorem, we give a lemma that will assist in its proof.

			\begin{flem}
				For $n\gg 0$, if $e(G_n) \ge (c+\varepsilon)\binom{n}{2}$ for some $c > 0$ and $\varepsilon > 0$, then there exists $H\subseteq G_n$ such that
				\begin{enumerate}
					\item $|H| \ge \sqrt{\varepsilon} n$.
					\item $\delta(H) \ge c|H|$.
				\end{enumerate}
			\end{flem}

			\begin{proof}
				Assume that $G_n$ itself does not satisfy the conclusions. Then, there exists some $x_n \in V(G_n)$ such that $d(x_n) < cn$.\\
				Let $H_{n-1} = G \setminus \{x_n\}$. If $H_{n-1}$ fails the second conclusion of the theorem, there exists $x_{n-1} \in V(H_{n-1})$ such that $d(x_{n-1}) < c (n-1)$.\\
				Repeating the above, we get a sequence of graphs $G_n = H_n \supsetneq H_{n-1} \supsetneq \cdots \supsetneq H_{\ell}$, where $V(H_{n-r}) \setminus V(H_{n-r-1}) = \{x_{n-r}\}$ for each $r$ and $\ell \ge \sqrt{\varepsilon} n$. Further, for each $i$, $d_{H_i}(x_i) < c i$ for $i = \ell,\ldots,n$.\\
				Now,
				\begin{align*}
					e(H_\ell) &> e(H_n) - \left(cn + c(n-1) + \cdots + c(\ell+1)\right) \\
					&= (c+\varepsilon)\binom{n}{2} - c \left(\frac{n(n+1)}{2} - \frac{\ell(\ell+1)}{2}\right) \\
					&= (c+\varepsilon)\binom{n}{2} - c \left(\binom{n}{2} + n - \binom{\ell+1}{2}\right) \\
					&= \varepsilon\binom{n}{2} - cn + c \binom{\ell+1}{2}.
				\end{align*}
				Now, the final expression must be at most $\binom{\ell}{2}$ (since $H_\ell$ has $\ell$ vertices).\\
				As a result, it would suffice to show that the above expression on taking $\ell = \lfloor \varepsilon n \rfloor$ is greater than $\binom{\ell}{2}$ for sufficiently large $n$ (this implies that the sequence must stop before reaching this $\ell$ due to one of the graphs satisfying the conclusions). Indeed, this is seen to be true as
				\begin{align*}
					\varepsilon\binom{n}{2} - cn + c\binom{\ell+1}{2} &\ge \varepsilon\binom{n}{2} + c\left(\frac{(\lfloor\sqrt{\varepsilon}n\rfloor+1)\lfloor\sqrt{\varepsilon}n\rfloor}{2} - n\right) \\
					&\ge \varepsilon\binom{n}{2} & (\text{for sufficiently large $n$}) \\
					&\ge \binom{\ell}{2},
				\end{align*}
				completing the proof.
			\end{proof}

			% In particular, taking $c = 1 - \frac{1}{r} + \frac{\delta}{2}$ reduces v2 to v1.

			% \begin{proof}[Proof of \nameref{theo: erdos-stone v2}]
			% 	We shall reduce this Theorem to an instance of \nameref{theo: erdos-stone v1}. Let $G_n$
			% \end{proof}


			% \begin{flem}
			% 	Let $d$ be the average degree of $G$. Then $G$ contains a (non-empty) subgraph $H$ with $\delta(H) \ge d/2$.
			% \end{flem}

	\subsection{An Introduction to Random Graphs}

		\begin{fdef}[Erd\H{o}s-R\'{e}nyi Model]
			Fix $0\le p\le 1$. The \textbf{Erd\H{o}s-R\'{e}nyi random graph model}, denoted $G(n,p)$ is the random variable which is a graph with vertex set $[n]$, such that for each $\{i,j\} \in \binom{[n]}{2}$, $\{i,j\}$ is an edge with probability $p$, independently across distinct pairs.
		\end{fdef}

		% That is, $G(n,p)$ is a distribution on all graphs with vertex set $[n]$. 
		So, for any graph $H$ on vertex set $[n]$,
		\[ \Pr(G(n,p) = H) = p^{e(H)} (1-p)^{\binom{n}{2} - e(H)}. \]

		\begin{remark}
			If $p=1/2$, this is the uniform distribution on the set of graphs on $[n]$.
		\end{remark}

		A recurring theme in probability theory is that random objects tend to behave very nicely given a large number of samples (along the lines of the laws of large numbers and the central limit theorem).

		\subsubsection{A motivating extremal problem}

			To understand why random graphs are important, let us look at $\exx(n;C_{2k})$. We have already seen that $\exx(n;C_{2k+1}) \ge \lfloor n^2/4 \rfloor$. Is it true that $\exx(n;C_{2k+1}) = \lfloor n^2 / 4 \rfloor$ for $n \gg 0$?\\
			While \Cref{theo: ESS exx square tight coloring} does say that $\exx(n;C_{2k+1}) \le (1/4 + o(1)) n^2$, it \emph{could} still be as large as $n^2/4 + n^{1.99}$, say.\\
			It turns out, however, that our claim happens to be true, as we shall see later.\\

			Before moving to this however, let us look at $\exx(n;C_{2k})$, and at the very least get a lower bound on this quantity.

			\begin{ftheo}
				For any $k$, there exists a constant $c$ such that for $n \gg 0$, there is a $C_k$-free $G_n$ with
				\[ e(G_n) \geq c \cdot n^{1 + 1/(k-1)}. \]
			\end{ftheo}

			The above result does not give anything useful for $k$ odd.

			\begin{proof}
				Consider $G(n,p)$ for some $p$ we shall fix later. Let $N(G)$ be the number of copies of $C_k$ in a given graph $G$.\\
				Given a cycle $(v_1,\ldots,v_k)$, observe that the sequences $(v_2,v_3,\ldots,v_k, v_1)$ and $(v_k,v_{k-1},\ldots,v_1)$ determine the same cycle. That is, performing cyclic shifts of a sequence of vertices or reversing their order around gives the same cycle.\\
				Let $\mathcal{C}$ be the set of all these cycles.\footnote{This can be made more formal by taking all length $k$ sequences of $[n]$ consisting of distinct elements and considering the equivalence classes formed by the equivalence relation defined in the previous paragraph.} By our observation,
				\[ |\mathcal{C}| = \frac{n(n-1)\cdots(n-k+1)}{2k} = \frac{n!}{2k\cdot k!}. \]
				Now,
				\begin{align*}
					\expec[N(G(n,p))] &= \expec\left[ \sum_{(v_1,\ldots,v_k) \in \mathcal{C}} \indic_{v_1v_2, v_2v_3, \ldots, v_{k-1}v_k, v_kv_1\text{ are edges}} \right] \\ 
					&= \sum_{(v_1,\ldots,v_k) \in \mathcal{C}} \Pr\left[ \indic_{v_1v_2, v_2v_3, \ldots, v_{k-1}v_k, v_kv_1\text{ are edges}} \right] & \text{(linearity of expectation)} \\
					&= \sum_{(v_1,\ldots,v_k) \in \mathcal{C}} p^k \\
					&= \frac{n!}{2k \cdot k!} p^k.
				\end{align*}
				This quantity is obviously less than $(np)^k / 2k$. If $p$, and thus the expectation is small, we expect to not see many $C_k$s. On the other hand,
				\begin{align*}
					\expec[e(G(n,p))] &= \expec \left[ \sum_{\{i,j\} \in \binom{[n]}{2}} \indic_{ij\text{ is an edge}} \right] \\
					&= \binom{n}{2} p.
				\end{align*}
				Given a graph, if we delete an (arbitrary) edge from each copy of $C_k$ in it, we will be left with no cycles. That is, given any graph $G$, there is a graph on the same vertex set with $e(G) - N(G)$ edges that is $C_k$-free. Inspired by this, by the linearity of expectation,
				\[ \expec[e(G(n,p)) - N(G(n,p))] \ge \binom{n}{2} p - \frac{(np)^k}{2k}. \]
				If we set $p = \left(\frac{k}{2}\right)^{1/(k-1)} n^{-1 + 1/(k-1)}$, then the above quantity is at least $n(n-1)p/4 \ge c \cdot n^{1 + 1/(k-1)}$ for an appropriate constant $c$ and $n\gg 0$, completing the proof.
			\end{proof}


\bibliographystyle{alpha}
\bibliography{references}


\end{document}