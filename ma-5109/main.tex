\documentclass{article}
\usepackage[T1]{fontenc}
\usepackage[utf8]{inputenc}
\usepackage{ma-5109}

\begin{document}

\thispagestyle{empty}
\titleBC

\setcounter{section}{-1}

\section{Notation}
	
	We use $[n]$ to represent $\{1,2,\ldots,n\}$.\\
	For integers $a$ and $b$, $[a,b]$ means $\{a,a+1,\ldots,b\}$.\\
	A graph $G_n$ is a graph with $n$ vertices.\\
	Given a graph $G$, $e(G)$ is the number of edges $G$ has.\\
	For a vertex $v$, denote by $N(v)$ the set of \emph{neighbours} of $v$ -- all the vertices that have an edge to $v$.\\
	For a vertex $v$, denote by $d(v) = |N(v)|$ the \emph{degree} of $v$ -- the number of edges incident on it.\\
	For $v \in V$ and $K \subseteq V$, $d(v,K)$ is the number of edges
	\[ \left|\{ u \in K : uv \in E \}\right| \]
	from $v$ into $K$.

\section{Introduction}

	\subsection{What are Graphs?}

		\begin{fdef}
			A (simple undirected) \textbf{graph} $G$ is an ordered pair $(V,E)$ where $V$ is a finite set called the \emph{vertex set} and $E$, called the \emph{edge set}, is a subset of $\binom{V}{2}$, where $\binom{S}{k}$ represents the set of all $k$-element subsets of $S$.
		\end{fdef}

		We typically represent graphs pictorially, showing vertices as dots and edges as arcs joining the vertices present in the corresponding subset.

		A few important graphs are:
		\begin{itemize}
			\item the \emph{null graph} with vertex set $V$, where $E = \emptyset$.
			\item the \emph{complete graph} $K_n$, where $V = [n]$ and $E = \binom{[n]}{2}$.
			\item the \emph{complete bipartite graph} $K_{m,n}$, where $V = A \cup B$ with $|A|=m$, $|B|=n$, and $A,B$ are disjoint, and $E = \{\{a,b\} : a \in A, b \in B\}$.
			\item the \emph{path graph} of length $n$, where $V = [n+1]$ and $E = \{\{m,m+1\} : m \in [n]\}$.
			\item the \emph{cycle} of length $n$, where $V = [n]$ and $E = \{\{l,m\} : l,m\in [n] , (m - l) \equiv 1 \pmod{n}\}$.
		\end{itemize}

		Now, consider the graph $G$ with vertex set $[4]$ and edge set $\{\{1,3\},\{3,2\},\{2,4\}\}$. This graph appears to be the same as the path graph of length $3$, but how do we make this correspondence more concrete?\\
		Relabeling vertices doesn't create a ``new'' graph.

		\begin{definition}[Graph Isomorphism]
			Two graphs $G = (V,E)$ and $G' = (V',E')$ are said to be \textbf{isomorphic} and we write $G \simeq G'$ if there exists a bijection $f : V \to V'$ such that there is an edge between two vertices $u$ and $v$ in $G$ if and only if there is an edge between $f(u)$ and $f(v)$ in $G'$.
		\end{definition}

		If two graphs are isomorphic, they are identical for our purposes (we only care about graphs up to isomorphism).

		\begin{definition}[Subgraph]
			Given a graph $G = (V,E)$, a \textbf{subgraph} $H = (V',E')$ is a graph such that $V' \subseteq V$ and $E' \subseteq E$. Given $V' \subseteq V$, the subgraph \emph{induced} by $V'$ on $G$ is that with vertex set $V'$ and edge set $\binom{V'}{2} \cap E$.
		\end{definition}

	\subsection{The birth of Extremal Graph Theory}

		Extremal graph theory is motivated by the following simple problem:

		\begin{quote}
			At most how many edges can a graph $G_n$ have if it contains no triangles?
		\end{quote}

		More precisely, what is
		\[ \max_{\substack{\text{no subgraph of $G_n$} \\ \text{is isomorphic to $K_3$}}} e(G_n)? \]

		Clearly, this number is well-defined since a graph on $n$ vertices cannot have more than $\binom{n}{2}$ vertices.

		A simple observation is that any complete bipartite graph has no triangles: if there were a triangle, then two vertices would be in the same ``part'', which contradicts the existence of edges only between the two parts.\\
		As a consequence, for any $1 \le m \le n$, it is possible to construct $m \times (n-m)$ edges (with this bound being attained for $K_{m,n-m}$). In particular, it is possible to construct a graph with $\lfloor n^2 / 4 \rfloor$ edges.

		\begin{ftheo}[Mantel's Theorem]
		\label{Mantel's Theorem}
			If $G_n$ has no triangle, then
			\[ e(G_n) \le \left\lfloor \frac{n^2}{4} \right\rfloor. \]
			Further, equality is attained iff $G_n \simeq K_{\lfloor n/2\rfloor , \lceil n/2 \rceil}$.
		\end{ftheo}
		\begin{proof}
			Suppose $G_n$ has no triangles. Saying that $G_n$ has no triangles is equivalent to saying that for distinct adjacent $u,v$, $N(u) \cap N(v) = \emptyset$.\\
			So, $d(u) + d(v) \le n$. Therefore,
			\begin{align*}
				ne(G_n) &\stackrel{\small{(1)}}{\ge} \sum_{uv \in E} d(u) + d(v) \\
					&= \sum_{uv \in E} |N(u) \cup N(v)| \\
					&= \left| (e, w) : e = uv \in E, w \in N(u) \cup N(v) \right| \\
					&= \sum_{u\in V} |\{ (e , w) : w \in N(u) , e = uv \in E \}| \\
					&= \sum_{u \in V} |\{ (v,w) : v,w \in N(u) \}| \\
					&\stackrel{\small{(2)}}{=} \sum_{u \in V} d(u)^2 \\
					&\stackrel{\small{(3)}}{\ge} \frac{1}{n} \left( \sum_{u \in V} d(u) \right)^2 \\
					&\stackrel{\small{(4)}}{=} \frac{4 e(G_n)^2}{n},
			\end{align*}
			where $(2)$ follows from the changing the main thing being summed over to $u$, the ``middle'' vertex in the $L$-like structure, $(3)$ follows from the \href{https://en.wikipedia.org/wiki/Cauchy-Schwarz_inequality}{Cauchy-Schwarz inequality}, and $(3)$ follows from the \href{https://en.wikipedia.org/wiki/Handshaking_lemma}{handshaking lemma}.\\

			What happens when equality is attained? Let us look at the case where $n$ is even.\\
			$(1)$ is only tight when $d(u) + d(v) = n$ for all edges $uv$ and $(3)$ is only tight when $d(u)$ is a constant (independent of $u$). This implies that $d(u) = \frac{n}{2}$ for every $u\in V$. Now, if $uv$ is an edge, $N(u) \cap N(v) = \emptyset$ implies that $N(u) \cup N(v) = V$, and so $G_n = K_{\frac{n}{2}, \frac{n}{2}}$.\\
			The case where $n$ is odd is analyzed similarly, with slight nuances in $(3)$ since exact equality is not attained.
		\end{proof}
		
		While the above is one of the early results in extremal graph theory, the subject was only really born due to Tur\'{a}n in the following result.

		\begin{ftheo}[Tur\'{a}n's Thoerem]
			If $G_n$ has no $K_{r+1}$ ($r \ge 2$), then $e(G_n) \le t_r(n)$, with equality attained iff $G \simeq T_r(n)$.
		\end{ftheo}

		The version for $r=2$ is just a triangle-free graph and is the same as \nameref{Mantel's Theorem}. In the proof of this, we split the vertex set into two parts and dumped all the edges between these parts.\\
		If we want to avoid $K_4$ ($r=3$), then perhaps we could split the vertex set into three parts and dump all the edges between these parts.\\
		In general, we want to partition $V$ of size $n$ into $r$ ``almost equal'' parts and set only those edges between vertices in distinct parts -- such a graph is known as the \textbf{Tur\'{a}n graph} $T_r(n)$ and the number of edges $e(T_r(n))$ is the \textbf{Tur\'{a}n number} $t_r(n)$.\\
		In particular, when $r \mid n$,
		\[ t_r(n) = \binom{r}{2} \left(\frac{n}{r}\right)^2 = \frac{n^2}{2} \left(1 - \frac{1}{r}\right). \]

		Here, we give three proofs of Tur\'{a}n's Theorem.

		\begin{proof}[Proof 1 of Tur\'{a}n's Theorem]
			We perform strong induction on $n+r$. We have already proved the result for $r=2$.\\
			Suppose $e(G_n) \ge t_r(n)$ and $G_n$ is $K_{r+1}$-free, where $r > 2$. We wish to prove that $G \simeq T_r(n)$.\\
			Since $t_r(n) \ge t_{r-1}(n)$ (check this!), the inductive hypothesis implies that $G$ has a copy $K \subseteq V$ of $K_r$. Observe that for $v \not\in K$, $d(v,K) \le r-1$ -- otherwise, there would be a copy of $K_{r+1}$ in $G$.\\
			As a result, $e(V \setminus K, K) \le (r-1)(n-r)$. By the induction hypothesis, $e(V\setminus K, V\setminus K) \le t_r(n-r)$. Therefore,
			\[ t_r(n) \le e(G_n) \le t_r(n-r) + (r-1)(n-r) + \binom{r}{2}. \]
			However, as can be checked manually, $t_r(n-r) + (r-1)(n-r) + \binom{r}{2} = t_r(n)$!\\
			It follows that equality holds everywhere -- $e(G_n) = t_r(n)$, $e(V \setminus K) = t_r(n-r)$, and $d(v,K) = r-1$ for all $v \in V \setminus K$.\\
			This graph is then isomorphic to $T_r(n)$ -- for each $v \in V \setminus K$, we can put the vertex in $K$ that is not adjacent to $v$ in the same bucket as $v$. Then, the only edges are those between distinct buckets (Why?).
		\end{proof}


\end{document}