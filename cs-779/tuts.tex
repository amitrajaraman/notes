\documentclass{article}
\usepackage[T1]{fontenc}
\usepackage[utf8]{inputenc}
\newcommand{\myname}{Amit Rajaraman}
\newcommand{\topicname}{CS 779 : Tutorial solutions}
\usepackage{../generic}
\usepackage{titling}
\usepackage{float}
\setlength{\droptitle}{-10cm}
\allowdisplaybreaks

\newcommand{\lcm}{\operatorname{lcm}}

\begin{document}

\thispagestyle{empty}
\titleBC

\tableofcontents

\vspace{1cm}

\clearpage

\section{Tutorial 1: Linear Algebra Methods}

	\begin{exercise}
		Prove that the maximum number of subsets of $[n]$ with pairwise non-empty intersection is $2^{n-1}$.
	\end{exercise}
	\begin{solution*}
		$2^{n-1}$ is clearly attainable by taking $\{S \subseteq [n] : 1 \in S\}$. Furthermore, this is an upper bound since if $\mathcal{S}$ is a family of subsets with pairwise non-empty intersection, then $\mathcal{S}' = \{S^c : S \in \mathcal{S}\}$ has zero intersection with $\mathcal{S}$ and is of the same size, so $2|\mathcal{S}| = |\mathcal{S}'|+|\mathcal{S}| \le 2^n$.
	\end{solution*}

	\begin{exercise}
		\label{ex1.2}
		Suppose you have a set system with $m$ sets $(A_i)_{i=1}^m$ such that $|A_i|$ is odd for each $i$ and $|A_i \cap A_j|$ is even for any $i \ne j$. Prove that $m \le n$.
	\end{exercise}
	\begin{solution*}
		Consider the $m \times n$ matrix $M$ where $M_{ij}$ is $1$ if $j \in A_i$ and is $0$ otherwise. Then,
		\[ (MM^\top)_{ij} = \sum_{k \in [n]} M_{ik} M_{jk} = |A_i \cap A_j|. \]
		In particular, all the diagonal entries of $MM^\top$ are odd and all off-diagonal entries are even. Using this, it is not too difficult to show that $\det(MM^\top) \ne 0$ (for an easy solution* of this, note that modulo $2$, $MM^\top$ is congruent to the identity, which has nonzero determinant). Therefore, $m = \rank(MM^\top) = \rank(M)$, so $m \le n$.
	\end{solution*}

	\begin{exercise}
		\label{ex1.3}
		Prove that for matrices $A,B$, $\rank(A+B) \le \rank(A)+\rank(B)$.
	\end{exercise}
	\begin{solution*}
		It suffices to show that any column of $A+B$ is present in the space spanned by the column of $A$ and $B$. This is straightforward since any column of $A+B$ is just the sum of the two corresponding columns in $A$ and $B$.
	\end{solution*}

	\begin{exercise}
		Suppose you have $A+A^\top = J-I$, where $J$ is the all ones matrix. Prove that $\rank(A) \ge n/2$.
	\end{exercise}
	\begin{solution*}
		Using the previous exercise, we have $n = \rank(J-I) = \rank(A+A^\top) \le \rank(A) + \rank(A^\top) = 2\rank(A)$.
	\end{solution*}

	\begin{exercise}
		\label{ex1.5}
		Suppose you have $A+A^\top = J-I$, where $J$ is the all ones matrix. Show that if $\rank(A) < n-1$, there is a vector $x$ such that $Ax = 0$, $x \ne 0$, and $\mathbf{1}^\top x = 0$. Using this, prove that $\rank(A) \ge n-1$.
	\end{exercise}
	\begin{solution*}
		Suppose $\rank(A) < n-1$. Then, $\dim \ker A \ge 2$. We also have $\dim \mathbf{1}^\perp = n-1$. Therefore, $\ker A$ and $\mathbf{1}^\perp$ have nonzero intersection, and say $x \ne 0$ is in both. $x$ satisfies the conditions mentioned in the question. Now,
		\begin{align*}
			0 &= x^\top (Ax) + (x^\top A^\top) x \\
				&= x^\top (J-I) x \\
				&= \left(\sum_i x_i\right)^2 - \left(\sum_i x_i^2\right) = - \sum_i x_i^2,
		\end{align*}
		so $x = 0$, a contradiction. Therefore, $\rank(A) \ge n-1$.
	\end{solution*}

	\begin{exercise}
		Suppose $B_1,\ldots,B_m$ are complete bipartite graphs whose edge disjoint union yields the complete graph $K_n$. Show that $m \ge n-1$.
	\end{exercise}
	\begin{solution*}
		Suppose that $B_i$ corresponds to the complete bipartite graph between sets $X_i,Y_i \subseteq [n]$, where $X_i \cap Y_i = \emptyset$. As a graph on vertex set $[n]$, on setting $M_i = \indic_X \indic_Y^\top$, $B_i$ has adjacency matrix $M_i + M_i^\top$. Note that $\rank(M_i) = 1$ for all $i$, since $\indic_Y^\perp \subseteq \ker M_i$. Because the edge disjoint union of the $B_i$ is $K_n$, we have $(\sum_i M_i) + (\sum_i M_i)^\top = J-I$. Using the previous exercise, $\rank(\sum_i M_i) \ge n-1$. Using \Cref{ex1.3} and the observation that $\rank(M_i) = 1$ for all $i$, this implies that $m = \sum_{i=1}^m \rank(M_i) \ge n-1$, completing the proof.
	\end{solution*}

	\begin{exercise}
		Suppose you have a set system of $m$ sets such that for every pair of sets, the intersection size is fixed as $\lambda \ge 1$. Prove that $m \le n$.
	\end{exercise}
	\begin{solution*}
		Let the set system be $(A_{i})_{i=1}^m$. The size of at most one set is equal to $\lambda$. Furthermore, if $|A_1| = \lambda$, then $A_i \setminus A_1$ are disjoint for distinct $i$, so $m-1 \le n-\lambda$. Thus, we may assume that the size of every set is greater than $\lambda$.\\
		Define the matrix $M$ exactly as in \Cref{ex1.2}. We have that the off-diagonal entries of $M$ are equal to $\lambda$. Now, $MM^\top = \lambda J + D$, for some diagonal matrix $D$ with all positive diagonal entries. We wish to show that $\rank(\lambda J + D) = m$. Let $x \ne 0$ in $\R^n$, and let $u,v$ be the components of $x$ along and orthogonal to $\mathbf{1}$ respectively, such that $u = t\mathbf{1}$. Then,
		\begin{align*}
			(\lambda J + D)x &= (\lambda J + D)(u+v) \\
				&= n\lambda u + D(u+v) \\
				&= D (D^{-1}n\lambda u + u+v).
		\end{align*}
		When $t = 0$, this is clearly nonzero as $v \ne 0$. Otherwise, to conclude, note that
		\[ \sum_i (D^{-1}n\lambda u + u + v)_i = \sum_i (D_{ii}^{-1} n \lambda + 1) u_i + v_i = \sum_i t(D_{ii}^{-1}n\lambda + 1),  \]
		which is nonzero as $d_{ii},\lambda > 0$ and $t \ne 0$.
	\end{solution*}

\section{Tutorial 2: Polynomial Space Methods}

	\begin{exercise}
		\label{ex:2.1}
		Find the dimension of the space spanned by the following polynomials over the given field.
		\begin{enumerate}[label=(\alph*)]
			\item $x_1, x_2, x_1x_2, x_1^2x_2, 1, (x_1+x_2)^2, x_1^2+x_2^2$ over $\R$ and over $\F_2$.
			\item $x_1^{i_1}x_2^{i_2}\cdots x_n^{i_n}$, where $i_1 + \cdots + i_n = m$ over $\R$ and over $\F_2$.
		\end{enumerate}
	\end{exercise}
	\begin{solution*}
		\begin{enumerate}[label=(\alph*)]
			\item Over $\R$, it is clear that $(x_1 + x_2)^2 = (x_1^2 + x_2^2) + 2(x_1x_2)$, and the collection formed by removing $(x_1 + x_2)^2$ is linearly independent, so the dimension of the space is $6$.\\
			Since linear dependence requires that there exists a linear combination of the polynomials which is equal to the zero polynomial (in the sense that every coefficient is $0$), and not merely a polynomial that evaluates to $0$ everywhere, the dimension of this space is $6$ as well.

			\item Over $\R$, all these monomials are linearly independent, so the dimension is the number of ways of choosing $n$ non-negative numbers that sum to $m$. This is a routine exercise in combinatorics, with the answer being $\binom{m+n-1}{m}$.\\
			As in the first part, the dimension over $\F_2$ is $\binom{m+n-1}{m}$ as well.
		\end{enumerate}
	\end{solution*}

	\begin{exercise}
		Given $m$ sets with sizes greater than $d$ and pairwise intersection $d$, prove that $m \le (n+1)$.\\
		\emph{Hint.} \textcolor{gray!20}{Associate a polynomial to each set so that the polynomials are linearly independent. Give an upper bound on the space spanned by these polynomials.}
	\end{exercise}
	\begin{solution*}
		Let $A_1,\ldots,A_m$ be sets of the above form. Associate to each set the polynomial
		\[ p_i(x) = \sum_{j \in A_m} x_j - d. \]
		Let $u_i$ be the indicator vector of $A_i$, equal to $1$ at precisely those coordinates $j$ in $A_i$. Note that $p_i(u_j) \ne 0$ iff $i = j$, so the $p_i$ are linearly independent. Furthermore, the span of the $p_i$ is of dimension at most $n+1$, corresponding to $1$ and the $n$ $x_j$. It follows that $m \le (n+1)$.
	\end{solution*}

	\begin{exercise}
		\phantom{pain}
		\begin{enumerate}[label=(\alph*)]
			\item How do we define the distance between a pair of points in $\R^n$?
			\item Construct as many points as you can so that the distance between a pair is one of two distances, either $d_1$ or $d_2$. You may also choose $d_1$ and $d_2$ to maximize the number.
			\item Consider $m$ points with exactly two pairwise distances. Associate polynomials $p_i(x)$ to each point such that the polynomials are linearly independent.
			\item Deduce an upper bound on the dimension of the span of your polynomials. What does this imply about the number of points with exactly two pairwise distances?
		\end{enumerate}
	\end{exercise}
	\begin{solution*}
		\begin{enumerate}[label=(\alph*)]
			\item Given $x,y \in \R^n$, the $L^2$ distance between them is given by $\|x-y\|_2 = (x-y)^\top(x-y) = \sum_{i=1}^n (x_i - y_i)^2$.
			\item 
			\item Let the points be $u_1,\ldots,u_m$. To each point, associate the polynomial $p_i(x) = (\|x-u_i\|_2^2 - d_1^2)(\|x-u_i\|_2^2 - d_2^2)$. Note that $p_i(u_j) \ne 0$ iff $i = j$. It follows that the polynomials are linearly independent.
			\item We have
			\[ \|x-u\|^2 - d^2 = \sum x_k^2 + \sum x_k (-2u_k) + \sum u_k^2 - d^2, \]
			a linear combination of $\sum x_k^2$, the $x_k$, and $1$. It follows that each $p_i$ is a linear combination of $\left(\sum_{k=1}^n x_k^2\right)^2$, $x_j \left( \sum_{k=1}^n x_k^2 \right)$, $x_j^2$, $x_jx_t$, $x_j$, and $1$, where $j,k$ range over $n$ with $j \ne k$. Therefore, the dimension of the span of $p_i$ is at most
			\[ 1 + n + n + \binom{n}{2} + n + 1 = \frac{n^2}{2} + \frac{5n}{2} + 2. \]
			Since the polynomials are linearly independent, this implies that the number of points with exactly two pairwise distances is at most the above quantity. 
		\end{enumerate}
	\end{solution*}

	\begin{exercise}
		A polynomial is called multilinear if the degree of each variable is at most one. What is the dimension of the space of multilinear polynomials of degree at most $d$ over $n$ variables?
	\end{exercise}
	\begin{solution*}
		The solution to this is near-identical to the second part of \Cref{ex:2.1}(b), with the answer being $\binom{n}{0} + \binom{n}{1} + \cdots + \binom{n}{d}$.
	\end{solution*}

	\begin{exercise}
		Consider $m$ sets $A_1,\ldots,A_m$ such that $|A_i| \equiv k \pmod{p}$ for some prime $p$. Assume that $|A_i \cap A_j| \in L \pmod{p}$ for some set $L$, such that $k \not\in L$ and $|L| = \ell$. Show that $m \le \binom{n}{0}+\cdots+\binom{n}{\ell}$.
	\end{exercise}
	\begin{solution*}
		For each set, associate the polynomial
		\[ q_i(x) = \prod_{u \in L} \left( -u + \sum_{j \in A_i} x_j \right) \]
		over $\F_p$. Denoting by $u_j$ the vector over $\F_p$ that is $1$ precisely at coordinates in $A_j$ and $0$ elsewhere, note that $q_i(u_j) \ne 0$ iff $i = j$. Now, consider the polynomial $p_i$ obtained by opening up the product in the above definition, and replacing any occurence of $x_j^t$ by $x_j$ for $t \ge 1$. Since any coordinate of the $u_j$ is $0$ or $1$, $p_i(u_j) = q_i(u_j)$ for any $j$. In particular, $p_i(u_j) \ne 0$ iff $i = j$ and so the $p_i$ are linearly independent. Furthermore, since the $p_i$ are multilinear, the dimension of their span is at most $\binom{n}{0} + \cdots + \binom{n}{\ell}$ as in the previous problem.
	\end{solution*}

	\begin{exercise}
		For a prime power $q = p^t$, prove that $\binom{r-1}{q-1}$ is divisible by $p$ iff $r$ is not divisible by $q$.
	\end{exercise}
	\begin{solution*}
		
	\end{solution*}

	\begin{exercise}
		Let $q = p^t$ and $k \in \Z$. Let $(A_i)_{i=1}^m$ be subsets of $[n]$ such that $|A_i| \equiv k \pmod{q}$ for each $i$ and $|A_i \cap A_j| \not\equiv k \pmod{q}$ for $i \ne j$. Then, show that $m \le \binom{n}{q-1} + \binom{n}{q-3} + \cdots$.
	\end{exercise}
	\begin{solution*}
		
	\end{solution*}

\section{Tutorial 3: Real Symmetric Matrices}

	\begin{exercise}
		Given an $n \times n$ real matrix $A$, a vector $v$ is called an eigenvector with eigenvalue $\lambda$ if it satisfies $Av = \lambda v$. Show that $\lambda$ is an eigenvalue iff the determinant of $A-\lambda\Id$ is $0$.
	\end{exercise}
	\begin{solution*}
		The determinant of $A-\lambda\Id$ is $0$ iff it is not full-rank, that is, there exists some $v$ such that $(A-\lambda\Id)v = 0$, which is equivalent to $\lambda$ being an eigenvalue of $A$.
	\end{solution*}

	\begin{exercise}
		Note that $\det(A-\lambda\Id)$ is a polynomial in $\lambda$ with real coefficients. Hence, it factors over the complex numbers. Let $v_i$ be the eigenvector corresponding to $\lambda_i$. Prove that $\Tr(A) = \sum_i \lambda_i$, and prove that the determinant is the product of the eigenvalues.
	\end{exercise}

	

\end{document}