\documentclass[dvipsnames]{beamer}
% \mode<presentation>{}
\usepackage[utf8]{inputenc}
\usepackage{amsmath, amssymb, amsfonts, amsthm, mathtools, mathrsfs}
\setbeamertemplate{theorems}[numbered]
\title{Coordinate Hit-and-run}
\author{Amit Rajaraman}
\date{\today}
% \institute{IIT Bombay}


\usetheme{Madrid}


\usepackage{parskip}
\usepackage{tcolorbox}
\usepackage{tikz-cd}
% \usepackage{commands}
\usepackage{graphicx}
\usepackage{soul}

\usepackage{tikz}
\usetikzlibrary{topaths,calc}
\usepackage{caption}
\usepackage{subcaption}
\usepackage{cancel}
\usepackage{commath}
\usepackage{sansmathaccent}
\pdfmapfile{+sansmathaccent.map}
\usepackage[absolute,overlay]{textpos}
\usepackage{framed,fancybox}
\usepackage{tcolorbox}
\usepackage{enumitem}
\definecolor{foo}{rgb}{0.2,0.2,0.7}

\newcommand\colorfbox[3]{%
 {\color{#1}\fbox{\parbox{\dimexpr\linewidth-2\fboxsep-2\fboxrule\relax}{\color{#2}#3}}}%
}


% \newenvironment{cframed}
%   {\def\FrameCommand{\fboxsep=\FrameSep\fcolorbox{blue}{white}}%
%     \MakeFramed {\advance\hsize-\width \FrameRestore}}
%   {\endMakeFramed}

% \renewcommand\fbox[1]{\Ovalbox{#1}}
\newcommand\MyText[1]{
  \begin{textblock*}{5cm}(.7\textwidth,0.7cm)
    \colorfbox{foo}{black}{#1}\phantom{.}
    % \begin{cframed}
    % 	#1
    % \end{cframed}
  \end{textblock*}
}

\setbeamerfont{bibliography entry author}{size=\small}
\setbeamerfont{bibliography entry title}{size=\small}
\setbeamerfont{bibliography entry location}{size=\small}
\setbeamerfont{bibliography entry note}{size=\small}
\setbeamerfont{bibliography item}{size=\small}


% \tikzset{
%     invisible/.style={opacity=0},
%     visible on/.style={alt={#1{}{invisible}}},
%     alt/.code args={<#1>#2#3}{%
%       \alt<#1>{\pgfkeysalso{#2}}{\pgfkeysalso{#3}}%
%   }
% }

\makeatletter
\newenvironment<>{proofs}[1][\proofname]{%
	\par
	\def\insertproofname{#1\@addpunct{.}}%
	\usebeamertemplate{proof begin}#2}
  {\usebeamertemplate{proof end}}
\makeatother


\newcommand{\R}{\mathbb{R}}
\newcommand{\Rp}{\mathbb{R}^+}
\newcommand{\Rn}{\mathbb{R}^n}
\newcommand{\C}{\mathbb{C}}
\newcommand{\N}{\mathbb{N}}
\newcommand{\Z}{\mathbb{Z}}
\newcommand{\Q}{\mathbb{Q}}
\newcommand{\F}{\mathbb{F}}
\newcommand{\GL}{\operatorname{GL}}
\newcommand\restr[2]{\ensuremath{\left.#1\right|_{#2}}}
\newcommand{\End}{\operatorname{End}}
\newcommand{\bigplus}{\scalerel*{+}{\textstyle\sum}}
% \newcommand{\ker}{\operatorname{ker}}
\newcommand{\im}{\operatorname{im}}
\newcommand{\Hom}{\operatorname{Hom}}
\newcommand{\diag}{\operatorname{diag}}
\newcommand{\Tr}{\operatorname{Tr}}
\newcommand{\incl}{\hookrightarrow}
\newcommand{\Id}{\operatorname{Id}}
\newcommand{\Cl}{\operatorname{Cl}}
\newcommand{\A}{\mathbb{A}}


% \setbeamercolor{footline}{fg=brown}
% \setbeamerfont{footline}{series=\bfseries}
% \addtobeamertemplate{navigation symbols}{}{%
%     \usebeamerfont{footline}%
%     \usebeamercolor[fg]{footline}%
%     \hspace{1em}%
%     [\insertframenumber/\inserttotalframenumber]
% }

\theoremstyle{definition}
\newtheorem{thm}{Theorem}
\newtheorem{defn}[thm]{Definition}
\newtheorem{prop}[thm]{Proposition}
\newtheorem{cor}[thm]{Corollary}
\newtheorem{caution}[thm]{Caution}
\newtheorem{ques}{Question}
\newtheorem*{ques*}{Question}
\newtheorem*{alg}{Algorithm}
\newtheorem*{fac}{Fact}
\newtheorem*{ex}{Example}
\newtheorem{lem}[thm]{Lemma}


% \DeclareMathOperator{\len}{len}
% \newcommand{\md}[1]{\left\lvert #1 \right\rvert}

\AtBeginSection[]
{
  \begin{frame}
	\frametitle{Table of Contents}
	\tableofcontents[currentsection]
  \end{frame}
}

\begin{document}
\begin{frame}
	\titlepage
\end{frame}

\begin{frame}
	\frametitle{Table of Contents}
	\tableofcontents
\end{frame}

%%% PERHAPS MENTION IN THE BEGINNING THAT ALL V-SPACES ARE OVER C AND ALL GROUPS ARE FINITE

\section{Introduction}

\begin{frame}{Representations}
	\begin{defn}[Representation]
		A \emph{representation} of a group $G$ is a homomorphism $\varphi : G \to \GL(V)$ for some finite-dimensional vector space $V$ over $\C$.\pause The \emph{degree} of $\varphi$ is the dimension of $V$.
	\end{defn}
	\pause
	Representations can be thought of as group actions $G \to S_X$, with the additional specification that the images are not just bijections, they are isomorphisms.
\end{frame}

\begin{frame}{Equivalence of representations}
	We would like that representations are the same even when expressed over different bases of $V$.
	\pause
	\begin{defn}[Equivalence]
		Two representations $\varphi : G \to \GL(V)$ and $\psi : G \to \GL(W)$ are said to be \emph{equivalent} if there exists an isomorphism (an \emph{equivalence}) $T : V \to W$ such that $\psi_g = T\varphi_gT^{-1}$ for all $g \in G$. If this is the case, we write $\varphi \sim \psi$.
	\end{defn}
	\pause

	\begin{center}
	\begin{tikzcd}[ampersand replacement=\&]
		V \arrow[r, "\varphi_g"] \arrow[d, "T"] \& V \arrow[d, "T"] \\
		W \arrow[r, "\psi_g"] \& W
	\end{tikzcd}
	\end{center}
\end{frame}

\begin{frame}{Irreducible representations}
		\begin{defn}[Invariant subspace]
			Let $\varphi : G \to \GL(V)$ be a representation. A subspace $W \le V$ is said to be \emph{$G$-invariant} with respect to $\varphi$ if for all $g \in G$ and $w \in W$, $\varphi_g(W) = W$.
		\end{defn}

		Note that if $W \le V$ is a $G$-invariant subspace, then $\restr{\varphi}{W} : G \to \GL(W)$ defined by $(\restr{\varphi}{W})_g(w) = \varphi_g(w)$ is a representation!

		\begin{defn}[Irreducible representation]
			A non-zero representation $\varphi : G \to \GL(V)$ is said to be \emph{irreducible} if the only $G$-invariant subspaces of $V$ are $0$ and $V$.
		\end{defn}

\end{frame}

\begin{frame}{Direct sum}
	\begin{defn}[Direct sum]
		Let $\varphi^{(1)} : G \to \GL(V_1)$ and $\varphi^{(2)} : G \to \GL(V_2)$ be representations. Then, their (external) \emph{direct sum} is the representation $\varphi^{(1)} \oplus \varphi^{(2)} : G \to \GL(V_1 \oplus V_2)$ defined by
		\[ \left(\varphi^{(1)} \oplus \varphi^{(2)}\right)_g(v_1,v_2) = (\varphi^{(1)}_g(v_1), \varphi^{(1)}_g(v_2)) \]
		for all $g \in G$ and $(v_1,v_2) \in V_1 \oplus V_2$.
	\end{defn}
	The above is more natural to picture using matrices.\\
	If $V_1 = \GL_m(\C)$ and $V_2 = \GL_n(\C)$ above, then each $\varphi^{(i)}_g$ can be expressed as a matrix. The matrix in $\GL_{m+n}(\C)$ corresponding to their direct sum is then given by
	\[ \left( \varphi^{(1)} \oplus \varphi^{(2)} \right)_g = \begin{pmatrix} \varphi^{(1)}_g &  \\  & \varphi^{(2)}_g \end{pmatrix}, \]
	where the empty cells are appropriately sized $0$ matrices.
\end{frame}

\begin{frame}{More on decomposing representations}
	\begin{defn}[Complete Reducibility]
		Let $G$ be a group. A representation $\varphi : G \to \GL(V)$ is said to be \emph{completely reducible} if $V = V_1 \oplus \cdots \oplus V_n$ where each $V_i$ is $G$-invariant and $\restr{\varphi}{V_i}$ is irreducible for each $i$.
	\end{defn}

	Note that even an irreducible representation is completely reducible. All we desire is that it can be ``decomposed'' into the direct sum of irreducible representations.

	\begin{defn}[Decomposability]
		A non-zero representation $\varphi$ is said to be \emph{decomposable} if $V = V_1 \oplus V_2$ for some non-zero $G$-invariant subspaces $V_1,V_2 \le V$. Otherwise, $\varphi$ is said to be \emph{indecomposable}.
	\end{defn}
\end{frame}

\begin{frame}

	Recall that a matrix $U \in \GL_n(\C)$ is said to be \emph{unitary} if $U^*U = I_n$.

	\begin{defn}[Unitary]
		Let $V$ be an inner product space. A representation $\varphi : G \to \GL(V)$ is said to be \emph{unitary} if $\varphi_g$ is unitary for every $g \in G$.
	\end{defn}

	\begin{lem}
		Any representation of a finite group $G$ is equivalent to a unitary representation.
	\end{lem}
\end{frame}

\begin{frame}{Maschke's Theorem}
	\begin{thm}[Maschke's Theorem]
		Every representation of a finite group is completely reducible.
	\end{thm}
\end{frame}

%%% MAYBE 2-3 SLIDES OF PROOF SKETCH, INSERT THE PREV LEMMA HERE INSTEAD!

\section{Morphisms}

\begin{frame}
	\begin{defn}[Morphism]
		Let $\varphi : G \to \GL(V)$ and $\rho : G \to \GL(W)$ be representations. A \emph{morphism} from $\varphi$ to $\rho$ is a linear map $T : V \to W$ such that the following diagram commutes for all $g \in G$.
		\begin{center}
		\begin{tikzcd}[ampersand replacement=\&]
			V \arrow[r, "\varphi_g"] \arrow[d, "T"] \& V \arrow[d, "T"] \\
			W \arrow[r, "\rho_g"] \& W
		\end{tikzcd}
		\end{center}
		The set of all morphisms from $\varphi$ to $\rho$ is denoted $\Hom_G(\varphi,\rho)$.
	\end{defn}
	Moreover, note that if $T$ is an isomorphism (between $V$ and $W$) in the above, it is an equivalence.
\end{frame}

\begin{frame}
	\begin{lem}
		Let $\varphi,\rho$ be irreducible representations of a group $G$.
		\begin{enumerate}[label = (\alph*)]
			\item If $\varphi \not\sim \rho$, $\Hom_G(\varphi,\rho) = 0$.
			\item $\Hom_G(\varphi,\varphi) = \{\lambda I : \lambda \in \C\}$.
		\end{enumerate}
	\end{lem}

	\begin{cor}
		Let $\varphi^{(1)},\ldots,\varphi^{(s)}$ be pairwise inequivalent irreducible representations of $G$. Set
		\[ \varphi = \underbrace{\varphi^{(1)} \oplus \cdots \oplus \varphi^{(1)}}_{m_1} \oplus \cdots \oplus \underbrace{\varphi^{(s)} \oplus \cdots \oplus \varphi^{(s)}}_{m_s}. \]
		Then, $\dim \Hom_G(\varphi^{(r)},\varphi) = m_r$ for each $r$.
	\end{cor}
\end{frame}

\begin{frame}{The group algebra}
	\begin{defn}
		Let $G$ be a group. Define the \emph{group algebra} $L(G) = \C^G$. We endow it with the inner product
		\[ \langle f_1,f_2\rangle = \frac{1}{|G|} \sum_{g \in G} f_1(g) \overline{f_2(g)}. \]
	\end{defn}
\end{frame}

\begin{frame}
	\begin{thm}
			Let $\varphi : G \to U_n(\C)$ and $\rho : G \to U_m(\C)$ be inequivalent irreducible unitary representations of a group $G$. Then,
			\begin{enumerate}[label = (\alph*)]
				\item $\langle \varphi_{ij} , \rho_{kl} \rangle = 0$.
				\item $\langle \varphi_{ij} , \varphi_{kl} \rangle = \begin{cases} 1/n , & (i,j) = (k,l), \\ 0, & \text{otherwise.} \end{cases}$
			\end{enumerate}
			In particular, the set $\{ \varphi_{ij} : 1 \le i,j \le n \} \cup \{ \rho_{kl} : 1 \le k,l \le m \}$ is a linearly independent set.
	\end{thm}
\end{frame}

\begin{frame}
	\begin{thm}
	Let $G$ be a (finite) group.
	\begin{enumerate}[label = (\alph*)]
		\item There are finitely many equivalence classes of irreducible representations of $G$.
		\item Let $\varphi^{(1)},\ldots,\varphi^{(s)}$ be a transversal of unitary irreducible representations of $G$. Set $d_i = \deg \varphi^{(i)}$. Then, the set of functions
		\[ \{ \sqrt{d_k} \varphi^{(k)}_{ij} : 1 \le k \le s, 1 \le i,j \le d_k \} \]
		is orthonormal.
	\end{enumerate}
\end{thm}
\end{frame}

\section{Character Theory}

\section{Fourier analysis}

\section{Case study: Abelian groups}

% \begin{frame}{The problem}
% 	% The B\'{a}r\'{a}ny-Fur\"{e}di Theorem in \cite{barany-furedi} shows that there exists no deterministic polynomial time algorithm to approximate the volume of a convex body. \pause \\
% 	\begin{ques*}
% 		Input: A convex body $K \subseteq \Rn$ with a well-guaranteed membership oracle.

% 		Output: A probability distribution on $\Rn$ that is at total variation distance at most $\epsilon$ from the uniform distribution on $K$.
% 	\end{ques*}
% 	Denote by $\pi_K$ the uniform distribution on $K$.
% \end{frame}

% \begin{frame}{}
%   \centering \Huge
%   \emph{Thank you!}
% \end{frame}


% \begin{frame}[allowframebreaks,t]{References}
% 	\bibliographystyle{alpha}
% 	\small\bibliography{references}
% \end{frame}

\end{document}