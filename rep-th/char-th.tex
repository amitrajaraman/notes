\section{Character Theory and Orthogonality Relations}

	\subsection{Morphisms}

		\begin{fdef}[Morphism]
			Let $\varphi : G \to \GL(V)$ and $\rho : G \to \GL(W)$ be representations. A \emph{morphism} from $\varphi$ to $\rho$ is a linear map $T : V \to W$ such that the following diagram commutes for all $g \in G$.
			\begin{center}
			\begin{tikzcd}
				V \arrow[rr, "\varphi_g"] \arrow[dd, "T"] & & V \arrow[dd, "T"] \\
				& & \\
				W \arrow[rr, "\rho_g"] & & W
			\end{tikzcd}
			\end{center}
			The set of all morphisms from $\varphi$ to $\rho$ is denoted $\Hom_G(\varphi,\rho)$.
		\end{fdef}

		By definition, $\Hom_G(\varphi,\rho) \subseteq \Hom(V,W)$.\\

		Recall that any representation is just a special group action of $G$ on the vector space of interest. Based off this, writing $gv$ instead of $\varphi_g v$, the definition of a morphism can be alternatively written as saying that $Tgv = gTv$ for all $g\in G, v \in V$.\footnote{the first $g$ is a $\varphi_g$ and the second is a $\rho_g$}.

		Also observe that if $T \in \Hom_G(\varphi,\rho)$ is an isomorphism, then $\varphi \sim \rho$.

		\begin{remark}
			$T \in \Hom(V,V)$ is in $\Hom_G(\varphi,\varphi)$ iff it commutes with every $\varphi_g$. In particular, the identity map is an element of $\Hom_G(\varphi,\varphi)$.
		\end{remark}

		\begin{fprop}
			Let $\varphi : G \to \GL(V)$ and $\rho : G \to \GL(W)$ be representations, and $T \in \Hom_G(\varphi,\rho)$. $\ker T$ and $\im T$ are $G$-invariant subspaces of $V$ and $W$ with respect to $\varphi$ and $\rho$ respectively.
		\end{fprop}
		\begin{proof}
			
		\end{proof}