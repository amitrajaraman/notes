\section{Character Theory and Orthogonality Relations}

	\subsection{Morphisms}

		\begin{fdef}[Morphism]
			Let $\varphi : G \to \GL(V)$ and $\rho : G \to \GL(W)$ be representations. A \emph{morphism} from $\varphi$ to $\rho$ is a linear map $T : V \to W$ such that the following diagram commutes for all $g \in G$.
			\begin{center}
			\begin{tikzcd}
				V \arrow[rr, "\varphi_g"] \arrow[dd, "T"] & & V \arrow[dd, "T"] \\
				& & \\
				W \arrow[rr, "\rho_g"] & & W
			\end{tikzcd}
			\end{center}
			The set of all morphisms from $\varphi$ to $\rho$ is denoted $\Hom_G(\varphi,\rho)$.
		\end{fdef}

		By definition, $\Hom_G(\varphi,\rho) \subseteq \Hom(V,W)$.\\

		Recall that any representation is just a special group action of $G$ on the vector space of interest. Based off this, writing $gv$ instead of $\varphi_g v$, the definition of a morphism can be alternatively written as saying that $Tgv = gTv$ for all $g\in G, v \in V$.\footnote{the first $g$ is a $\varphi_g$ and the second is a $\rho_g$}.

		Also observe that if $T \in \Hom_G(\varphi,\rho)$ is an isomorphism, then $\varphi \sim \rho$.

		\begin{remark}
			$T \in \Hom(V,V)$ is in $\Hom_G(\varphi,\varphi)$ iff it commutes with every $\varphi_g$. In particular, the identity map is an element of $\Hom_G(\varphi,\varphi)$.
		\end{remark}

		\begin{fprop}
			Let $\varphi : G \to \GL(V)$ and $\rho : G \to \GL(W)$ be representations, and $T \in \Hom_G(\varphi,\rho)$. $\ker T$ and $\im T$ are $G$-invariant subspaces of $V$ and $W$ with respect to $\varphi$ and $\rho$ respectively.
		\end{fprop}
		\begin{proof}
			Let $v \in \ker T$. Then, for $g \in G$,
			\[ T (\varphi_g v) = \rho_g T v = 0,  \]
			so $\varphi_g v \in \ker T$. Similarly, for $w \in \im T$, letting $v \in V$ such that $Tv = w$, 
			\[ \rho_g w = \rho_g T v = T (\varphi_g v) \in \im T.  \]
		\end{proof}

		We had mentioned earlier that $\Hom_G(\varphi,\rho) \subseteq \Hom(V,W)$. In fact, the following stronger statement is true.

		\begin{fprop}
			\label{prop: hom is a subspace}
			Let $G$ be a group and $\varphi : G \to \GL(V)$, $\rho : G \to \GL(W)$ be representations. Then $\Hom_G(\varphi,\rho)$ is a subspace of $\Hom(V,W)$.
		\end{fprop}
		\begin{proof}
			Clearly, $0 \in \Hom_G(\varphi,\rho)$. If $S,T \in \Hom_G(\varphi,\rho)$ and $\alpha \in \C$, then for any $g \in G$ and $v \in V$,
			\begin{align*}
				(S + \alpha T) \varphi_g v &= S \varphi_g v + \alpha T \varphi_g v \\
					&= \rho_g S v + \alpha \rho_g T v \\
					&= \rho_g S v + \rho_g (\alpha T) v = \rho_g (S + \alpha T) v, 
			\end{align*}
			so $S + \alpha T \in \Hom_G(\varphi,\rho)$.
		\end{proof}

		Another expected result is that the homomorphism subspaces of equivalent representations are isomorphic.

		\begin{prop}
			\label{prop: homs of equivalent are isomorphic}
			Let $G$ be a group and $\varphi_i : G \to \GL(V_i)$, $\rho_i : G \to \GL(W_i)$ be representations for $i = 1,2$. If $\varphi^{(1)} \sim \varphi^{(2)}$ and $\rho^{(1)} \sim \rho^{(2)}$, then $\dim \Hom_G(\varphi^{(1)},\rho^{(1)}) = \dim \Hom_G(\varphi^{(2)},\rho^{(2)})$.
		\end{prop}
		\begin{proof}
			Let $P : V_1 \to V_2$ and $R : W_1 \to W_2$ be corresponding equivalences. Consider $\Phi : \Hom_G (\varphi^{(1)},\rho^{(1)}) \to \Hom_G (\varphi^{(2)},\rho^{(2)})$ defined by $\Phi(S) = R \circ S \circ P^{-1}$. We claim that $\Phi$ is an isomorphism between the subspaces. Let us first show that this does indeed map into $\Hom_G(\varphi^{(2)},\rho^{(2)})$. We have that for any $g \in G$ and $v \in V_1$,
			\begin{align*}
				\Phi(S) (\varphi^{(2)})_g v &= R S P^{-1} (\varphi^{(2)})_g v \\
					&= R S (\varphi^{(1)})_g P^{-1} v & \text{($P^{-1}$ is an equivalence)} \\
					&= R (\rho^{(1)})_g S P^{-1} v & (S \in \Hom_G(\varphi^{(1)},\rho^{(1)})) \\
					&= (\rho^{(2)})_g R S P^{-1} v & \text{($R$ is an equivalence)} \\
					&= (\rho^{(2)})_g \Phi(S) v.
			\end{align*}
			It is clear that $\Phi$ is linear, and further that $\Phi$ is a bijection as an inverse is easily constructed similarly.
		\end{proof}

		\begin{flem}[Schur's Lemma]
			\label{lem: schurs lemma}
			Let $G$ be a group, $\varphi : G \to \GL(V)$ and $\rho : G \to \GL(W)$ be irreducible representations of $G$, and $T \in \Hom_G(\varphi,\rho)$. Then, either $T$ is an equivalence or $T = 0$.
		\end{flem}
		\begin{proof}
			Suppose that $T \ne 0$. It suffices to show that $T$ is a bijection. If $\ker T \ne 0$, then we have a nonzero proper subspace $\ker T$ that is $G$-invariant (with respect to $\varphi$), contradicting irreducibility (of $\varphi$). Therefore, $T$ is injective. Similarly, $\im T \ne 0$ and if $\im T \ne W$, we have a nonzero proper subspace $\im T$ that is $G$-invariant (with respect to $\rho$), contradicting irreducibility (of $\rho$). Therefore, $T$ is surjective, completing the first part of the proof.
		\end{proof}

		\begin{fcor}
			\label{cor: schurs corollary}
			Let $G$ be a group, $\varphi : G \to \GL(V)$ and $\rho : G \to \GL(W)$ be irreducible representations of $G$, and $T \in \Hom_G(\varphi,\rho)$.
			\begin{enumerate}[label=(\alph*)]
				\item If $\varphi \not\sim \rho$, then $\Hom_G(\varphi,\rho) = 0$.
				\item If $\varphi = \rho$, $T = \lambda I$ for some $\lambda \in \C$. That is, $\Hom_G (\varphi,\varphi)$ is one-dimensional with basis $\{I\}$.
			\end{enumerate}
		\end{fcor}
		\begin{proof}
			(a) is immediate from Schur's Lemma.\\
			For (b), let $\lambda$ be an eigenvalue of $T$ (which exists since $\C$ is algebraically closed). Recall that $I \in \Hom_G(\varphi,\varphi)$. It follows from \Cref{prop: hom is a subspace} that $T - \lambda I \in \Hom_G(\varphi,\varphi)$. By the definition of an eigenvalue, $T - \lambda I$ is not invertible. Therefore, $T - \lambda I = 0$, proving the required.
		\end{proof}

		Next, let us show that a direct sum of representations corresponds to a direct sum of their $\Hom$s as well.

		\begin{fprop}
			\label{prop: direct sum of reps direct sum of Homs}
			Let $\varphi : G \to \GL(V)$ and $\rho_i : G \to \GL(W_i)$ be representations for $i = 1,2$. It is true that
			\[ \Hom_G(\varphi,\rho^{(1)} \oplus \rho^{(2)}) \cong \Hom_G(\varphi,\rho^{(1)}) \oplus \Hom_G(\varphi,\rho^{(2)}). \]
			In particular,
			\[ \dim \Hom_G(\varphi,\rho^{(1)} \oplus \rho^{(2)}) = \dim \Hom_G(\varphi,\rho^{(1)}) + \dim \Hom_G(\varphi,\rho^{(2)}). \]
		\end{fprop}
		\begin{proof}
			Let $T \in \Hom_G(\varphi,\rho^{(1)}\oplus\rho^{(2)}) \subseteq \Hom(V,W_1\oplus W_2)$. Letting $\pi_i$ denote the projection maps, $\pi_i \circ T : V \to W_i$ is linear for $i = 1,2$. Further, $(\pi_i \circ T) \in \Hom_G(\varphi,\rho_i)$ because
			\[ (\pi_i \circ T) \varphi_g v = \pi_i (\rho^{(1)} \oplus \rho^{(2)})_g T v = (\rho_i)_g T v. \]
			On the other hand, given morphisms $T_i \in \Hom_G(V,W_i)$ for $i = 1,2$, $T : V \to W$ defined by $T(v) = (T_1(v), T_2(v))$ is also a morphism. As a result, the correspondence $(T_1,T_2) \mapsto T$ is bijective and $\C$-linear, so is an isomorphism.
		\end{proof}

		\begin{fcor}
			Let $\varphi^{(1)},\ldots,\varphi^{(s)}$ be pairwise inequivalent irreducible representations of $G$. Set
			\[ \varphi = \underbrace{\varphi^{(1)} \oplus \cdots \oplus \varphi^{(1)}}_{m_1} \oplus \cdots \oplus \underbrace{\varphi^{(s)} \oplus \cdots \oplus \varphi^{(s)}}_{m_s}. \]
			Then,
			\[ \dim \Hom_G (\varphi^{(r)}, \varphi) = m_r \]
			for $1 \le r \le s$.
		\end{fcor}
		\begin{proof}
			We have
			\begin{align*}
				\dim \Hom_G (\varphi^{(r)}, \varphi) &= \sum_{i=1}^s m_i \dim \Hom_G (\varphi^{(r)},\varphi^{(i)}) & \text{(by \Cref{prop: direct sum of reps direct sum of Homs})} \\
					&= m_r & \text{(by \Cref{cor: schurs corollary})}. \qedhere
			\end{align*}
		\end{proof}

		The above says that if we know that a representation is completely reducible and we know the (pairwise inequivalent and irreducible) representations that occur in a decomposition, then the number of times each representation occurs is fixed as well.

		\begin{fcor}
			\label{cor: completely reducible unique decomp}
			Let $\varphi^{(1)},\ldots,\varphi^{(s)}$ and $\psi^{(1)},\ldots,\psi^{(r)}$ be pairwise\footnote{the two lists are separately pairwise inequivalent} inequivalent irreducible representations of $G$. Let $\varphi$ be a representation of $G$ such that
			\[ \varphi \cong \bigoplus_{i=1}^{s} (\varphi^{(i)})^{m_i} \cong \bigoplus_{j=1}^r (\psi^{(j)})^{n_j} \]
			where $m_i,n_j > 0$.
			Then, $r=s$ and there is a permutation $\sigma$ of $[r]$ such that $\varphi^{(i)} \sim \psi^{(\sigma_i)}$ and $m_i = n_{\sigma(i)}$.
		\end{fcor}
		\begin{proof}
			It suffices to show that each $\varphi_i$ is equivalent to some $\psi_j$. Indeed, pairwise inequivalence then implies that $r=s$, and the previous corollary shows that $m_i = n_j$. Suppose instead that $\varphi^{(1)}$ is not equivalent to any $\psi_j$. Then, denoting by $(psi^{(j)})^{n_j}$ the direct sum of $n_j$ $\psi^{(j)}$s,
			\begin{align*}
				m_1 &= \Hom(\varphi^{(1)}, \varphi) \\
					&= \Hom(\varphi^{(1)}, \bigoplus_{j=1}^s (\psi^{(j)})^{n_j}) \\
					&= \sum_{j=1}^s n_j \Hom(\varphi^{1},\psi^{(j)}) = 0,
			\end{align*}
			leading to a contradiction and completing the proof.
		\end{proof}

		Compare this to \nameref{theo: maschkes theorem}. There we had that any representation of a finite group is completely reducible. Here, we have shown that the decomposition of any completely reducible representation is ``unique''!\\

		Recall \Cref{prop: cyclic group degree one}.

		\begin{ftheo}[Irreducible representations of finite abelian groups]
			\label{theo: irreducible reps of finite abelian groups}
			Let $G$ be an abelian group. Any irreducible representation of $G$ has degree $1$.
		\end{ftheo}
		\begin{proof}
			Let $\varphi : G \to \GL(V)$ be an irreducible representation.\\
			For any $h \in G$, for all $g \in G$ $\varphi_h \varphi_g = \varphi_g \varphi_h$, so $\varphi_h \in \Hom_G(\varphi,\varphi)$. \Cref{cor: schurs corollary}(b) then shows that $\varphi_h = \lambda_h I$ for some $\lambda_h \in \C$ (this uses that $\varphi$ is irreducible). Fix any $v \in V$. Then, $\varphi_h v = \lambda_h I v = \lambda_h v \in \C v$, so $\C v$ is a $G$-invariant subspace. By irreducibility, $V = \C v$ and is thus one-dimensional.
		\end{proof}

		Further recall that we had characterized the degree one representations of an abelian group in \Cref{cor: deg one reps of abelian group}.

		\begin{corollary}
			\label{cor: rep of finite abelian group is diagonalizable}
			Let $G$ be a finite abelian group and $\varphi : G \to \GL_n(\C)$ a representation. Then, there exists invertible $T$ such that $T^{-1}\varphi_g T$ is diagonal for all $g \in G$.
		\end{corollary}
		Note that $T$ is independent of $g$.
		\begin{proof}
			Since $G$ is finite and abelian, we can write using \Cref{theo: maschkes theorem,theo: irreducible reps of finite abelian groups} that
			\[ \varphi = \varphi^{(1)} \oplus \cdots \oplus \varphi^{(n)} \]
			where each $\varphi^{(i)}$ is degree $1$. If $T$ is an isomorphism giving the above equivalence, then
			\[ T^{-1} \varphi_g T = \diag(\varphi^{(1)}_g,\ldots,\varphi^{(n)}_g). \]
		\end{proof}

		\begin{corollary}
			\label{cor: finite order diagonalisable}
			Let $A \in \GL_n(\C)$ be of finite order. Then $A$ is diagonalisable.
		\end{corollary}
		\begin{proof}
			Consider the representation $\varphi : \Z/n\Z \to \GL_n(\C)$ defined by $\overline{k} \mapsto A^k$. \Cref{cor: rep of finite abelian group is diagonalizable} implies that $\varphi_{\overline{1}} = A$ is diagonalisable (in fact, $I,A,\ldots,A^{n-1}$ are simultaneously diagonalisable).
		\end{proof}

	\subsection{The Orthogonality Relations}

		\begin{tcolorbox}[colframe=red!25,colback=red!25]
			\underline{\textbf{For the remainder of this report, assume that any group $G$ is finite}} unless otherwise mentioned.
		\end{tcolorbox}

		\begin{fdef}
			Let $G$ be a group. Define the \emph{group algebra} $L(G) \coloneqq \C^G$. $L(G)$ is a vector space over $\C$ in the natural sense. It is also an inner product space when equipped with the inner product
			\[ \langle f_1,f_2 \rangle = \frac{1}{|G|} \sum_{g \in G} f_1(g) \overline{f_2(g)}. \]
			In particular, the \emph{norm} of $f \in L(G)$ is $\norm{f} = \sqrt{\langle f,f\rangle}$.
		\end{fdef}

		Note that the sum involved in $\langle f_1,f_2\rangle$ makes sense because $G$ is finite. Given a representation $\varphi : G \to \GL_n(\C)$, we get $n^2$ elements $\varphi_{ij} : G \to \C$ corresponding to the $n^2$ entries of the matrix. We shall study $\varphi_{ij}$ when $\varphi$ is irreducible and unitary.

		\begin{fprop}
			\label{prop def: sharp}
			Let $\varphi : G \to \GL(V)$ and $\rho : G \to \GL(W)$ be representations. Define for any linear transformation $T : V \to W$
			\[ T^\sharp = \frac{1}{|G|} \sum_{g \in G} \rho_{g^{-1}} T \varphi_g \in \Hom_G(\varphi,\rho). \]
			Then,
			\begin{enumerate}[label=(\alph*)]
				\item $T^\sharp \in \Hom_G(\varphi,\rho)$,
				\item if $T \in \Hom_G(\varphi,\rho)$, then $T^\sharp = T$, and
				\item the map $P : \Hom_\C(V,W) \to \Hom_G(\varphi,\rho)$ defined by $T \mapsto T^\sharp$ is a surjective linear map.
			\end{enumerate}
		\end{fprop}
		\begin{proof}
			\phantom{pain}
			\begin{enumerate}
				\item For any $h \in H$,
				\begin{align*}
					T^\sharp \varphi_h &= \frac{1}{|G|} \sum_{g \in G} \rho_{g^{-1}} T \varphi_g \varphi_h \\
						&= \frac{1}{|G|} \sum_{g \in G} \rho_{g^{-1}} T \varphi_{gh} \\
						&= \frac{1}{|G|} \sum_{g' \in G} \rho_{hg'^{-1}} T \varphi_{g'} & (g \mapsto gh \text{ defines a bijection }G\to G) \\
						&= \rho_h \frac{1}{|G|} \sum_{g' \in G} \rho_{g'^{-1}} T \varphi_{g'} = \rho_h T^\sharp.
				\end{align*}

				\item If $T \in \Hom_G(\varphi,\rho)$, then
				\begin{align*}
					T^\sharp &= \frac{1}{|G|} \sum_{g \in G} \rho_{g^{-1}} T \varphi_g \\
						&= \frac{1}{|G|} \sum_{g \in G} \rho_{g^{-1}} \rho_g T \\
						&= \frac{1}{|G|} \sum_{g \in G} T = T.
				\end{align*}

				\item (b) shows that $P$ is surjective. For linearity, we have that for any $T_1,T_2 \in \Hom_\C(V,W)$ and $c \in \C$,
				\begin{align*}
					P(cT_1 + T_2) &= \frac{1}{|G|} \sum_{g \in G} \rho_{g^{-1}} (cT_1 + T_2) \varphi_g \\
						&= c \frac{1}{|G|} \sum_{g \in G} \rho_{g^{-1}} T_1 \varphi_g + \frac{1}{|G|} \sum_{g \in G} \rho_{g^{-1}} T_2 \varphi_g \\
						&= cP(T_1) + P(T_2). \qedhere
				\end{align*}
			\end{enumerate}
		\end{proof}

		\begin{fprop}
			\label{prop: schur lemma but sharp}
			Let $\varphi : G \to \GL(V)$ and $\rho : G \to \GL(W)$ be irreducible representations and let $T : V \to W$ be a linear map. Then,
			\begin{enumerate}[label=(\alph*)]
				\item if $\varphi \not\sim \rho$, $T^\sharp = 0$ and
				\item if $\varphi = \rho$, $T^\sharp = \frac{\Tr T}{\deg \varphi} I$.
			\end{enumerate}
		\end{fprop}
		\begin{proof}
			\phantom{pain}
			\begin{enumerate}[label = (\alph*)]
				\item This is straightforward on an application of \nameref{lem: schurs lemma} since $T^\sharp \in \Hom_G(\varphi,\rho)$.
				\item Again, by \nameref{lem: schurs lemma}, we have that $T^\sharp = \lambda I$ for some $\lambda I \in \C$. Now, note that
				\[ \Tr T^\sharp = \Tr(\lambda I) = \lambda \deg \varphi. \]
				We also have
				\begin{align*}
					\Tr T^\sharp &= \Tr \left( \frac{1}{|G|} \sum_{g \in G} \varphi_{g^{-1}} T \varphi_g \right) \\
						&= \frac{1}{|G|} \sum_{g \in G} \Tr \left( \varphi_{g^{-1}} T \varphi_g \right) \\
						&= \frac{1}{|G|} \sum_{g \in G} \Tr \left( \varphi_{g^{-1}} \varphi_g T \right) & (\Tr(ABC) = \Tr(ACB)) \\
						&= \Tr T,
				\end{align*}
				so the required follows.
			\end{enumerate}
		\end{proof}

		Suppose that $V = \C^n$ and $W = \C^m$. $P$ is then a linear form from $\GL(V,W) = M_{m \times n}(\C)$ to itself. A natural question to ask is: how do we represent $P$ as a matrix with respect to the standard basis vectors $E_{ij}$? (Recall that $E_{ij}$ is the $m \times n$ matrix with $1$ in the $(i,j)$th entry and $0$ elsewhere)\\
		It is a straightforward computational task to check that if $A = (a_{ij}) \in M_{r \times m}(\C)$, $E_{ki} \in M_{m \times n}(\C)$, and $B = (b_{ij}) \in M_{n \times s}(\C)$, then
		\begin{equation}
			(AE_{ki}B)_{lj} = a_{lk} b_{ij}.
		\end{equation}

		\begin{flem}
			\label{lem: schur orthogonality lemma}
			Let $\varphi : G \to U_n(\C)$ and $\rho : G \to U_m(\C)$ be unitary representations of $G$. Let $A = E_{ki} \in M_{m \times n}(\C)$. Then,
			\[ A_{lj}^\sharp = \langle \varphi_{ij} , \rho_{kl} \rangle. \]
		\end{flem}
		\begin{proof}
			Let $g \in G$. Because $\rho_g$ is unitary, $\rho_{g^{-1}} = \rho_g^*$. As a result, $(\rho_{g^{-1}})_{lk} = \overline{(\rho_g)_{kl}}$. Consequently,
			\begin{align*}
				(A^\sharp)_{lj} &= \frac{1}{|G|} \sum_{g \in G} (\rho_{g^{-1}} A \varphi_g)_{lj} \\
					&= \frac{1}{|G|} \sum_{g \in G} (\rho_{g^{-1}})_{lk} (\varphi_{g})_{ij} \\
					&= \frac{1}{|G|} \sum_{g \in G} \overline{(\rho_{g})_{kl}} (\varphi_{g})_{ij} \\
					&= \langle \varphi_{ij} , \rho_{kl} \rangle. \qedhere
			\end{align*}
		\end{proof}

		\begin{ftheo}[Schur's Orthogonality Relations]
			\label{theo: schurs orthogonality relations}
			Let $\varphi : G \to U_n(\C)$ and $\rho : G \to U_m(\C)$ be inequivalent irreducible unitary representations of a group $G$. Then,
			\begin{enumerate}[label=(\alph*)]
				\item $\langle \varphi_{ij} , \rho_{kl} \rangle = 0$.
				\item $\langle \varphi_{ij} , \varphi_{kl} \rangle = \begin{cases} 1/n , & (i,j) = (k,l), \\ 0, & \text{otherwise.} \end{cases}$
			\end{enumerate}
			In particular, the set $\{ \varphi_{ij} : 1 \le i,j \le n \} \cup \{ \rho_{kl} : 1 \le k,l \le m \}$ is a linearly independent set.
		\end{ftheo}
		\begin{proof}
			\phantom{pain}
			\begin{enumerate}[label = (\alph*)]
				\item 
				Let $A = E_{ki} \in M_{m \times n}(\C)$. By \Cref{prop: schur lemma but sharp}, $A^\sharp = 0$ because $\varphi \not\sim \rho$, so in particular, using \Cref{lem: schur orthogonality lemma}, $\langle \varphi_{ij} , \rho_{kl} \rangle = (A^\sharp)_{lj} = 0$.

				\item
				Let $A = E_{ki} \in M_{n \times n}(\C)$. By \Cref{prop: schur lemma but sharp}, $A^\sharp = \frac{\Tr A}{n} I$. $\Tr A$ is $1$ if $k = i$ and $0$ otherwise. We also have $\langle \varphi_{ij} , \rho_{kl} \rangle = \left(\frac{\Tr A}{n} I\right)_{lj}$, which is zero if $l \ne j$. That is, the quantity of interest is equal to $1/n$ if $k = i$ and $l = j$ and $0$ otherwise. \qedhere
			\end{enumerate}
		\end{proof}

		\begin{fcor}
			\label{cor: orthonormal set of fns for irred unitary rep}
			Let $\varphi$ be an irreducible unitary representation of $G$ of degree $n$. The set $\{ \sqrt{n} \varphi_{ij} : 1 \le i,j \le n \}$ of functions forms an orthonormal set.
		\end{fcor}

		\begin{fprop}
			\label{prop: finitely many irreducible representations}
			Let $G$ be a (finite) group. The following are true.
			\begin{enumerate}[label = (\alph*)]
				\item There are finitely many equivalence classes of irreducible representations of $G$.
				\item Let $\varphi^{(1)},\ldots,\varphi^{(s)}$ be a transversal of unitary representatives of irreducible representations of $G$. Set $d_i = \deg \varphi^{(i)}$. Then, the set of functions
				\[ \{ \sqrt{d_k} \varphi^{(k)}_{ij} : 1 \le k \le s, 1 \le i,j \le d_k \} \]
				is orthonormal.
			\end{enumerate}
		\end{fprop}
		\begin{proof}
			\phantom{pain}
			\begin{enumerate}[label = (\alph*)]
				\item By \Cref{lem: rep finite group equivalent to unitary}, any equivalence class of representations (any class of irreducible representations in particular) contains a unitary representation. As $\dim L(G) = |G|$, no linearly independent set of vectors in $L(G)$ can contain more than $|G|$ elements. Because orthonormal sets are linearly independent, \Cref{cor: orthonormal set of fns for irred unitary rep,theo: schurs orthogonality relations}(a) show that there can only be finitely many classes of irreducible representations.

				\item This again directly follows from \Cref{cor: orthonormal set of fns for irred unitary rep,theo: schurs orthogonality relations}(a).
			\end{enumerate}
		\end{proof}

		In particular, using the same notation as the above proposition, we have that
		\begin{equation}
			s \le d_1^2 + d_2^2 + \cdots + d_s^2 \le |G|. \label{eqn: sum of degree squared of irreds inequality}
		\end{equation}
		Indeed, the lower bound is obvious as each $d_i \ge 1$. For the upper bound, each representation of degree $d_k$ corresponds to $d_k^2$ many orthonormal functions, so the overall set of representations corresponds to $\sum d_i^2$ orthonormal functions, which can be at most $\dim L(G) = |G|$. This also says that the degree of any irreducible representation is at most $\sqrt{|G|}$.

		In fact, we shall see later that it is \emph{exactly} $|G|$.

	\subsection{Characters and Class Functions}

		Recall the remark after \Cref{cor: completely reducible unique decomp}, which said that the decomposition given by \nameref{theo: maschkes theorem} is unique. In this section, we shall prove a stronger version of the same (explicitly finding the number of irreducible representations), arriving at some interesting results along the way.\\
		Given an endomorphism of a finite dimensional vector space, we can talk about its trace, which is just the trace of any matrix representation after fixing an ordered basis. It is not too difficult to see that this trace is basis-invariant. We extensively use this fact in this section, namely that $\Tr(ABC) = \Tr(ACB)$ so if $C = A^{-1}$ then $\Tr(ABA^{-1}) = \Tr(B)$.

		\begin{fdef}[Character]
			Let $\varphi : G \to \GL(V)$ be a representation. The \emph{character} $\chi_\varphi : G \to \C$ of $\varphi$ is defined by $\chi_{\varphi}(g) = \Tr \varphi_g$. The character of an irreducible representation is called an \emph{irreducible character}.
		\end{fdef}

		As mentioned, the character does not depend on the basis we choose, so we may assume that we are talking about matrix representations. If $\varphi : G \to \GL_n(\C)$ is a representation given by $\varphi_g = ((\varphi_g)_{ij})$, $\chi_\varphi(g) = \sum_{i=1}^n (\varphi_g)_{ii}$.\\

		Occasionally, we cut out the explicit writing of the representation and directly refer to characters of a group. The degree of this character is just the degree of the corresponding representation.

		\begin{remark}
			If $z : G \to \C^* \incl \C$ is a degree $1$ representation, then $\chi_z = z$. Henceforth, we treat degree $1$ representations and their characters as the same.
		\end{remark}

		\begin{prop}
			\label{prop: character of 1}
			If $\varphi : G \to \GL(V)$ is a representation, $\chi_\varphi(1) = \deg \varphi = \dim V$.
		\end{prop}
		\begin{proof}
			Indeed, $\varphi_1 = \Id_V$ so $\chi_\varphi(1) = \Tr \varphi_1 = \Tr \Id_V = \dim V = \deg \varphi$.
		\end{proof}

		\begin{prop}
			\label{prop: character same under equiv}
			If $\varphi$ and $\rho$ are equivalent representations, $\chi_\varphi = \chi_\rho$.
		\end{prop}
		\begin{proof}
			We may assume that $\varphi,\rho : G \to \GL_n(\C)$. If $T \in \GL_n(\C)$ is an invertible matrix such that $\varphi_g = T \rho_g T^{-1}$ for all $g \in G$, then
			\[ \chi_\varphi(g) = \Tr \varphi_g = \Tr (T \rho_g T^{-1}) = \Tr \rho_g = \chi_\rho(g). \qedhere \]
		\end{proof}

		\begin{corollary}
			\label{cor: character sum of roots of unity}
			Let $G$ be a group of order $n$ and $\chi$ a character of degree $m$ of $G$. Then, $\chi(g)$ is a sum of $m$ $n$th roots of unity for each $g \in G$.
		\end{corollary}
		\begin{proof}
			Because characters are invariant under equivalence, let us assume that the representation is of the form $\varphi : G \to \GL_m(\C)$. Fix $g \in G$. Then, $\varphi_g^n = I$ so $\varphi_g$ is diagonalisable by \Cref{cor: finite order diagonalisable}. So, we may assume that $\varphi_g$ is diagonal. It has eigenvalues $(\lambda_i)_{i=1}^m$, where each $\lambda_i$ is an $n$th root of unity. The desideratum follows.
		\end{proof}

		A proof similar to \Cref{prop: character same under equiv} also shows the following.

		\begin{fprop}
			Let $\chi$ be a character of $G$. Then, $\chi$ is constant on conjugacy classes of $G$.
		\end{fprop}
		\begin{proof}
			Let $g,h \in G$ and $\varphi$ be a representation corresponding to $\chi$. Then,
			\begin{align*}
				\chi(g) &= \Tr \varphi_g \\
					&= \Tr (\varphi_{h^{-1}} \varphi_g \varphi_h) \\
					&= \Tr \varphi_{h^{-1}gh} = \chi(h^{-1}gh). \qedhere
			\end{align*}
		\end{proof}

		Functions such as these have a name of their own.

		\begin{fdef}[Class function]
			A function $f : G \to \C$ is said to be a \emph{class function} if $f(g) = f(h^{-1}gh)$ for all $g,h \in G$. The set of all class functions is denoted $Z(L(G))$.
		\end{fdef}

		Given a conjugacy class $C \subseteq G$ and a class function $f$, we denote by $f(C)$ the constant value taken by $f$ on $C$.

		\begin{prop}
			$Z(L(G))$ is a subspace of $L(G)$.
		\end{prop}
		We omit the proof of the above as it is very straightforward.

		\begin{fdef}
			Given a group $G$, the set of conjugacy classes of $G$ is denoted $\Cl(G)$. For $C \in \Cl(G)$, we define $\delta_C : G \to \C$ by
			\[ \delta_C(g) = \begin{cases} 1, & g \in C, \\ 0, & \text{otherwise.} \end{cases} \]
		\end{fdef}

		That is, $\delta_C$ is the indicator function of $C$.

		\begin{fprop}
			\label{prop: dim ZLG}
			The set $B = \{ \delta_C : C \in \Cl(G) \}$ is a basis of $Z(L(G))$. In particular, $\dim Z(L(G)) = |\Cl(G)|$.
		\end{fprop}
		\begin{proof}
			It is clear that $\delta_C \in \Cl(G)$ for each $C \in \Cl(G)$.

			To show that $B$ spans $Z(L(G))$, note that for any $f \in Z(L(G))$,
			\[ f = \sum_{C \in \Cl(G)} f(C) \delta_C. \]
			This is easily checked by computing both sides at an arbitrary $g \in G$.

			To show linear independence on the other hand, we have for $C,C' \in \Cl(G)$
			\[ \langle \delta_C , \delta_{C'} \rangle = \sum_{g \in G} \delta_C(g) \overline{\delta_{C'}(g)} = \begin{cases} 0 , & C \ne C', \\ \frac{|C|}{|G|}, & C = C' \end{cases} \]
			and any set of orthogonal nonzero vectors is linearly independent.

			The desideratum follows.
		\end{proof}

		\begin{ftheo}
			\label{theo: inner product of irreducible characters}
			Let $\varphi,\rho$ be irreducible representations of $G$. Then
			\[ \langle \chi_\varphi , \chi_\rho \rangle = \begin{cases} 1 , & \varphi \sim \rho, \\ 0 , & \varphi \not\sim \rho. \end{cases} \]
			Thus, the set of irreducible characters of $G$ forms an orthonormal set of class functions.
		\end{ftheo}
		\begin{proof}
			By \Cref{lem: rep finite group equivalent to unitary,prop: character same under equiv}, we may assume that $\varphi : G \to U_n(\C)$ and $\rho : G \to U_m(\C)$. We have
			\begin{align*}
				\langle \chi_\varphi , \chi_\rho \rangle &= \frac{1}{|G|} \sum_{g \in G} \chi_\varphi(g) \overline{\chi_\rho(g)} \\
					&= \frac{1}{|G|} \sum_{g \in G} \left(\sum_{1 \le i \le n} \varphi_{ii}(g)\right) \overline{\left( \sum_{1 \le j \le m} \rho_{jj}(g) \right)} \\
					&= \sum_{\substack{1 \le i \le n \\ 1 \le j \le m}} \frac{1}{|G|} \sum_{g \in G} \varphi_{ii}(g) \overline{\rho_{jj}(g)} \\
					&= \sum_{\substack{1 \le i \le n \\ 1 \le j \le m}} \langle \varphi_{ii} , \rho_{jj} \rangle.
			\end{align*}
			Recall \nameref{theo: schurs orthogonality relations}. If $\varphi \not\sim \rho$, then it immediately follows that the above quantity of interest is $0$. If $\varphi \sim \rho$ on the other hand, we may assume that $\varphi = \rho$ by \Cref{prop: character same under equiv}. We then again have by Schur's orthogonality relations that the summand is nonzero only when $i = j$, and in this case it is just equal to $1/n$. The overall sum is then $n \cdot 1/n = 1$, completing the proof.
		\end{proof}

		\begin{fcor}
			Given two inequivalent irreducible representations $\varphi , \rho$ of $G$, $\chi_{\varphi} \ne \chi_{\rho}$.
		\end{fcor}
		\begin{proof}
			We have $\langle \chi_\varphi , \chi_\rho \rangle = 0$, but if $\chi_\varphi = \chi_\rho$ we also have $\langle \chi_\varphi , \chi_\varphi \rangle = 1$.
		\end{proof}

		\begin{fcor}
			\label{cor: irred equivalent iff same char}
			Two irreducible representations are equivalent if and only if they have the same character.
		\end{fcor}

		In \Cref{cor: equivalent iff same char}, we shall see that the above holds in more generality.\\

		Consequently, there are at most $|\Cl(G)|$ equivalence classes of irreducible representations.

		\begin{definition}
			If $V$ is a vector, $\varphi$ is a representation, and $m \in \N$, then
			\[ mV = \underbrace{V \oplus \cdots \oplus V}_{m} \text{ and } m\varphi = \underbrace{\varphi \oplus \cdots \oplus \varphi}_{m}. \]
			$0V$ is the zero vector space and $0\varphi$ is the degree zero representation.
		\end{definition}

		Now, we would like to show the uniqueness of decomposition, just as we did in \Cref{cor: completely reducible unique decomp}. Indeed, this is easier now since we have a finite number of irreducible representations. Suppose we are given a transerversal $\varphi^{(1)},\ldots,\varphi^{(s)}$ of irreducible representations, and let
		\[ \varphi \sim m_1 \varphi^{(1)} \oplus m_2 \cdots \oplus m_s \varphi^{(s)}. \]

		\begin{lemma}
			Let $\varphi = \rho \oplus \psi$. Then $\chi_{\varphi} = \chi_\rho + \chi_\psi$.
		\end{lemma}
		\begin{proof}
			We may suppose that $\rho : G \to \GL_n(\C)$ and $\psi : G \to \GL_m(\C)$. The block matrix form of $\varphi : G \to \GL_{n+m}(\C)$ can then be written as
			\[ \varphi_g = \begin{bmatrix} \rho_g & \\ & \psi_g \end{bmatrix}, \]
			and the required immediately follows.
		\end{proof}

		As an immediate consequence of the above lemma and \Cref{theo: inner product of irreducible characters}, we get the following.

		\begin{ftheo}
			 Suppose we are given a transerversal $\varphi^{(1)},\ldots,\varphi^{(s)}$ of irreducible representations, and let $\varphi$ be a representation such that
			\[ \varphi \sim m_1 \varphi^{(1)} \oplus m_2 \cdots \oplus m_s \varphi^{(s)}. \]
			Then, $m_i = \langle \chi_\varphi , \chi_{\varphi^{(i)}}$.
		\end{ftheo}

		That is, as we saw earlier, the decomposition of $\varphi$ into irreducible representations is ``unique''.
		
		\begin{fcor}
			\label{cor: equivalent iff same char}
			$\varphi$ is determined up to equivalence by its character.
		\end{fcor}

		The above follows quite directly from the fact that the decomposition is unique.

		\begin{fcor}
			\label{cor: character inner product natural}
			\phantom{pain}
			\begin{enumerate}
				\item $\norm{\chi}^2 \in \N$, and $\norm{\chi} = 1$ iff $\chi$ is irreducible.
				\item $\langle \chi_1 , \chi_2 \rangle \in \N_0$. Note that the characters themslves need not necessarily be real-valued.
			\end{enumerate}
		\end{fcor}
		
		To see this, note that if
		\[ \rho_1 \sim m_1 \varphi^{(1)} \oplus \cdots \oplus m_s \varphi^{(s)} \]
		and
		\[ \rho_2 \sim n_1 \varphi^{(1)} \oplus \cdots \oplus n_s \varphi^{(s)}, \]
		then $\langle \chi_{\rho_1} , \chi_{\rho_2} \rangle = \sum_i m_i n_i$.

		\begin{fcor}
			Let $z : G \to \C^*$ be a degree one representation and $\rho : G \to \GL_n(\C)$ be a representation. Consider $\varphi : G \to \GL_n(\C)$ defined by $\varphi_g = z_g \rho_g$. Then,
			\begin{enumerate}[label=(\alph*)]
				\item $\varphi$ is a representation,
				\item $\chi_\varphi = z \chi_\rho$ and $\norm{\chi_\varphi} = \norm{\chi_\rho}$,
				\item $\varphi$ is irreducible iff $\rho$ is, and
				\item if there exists $g_0 \in G$ such that $z_{g_0} \ne 1$ and $\chi_\varphi(g_0) \ne 0$, then $\rho \not\sim \varphi$.
			\end{enumerate}
		\end{fcor}
		\begin{proof}
			\phantom{pain}
			\begin{enumerate}[label=(\alph*)]
				\item This is direct as $z_{g_1} \rho_{g_2} = \rho_{g_2} z_{g_1}$.

				\item We have $\chi_\varphi(g) = \Tr(\varphi_g) = \Tr(z_g \rho_g) = z_g \Tr(\rho_g)$, so $\chi_\varphi = (z \chi_\rho)(g)$.\\
				Because $G$ is finite, we have $z_g^{|G|} = 1$, and so $|z_g| = 1$. Consequently, $\norm{\chi_\varphi(g)} = \norm{z_g \chi_\rho(g)} = \norm{\chi_\rho(g)}$. As a result, $\norm{\chi_\varphi} = \norm{\chi_\rho}$ as well.

				\item Since $\varphi$ (resp. $\rho$) is irreducible iff $\norm{\chi_\varphi}$ (resp. $\norm{\chi_\rho}$) is $1$, we are done.

				\item This follows from \Cref{cor: equivalent iff same char} since in this case, $\chi_\varphi(g_0) \ne \chi_\rho(g_0)$.
			\end{enumerate}
		\end{proof}


		\begin{fdef}
			Let $G$ be a finite group and $\varphi^{(1)},\ldots,\varphi^{(s)}$ be a transversal of irreducible unitary representations of $G$. If $\rho \sim m_1 \varphi^{(1)} \oplus \cdots \oplus m_s \varphi^{(s)}$, then $m_i$ is said to be the \emph{multiplicity} of $\varphi^{(i)}$ in $\rho$. If $m_i > 0$, $\varphi^{(i)}$ is said to be an \emph{irreducible constituent} of $\rho$.
		\end{fdef}

		Using the notation of the above definition, $\deg \rho = \sum m_i \deg \varphi^{(i)}$.\\
		Letting $m_i = \langle \chi_\rho , \chi_{\varphi^{(i)}} \rangle$, we have
		\[ \rho \sim m_1 \varphi^{(1)} \oplus \cdots \oplus m_s \varphi^{(s)}. \]


	\subsection{The Regular Representation}

		Recall linearisation.

		\begin{fdef}[Regular representation]
			Let $G$ be a finite group. The \emph{regular representation} of $G$ is the homomorphism $L : G \to \GL(\C G)$ defined by
			\[ L_g \left( \sum_{h \in G} c_h h \right) = \sum_{h \in G} c_h gh = \sum_{x \in G} c_{g^{-1}x} x \]
			for $g \in G$.
		\end{fdef}

		The above representation can be understood very simply on noting that given a standard basis vector $h \in G$ of $\C G$, $L_g h$ is just $gh$, that is, it permutes the basis vectors. It acts on arbitrary elements by extending this map linearly to $\C G$. This may be used to check that the regular representation is indeed a representation. \\
		Clearly, $\deg L = |G|$. As a result, by \Cref{eqn: sum of degree squared of irreds inequality}, $L$ is \emph{not} irreducible.

		\begin{fprop}
			The regular representation is a unitary representation of $G$.
		\end{fprop}
		\begin{proof}
			Fix $g \in G$. We have
			\begin{align*}
				\left\langle L_g \sum_{h \in G} c_h h , L_g \sum_{h \in G} d_h h \right\rangle &= \left\langle \sum_{h \in G} c_{g^{-1}h} h , \sum_{h \in G} d_{g^{-1}h} h \right\rangle \\
					&= \sum_{h \in G} c_{g^{-1} h} d_{g^{-1} h} \\
					&= \sum_{h \in G} c_h d_h = \left\langle \sum_{h \in G} c_h h , \sum_{h \in G} d_h h \right\rangle.
			\end{align*}
		\end{proof}

		\begin{fprop}
			The character of the regular representation $L$ is given by
			\[ \chi_L(g) = \begin{cases} |G| , & g = 1, \\ 0 , & \text{otherwise.} \end{cases} \]
		\end{fprop}
		\begin{proof}
			By \Cref{prop: character of 1}, $\chi_L(1) = |G|$. Let $g \ne 1$ and fix an ordering $(g_1,\ldots,g_n)$ of $|G|$. We claim that all the diagonal entries of the matrix representation $[L_g]$ of $L_g$ with respect to this basis are $0$. Indeed, for any $g_i$, $gg_i \ne g_i$, so the $i$th entry of the $i$th column is $0$. It follows that $\chi_L(g) = \Tr [L_G] = 0$.
		\end{proof}

		The above can be used in conjunction with \Cref{cor: character inner product natural} to give an alternate proof that $L$ is not irreducible.

		For the remainder of this subsection, fix a finite group $G$, $\varphi^{(1)},\ldots,\varphi^{(s)}$ as a transversal of irreducible unitary representations of $G$, $d_i = \deg \varphi^{(i)}$, and let $\chi_i = \chi_{\varphi^{(i)}}$.\\

		We shall now show that the second inequality is in fact an equality in \Cref{eqn: sum of degree squared of irreds inequality}.

		\begin{ftheo}
			\label{theo: sum of degree squared of irreds}
			Let $L$ denote the regular representation of $G$. Then,
			\[ L \sim d_1 \varphi^{(1)} \oplus \cdots \oplus d_s \varphi^{(s)}. \]
			In particular,
			\[ |G| = \sum d_i^2. \]
		\end{ftheo}
		\begin{proof}
			It suffices to show that $\langle \chi_L , \chi_i \rangle = d_i$. Indeed, this is immediate as
			\[ \langle \chi_L , \chi_i \rangle = \frac{1}{|G|} \sum_{g \in G} \chi_L(g) \overline{\chi_i(g)} = \frac{1}{|G|} \chi_L(1) \chi_i(1) = d_i. \]
		\end{proof}

		Using \Cref{prop: finitely many irreducible representations}, we get the following.

		\begin{fcor}
			\label{cor: orthonormal basis of LG}
			The set $B = \{ \sqrt{d_k} \varphi_{ij}^{(k)} : 1 \le k \le s, 1 \le i,j \le d_k \}$ is an orthonormal basis of $L(G)$.
		\end{fcor}

		\begin{fprop}
			\label{cor: orthonormal basis of ZLG}
			The set $B = \{\chi_1 , \ldots, \chi_s\}$ is an orthonormal basis of $Z(L(G))$.
		\end{fprop}
		\begin{proof}
			Assume that $\varphi^{(i)} : G \to U_{d_i}(\C)$.\\
			Recall by \Cref{theo: inner product of irreducible characters} that $B$ is an orthonormal set. We are done if we show that it is a basis. Let $f \in Z(L(G)) \le L(G)$. By \Cref{cor: orthonormal basis of LG}, we have constants $c_{ij}^{(k)} \in \C$ such that
			\[ f = \sum c_{ij}^{(k)} \varphi_{ij}^{(k)}. \]
			Let $x \in G$. We have
			\begin{align*}
				f(x) &= \frac{1}{|G|} \sum_{g \in G} f(g^{-1}xg) & (f \in Z(L(G))) \\
					&= \frac{1}{|G|} \sum_{g \in G} \sum_{i,j,k} c_{ij}^{(k)} \varphi_{ij}^{(k)}(g^{-1}xg) \\
					&= \sum_{i,j,k} c_{ij}^{(k)} \frac{1}{|G|} \sum_{g \in G} \varphi_{ij}^{(k)}(g^{-1}xg) \\
					&= \sum_{i,j,k} c_{ij}^{(k)} \left[\frac{1}{|G|} \sum_{g \in G} \varphi^{(k)}(g^{-1})\varphi^{(k)}(x)\varphi^{(k)}(g)\right]_{ij} \\
					&= \sum_{i,j,k} c_{ij}^{(k)} \left[\left(\varphi^{(k)}(x)\right)^\sharp\right]_{ij} & (\text{recall \Cref{prop def: sharp}}) \\
					&= \sum_{i,j,k} c_{ij}^{(k)} \frac{\Tr \varphi^{(k)}(x)}{d_k} I_{ij} & (\text{recall \Cref{prop: schur lemma but sharp}}) \\
					&= \sum_{i,k} \frac{c_{ii}^{(k)}}{d_k} \chi_k(x) \\
					&= \sum_k \left( \sum_i \frac{c_{ii}^{(k)}}{d_k} \right) \chi_k(x),
			\end{align*}
			so $B$ is a basis and we are done.
		\end{proof}

		Now use \Cref{prop: dim ZLG} to get the following.

		\begin{fcor}
			\label{cor: number of irred reps is conjug}
			There are precisely $|\Cl(G)|$ equivalence classes of irreducible representations of a group $G$.
		\end{fcor}

		Recall that $|\Cl(G)| = |G|$ iff $G$ is abelian.

		\begin{fcor}
			\label{cor: equivalence classes of irreds for abelian}
			$G$ has $|G|$ equivalence classes of irreducible representations iff $G$ is abelian.
		\end{fcor}

		In the above scenario, we have $|G| = d_1^2 + \cdots + d_{|G|}^2$, so we get the following.

		\begin{fcor}
			\label{cor: abelian iff irred degree one}
			$G$ is abelian iff all its irreducible representations have degree one.
		\end{fcor}


		\begin{fdef}
			Let $G$ be a finite group with irreducible characters $\chi_1,\ldots,\chi_s$ and conjugacy classes $C_1,\ldots,C_s$. The \emph{character table} of $G$ is the $s \times s$ matrix $\mathsf{X}$ with $\mathsf{X}_{ij} = \chi_i(C_j)$.
		\end{fdef}

		The above table is square because of \Cref{cor: number of irred reps is conjug}.

		\begin{fprop}
			Let $C,C'$ be conjugacy classes of $G$ and $g \in C, h \in C'$. Then,
			\[ \sum_{i=1}^s \chi_i(g) \overline{\chi_i(h)} = \begin{cases} |G|/|C| , & C = C', \\ 0, & \text{otherwise.} \end{cases} \]
			Consequently, the columns of the character table are orthogonal, and it is invertible.
		\end{fprop}
		\begin{proof}
			Recall \Cref{cor: orthonormal basis of ZLG} and also that $\delta_{C'} \in Z(L(G))$. So,
			\[ \delta_{C'} = \sum_{i=1}^s \langle \delta_{C'} , \chi_i \rangle \chi_i. \]
			So,
			\begin{align*}
				\delta_{C'}(g) &= \sum_{i=1}^s \frac{1}{|G|} \sum_{x \in G} \delta_{C'}(x) \overline{\chi_i(x)} \chi_i(g) \\
					&= \frac{1}{|G|} \sum_{i=1}^s \sum_{x \in C'} \overline{\chi_i(x)} \chi_i(g) \\
					&= \frac{|C'|}{|G|} \sum_{i=1}^s \overline{\chi_i(h)} \chi_i(g) & (\chi_i \text{ is a class function}) \\
				\sum_{i=1}^s \overline{\chi_i(h)} \chi_i(g) &= \frac{|G|}{|C'|} \delta_{C'}(g).
			\end{align*}
			The desideratum follows.
		\end{proof}

	\subsection{Representations of abelian groups}

		We conclude this section by completing our discussion of representations of abelian groups. For this subsection, let $G$ be an abelian group.\\
		By \Cref{prop: cyclic group degree one,cor: equivalence classes of irreds for abelian}, we know that the $|G|$ degree one representations of $G$ are precisely the irreducible representations of $G$.\\
		Recall from \Cref{prop: deg-one reps of ZnZ} that we know all of these representations when $G = \Z/n\Z$. By the structure theorem of finite abelian groups, we get a complete description of the irreducible representations of $G$ for any abelian group in general using the following lemma.

		\begin{flem}
			Let $G_1,G_2$ be finite abelian groups. with $m = |G_1|$ and $n = |G_2|$. Suppose that $\rho_1,\ldots,\rho_m$ and $\varphi_1,\ldots,\varphi_n$ are all the irreducible representations of $G_1$ and $G_2$ respectively. The functions $\alpha_{ij} : G_1 \times G_2 \to \C$ for $1 \le i \le m$ and $1 \le j \le n$ defined by
			\[ \alpha_{ij}(g_1,g_2) = \rho_i(g_1) \varphi_j(g_2) \]
			form a complete set of irreducible representations of $G$.
		\end{flem}
		\begin{proof}
			Note that $\alpha_{ij}(g,1) = \rho_i(g)$ and $\alpha_{ij}(1,g) = \varphi_j(g)$. This gives that all the $\alpha_{ij}$ are distinct as $\alpha_{ij}$ and $\alpha_{kl}$ are identical iff $\rho_i$ and $\rho_k$ are identical and $\varphi_j$ and $\varphi_l$ are identical.\\
			Further, because each $\alpha_{ij}$ is degree-one, it suffices to show that each $\alpha_{ij}$ is a homomorphism. This is immediate as by commutativity,
			\begin{align*}
			 	\alpha_{ij}((g_1,g_2)(g_1',g_2')) &= \alpha_{ij}(g_1g_1',g_2g_2') \\
			 		&= \varphi_i(g_1g_1') \rho_j(g_2g_2') \\
			 		&= \varphi_i(g_1) \varphi_i(g_1') \rho_j(g_2) \rho_j(g_2') \\
			 		&= \varphi_i(g_1) \rho_j(g_2) \varphi_i(g_1') \rho_j(g_2') = \alpha_{ij}(g_1,g_2) \alpha_{ij}(g_1',g_2'). \qedhere
			 \end{align*} 
		\end{proof}

		Further observe that because degree-one representations are the same as their characters, the above can be used quite easily to get the character table of the product.

	\subsection{The Dimension Theorem}

		Recall the previous section where we showed that a given group $G$ has only finitely many irreducible representations. In this section, we shall show that the degree of any irreducible representation divides the order of the group.\\
		Also recall algebraic integers (\Cref{def: algebraic integer}).

		\begin{fprop}
			\label{prop: character alg int}
			Let $\chi$ be a character of $G$. Then, $\chi(g)$ is an algebraic integer for all $g \in G$.
		\end{fprop}
		\begin{proof}
			Recall that any root of unity is an algebraic integer. The required then follows on using \Cref{cor: character sum of roots of unity,prop: alg int subring}.
		\end{proof}

		For the remainder of this section, let $G$ be a finite group with conjugacy classes $\{C_i\}_{i=1}^s$ with $C_1 = \{1\}$. For $i \in [s]$, let $h_i = |C_i|$. Let $\varphi : G \to \GL(V)$ denote a degree $d$ representation and $\chi_i = \chi_\varphi(C_i)$. Finally, let $T_i = \sum_{x \in C_i} \varphi_x$.

		\begin{flem}
			If $\varphi$ is irreducible, $T_i = \frac{h_i}{d} \chi_i \cdot I$.
		\end{flem}
		\begin{proof}
			First, for any $g \in G$,
			\[ \varphi_g T_i \varphi_{g^{-1}} = \varphi_g \left( \sum_{x \in C_i} \varphi_x \right) \varphi_{g^{-1}} = \sum_{x \in C_i} \varphi_{gxg^{-1}} = \sum_{y \in C_i} \varphi_y = T_i, \]
			so $T_i \in \Hom(\varphi,\varphi)$. By \Cref{cor: schurs corollary}, $T_i = \lambda I$ for some $\lambda \in \C$. Now,
			\[ \lambda = \frac{1}{d} \Tr T_i = \frac{1}{d} \sum_{x \in C_i} \Tr \varphi_x = \frac{1}{d} \sum_{x \in C_i} \chi_i = \frac{h_i}{d} \chi_i, \]
			completing the proof.
		\end{proof}

		\begin{flem}
			$T_i \circ T_j = \sum_{k=1}^s a_{ijk} T_k$ for some $\{a_{ijk}\}_{1\le i,j,k\le s} \subseteq \Z$.
		\end{flem}
		Note that $\varphi$ is not assumed to be irreducible in this lemma.
		\begin{proof}
			First,
			\[ T_i \circ T_j = \left(\sum_{x \in C_i} \varphi_x\right) \circ \left(\sum_{y \in C_j} \varphi_y\right) = \sum_{\substack{x \in C_i \\ y \in C_j}} \varphi_{xy} = \sum_{g \in G} a_{ijg} \varphi_g, \]
			where $a_{ijg} = |\{(x,y) \in C_i \times C_j : xy = g\}|$. Let $X_{ijg}$ be this set. Suppose that $g_1,g_2 \in C_k$, and let $g_2 = kg_1k^{-1}$. Observe then that the function $X_{ijg_1} \to X_{ijg_2}$ defined by $(x,y) \mapsto (kxk^{-1},kyk^{-1})$ is a bijection. Indeed, it has inverse $(x,y) \mapsto (k^{-1}xk,k^{-1}yk)$. So, $a_{ijg_1} = a_{ijg_2}$. Letting the value of $a_{ijg}$ for $g \in C_k$ be $a_{ijk}$, we get
			\[ T_i \circ T_j = \sum_{g \in G} a_{ijg} \varphi_g = \sum_{k=1}^s \sum_{g \in C_k} a_{ijk} \varphi_g = \sum_{k=1}^s a_{ijk} T_k. \]
		\end{proof}

		Combining the two lemmas, we get the following.

		\begin{fcor}
			For some $\{a_{ijk}\}_{1\le i,j,k\le s} \subseteq \Z$,
			\[ \left( \frac{h_i}{d} \chi_i \right) \left( \frac{h_j}{d} \chi_j \right) = \sum_{k=1}^s a_{ijk} \frac{h_k}{d} \chi_k. \]
		\end{fcor}

		\begin{flem}
			\label{lemma: hi chii di alg int}
			If $\varphi$ is irreducible, $h_i \chi_i/d_i$ is an algebraic integer for every $i$.
		\end{flem}
		\begin{proof}
			Using the previous corollary, it is not too difficult to come up with an appropriate integer matrix in the context of \Cref{prop: alg int iff integer eigenvalue}.
		\end{proof}

		\begin{ftheo}[Dimension Theorem]
			\label{theo: dimension th}
			Let $\varphi$ be an irreducible degree $d$ representation of $G$. Then, $d$ divides $|G|$.
		\end{ftheo}
		\begin{proof}
			By \Cref{theo: inner product of irreducible characters}, $\langle \chi_\varphi , \chi_\varphi \rangle = 1$. So,
			\[ \frac{|G|}{d} = \frac{|G|}{d} \cdot \frac{1}{|G|} \sum_{g \in G} \chi_\varphi(g) \overline{\chi_\varphi(g)} = \sum_{g \in G} \frac{\chi_\varphi(g)}{d} \overline{\chi_\varphi(g)} = \sum_{i=1}^s \sum_{g \in C_i} \frac{\chi_\varphi(g)}{d} \overline{\chi_\varphi(g)} = \sum_{i=1}^s \left( \frac{h_i\chi_i}{d} \right) \overline{\chi_i}. \]
			Note that $h_i \chi_i / d$ is an algebraic integer by \Cref{lemma: hi chii di alg int}, and $\overline{\chi_i}$ is an algebraic integer by \Cref{prop: character alg int} (recall that $\A$ is closed under conjugation). By \Cref{prop: alg int subring}, $|G|/d$ is an algebraic integer too. However, this is rational, so the desideratum follows from \Cref{prop: rational alg ints are ints}.
		\end{proof}

		\begin{fcor}
			Let $p,q$ be primes with $p \le q$ and $q \not\equiv 1 \pmod p$. Then, any group $G$ of order $pq$ is abelian. In particular, so are groups of order $p^2$.
		\end{fcor}
		\begin{proof}
			Let $d_1,\ldots,d_s$ be the degrees of the irreducible representations of $G$. We shall show that $d_i = 1$ for all $i$, then use \Cref{cor: abelian iff irred degree one}. Let us assume without loss of generality that $d_1 = 1$. We have
			\[ pq = 1 + d_2^2 + \cdots + d_s^2. \]
			By the \nameref{theo: dimension th}, $d_i \in \{1,p,q,pq\}$ for all $i$. In fact, because $p \le q$, $d_i \in \{1,p\}$. Let $m$ be the number of representations of degree $1$ and $n$ that of degree $p$. We have $pq = m + np^2$. So, $p \mid m$. Let $m = pm'$. We have $q = m' + np$. By \Cref{cor: number of degree one reps}, $m \mid pq$, so $m' \mid q$. As a result, $m'\in\{1,q\}$. However, $m'$ cannot be $1$ because that with the previous equation would contradict the fact that $q \not\equiv 1 \pmod p$. Therefore, $m' = q$ and $m = pq$, completing the proof.
		\end{proof}

		Recall that the above is a basic fact from group theory which is typically proved using the Sylow theorems.

		\begin{porism}
			Let $G$ be a group of order $pq$ for primes $p<q$. Then, all irreducible representations of $G$ have degree either $1$ or $p$. Moreover, $G$ has an irreducible representation of degree $p$ iff it is non-abelian. 
		\end{porism}

		Noting that $m = |G/[G,G]|$ and $p \mid m$ in the proof of the previous corollary, we get the following.

		\begin{fpor}
			Let $G$ be a group of order $pq$ for primes $p < q$. Then, $|[G,G]| \in \{1,q\}$. Moreover, $|[G,G]| = q$ iff $G$ is non-abelian.
		\end{fpor}

		As before, this can be proved using elementary group theory as well. We leave the details of this as an exercise to the reader.

\clearpage