\section{Introduction}

\begin{exercise}
	Recall that the number of $k$-subsets of $[n]$ is $\binom{n}{k}$. Given a $k$-subset $S = \{x_1,\ldots,x_k\}$ of $[n]$, we write $S_< = \{x_1,\ldots,x_k\}_<$ to denote that $x_1<x_2<\cdots<x_k$. Determine the number of $k$-subsets $\{x_1,\ldots,x_k\}_<$ of $[n]$ such that $x_i \cong i \mod 2$.
\end{exercise}
For example, for $n=6$ and $k=3$, we have the subsets $\{1,4,5\},\{1,2,3\},\{1,2,5\},\{3,4,5\}$.


Broadly, there are three types of ``answers'': a formula, a recurrence, and a generating function. A great example of the second and third is the following.\\
% Euler's Theorem
$p(n)$, the number of number partitions of $n$, is given by the generating function
\[ \sum_{n \ge 0} p(n) x^n = \prod_{i \ge 1} \frac{1}{1-x^i}. \]
Using this, a recursion may be obtained as well.
We do \emph{not} plug in values for $x$ in the above. We merely look at the coefficient of $x^n$ in it. We want the coefficient to be a finite sum for all $n$. If it is an infinite sum, convergence issues may arise.

% www.math.iitb.ac.in/~krishnan/phd-2022/

% The number of derangements of $[n]$ is the integer closest to $n!/e$.
% Bell number B_n (number of set partitions of $[n]$) : \sum_{n \ge 0} B_n x^n / n! = e^{e^x - 1}.

\subsection{Counting in $S_n$}

	Recall that $S_n$ is generated by transpositions. A transposition $(i,j)$ is a permutation $\sigma$ defined by
	\[ \sigma(k) = \begin{cases} j, & k=i, \\ i, & k=j, \\ k, & \text{otherwise.} \end{cases} \]
	In fact, $S_n$ is generated by just ``adjacent transpositions'' $S_i = (i,i+1)$ for $1 \le i < n$.
	We have
	\begin{align*}
		S_i^2 &= \Id \\
		S_i S_{i+1} S_i &= S_{i+1} S_i S_{i+1} \\
		S_i S_j &= S_j S_i \text{ if $|i-j| > 2$.}
	\end{align*}
	What is the minimum number of adjacent transpositions required to generate a given transposition? This number is referred to as the \emph{length} $\ell(\pi)$ of the transposition $\pi$. What is $\sum_{\pi \in S(n)} q^{\ell(\pi)}$?


	% PRESENT PROOF OF THE FIRST ON TUESDAY -- 02/08/2022