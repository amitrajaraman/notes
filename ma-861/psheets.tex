\section{Problem Sheets}

	\subsection{Problem Sheet 1}

		\begin{problem}
			Let $S(n,k)$ and $s(n,k)$ be Stirling numbers of the second and first kind respectively. Show that for all $n,k$, we have $s(n,k) \ge S(n,k)$.
		\end{problem}
	
		\begin{solution*}
			Let $X_{S(n,k)}$ be the set of partitions of $[n]$ into exactly $k$ parts and $X_{s(n,k)}$ the number of permutations of $[n]$ with exactly $k$ cycles. Recall that by definition, $|X_{S(n,k)}| = S(n,k)$ and $|X_{s(n,k)}| = s(n,k)$. It suffices to demonstrate an injection $f$ from $X_{S(n,k)}$ to $X_{s(n,k)}$. We do so as follows. Let $\{\{x_{1,1},\ldots,x_{1,n_1}\},\ldots,\{x_{k,1},\ldots,x_{1,n_k}\}\}$ be a partition of $[n]$ into exactly $k$ parts, where $x_{i,j_1} < x_{i,j_2}$ for $j_1 < j_2$. Then, we have a corresponding permutation of $[n]$ with exactly $k$ cycles given by $(x_{1,1},\ldots,x_{1,n_1})\cdots(x_{k,1},\ldots,x_{k,n_k})$. This map is clearly an injection, so we are done.
		\end{solution*}

		\begin{problem}
			Show that
			\[ S(n,k) = \sum_{r=1}^k (-1)^{k-r} \frac{r^n}{r!(k-r)!}. \]
		\end{problem}
		\begin{solution*}
			$S(n,k)$ is merely $1/k!$ times the number of surjective functions from $[n]$ to $[k]$ (because the ordering of the partitions does not matter). The set of functions that are \emph{not} surjective is
			\[ \bigcup_{i \in [n]} \{ f \in [k]^{[n]} : i \not\in \operatorname{Im}(f) \}. \]
			The size of the above is quite easily determined by the inclusion-exclusion principle to get
			\[ k^n - k!S(n,k) = \sum_{r=1}^{k-1} (-1)^{r+1} \cdot \underbrace{\binom{k}{r}}_\text{choosing $r$ elements in $[k]$ to ``avoid''} \cdot \underbrace{(k-r)^n}_\text{counting functions that avoid the chosen}, \]
			and the desideratum immediately follows.
		\end{solution*}

		\begin{problem}
			Let $A_n(y) = \sum_{k} S(n,k) y^k$. Show that $A_n(y) = (y + yD)^n1$ where $D = \od{}{y}$ is the derivative operator.
		\end{problem}
		\begin{solution*}
			First, recall that $S(n+1,k+1) = S(n,k) + (k+1) \cdot S(n,k+1)$ -- the $S(n,k)$ corresponds of those partitions where $n+1$ is in a part of its own, and the $(k+1)S(n,k+1)$ corresponds to those partitions where this is not the case, so we can consider any partition of $[n]$ into $k+1$ parts, then decide which of the $k+1$ parts to place $n+1$ in.\\
			We have
			\begin{align*}
				(y+yD)A_n(y) &= (y+yD) \sum_{k=1}^n S(n,k) y^k \\
					&= \sum_{k=1}^n S(n,k) (y^{k+1} + ky^k) \\
					&= S(n,1) y + \sum_{k=2}^{n} y^k (S(n,k-1) + (k-1) S(n,k)) + S(n,n) y^{n+1} \\
					&= S(n+1,1) y + \sum_{k=2}^n S(n+1,k) y^k + S(n+1,n+1) y^{n+1} = A_{n+1}(y).
			\end{align*}
			The required follows inductively.
		\end{solution*}

		\begin{problem}
			Let $D_n$ be the number of derangements in $\mathfrak{S}_n$ and let $D(x) = \sum_{n \ge 0} D_n x^n/n!$ be its egf. Determine $D(x)$.
		\end{problem}
		\begin{solution*}
			A permutation $\pi \in \mathfrak{S}_n$ is a derangement iff it has no cycles of length $1$. Define $f : \N \to \N_0$ by
			\[ f(k) = \begin{cases} 0 , & k = 1, \\ 1 , & \text{otherwise.} \end{cases} \]
			By the earlier observation, $\pi \in \mathfrak{S}_n$ is a derangement iff $f(|C_1|)\cdots f(|C_k|) = 1$ where $C_1,\ldots,C_k$ are the cycles of $\pi$.
			Using Corollary 5.1.9 in \cite{ec2}, we get that
			\[ D(x) = \exp \left(\sum_{n \ge 2} \frac{x^n}{n}\right) = \exp \left( -x - \log(1-x) \right) = \frac{e^{-x}}{1-x} . \]
		\end{solution*}

		\begin{problem}
			Let $s(n,2)$ denote the number of $\pi \in \mathfrak{S}_n$ with $2$ cycles in its cyclic decomposition and let $H_n$ denote the $n$th harmonic number. Show that $s(n+1,2) = H_n \times n!$.
		\end{problem}
		\begin{solution*}
			It is easily checkable that the number of cyclic permutations of $[k]$ is $(k-1)!$. We have
			\begin{align*}
				s(n+1,2) &= \sum_{k=1}^{\lfloor (n+1)/2 \rfloor} \binom{n+1}{k} (k-1)! (n+1-k-1)! \\
					&= \sum_{k=1}^{\lfloor (n+1)/2 \rfloor} \frac{(n+1)!}{k(n+1-k)} \\
					&= \sum_{k=1}^{\lfloor (n+1)/2 \rfloor} n! \left( \frac{1}{k} + \frac{1}{n+1-k} \right) \\
					&= n! \left(\sum_{k=1}^{\lfloor (n+1)/2 \rfloor} \frac{1}{k} + \sum_{k=\lceil (n+1)/2 \rceil}^{n} \frac{1}{k} \right) \\
					&= \begin{cases} n!H_n, & \text{$n$ is even}, \\ n! (H_n + \frac{2}{n+1}), & \text{$n$ is odd.} \end{cases} 
			\end{align*}
		\end{solution*}

		\begin{problem}
			For a fixed positive integer $k$, consider the egf $f_k(x) = \sum_{n \ge 0} s(n,k) x^n/n!$. Show that
			\[ f_k(x) = \frac{1}{k!} \ln\left( \frac{1}{1-x} \right)^k. \]
		\end{problem}
		\begin{solution*}
			Define $g : \N \to \N_0$ by
			\[ g(r) = \begin{cases} 1, & r = k, \\ 0, & \text{otherwise.} \end{cases} \]
			Observe that
			\[ s(n,k) = \sum_{\pi \in \mathfrak{S}_n} g(r), \]
			where $C_1,\ldots,C_r$ are the cycles in $\pi$. We now use Corollary 5.1.8 in \cite{ec2} with $f$ being the function that takes the constant $1$. We have $E_g(x) = \frac{1}{(k-1)!} x^k$, so we get that
			\[ f_k(x) = \frac{1}{(k-1)!} \left( \sum_{n \ge 1} \frac{x^n}{n} \right)^k = \frac{1}{(k-1)!} \ln \left( \frac{1}{1-x} \right)^k. \]
		\end{solution*}

		\begin{problem}
			Find $\sum_{k=0}^n (-1)^k s(n,k)$.
		\end{problem}
		\begin{solution*}
			Recall that a permutation in $\mathfrak{S}_n$ with $k$ cycles has sign $(-1)^{n-k}$. We have
			\begin{align*}
				\sum_{k=0}^n (-1)^k s(n,k) &= \sum_{k=0}^n \sum_{\substack{\pi \in \mathfrak{S}_n \\ \text{$\pi$ has exactly $k$ cycles}}} (-1)^k \\
					&= (-1)^n \sum_{\pi \in \mathfrak{S}_n} \sign(\pi).
			\end{align*}
			For $n=1$, this is $-1$. Otherwise, note that $\pi \mapsto (1,2) \pi$ is a bijection between odd and even permutations. As a result, the above sum is equal to $0$.
		\end{solution*}

		\begin{problem}
			Show that $S(n+1,k+1) = \sum_{m=0}^n \binom{n}{m} S(m,k)$.
		\end{problem}
		\begin{solution*}
			We shall count the partitions of $[n+1]$ into exactly $k+1$ parts based on the part $n+1$ is in. We have
			\[ S(n+1,k+1) = \sum_{S \subseteq [n]} S(|[n]\setminus S|,k) = \sum_{m=0}^n \binom{n}{m} S(n-m,k) = \sum_{m=0}^n \binom{n}{m} S(m,k). \]
		\end{solution*}

		% \begin{problem}
		% 	Find the number of ways in which a product of $n$ distinct primes can be factored.
		% \end{problem}
		% \begin{solution*}
		% 	$s(n,k)$ trivially?
		% \end{solution*}

		\begin{problem}
			Show that for $n \ge 1$, the $S_{n,k}$ as $k$ varies has either a unique maximum value or has at most two equal values.
		\end{problem}
		\begin{solution*}
			
		\end{solution*}

	\subsection{Problem Sheet 2}

		\begin{problem}
			\label{problem: recurrence of Chebyshev of the first kind}
			Show that
			\[ T_n(x) = 2xT_{n-1}(x) - T_{n-2}(x). \]
			when $n \ge 2$ and $T_0(x) = 1, T_1(x) = x$.
		\end{problem}
		\begin{solution*}
			Let $\cos \theta = x$. We have
			\begin{align*}
				T_n(x) = \cos n\theta &= \cos (n-1)\theta \cos \theta - \sin (n-1)\theta \sin \theta \\
					&= x T_{n-1}(x) - ( \sin (n-2)\theta \cos\theta + \cos (n-2)\theta \sin\theta ) \sin\theta \\
					&= x T_{n-1}(x) - T_{n-2}(x) (1-x^2) - x (\sin\theta \sin (n-2)\theta) \\
					&= x T_{n-1}(x) + x^2 T_{n-2}(x) - T_{n-2}(x) - x (\cos\theta \cos (n-2)\theta - \cos (n-1)\theta) \\
					&= 2x T_{n-1}(x) - T_{n-2}(x). \qedhere
			\end{align*}
		\end{solution*}

		\begin{problem}
			Show that
			\begin{enumerate}[label=(\alph*)]
				\item $T_n(1) = 1$ \text{ and}
				\item $T_n(-1) = (-1)^n$.
			\end{enumerate}
		\end{problem}
		\begin{solution*}
			This immediately follows since $T_n(\cos\theta) = \cos n\theta$, so $T_n(\cos 0) = \cos (n\cdot 0) = 1$ and $T_n(\cos \pi) = \cos(n\pi) = (-1)^n$. They can also be easily proved inductively.
		\end{solution*}

		\begin{problem}
			Show that
			\begin{enumerate}[label=(\alph*)]
				\item $U_n(1) = n+1$.
				\item $U_n(-1) = (-1)^n (n+1)$
			\end{enumerate}
		\end{problem}
		\begin{solution*}
			We prove this inductively. Both statements are trivially true for $n=0,1$. For $n\ge 2$, inductively, we have
			\[ U_n(1) = 2U_{n-1}(1) - U_{n-2}(1) = 2n - (n-1) = n+1 \]
			and
			\[ U_n(-1) = -2U_{n-1}(1) - U_{n-2}(1) = (-1)^{n} \cdot 2n + (-1)^{n-1} (n-1) = (-1)^n (n+1). \]
		\end{solution*}

		\begin{problem}
			Show that
			\[ \frac{1}{\iota^n} U_n(\iota/2) = f_{n+1}. \]
		\end{problem}
		\begin{solution*}
			Again, we prove this inductively. We have $U_0(\iota/2) = 1 = f_1$ and $U_1(\iota/2) = \iota = \iota f_2$. For $n \ge 2$, we inductively have
			\[ U_n(\iota/2) = \iota U_{n-1}(\iota/2) - U_{n-2}(\iota/2) = \iota^n f_{n} - \iota^{n-2} f_{n-1} = \iota^n (f_n + f_{n-1}) = \iota^n f_{n+1}. \]
		\end{solution*}

		\begin{problem}
			Show that if $m,n \ge 1$,
			\[ T_{m+n}(x) = T_m(x)U_n(x) - T_{m-1}(x)U_{n-1}(x). \]
		\end{problem}
		\begin{solution*}
			This may be checked manually for $m+n=2,3$. We perform induction on $m+n$. We have that
			\begin{align*}
				T_{m+n}(x) &= 2xT_{m+n-1}(x) - T_{m+n-2}(x) \\
					&= 2x (T_{m-1}(x)U_n(x) - T_{m-2}(x)U_{n-1}(x)) - (T_{m-1}(x)U_{n-1}(x) - T_{m-2}(x)U_{n-2}(x)) \\
					&= 2x T_{m-1}(x) U_n(x) - T_{m-1}(x)U_{n-1}(x) - T_{m-2}(x) (2x U_{n-1}(x) - U_{n-2}(x)) \\
					&= 2x T_{m-1}(x) U_n(x) - T_{m-1}(x)U_{n-1}(x) - T_{m-2}(x) U_n(x) \\
					&= U_n(x) (2xT_{m-1}(x) - T_{m-2}(x)) - T_{m-1}(x)U_{n-1}(x) \\
					&= T_m(x) U_n(x) - T_{m-1}(x) U_{n-1}(x).
			\end{align*}
		\end{solution*}

		\begin{problem}
			Similar to the tiling combinatorial model for $U_n$, get a combinatorial model for $T_n$.
		\end{problem}
		\begin{solution*}
			One can do this in a manner identical to that of $U_n$, except that a square piece has weight $x$ if it is at the leftmost $(1,1)$ position.
		\end{solution*}