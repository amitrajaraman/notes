\section{Introduction}

Manifolds are essentially like curves or surfaces, except that they might be of higher dimension. Loosely speaking, an $n$-dimensional manifold is something that, at every point, is locally like $\Rn$.\\
So for example, manifolds of dimension $1$ would be lines and curves. Manifolds of dimension $2$ are surfaces, such as the unit sphere in $\R^3$, torii, paraboloids, etc.\\
To help us formalize this notion, let us make concrete what ``locally like $\Rn$'' means.

\begin{fdef}
	A topological space $M$ is said to be \textit{locally Euclidean of dimension $n$} if every point in $M$ has a neighbourhood homeomorphic to an open subset of $\Rn$.
\end{fdef}

Further, we see that since any open subset of $\Rn$ contains (an affine shift) of the Euclidean ball, the following follows as well.

\begin{lemma}
	Let $M$ be a topological space. The following are equivalent.
	\begin{enumerate}[(a)]
		\item $M$ is locally Euclidean of dimension $n$.
		\item Every point of $M$ has a neighbourhood homeomorphic to the open ball in $\Rn$.
		\item Every point of $M$ has a neighbourhood homeomorphic to $\Rn$.
	\end{enumerate}
\end{lemma}

We represent the open ball in $\Rn$ by $\B^n$.\\
The proof of the above is not too difficult -- we can use the fact that in $\Rn$, \textit{translations} ($x\mapsto x+x_0$), \textit{dilations} ($x\mapsto cx$), and \textit{boundedness} are not topological properties (that is, they need not be preserved under homeomorphism). In particular, $\B^n$ is homeomorphic to $\Rn$.\\

Suppose $U$ is locally Euclidean of dimension $n$ and $U\subseteq M$ is open and homeomorphic to some open subset of $\Rn$. In this context, $U$ is called a \textit{coordinate domain} and any homeomorphism $\varphi$ from $U$ to an open subset of $\Rn$ is called a \textit{coordinate map}. $(U,\varphi)$ is then called a \textit{coordinate chart}. If $U$ is homeomorphic to $\B^n$, it is called a \textit{coordinate ball} (or sometimes, a \textit{coordinate disk} in dimension $2$).\\
If $p\in M$ and $U\ni p$ is a coordinate domain, it is called a \textit{coordinate neighbourhood} or \textit{Euclidean neighbourhood of $p$}.

Observe that the definition of being locally Euclidean even extends to dimension $0$ -- this is equivalent to saying that the space is discrete.

\begin{fdef}[Manifold]
	An \textit{$n$-dimensional topological manifold} is a second countable Hausdorff space that is locally Euclidean of dimension $n$.
\end{fdef}

We often refer to them as $n$-dimensional manifolds, $n$-manifolds, or even just manifolds if the dimension is known or unimportant.

First of all, note that any manifold has \textit{a} well-defined dimension. However, this begs the question -- does a manifold have a \textit{unique} dimension? This might seem quite obvious, but proving it is quite non-trivial. We state the result here, and return to it much later.

\begin{ftheo}
	A non-empty topological space cannot be both an $m$-manifold an an $n$-manifold for some $m\neq n$.
\end{ftheo}

The proof of the above is quite obvious in the case where one of $m$ and $n$ is $0$. Indeed, this would mean that the space is discrete, but a singleton set (that is open), does not contain any open subset homeomorphic to $\B^n$.

At the beginning of this section, we described how manifolds can be thought of as curves or surfaces, but of a higher dimension. Is this true in general? It turns out that it is, and any $n$-manifold is homeomorphic to some subset of a Euclidean space.

In the definition of a manifold, some authors require separability instead of second countability. This is in fact equivalent to our definition, as is not too difficult to show (one direction is obvious from \Cref{theo: second countable then}).