%!TEX root = main.tex

% \section{The Delsarte and Schrijver Bounds}

\clearpage
\section{$*$-algebras of matrices}

	Denote by $\mathcal{M}_n(\C)$ the $\C$-vector space of all $n \times n$ complex matrices.

	\begin{fdef}
		A subspace $\mathcal{A} \subseteq \mathcal{M}_n(\C)$ is said to be a \emph{$*$-algebra of matrices} if
		\begin{enumerate}[label=(\alph*)]
			\item $\mathcal{A}$ is closed under multiplication, in that if $A,B \in \mathcal{A}$, then $AB \in \mathcal{A}$, and
			\item $\mathcal{A}$ is closed under conjugate transposes, in that if $A = (a_{ij}) \in \mathcal{A}$, then $A^\dagger = (\overline{a_{ji}}) \in \mathcal{A}$.
			\item $\Id \in \mathcal{A}$.
		\end{enumerate}
	\end{fdef}

	That is, it is a subalgebra that is closed under conjugate transposes. $*$-algebras are also sometimes referred to as \emph{self-adjoint algebras}.\\

	Let $q$ be a prime power. Denote by $B_q(n)$ the set of all subspaces of $\F_q^n$ and $B_q(n,k)$ the set of all $k$-dimensional subspaces of $\F_q^n$. It is not too difficult to show that
	\[ |B_q(n,k)| = \binom{n}{k}_q = \frac{(q^n-1)(q^n-q)(q^n-q^2)\cdots(q^n-q^{n-k+1})}{(q^k-1)(q^k-q)(q^k-q^2)\cdots(q^k-q^{k-1})}. \]
	We had also considered this quantity $\binom{n}{k}_q$ in Section 1.4 of \href{https://amitrajaraman.github.io/notes/ma-861/main.pdf}{Combinatorics I}. Recall the $q$-Pascal recurrence
	\begin{equation}
		\label{eqn: q-pascal}
		\binom{n+1}{k}_q = \binom{n}{k-1}_q + q^k \binom{n}{k}_q
	\end{equation}
	for $n\ge 0, k \ge 1$ with $\binom{n}{0}_q = 1$ and $\binom{0}{k} = \delta_{0,k}$. Is there a way to see this recurrence more directly using the subspace perspective of the $q$-binomial coefficient? If we have a (size $k$) basis of a $k$-dimensional subspace of $\F_q^n$, and consider the $k \times n$ matrix with rows equal to the vectors in this basis, we may bring this matrix to a \emph{unique} row-reduced echelon form (independent of the basis used) using row operations wherein
	\begin{enumerate}[label=(\roman*)]
		\item all rows are nonzero,
		\item the first non-zero entry in every row is a $1$. Suppose this entry occurs in column $C_i$ in row $i$,
		\item $C_1 < C_2 < \cdots C_k$, and
		\item the submatrix comprising the $\{C_1,\ldots,C_k\}$ columns is a $k \times k$ identity matrix.
	\end{enumerate}
	So, we can count $k \times n$ matrices in RREF instead of subspaces. \Cref{eqn: q-pascal} then follows immediately by considering whether the last column is pivotal or not.

	\begin{fdef}
		Let $A$ be Hermitian. Then, $\langle A\rangle$, the $*$-algebra generated by $A$, is $\Span\{\Id,A,A^2,\ldots\}$.
	\end{fdef}

	Note that this algebra is abelian. Furthermore, by the spectral theorem, $\dim(\langle A\rangle)$ is the number of distinct eigenvalues of $A$.\\ % *** EXPLAIN ***

	If $A \in \mathcal{M}^n(\C)$ is such that $PAP^{-1}$ is Hermitian, then $P\langle A\rangle P^{-1}$ is also a $*$-algebra.
	
	\begin{fex}[$*$-algebras on graphs]
		\label{exmp: star-alg}
		Let $G = (V,E)$ be a graph and $A$ its adjacency matrix. $\langle A\rangle$ is called the \emph{adjacency algebra} of $G$. \\

		More specifically, consider the $n$-cube graph $C_n$ with vertex set $B(n)=2^{[n]}$ and an edge between $X,Y$ if $|X \triangle Y| = 1$. Although $\langle A\rangle$ is a $*$-algebra of $2^n \times 2^n$ matrices, its dimension turns out to be only $n+1$. The fact that we only require $n+1$ parameters to describe an arbitrary element of $\langle A\rangle$ is key to the Delsarte bound on binary code size we shall study later.\\

		Let $k \le n/2$. The Johnson graph has vertex set $B(n,k) = \binom{[n]}{k}$ and an edge between $X,Y$ if $|X \cap Y| = k-1$. The dimension of this graph's adjacency algebra turns out to be $k+1$.\\

		The Grassmann graph $J_q(n,k)$ has vertex set $B_q(n,k)$ with $X,Y \in B_q(n,k)$ adjacent iff $\dim(X \cap Y) = k-1$. It turns out that the dimension of this graph's adjacency algebra is $k+1$ as well. Interestingly, the proof for this ends up just being a ``$q$-analogue'' of the proof for the Johnson graph.\\

		The $q$-analogue of the $n$-cube $C_q(n)$ has vertex set $B_q(n)$ with $X,Y$ adjacent iff $|\dim X - \dim Y| = 1$. We do not know the dimension of this graph's adjacency algebra! The adjacency matrix seems difficult to study (and is perhaps not even the right object to study). We shall instead study a weighted adjacency matrix of $C_q(n)$.
	\end{fex}

	All the above examples are commutative. \href{https://amitrajaraman.github.io/notes/rep-th/main.pdf}{Recall} that a \emph{unitary representation} of a group $G$ is a group homomorphism $\varphi : G \to \mathcal{U}_n(\C)$.

	\begin{ftheo}
		Let $\varphi$ be a unitary representation of a group $G$. Then,
		\[ \mathcal{A} = \{A \in \mathcal{M}_n(\C) : A \varphi(g) = \varphi(g) A \text{ for all } g \in G \} \]
		is a $*$-algebra called the \emph{commutant} of $\varphi$.
	\end{ftheo}
	\begin{proof}
		It is obvious that $\mathcal{A}$ is a subspace that is closed under multiplication. We have for $A \in \mathcal{A}, g \in G$ that
		\[ \varphi(g^{-1}) = \varphi(g)^{-1} = \varphi(g)^\dagger, \]
		so
		\[ A^\dagger \varphi(g) = (\varphi(g)^\dagger A)^\dagger = (\varphi(g^{-1}) A)^\dagger = (A \varphi(g)^{-1})^\dagger = \varphi(g) A^\dagger, \]
		which easily yields the desideratum.
	\end{proof}

	The above $*$-algebra may possibly be non-commutative. Suppose that $G$ acts on a set $S$. For each $g$, we can denote the group action by an $S \times S$ permutation matrix $\rho(g)$, with $(\rho(g))_{gs,s} = 1$. This gives a \emph{representation} $\rho : G \to \mathcal{U}_{S}(\C)$ -- any group action thus yields a $*$-algebra.\\
	We would like to analyze the set of matrices which commute with all $\rho(g)$. Let $G$ act on the sets $S,T$, and let $\rho : G \to \mathcal{U}_S(\C), \tau : G \to \mathcal{U}_T(\C)$ be the corresponding maps. Consider
	\[ \mathcal{A} = \left\{ M \in \mathcal{M}_{T \times S}(\C) : M \rho(g) = \tau(g) M \text{ for all $g \in G$} \right\}. \]
	Finally, we shall set $S = T$ so that it is a $*$-algebra, which we denote $\Hom_G(S,S)$.

	\begin{flem}
		Let $M \in \mathcal{M}_{T \times S}(\C)$. Defining $\mathcal{A}$ as above, $M \in \mathcal{A}$ iff $M_{t,s} = M_{gt,gs}$ for all $g \in G, t \in T,s \in S$.
	\end{flem}
	\begin{proof}
		The $t,s$th entry of $M\rho(g)$ is equal to $M_{t,gs}$, and that of $\tau(g)M$ is $M_{g^{-1}t,s}$. The required follows.
	\end{proof}

	Now, the two actions induce an action on $T \times S$. $M$ belongs to $\mathcal{A}$ iff it is constant on the orbits of this action. Consequently, the dimension of $\mathcal{A}$ is the number of orbits of the action of $G$ on $T \times S$, with a basis being the set of matrices $M_j$ which are equal to $1$ on precisely those cells in the same orbit $\theta_j$ and $0$ elsewhere.\\
	This basis of $\mathcal{A}$ is called its \emph{orbital basis}.

	\begin{flem}[Gelfand's Lemma]
		\label{lem: gelfands lemma}
		Let $T = S$ in the above discussion. If each $M_j$ is symmetric, $\mathcal{A}$ is commutative.
	\end{flem}
	\begin{proof}
		Since each $M_j$ is symmetric and orthogonal, all matrices in $\mathcal{A}$ are symmetric. We are done if we show that a $*$-algebra of symmetric matrices is commutative. Indeed, $MN = (MN)^\top = N^\top M^\top = NM$.
	\end{proof}

	The converse does \emph{not} hold.

	\begin{fex}[The converse of Gelfand's lemma is not true]
		Let $G$ be a finite group. $G \times G$ acts on $G$ by $(g,h) \cdot a = gah^{-1}$. What is the orbital basis of the commutant of this action?\\
		Let $(a,b),(c,d) \in G \times G$. Then, $(a,b) \sim (c,d)$ iff $ab^{-1}$ and $cd^{-1}$ are conjugates in $G$.\\
		The former is true iff for some $g,h \in G$, $gah^{-1} = c$ and $gbh^{-1} = d$. Equivalently, $ga = ch$ and $b^{-1}g^{-1} = h^{-1}d^{-1}$. Multiplying the two, this implies that $gab^{-1}g^{-1} = cd^{-1}$, that is, $ab^{-1}$ and $cd^{-1}$ are conjugates. For the backward direction, if we have $gab^{-1}g^{-1} = cd^{-1}$. Setting $h = gac^{-1}$, the previous equation implies that $h = d^{-1}gb$. This directly implies that $gah^{-1} = c$ and $gbh^{-1} = d$.\\
		Let the conjugacy classes of $G$ be $C_1,\ldots,C_t$. Consider the $G \times G$ matrices $A_j$ by
		\[ A_j(g,h) = \begin{cases} 1, & gh^{-1} \in C_j, \\ 0, & \text{otherwise.} \end{cases} \]
		In the case where each element of the group is conjugate to its inverse, we can use \nameref{lem: gelfands lemma} to conclude that each $A_j$ is symmetric so $\mathcal{A}$ is abelian. An example of such a group is the symmetric group $S_n$, and the dimension of the resulting $\mathcal{A}$ is $p(n)$, the number of number partitions of $n$.\\

		However, $\mathcal{A}$ is commutative for \emph{any} $G$, even in the case where the orbital matrices are not symmetric. As before, let $C_1,\ldots,C_t$ be the conjugacy classes of $G$, and consider the orbital matrices $A_1,\ldots,A_t$, where $(A_r)_{gh} = 1$ iff $gh^{-1} \in C_r$ and $0$ otherwise. It suffices to show that the orbital matrices commute. Let us show that $A_1,A_2$ commute. We have
		\[ (A_1A_2)_{ab} = |\{x \in G : ax^{-1} \in C_1, xb^{-1} \in C_2\}| \]
		and
		\[ (A_2A_1)_{ab} = |\{x \in G : xb^{-1} \in C_1, ax^{-1} \in C_2\}|. \]
		It is easily checked that a bijection between these two sets is given by $x \mapsto ax^{-1}b$, proving the claim.
	\end{fex}

	Let us get back to our earlier discussion in \Cref{exmp: star-alg}. Think of $B(n)$ as $\{0,1\}^n$. Consider the \emph{hyperoctahedral group} $H_n$, which has base set equal to $S_2^n \times S_n$, with elements denoted $(\sigma_1,\sigma_2,\ldots,\sigma_n,\pi)$. This group acts on $B(n)$ by first permuting the $n$ coordinates according to $\pi$, then deciding whether or not to flip the entries based on the $(\sigma_i)$. Note that adjacency is preserved under the group action. In fact, $H_n$ is the set of all permutations that preserve adjacency.\\
	The group action can be thought of as first taking the vertex to any other arbitrary vertex, then permuting the $n$ outgoing edges in some manner -- these two together further determine the group element.\\

	Let $\alpha,\beta,\alpha',\beta' \in B(n)$. We denote by $d(\alpha,\beta)$ the set of coordinates where $\alpha,\beta$ differ. We write $(\alpha,\beta) \sim (\alpha',\beta')$ if the two are in the same $H_n$-orbit.

	\begin{flem}
		$(\alpha,\beta)$ and $(\alpha',\beta')$ are in the same $H_n$-orbit iff $d(\alpha,\beta) = d(\alpha',\beta')$.
	\end{flem}
	\begin{proof}
		The forward direction is straightforward -- permuting the coordinates leaves the distance the same and flipping a select set of coordinates of both also leaves the distance unchanged.\\
		For the backward direction, suppose $d(\alpha,\beta) = d(\alpha',\beta') = k$. Consider the permutation applied to $\alpha$ which has all $0$s at the start then all $1$s. Then, flip all the $1$s in $\alpha$. Consider the element $\beta''$ obtained by performing the same operations on $\beta$. Due to the first part, $\beta''$ has exactly $k$ $1$s. Next, permute the coordinates of $\beta''$ to get $\beta'''$, which has all $0$s at the start then all $1$s. $(0,\beta''')$ is in the same orbit as $(\alpha,\beta)$. By performing similar operations, it is also in the same orbit as $(\alpha',\beta')$, completing the proof. 
	\end{proof}

	Let $A_0,A_1,\ldots,A_n$ be the $n$ orbital bases of $B(n) \times B(n)$ under the group action $H_n$, defined by
	\[ A_j(\alpha,\beta) = \begin{cases} 1, & d(\alpha,\beta) = j, \\ 0, & \text{otherwise.} \end{cases} \]
	Going back to the perspective of $B(n)$ containing subsets of $[n]$,
	\[ A_j(X,Y) = \begin{cases} 1, & |X\triangle Y| = j, \\ 0, & \text{otherwise.} \end{cases} \]
	Note that $A_1$ is the adjacency matrix $A$ of the $n$-cube graph $C(n)$!
	% CAL A HAS SOME SPECIAL NAME HERE, CHECK NOTES
	\begin{fprop}
		\label{prop: ncube-adj-eigenvals}
		It holds that $\langle A\rangle = \Span\{A_0,A_1,\ldots,A_n\}$.
	\end{fprop}
	\begin{proof}
		Denote by $\mathcal{A}$ the algebra on the right, which is the commutant of the $H_n$ action on $B(n)$. Because $A_1 = A$ is in $\mathcal{A}$, $\langle A\rangle \subseteq \mathcal{A}$. It remains to show the reverse containment, which is implied if we show that $A_j \in \langle A\rangle$ for each $j$. If $A_j \in \langle A\rangle$, then $AA_j$ is just some linear combination of $A_0,A_1,\ldots,A_{j+1}$ (with a positive coefficient on $A_{j+1}$), completing the proof.
	\end{proof}

	\begin{fcor}
		The adjacency matrix $A$ of the $n$-cube graph has $n+1$ distinct eigenvalues.
	\end{fcor}

	A natural next question is: what are these $n+1$ eigenvalues, and what are each of their eigenspaces and multiplicities?\\
	As a little spoiler, we answer these questions: the eigenvectors are $n-2k$ for $k=0,1,\ldots,n$, with $n-2k$ having multiplicity $\binom{n}{k}$. We shall prove this later in \Cref{subsec:delsarte}. \\

	Let us next go back to the example of $B(n,k)$. $S_n$ acts on $B(n,k)$ with $\pi \cdot \{i_1,\ldots,i_k\} = \{\pi(i_1),\ldots,\pi(i_k)\}$. What are the orbits of this $S_n$-action on $B(n,k) \times B(n,k)$?

	\begin{flem}
		Let $(X,Y), (X',Y') \in B(n,k) \times B(n,k)$. Then, $(X,Y) \sim (X',Y')$ iff $|X \cap Y| = |X' \cap Y'|$.
	\end{flem}
	The proof of the above is straightforward, and we omit it. Note in particular that $(X,Y) \sim (Y,X)$, so each orbital matrix is symmetric. Therefore,
	\[ \mathcal{A} = \Hom_{S_n}(B(n,k),B(n,k)) \]
	is commutative. We have for any sets $X,Y$ of size $k$ that
	\[ \max\{0,2k-n\} \le |X \cap Y| \le k. \]
	Therefore, $\dim \mathcal{A} = 1 + \min\{k,n-k\}$. Let $\{A_k,A_{k-1},\ldots,A_{\max\{0,2k-n\}}\}$ be the orbital basis of $\mathcal{A}$ with $A_j(X,Y) = 1$ if $|X \cap Y| = j$ and $0$ otherwise. Then,$A_k = \Id$ and $A_{k-1} = A$ is the adjacency matrix of the Johnson graph $J(n,k)$!

	\begin{fprop}
		\label{prop: johnson-adj-eigenvals}
		It holds that $\langle A\rangle = \Span\{A_k,A_{k-1},\ldots,A_{\max\{0,2k-n\}}\}$.
	\end{fprop}
	The proof is very similar to that of \Cref{prop: ncube-adj-eigenvals}.

	\begin{fcor}
		The adjacency matrix $A$ of the Johnson graph $J(n,k)$ has $1+\min\{k,n-k\}$ distinct eigenvalues.
	\end{fcor}
	In the case where $k \le n-k$, the multiplicities of the eigenvalues of the graph are $\binom{n}{0}, \binom{n}{1} - \binom{n}{0}, \binom{n}{2} - \binom{n}{1}, \ldots, \binom{n}{k}-\binom{n}{k-1}$. We shall prove this and find the corresponding eigenspaces later in \Cref{subsec:schrijver,subsec:johnson-schemes}.\\

	When we deal with $B_q(n,k)$, the collection of $k$-dimensional subspaces of $\F_q^n$, we shall take the action of $\GL_n(\F_q)$ defined by
	\[ MX = M(X) = \{Mv : v \in X\} \]
	Once more, we get results as in the Johnson graph.

	\begin{flem}
		Let $(X,Y), (X',Y') \in B_q(n,k) \times B_q(n,k)$. Then, $(X,Y) \sim (X',Y')$ iff $\dim(X \cap Y) = \dim(X' \cap Y')$.
	\end{flem}
	So, the Grassmann graph with adjacency matrix $A$ and corresponding adjacency algebra $\mathcal{A}$ has $\dim \mathcal{A} = 1+\min\{k,n-k\}$ as well. Letting $\{A_k,A_{k-1},\ldots,A_{\max\{0,2k-n\}}\}$ be the orbital basis of $\mathcal{A}$ with $A_j(X,Y) = 1$ if $\dim(X \cap Y) = j$ and $0$ otherwise, we again get the following.

	\begin{fprop}
		\label{prop: johnson-adj-eigenvals}
		It holds that $\langle A\rangle = \Span\{A_k,A_{k-1},\ldots,A_{\max\{0,2k-n\}}\}$.
	\end{fprop}

	\begin{fcor}
	 	The adjacency matrix $A$ of the Grassmann graph $J_q(n,k)$ has $1+\min\{k,n-k\}$ distinct eigenvalues. 
	 \end{fcor}
	 
	 The multiplicity of the eigenvalues (when $k \le n/2$) end up being $\binom{n}{0}_q, \binom{n}{1}_q-\binom{n}{0}_q, \binom{n}{2}_q-\binom{n}{1}_q, \ldots, \binom{n}{k}_q-\binom{n}{k-1}_q$.\\

	 So far, all examples have been commutative.

	 \begin{fex}[Non-commutative $*$-algebras]
	 	\label{ex:non-comm-alg}
	 	Consider the action of $S_n$ on $B(n)$, with $\pi \{i_1,\ldots,i_k\} = \{\pi(i_1),\ldots,\pi(i_k)\}$. Similar to what we have already seen, $(X,Y) \sim (X',Y')$ iff $|X| = |X'|$, $|Y| = |Y'|$, and $|X \cap Y| = |X' \cap Y'|$. Consider the $B(n) \times B(n)$ matrix $M_{i,j,t}$ defined by
	 	\[
	 	M_{i,j,t}(X,Y) =
	 	\begin{cases}
	 		1, & |X|=i, |Y|=j, |X \cap Y| = t, \\ 0, & \text{otherwise,}
	 	\end{cases}
	 	\]
	 	for any choice of $i-t\ge 0$, $j-t\ge 0$, and $i+j-t \le n$. The number of ways of choosing such $i,j,t$ is $\binom{n+3}{3}$ -- we would like to find the number of solutions to $(i-t) + (j-t) + t + r = n$, where $i-t,j-t,t,r \ge 0$.\\
	 	Therefore, setting $\mathcal{A} = \Hom_{S_n}(B(n),B(n))$, we have $\dim \mathcal{A} = \binom{n+3}{3}$. Further note that $\mathcal{A}$ is non-commutative. Indeed, $M_{2,3,1} M_{3,4,2} \ne 0$ but $M_{3,4,2} M_{2,3,1} = 0$.\\

	 	The $q$-analogue of the above example is as follows. Let $\GL_n(\F_q)$ act on $B_q(n)$, and define $M_{i,j,t}(q)$ by
	 	\[ M_{i,j,t}(q)(X,Y) = \begin{cases} 1, & \dim X = i, \dim Y = j, \dim (X \cap Y) = t, \\ 0, & \text{otherwise.} \end{cases} \]
	 	Again, we have $\dim \mathcal{A} = \binom{n+3}{3}$.
	 \end{fex}

	So far, this idea of translating proofs to proofs in the setting of $q$-analogues seems pretty straightforward. However, things don't work out as well when we try to go from $C(n)$ to $C_q(n)$. The issue is that $H_n$ does not have a neat $q$-analogue. Later, we shall look at a $q$-analogue of $\Hom_{H_n}(B(n),B(n))$ that does not come from a group action.

	\begin{fex}
		Consider $K_{2n}$, the complete graph on $2n$ vertices. It is not too difficult to show that the number of perfect matchings of $K_{2n}$ is $\frac{(2n)!}{n!2^n} = (2n)!!$. Denote the set of all perfect matchings on $K_{2n}$ by $\PM_{2n}$. $S_{2n}$ acts on $\PM_{2n}$ in an obvious manner, by mapping the matching $\{i_1j_1,i_2j_2,\ldots,i_nj_n\}$ to $\{\pi(i_1)\pi(j_1),\ldots,\pi(i_n)\pi(j_n)\}$. What are the $K_{2n}$ orbits on $\PM_{2n} \times \PM_{2n}$?\\
		Let $M_1,M_2 \in \PM_{2n}$. It is not too difficult to see that $M_1 \cup M_2$ comprises of ``alternating cycles'', namely even cycles whose edges alternate between being in $M_1,M_2$ (such a cycle may also be a $2$-cycle with two edgees between two vertices, one of which is in $M_1$ and the other in $M_2$).  This induces a number partition of $n$, based on the number of cycles of size $2k$ for $1 \le k \le n$. Call this partition $d(M_1,M_2)$.\\
		
		We claim that $(M_1,M_2) \sim (M_3,M_4)$ iff $d(M_1,M_2) = d(M_3,M_4)$.\\
		The forward direction is direct since if we have $\pi (M_1,M_2) = (M_3,M_4)$, then $\pi$ applied to the vertices of the multigraph $M_1 \cup M_2$ gives $M_3 \cup M_4$ while having the same graph (up to isomorphism), so the partition remains the same. For the backward direction, just match up $M_1 \cup M_2$ and $M_3 \cup M_4$ in a way that cycle sizes agree.\\

		Therefore, the dimension of this $*$-algebra is $p(n)$, the number of partitions of $n$. Recall that this is the same as the number of partitions as the previous example when $G = S_n$. Further, since $d(M_1,M_2) = d(M_2,M_1)$, this algebra is commutative by \nameref{lem: gelfands lemma}.
	\end{fex}

	Much like the spectral theorem of normal matrices, there is a spectral theorem of $*$-algebras which ``diagonalizes'' them.

	\begin{ftheo}[Spectral theorem for commutative $*$-algebras]
		Let $\mathcal{A} \subseteq \mathcal{M}_n(\C)$ be a commutative $*$-algebra. Then, there exists an $n \times n$ unitary matrix $U$ and positive integers $q_1,\ldots,q_m$ (determined up to permutation) such that $U^\dagger \mathcal{A} U$ is the set of all \emph{$(q_0,\ldots,q_m)$-block diagonal matrices}, that is, the set of all matrices
		\[ \begin{pmatrix} C_1 & & & \\ & C_2 & & \\ & & \ddots & \\ & & & C_m \end{pmatrix}, \]
		where $C_k$ is a $q_k \times q_k$ scalar matrix. In particular, any element of $U^\dagger \mathcal{A} U$ is determined by the $m$ scalars corresponding to these blocks, so $\dim \mathcal{A} = m$ and $q_1 + \cdots + q_m = n$.
	\end{ftheo}
	\begin{proof}
		Instead of matrices in $\mathcal{M}_n(\C)$, we shall view the elements of $\mathcal{A}$ as linear operators on $\C^n$. We apply induction on $n$.\\
		First off, note that for any $S \in \mathcal{A}$, we have $S^\dagger \in \mathcal{A}$ by the definition of a $*$-algebra, and that $SS^\dagger = S^\dagger S$ since $\mathcal{A}$ is commutative. That is, all operators in $\mathcal{A}$ are normal. It follows by the spectral theorem that $\C^n$ can be decomposed into orthogonal eigenspaces of any such operator. \\
		Let $S,T \in \mathcal{A}$. Then, for any eigenvector $v \in \C^n$ of $S$ with eigenvalue $\lambda$, $S(Tv) = T(Sv) = \lambda (Tv)$, so eigenspaces of $S$ are invariant under $T$.\\
		Now, the base case $n=1$ is trivial. In general, let $S \in \mathcal{A}$ be a non-scalar matrix, and decompose $\C^n$ into an orthogonal direct sum $W_1 \oplus \cdots \oplus W_m$ of eigenspaces of $S$, where $m \ge 2$. As observed, each $W_i$ is invariant under operators in $\mathcal{A}$. Since $\dim W_i < n$, the result follows by the inductive hypothesis.
	\end{proof}

	\begin{corollary}
		Let $\mathcal{A}$ be a commutative $*$-algebra. Then there exist subspaces $W_1,\ldots,W_m$ of $\C^n$ that are (common) eigenspaces of any $A \in \mathcal{A}$.
	\end{corollary}

	There is also a more general spectral theorem for (not necessarily commutative) $*$-algebras, that we state without proof.

	\begin{ftheo}[Spectral theorem for $*$-algebras]
		\label{theo:noncomm-spec-thm}
		Let $\mathcal{A} \subseteq \mathcal{M}_n(\C)$ be a commutative $*$-algebra. Then, there exists an $n \times n$ unitary matrix $U$ and positive integers $p_1,\ldots,p_m$ and $q_1,\ldots,q_m$ (determined up to permutation) such that $U^\dagger \mathcal{A} U$ is the set of all \emph{$((p_0,q_0),\ldots,(p_m,q_m))$-block diagonal matrices}, that is, the set of all matrices
		\[ U^\dagger \mathcal{A} U = \begin{pmatrix} C_1 & & & \\ & C_2 & & \\ & & \ddots & \\ & & & C_m \end{pmatrix}, \]
		where $C_k$ is a block diagonal matrix
		\[ C_k = \begin{pmatrix} B_k & & & \\ & B_k & & \\ & & \ddots & \\ & & & B_k \end{pmatrix} \]
		consisting of $q_k$ repeated blocks of a $p_k \times p_k$ matrix $B_k$.	
		Furthermore, $\dim \mathcal{A} = p_1^2 + \cdots + p_m^2$ and $n = p_1q_1 + \cdots + p_mq_m$.
	\end{ftheo}

	In either spectral theorem, we say that we have a \emph{diagonalization} of $\mathcal{A}$ if we know the images $A \mapsto U^\dagger A U$ explicitly, and an \emph{explicit diagonalization} if we further know $U$.

\clearpage
\section{A primer on representation theory}

	\begin{fdef}
		A \emph{representation} of a group $G$ is a group homomorphism $\varphi : G \to \GL(V)$ for some finite-dimensional vector space $V$ over $\C$. Given such a representation, we say that $V$ is a \emph{$G$-module}.
	\end{fdef}
	The image of $g$ under $\varphi$ is denoted $\varphi_g$, but we usually abuse notation it like a group action. That is, we denote $(\varphi(g))(v)$ as $\varphi_g(v)$ or merely $g \cdot v$ or even $gv$ when the representation is clear from context.

	\begin{fex}
		Let $G$ be a group and $S$ a finite set such that $G$ acts on $S$. Consider the \emph{linearization} of $S$ or the \emph{permutation module} corresponding to $S$, which is the vector space with $S$ as a basis, that is,
		\[ \C[S] = \left\{ \sum_{s \in S} \alpha_s s : \alpha_s \in \C \right\}. \]
		The action of $G$ induces a representation on $\C[S]$, namely
		\[ g \cdot \left( \sum_s \alpha_s s \right) = \sum_s \alpha_s (g \cdot s). \] 
	\end{fex}

	\begin{fdef}
		Given a $G$-module $V$, a subspace $W \subseteq V$ is said to be a \emph{submodule} of $V$ if for all $w \in W$ and $g \in G$, $gw \in W$.
	\end{fdef}
	That is, it is invariant with respect to the representation.

	\begin{fdef}
		A $G$-module $V$ is said to be \emph{irreducible} if $\dim V > 0$ and it has no submodules other than $\{0\}$ and $V$.
	\end{fdef}
	More succinctly, an irreducible $G$-module is one with exactly two submodules. In particular, any one-dimensional module is irreducible.

	\begin{fex}
		\label{ex:sn-action-permutation-mod}
		Consider the obvious action of $S_n$ on $X = [n]$. Considering the permutation module $\C[X]$, the subspaces
		\begin{align*}
			V_1 &= \Span\{1+2+\cdots+n\} \text{ and} \\
			V_2 &= \{c_11 + c_22 + \cdots + c_nn : c_1 + \cdots + c_n = 0\}.
		\end{align*}
		Clearly, $V_1$ is irreducible. It turns out that $V_2$ is irreducible as well! Suppose instead that $W \ne 0$ is a submodule of $V_2$, containing $w = c_11 + \cdots + c_nn$ for some $(c_i)$ adding up to $0$. Suppose that $c_1 \ne 0$. We must have that some other $c_i$ is also nonzero and unequal to $c_1$; suppose that $c_2$ is so. Then,
		\begin{align*}
			w &= c_1 1 + c_2 2 + \cdots + c_nn \in W \\
			(1\; 2) w &= c_2 1 + c_1 2 + \cdots + c_n n \in W
		\end{align*}
		since $W$ is a submodule. Subtracting the two, we get that $(1-2) \in W$. Applying $(2\; j)$ for $j \ge 3$, we get that $(1-j) \in W$ for any $j = 2,3,\ldots,n$. Therefore, $\dim W = n-1$ so $W$ must be $V_2$.
	\end{fex}

	Ideally, we would like some result in the spirit of the prime factorization theorem, saying that any module can be decomposed into a direct sum of irreducible submodules in a ``unique'' fashion. We shall spend the remainder of this section developing this theorem.

	\begin{fdef}
		Let $V$ be a finite-dimensional vector space with an inner product $\langle \cdot,\cdot\rangle$. A \emph{unitary} representation is a group homomorphism $\varphi : G \to \mathcal{U}(V)$. In such a case, $V$ is called a \emph{unitary $G$-module}.
	\end{fdef}
	Above $\mathcal{U}(V)$ is the subgroup of matrices in $\GL(V)$ under which the inner product is preserved. That is, $\mathcal{U}(V)$ is the set of all matrices $A$ such that for any $v,w \in V$, $\langle v,w\rangle = \langle Av,Aw\rangle$.

	\begin{lemma}
		Let $V$ be a unitary $G$-module with $\dim V > 0$. Then, $V$ is a direct sum of irreducible submodules.
	\end{lemma}
	\begin{proof}
		If $V$ is irreducible, we are done. Suppose otherwise, and let $W \ne 0$ be a proper submodule of $V$. Consider $W^\perp = \{ v \in V : \langle v,w\rangle = 0 \}$. For any $v \in W^\perp$, $g \in G$, and $w \in W$, since $W$ is a submodule, $\langle gv,w\rangle = \langle v,g^{-1}w\rangle = 0$, so $gv \in W^\perp$. It follows that $W^\perp$ is a proper submodule of $V$. Induction on dimension completes the proof. 
	\end{proof}

	\begin{flem}
		Let $V$ be a $G$-module with $\dim V > 0$. Then, $V$ is a direct sum of irreducible submodules.
	\end{flem}
	\begin{proof}
		Let $(\cdot,\cdot)$ be any inner product on $V$. Consider the inner product $\langle \cdot,\cdot\rangle$ defined by
		\[ \langle v,w\rangle = \sum_{h \in G} ( hv,hw ). \]
		Note that $V$ is a unitary $G$-module with respect to $\langle \cdot,\cdot\rangle$. The desideratum follows by the previous lemma.
	\end{proof}

	This completes the first part of the statement we made earlier, showing that any module can be decomposed into a direct sum of irreducibles. Now, we would like to show that this decomposition is also unique in some sense.

	\begin{fdef}
		Given $G$-modules $V,W$, a linear map $f : V \to W$ is said to be \emph{$G$-linear} if $f$ commutes with the action of $G$, that is, $f(gv) = gf(v)$. We denote
		\[ \Hom_G(V,W) = \{ f : V \to W : f \text{ is $G$-linear} \}. \]
	\end{fdef}

	In some settings, $W$ may be a vector space of functions; in such cases, take care with the definition of $G$-linearity.

	\begin{flem}
		Let $V,W$ be irreducible $G$-modules and $f : V \to W$ be $G$-linear. Then, either $f \equiv 0$ or $f$ is an isomorphism.
	\end{flem}
	\begin{proof}
		Note that $\ker f$ and $\operatorname{im} f$ are respectively submodules of $V$ and $W$, so by irreducibility, they must each be equal to $0$ or the entire vector space. If $\ker f = V$, then $f \equiv 0$. If $\ker f = 0$, we must also have $\operatorname{im} f = W$ so $f$ is an isomorphism.
	\end{proof}

	\begin{flem}[Schur's Lemma]
		\label{schur's lemma}
		Let $V$ be an irreducible $G$-module and $f : V \to V$ be $G$-linear. Then, $f = \lambda I$ for some $\lambda \in \C$.
	\end{flem}
	\begin{proof}
		Let $\lambda$ be some eigenvalue of $f$. Then, $f - \lambda I$ is also $G$-linear and has nonzero kernel; by the previous lemma, it follows that it is identically $0$, completing the proof.
	\end{proof}

	\begin{fcor}
		Let $V,W$ be irreducible $G$-modules. Then,
		\[ \dim \Hom_G(V,W) = \begin{cases} 1, & V \cong W, \\ 0, & \text{otherwise.} \end{cases} \]
	\end{fcor}

	\begin{fcor}
		\label{irred-unique-inner-prod}
		A $G$-invariant inner product on an irreducible $G$-module is unique up to scaling.
	\end{fcor}
	\begin{proof}
		Let $\langle\cdot,\cdot\rangle$ and $[\cdot,\cdot]$ be two $G$-invariant inner products on an irreducible $G$-module $V$.
		Consider the linear map $\varphi : V \to V^*$ (where $V^*$ is the dual of $V$) defined by
		\[ \varphi(v)(u) = \langle v,u\rangle, \]
		and similarly $\psi : V \to V^*$ defined by $\psi(v)(u) = [v,u]$. Note that both $\varphi$ and $\psi$ are $G$-linear isomorphisms, where for $f \in V^*$ we define
		\[ (g \cdot f)(v) = f(g^{-1} \cdot v). \]
		It follows that $\psi^{-1} \circ \varphi : V \to V$ is a $G$-linear isomorphism. Irreducibility of $V$ with \nameref{schur's lemma} implies that $\psi^{-1} \circ \varphi = \lambda \Id$, which yields the desired.
	\end{proof}

	\begin{flem}
		Let $V,W$ be $G$-modules, and $W_1,W_2$ be $G$-submodules of $W$ such that $W = W_1 \oplus W_2$. Then,
		\[ \Hom_G(V,W_1\oplus W_2) \cong \Hom_G(V,W_1) \oplus \Hom_G(V,W_2). \]
		In particular,
		\[ \dim \Hom_G(V,W_1 \oplus W_2) = \dim \Hom_G(V,W_1) + \dim \Hom_G(V,W_2). \]
	\end{flem}
	\begin{proof}
		Let $\pi_1 : W \to W_1$ and $\pi_2 : W \to W_2$ denote the respective projection maps. Given $T \in \Hom_G(V,W_1\oplus W_2)$, we have $\pi_1 \circ T \in \Hom_G(V,W_1)$ and $\pi_2 \circ T \in \Hom_G(V,W_2)$. For the backward inclusion, given $T_1 \in \Hom_G(V,W_1), T_2 \in \Hom_G(V,W_2)$, the map $T$ defined by $T(v) = (T_1(v),T_2(v))$ is in $\Hom_G(V,W)$. This establishes an isomorphism between $\Hom_G(V,W)$ and $\Hom_G(V,W_1) \oplus \Hom_G(V,W_2)$, proving the claim.
	\end{proof}

	Given a vector space $V$, denote by $nV$ the direct sum of it with itself $n$ times. Also denote $0V = 0$.

	\begin{fcor}
		Let $V_1,\ldots,V_r$ be irreducible $G$-modules and $V,W$ be $G$-modules such that
		\begin{align*}
			V &\cong n_1V_1 \oplus n_2V_2 \oplus \cdots \oplus n_rV_r \text{ and} \\
			W &\cong m_1V_1 \oplus m_2V_2 \oplus \cdots \oplus m_rV_r, \\
		\end{align*} 
		where $n_i,m_i \ge 0$. Then,
		\[ \dim \Hom_G(V,W) = n_1m_1 + n_2m_2 + \cdots + n_rm_r. \]
	\end{fcor}

	\begin{fcor}
		Let $V$ be a $G$-module such that
		\[ V \cong n_1V_1 \oplus n_2V_2 \oplus \cdots \oplus n_rV_r, \]
		where $V_1,\ldots,V_r$ are irreducible $G$-modules, and $n_i > 0$ for each $i$. Then, the $(n_i,V_i)$ are determined by $V$ up to permutation and isomorphism. 
	\end{fcor}
	\begin{proof}
		This is immediate on noting that by the previous corollary, for any irreducible $W$, $W$ appears with multiplicity $n$ in a decomposition of $V$ iff $\dim \Hom_G(V,W) = n$.
	\end{proof}

	\begin{fdef}
		A $G$-module $V$ is \emph{multiplicity-free} iff for any irreducible $W$, $\dim \Hom_G(V,W) \in \{0,1\}$.
	\end{fdef}

	\begin{flem}
		\label{lem:multiplicity-free}
		Let $G$ act on a set $S$ and consider $\mathcal{A} = \Hom_G(S,S)$. Then, $\C[S]$ is multiplicity-free iff $\mathcal{A}$ is commutative.
	\end{flem}
	Suppose that $\C[S] \cong n_1 V_1 \oplus \cdots \oplus n_r V_r$. It is easy to see that
	\[ \mathcal{A} \cong \Hom_G(n_1 V_1, n_1 V_1) \oplus \cdots \oplus \Hom_G(n_r V_r, n_r V_r) \]
	is commutative iff each of the $r$ parts of the direct sum are commutative. The idea behind the proof is that each $\Hom_G(n_i V_i, n_i V_i)$ is essentially a $n_i \times n_i$ matrix, which is commutative iff $n_i = 1$.

	\begin{flem}
		Let $G$ act on sets $S,T$. Define the subspace of $\C[S]$
		\[ F(G,S) = \{v \in \C[S] : g \cdot v = v \text{ for all } g \in G \}. \]
		Similarly define $F(G,T)$. Suppose that $f : \C[S] \to \C[T]$ is $G$-linear. Then,
		\begin{enumerate}[label=(\alph*)]
			\item $f(F(G,S)) \subseteq F(G,T)$,
			\item if $f : \C[S] \to \C[T]$ is onto, so is $f : F(G,S) \to F(G,T)$, and
			\item if $f : \C[S] \to \C[T]$ is one-one, so is $f : F(G,S) \to F(G,T)$. 
		\end{enumerate}
	\end{flem}
	\begin{proof}
		$G$-linearity immediately implies the first part. For any $v \in f(F(G,S))$ and $g \in G$, we have $g \cdot f(v) = f(g \cdot v) = f(v)$, so $f(v) \in F(G,T)$. The third part is direct since the restriction of a one-one function is one-one.\\
		For ontoness, let $w$ be an arbitrary element in $F(G,T)$, and $v \in \C[S]$ such that $f(v) = w$. Then,
		\[ f\left(\frac{1}{|G|} \sum_{g \in G} g \cdot v\right) = \frac{1}{|G|} \sum_{g \in G} f(g \cdot v) = \frac{1}{|G|} \sum_{g \in G} g \cdot w = w \]
		and further, for any $h \in G$,
		\[ h \cdot \left(\frac{1}{|G|} \sum_{g \in G} g \cdot v\right) = \frac{1}{|G|} \sum_{g \in G} (hg) \cdot v = \frac{1}{|G|} \sum_{g \in G} g \cdot v \in F(G,S), \]
		completing the proof.
	\end{proof}

	A spiritual converse of the above is as follows.

	\begin{fcor}
		\label{cor:glinear-11}
		Using the notation of the above lemma, let $f : \C[S] \to \C[T]$ be $G$-linear. Suppose that for each $s \in S$, there exists a subgroup $G_s \subseteq G$ fixing $s$ such that $f : F(G_s,S) \to F(G_s,T)$ is one-one. Then, $f : \C[S] \to \C[T]$ is one-one.
	\end{fcor}
	\begin{proof}
		Suppose otherwise, and let $v = \sum_{r \in S} \alpha_r r$ be a nonzero vector in $\C[S]$ such that $f(v) = 0$. Suppose that $\alpha_s = 0$ for some $s \in S$. Let $v' = \frac{1}{|G|} \sum_{g \in G_S} g \cdot v$. As in the proof of the previous lemma, we have $f(v') = 0$, $v' \in F(G_S,S)$ and also $v' \ne 0$. This is a contradiction to the one-oneness of $f : F(G_s,S) \to F(G_s,T)$.
	\end{proof}

\section{The Delsarte bound}
\label{subsec:delsarte}

	\begin{fdef}
		A \emph{binary code} $C$ (of length $n$) is a non-empty proper subset of $B(n)$. Given $X,Y \in B(n)$, the \emph{Hamming distance} $d$ defined by $d(X,Y) = |X \triangle Y|$. The Hamming distance of a code $C$ is $d(C) = \min_{\substack{X,Y \in C \\ X \ne Y}} d(X,Y)$.
	\end{fdef}

	Codes are studied in great detail in coding theory, with the distance of a code being an indicator of how resistant it is to ``corruption''.

	\begin{fdef}
		Given $n,d$, $A(n,d)$ is the size of a largest binary code of length $n$ whose distance is at least $d$.
	\end{fdef}

	Given the previous paragraph, it should be of no surprise that $A(n,d)$ is of great interest to coding theorists. However, it turns out that computing it is \textsf{NP}-hard. We shall give an efficient algorithm to compute an upper bound on $A(n,d)$. While we do not provide any theoretical guarantee on how good this bound is, it turns out to be surprisingly effective in practice.\\

	Consider the graph $G$ on vertex set $B(n)$, where $X,Y$ are adjacent iff $d(X,Y) < d$. $A(n,d)$ is then precisely the size of a largest independent set on $G$. For $S \subseteq B(n)$ an independent set, let $\chi(S) \in \R^V$ be the indicator vector of $S$. Consider
	\[ M = \frac{1}{|S|} \chi(S) \chi(S)^\top. \]
	Then, $M$ is positive semidefinite, $M_{ij} = 0$ if $ij \not\in E$, $\Tr(M) = 1$, and $|S| = \sum_{i,j} M_{ij}$.

	\begin{fdef}[Semidefinite Program]
		Given matrices $C,X$, denote $\langle C,X\rangle = \sum_{i,j} C_{ij} X_{ij}$. A \emph{semidefinite program} is a program of the form
		\[
		\label{def:sdp}
		\begin{array}{ll@{}ll}
		\text{maximize}  & \displaystyle\langle C,X\rangle &\\
		\text{subject to} \displaystyle &X \pge 0  &\\
		& \displaystyle \langle A_i,X\rangle = b_i, & i \in [m]
		\end{array}
		\]
		where $X$ is a $n \times n$ matrix of variables $x_{ij}$, $A_i$ and $C$ are matrices (that are also part of the input of the program), and the $b_i$ are constants.
	\end{fdef}
	That is, a semidefinite program is just a linear program with an additional constraint that a matrix defined by the variables is positive semidefinite. It turns out that optima to semidefinite programs can be found in polynomial time (up to an error of $\epsilon$).

	Given the earlier discussion, it follows that the size of a largest independent set is bounded from above by the solution to the following semidefinite program.

	\begin{equation}
		\label{indep-sdp}
		\begin{array}{ll@{}ll}
		\text{maximize}  & \displaystyle  \langle J,M\rangle &\\
		\text{subject to} \displaystyle & M \pge 0, &\\
		& \displaystyle \Tr(M) = 1, & \\
		& M_{ij} = 0, & \qquad ij \in E.
		\end{array}
	\end{equation}
	However, note that for our graph $G$ on $B(n)$, this SDP is of exponential size in the input parameter $n$! The Delsarte bound takes advantage of the symmetries of the graph to bring this down to a \emph{linear} program whose size is polynomial in $n$.\\

	Recall the hyperoctahedral group $H_n$. For $\tau \in H_n$, let $\rho_\tau$ be the $B(n) \times B(n)$ permutation matrix that permutes vertices according to $\tau$. The key idea is that since $\tau$ is distance-preserving, if $C$ is a code with minimum distance at least $d$, so is $\tau(C)$. Therefore, for a given code $C$, instead of the $\chi(C) \chi(C)^\top$ we considered earlier, we shall instead look at
	\begin{equation}
		\label{eq:delsarte-key}
		M = \frac{1}{|C|} \sum_{\tau \in H_n} \rho_\tau \chi(C) \chi(C)^\top \rho_\tau^\top,
	\end{equation}
	which is positive semidefinite. Furthermore, since $M$ lives in a far lower-dimensional space than the $2^n \times 2^n$ space we had earlier. In fact, $M \in \Hom_{H_n}(B(n),B(n))$, so lives in only a $(n+1)$-dimensional space (recall that we had proved this back in \Cref{prop: ncube-adj-eigenvals})! Indeed, it is easy to show that for any $\sigma \in H_n$, $M$ commutes with the unitary matrix $P_\sigma$, since
	\begin{equation}
		\label{eq:1.4}
		P_\sigma M P_\sigma^\top = P_\sigma \left(\frac{1}{|C|} \sum_{\tau \in H_n} P_\tau \chi(C) \chi(C)^\top P_\tau^\top \right) P_\sigma^\top = \frac{1}{|C|} \sum_{\tau \in H_n} P_{\sigma \circ \tau} \chi(C) \chi(C)^\top P_{\sigma \circ \tau}^\top = M.
	\end{equation}
	Let $A_0,\ldots,A_n$ be the orbital basis of $\Hom_{H_n}(B(n),B(n))$, so any element in the commutant is of the form $\sum_{i=0}^n x_i A_i$. Let us next express the $x_i$ in terms of the code itself.

	\begin{fprop}
		\label{prop:delsarte-1}
		Let $\lambda_i$ be the number of pairs $(X,Y) \in C^2$ with $d(X,Y) = i$, and $\alpha_i = \lambda_i / |C| \binom{n}{i}$.  With $M$ defined as above,
		\[ M = n! (\alpha_0 A_0 + \alpha_1 A_1 + \cdots + \alpha_n A_n).  \]
	\end{fprop}
	\begin{proof}
		The number of $1$s in $A_i$ is $2^n \binom{n}{i}$. The number of $1$s in $\chi(C)\chi(C)^\top$ in the nonzero positions of $A_i$ is precisely $\lambda_i$. When we sum over the elements of $H_n$, this implies that the sum of elements of $M$ in the nonzero positions of $A_i$ is $2^n n! \lambda_i = 2^n n! \binom{n}{i} \alpha_i |C|$. Therefore, the $A_i$ term in $M$ has a coefficient of $(2^n n! \binom{n}{i} \alpha_i |C|)/(|C| 2^n \binom{n}{i}) = n! \alpha_i$, as desired.
	\end{proof}

	Therefore, the upper bound yielded by \cref{indep-sdp} is at most that of the following semidefinite program.
	\[
		\label{indep-sdp2}
		\begin{array}{ll@{}ll}
		\text{maximize}  & \sum_{i=0}^{n} \binom{n}{i} x_i  &\\
		\text{subject to} \displaystyle & x_i \ge 0 \quad \text{for all $i$,} &\\
		& x_0 = 1, x_1 = \cdots = x_{d-1} = 0, & \\
		& x_0 A_0 + x_1 A_1 + \cdots + x_n A_n \pge 0. &
		\end{array}
	\]

	However, the positive semidefiniteness constraint is still exponentially large! To get around this, recall that the $A_i$ have the same eigenspaces, and only $(n+1)$ distinct eigenvalues, so we can just manually check that all the eigenvalues of $\sum_{i=0}^n x_i A_i$ are non-negative. To do this, we must compute the eigenvalues of each $A_i$.\\
	Now, consider $\C^2$ with the basis $e_0 = \begin{pmatrix} 1 & 0 \end{pmatrix}$ and $e_1 = \begin{pmatrix} 0 & 1 \end{pmatrix}$. The matrix $\begin{pmatrix} 0 & 1 \\ 1 & 0 \end{pmatrix}$ has eigenvalues $1,-1$ with the respective eigenvectors being
	\[ u = \frac{e_0 + e_1}{\sqrt{2}} \text{ and } v = \frac{e_0 - e_1}{\sqrt{2}}. \]
	Now, consider the isomorphism $\C[B(n)] \to (\C^2)^{\otimes n}$ where each basis vector $X$ maps to $a_1 \otimes \cdots \otimes a_n$, with $a_i = e_1$ if $i \in X$ and $e_0$ otherwise.\\
	An alternate orthonormal basis of $\C[B(n)]$ is the set of $u_1 \otimes \cdots \otimes u_n$, where each $u_i$ is either $u$ or $v$.

	Now, consider the subspace $W_j$ spanned by all $u_1 \otimes \cdots \otimes u_n$, where exactly $j$ of the $u_i$ are $v$ (and the remaining are $u$). It may be checked that $W_j$ is an eigenspace of $A_i$, with the eigenvalue
	\[ \sum_{k=0}^{i} (-1)^k \binom{j}{k} \binom{n-j}{i-k}. \]
	In particular, the eigenvalues of $A = A_1$ are $n-2j$ with multiplicity $\dim W_j = \binom{n}{j}$. Therefore, an upper bound on $A(n,d)$ is given by the linear program
	\[
		\label{indep-sdp3}
		\begin{array}{ll@{}ll}
		\text{maximize}  & \sum_{i=0}^{n} \binom{n}{i} x_i  &\\
		\text{subject to} \displaystyle & x_i \ge 0 \quad \text{for all $i$,} &\\
		& x_0 = 1, x_1 = \cdots = x_{d-1} = 0, & \\
		& \displaystyle \sum_{i=0}^{n} x_i \left( \sum_{k=0}^{i} (-1)^k \binom{j}{k} \binom{n-j}{i-k} \right) \ge 0 & \qquad j \in [n].
		\end{array}
	\]

\clearpage
\section{The Schrijver bound}
\label{subsec:schrijver}

	The idea behind the Schrijver bound is that we split the sum in \cref{eq:delsarte-key} into two parts as 
	\[ |C| \cdot M = |\Pi| \cdot \frac{1}{|\Pi|} \sum_{\tau \in \Pi} \rho_\tau \chi(C) \chi(C)^\top \rho_\tau^\top + |H_n \setminus \Pi| \cdot \frac{1}{|H_n \setminus \Pi|} \sum_{\tau \in H_n \setminus \Pi} \rho_\tau \chi(C) \chi(C)^\top \rho_\tau^\top, \]
	where each of the two matrices live in a space of dimension polynomial in $n$. It is clear that the two are positive semidefinite.\\
	Here, $\Pi$ is defined as
	\[ \Pi = \{ \tau \in H_n : \tau(C) \ni \mathbf{0} \}. \]
	For $X \in B(n)$, consider
	\[ \Pi_X = \{ \tau \in H_n : \tau(X) = \mathbf{0} \}. \]
	Then, setting 
	\[ R_X = \frac{1}{|\Pi_X|} \sum_{\tau \in \Pi_X} \rho_{\tau} \chi(C) \chi(C)^\top \rho_\tau^\top, \]
	we have
	\[ R = \frac{1}{|\Pi|} \sum_{\tau \in \Pi} \rho_{\tau} \chi(C) \chi(C)^\top \rho_\tau^\top = \frac{1}{|C|} \sum_{X \in C} R_X. \]
	Set $\Pi' = H_n \setminus \Pi$. We similarly have
	\[ R' = \frac{1}{|\Pi'|} \sum_{\tau \in \Pi'} \rho_{\tau} \chi(C) \chi(C)^\top \rho_\tau^\top, \]
	so
	\[ |C|\cdot M = |\Pi| \cdot R + |\Pi'| \cdot R'. \]
	The space we shall consider is $\mathcal{A} = \Hom_{S_n}(B(n),B(n))$ -- recall from \Cref{ex:non-comm-alg} that this is a non-commutative $\binom{n+3}{3}$-dimensional $*$-algebra with basis $(M_{i,j,t})$. It is reasonably easy to show that $R,R' \in \mathcal{A}$ by a proof similar to \cref{eq:1.4}.

	\begin{fprop}
		Let $\lambda_{i,j,t}$ be the number of pairs $(X,Y,Z) \in C^3$ with $d(X,Y) = i, d(Y,Z) = j, d(Z,X) = i+j-2t$, and $\alpha_{i,j,t} = \lambda_i / |C| \binom{n}{i-t,t,j-t}$.  With $R,R'$ defined as above,
		\[ R = \sum_{i,j,t} \alpha_{i,j,t} M_{i,j,t}  \]
		and
		\[ R' = \frac{|C|}{2^n - |C|} \sum_{i,j,t} (\alpha_{i+j-2t,0,0} - \alpha_{i,j,t}) M_{i,j,t}. \]
	\end{fprop}
	\begin{proof}
		The sum of elements of $R_X$ in the nonzero positions of $M_{i,j,t}$ is precisely the number of $(Y,Z) \in C^2$ such that for some $\tau \in \Pi$, $|\tau(Y)| = i$, $|\tau(Z)| = j$, and $d(\tau(Y),\tau(Z)) = i+j-2t$, which is precisely the number of $(Y,Z) \in C^2$ such that $d(X,Y) = i$, $d(X,Z) = j$, and $d(Y,Z) = i+j-2t$. Summing over $X$ and dividing by $|C|$, this is exactly $\binom{n}{i-t,t,j-t} \alpha_{i,j,t}$. On the other hand, the sum of elements of $M_{i,j,t}$ is $\binom{n}{i-t,t,j-t}$. The first equation follows.\\
		Now, by \Cref{prop:delsarte-1}, we have
		\begin{align*}
			M &= n! \sum_{t=0}^{n} \alpha_t A_t \\
				&= n! \sum_{t=0}^{n} \alpha_{t,0,0} A_t \\
				&= n! \sum_{t=0}^{n} \alpha_{t,0,0} \sum_{i,j} M_{i,j,(i+j-t)/2} \\
				&= n! \sum_{i,j,t} \alpha_{i+j-2t,0,0} M_{i,j,t}.
		\end{align*}
		Therefore, using the expansion of $R$, we have
		\begin{align*}
			|\Pi| R + |\Pi'| R' &= |C| \cdot M \\
			n! |C| R + n! (2^n - |C|) R' &= n! |C| \sum_{i,j,t} \alpha_{i+j-2t,0,0} M_{i,j,t} \\
			R' &= \frac{|C|}{2^n - |C|} \sum_{i,j,t} (\alpha_{i+j-2t,0,0} - \alpha_{i,j,t}) M_{i,j,t}.
		\end{align*}
	\end{proof}

	Now, note that $|C| = \sum_{i=0}^{n} \binom{n}{i} \alpha_{i,0,0}$. So, the upper bound yielded by \cref{indep-sdp} is at most that by the following semidefinite program, where we have added a couple more constraints that may be proved using the definitions of $\alpha_{i,j,t}$.

	\begin{equation}
		\label{indep-sdp2}
		\begin{array}{ll@{}ll}
		\text{maximize}  & \sum_{i=0}^{n} \binom{n}{i} x_{i,j,t}  &\\
		\text{subject to} \displaystyle & x_{i,j,t} = 0 & \{i,j,i+j-2t\} \cap [d-1] \ne \emptyset, \\
		& \displaystyle x_{i,j,t} = x_{i',j',t'} & \text{$(i,j,i+j-2t)$ is a permutation of $(i',j',i'+j'-2t')$}, \\
		& \displaystyle 0 \le x_{i,j,t} \le x_{i,0,0} & \text{ for all $i,j,t$}, \\
		& \displaystyle x_{i,0,0} + x_{j,0,0} \le 1 + x_{i,j,t} & \text{ for all $i,j,t$}, \\
		& \displaystyle \sum_{i,j,t} x_{i,j,t} M_{i,j,t} \pge 0, & \\
		& \displaystyle \sum_{i,j,t} (x_{i+j-2t,0,0} - x_{i,j,t}) M_{i,j,t} \pge 0. &
		\end{array}
	\end{equation}

	To conclude, we must, as in the Delsarte bound, take advantage of symmetries to bring down the size of the PSD constraint. This is far more complicated here, however, since the algebra is non-commutative so we must deal with the \emph{block} diagonalization (recall the \nameref{theo:noncomm-spec-thm}). 

	\begin{ftheo}[Schrijver]
		\label{theo:schrijver}
		Let $\mathcal{A}_n = \Hom_{S_n}(B(n),B(n))$. Set $m = \lfloor n/2 \rfloor$, and $p_k = n-2k+1$ and $q_k = \binom{n}{k} - \binom{n}{k-1}$ for $k=0,1,\ldots,m$. Then, the following are true.
		\begin{enumerate}[label=(\alph*)]
			\item There exists a $B(n) \times S$ real unitary matrix $V$ (for some indexing set $S$ of size $2^n$) such that $V^\dagger \mathcal{A}_n V$ is equal to the set of all $S \times S$ $((p_0,q_0),(p_1,q_1),\ldots,(p_m,q_m))$-block-diagonal matrices.\\
			In particular, this implies that $p_0^2 + \cdots + p_m^2 = \dim \mathcal{A}_n = \binom{n+3}{3}$ and $p_0q_0 + \cdots + p_mq_m = 2^n$.
			\item ``Dropping'' the duplicated blocks in the above block-diagonalization, we get a PSDness-preserving $*$-algebra isomorphism
			\[ \Phi : \mathcal{A}_n \to \bigoplus_{k=0}^m \mathcal{M}_{p_k}(\C). \]
			\item Suppose that
			\[ \Phi\left( \sum_{r,s,t=0}^n x_{r,s,t} M_{r,s,t} \right) = (R_0,\ldots,R_m), \]
			where the rows and columns of $R_k \in \mathcal{M}_{p_k}(\C)$ are indexed by $k,k+1,\ldots,n-k$. Then, for $k \le i,j \le n-k$,
			\[ (R_k)_{ij} = \frac{1}{\sqrt{\binom{n-2k}{i-k}\binom{n-2k}{j-k}}} \sum_{u,t=0}^n (-1)^{u-t} \binom{u}{t} \binom{n-2k}{u-k} \binom{n-k-u}{i-u} \binom{n-k-u}{j-u} x_{i,j,t}. \]
		\end{enumerate}
	\end{ftheo}

	We shall spend the remainder of this section proving the above monster of a theorem.

	\begin{fdef}
		The \emph{up} linear operator $U : \C[B(n)] \to \C[B(n)]$ is defined by
		\[ X \mapsto \sum_{\substack{Y \supseteq X \\ |Y|=|X|+1}} Y. \]
		Similarly, the \emph{down} linear operator $D : \C[B(n)] \to \C[B(n)]$ is defined by
		\[ X \mapsto \sum_{\substack{Y \subset X \\ |Y|=|X|-1}} Y. \]
	\end{fdef}
	
	Despite the deceptive names, $U$ and $D$ are \emph{not} inverses of each other.

	\begin{flem}
		Let $k < n/2$ and consider the restriction $U : \C[B(n,k)] \to \C[B(n,k+1)]$ of the up operator. This map is one-one.
	\end{flem}
	\begin{proof}
		
	\end{proof}

	\begin{fdef}
		An element $v \in \C[B(n)]$ is said to be \emph{homogeneous} if $v \in \C[B(n,k)]$ for some $0 \le k \le n$. In this case, we say that the \emph{rank} of $v$ is $k$ and write $r(v) = k$.\\
		A \emph{symmetric Jordan chain} (SJC) is a sequence $(v_k,v_{k+1},\ldots,v_{n-k})$ of non-zero homogeneous elements of $\C[B(n)]$ such that $r(v_i) = i$ for $i = k,k+1,\ldots,n-k$, $U(v_i) = v_{i+1}$ for $i=k,k+1,\ldots,n-k-1$. and $U(v_{n-k}) = 0$.\\
		A \emph{symmetric Jordan basis} (SJB) of $\C[B(n)]$ is a basis of $\C[B(n)]$ consisting of a disjoint union of SJCs.
	\end{fdef}

	It is not difficult to see that in an SJB, the number of SJCs going from rank $k$ to $n-k$ is $\binom{n}{k}-\binom{n}{k-1}$ -- the chains starting at lower levels account for a $\binom{n}{k-1}$-dimensional subspace of $B(n,k) \subseteq B(n)$, so an appropriate number of SJCs have to start at this level.

	\begin{fex}
		An SJB of $\C[B(3)]$ consists of the chains $(\emptyset, \{1\}+\{2\}+\{3\}, 2(\{1,2\}+\{1,3\}+\{2,3\}), 6\{1,2,3\})$, $(2\{3\}-\{1\}-\{2\} , \{1,3\}+\{2,3\}-2\{1,2\})$, and $(\{2\}-\{1\} , \{2,3\} - \{1,3\})$.
	\end{fex}

	We endow $\C[B(n)]$ with the standard inner product defined by $\langle X,Y\rangle = \delta_{XY}$ for $X,Y \in B(n)$. The primary lemma in our proof will be the following.

	\begin{flem}
		\label{schrijver-lem}
		There exists an SJB $J(n)$ of $\C[B(n)]$ satisfying
		\begin{enumerate}[label=(\alph*)]
			\item The vectors in $J(n)$ are orthogonal with respect to the standard inner product $\langle \cdot,\cdot\rangle$.
			\item Let $0 \le k \le n/2$ and let $(v_k,\ldots,v_{n-k})$ be an SJC in $J(n)$ starting at rank $k$ and going to rank $n-k$. Then,
			\[ \frac{\|v_{i+1}\|}{\|v_i\|} = \sqrt{(i+1-k)(n-k-i)} \]
			for $k \le i \le n-k$.
		\end{enumerate}
	\end{flem}

	\begin{flem}
		\label{schrijver-lem1}
		For $0 \le k \le n$, set $m(k) = \min\{k,n-k\}$. For any $0 \le k \le n$, $\C[B(n,k)]$ can be decomposed into orthogonal mutually non-isomorphic irreducibles as $W_{k,0} \oplus W_{k,1} \oplus \cdots \oplus W_{k,m(k)}$, where $W_{k,r}$ is of dimension $\binom{n}{k} - \binom{n}{k-1}$. Furthermore, $W_{k,m(k)}$ and $W_{j,m(k)}$ are $S_n$-isomorphic for any $k \le j \le n-k$.
	\end{flem}

	Before proving this, let us first establish some consequences of this result. The proof of \Cref{theo:schrijver} just uses the change-of-basis matrix associated with the SJB $J(n)$. \\

	If we write the up operator $U$ with respect to the SJB $J(n)$, we get $q_k$ identical blocks of size $p_k \times p_k$. Each of these $p_k \times p_k$ blocks has a $1$ at the $ij$th entry if $j-i=1$ and $0$ elsewhere. It turns out that something similar is also true for the $M_{i,j,t}$, as we shall show in the proof of \nameref{theo:schrijver}.\\
	Suppose we normalize $J(n)$ to get an ortho\emph{normal} basis $J'(n)$ of $\C[B(n)]$. Let $(v_k,\ldots,v_{n-k})$ be an SJC in $J(n)$. For $i=k,\ldots,n-k$, set
	\[ v_i' = \frac{v_i}{\|v_i\|} \in J'(n) \]
	and
	\[ \alpha_i = \frac{\|v_{i+1}\|}{\|v_i\|} = \sqrt{(i+1-k)(n-k-i)}, \]
	with $\alpha_k = 0$. Then,
	\[ U(v_i') = \alpha_i v_{i+1}'. \]
	So, with respect to $J'(n)$, the matrix $U$ is again block-diagonal, with the block corresponding to $(v_k',\ldots,v_{n-k}')$ having $\alpha_i$ at the $ij$th block if $j-i=1$. \\
	Now, observe that with respect to the standard basis $B(n)$ of $\C[B(n)]$, the matrices for $U$ and $D$ are real and transposes of each other. Because $J'(n)$ is orthonormal, the corresponding matrices are adjoints even here! Therefore, $D(v_{i+1}') = \alpha_i v_i'$, and the subspace spanned by the normalized SJC $(v_k',\ldots,v_{n-k}')$ is closed under $D$.

	\begin{fprop}
		\label{prop:d-on-sjc}
		Let $(v_k,\ldots,v_{n-k})$ be an SJC in $J(n)$. Then, for $i=k,\ldots,n-k-1$, setting $\alpha_i = \|v_{i+1}\|/\|v_i\|$, $D(v_{i+1}) = \alpha_i^2 v_i$.
	\end{fprop}
	The proof is immediate from the previous discussion.

	\begin{proof}[Proof of \Cref{schrijver-lem1}]
		Recall that $\Hom_{S_n}(B(n,k),B(n,k))$ is commutative and has dimension $1+\{k,n-k\}$. It follows by \Cref{lem:multiplicity-free} that $\C[B(n,k)]$ is the direct sum of $1+\{k,n-k\}$ mutually non-isomorphic irreducibles. Further note that because the $S_n$ action on $B(n,k)$ results in a unitary representation of $\C[B(n,k)]$, these irreducibles can be taken to be orthogonal. \\
		For $0 \le k \le j \le n-k \le n$, it is easily checked that the $S_n$ action on $B(n,k) \times B(n,j)$ has $1+k$ irreducibles; the idea is the same as that in \Cref{ex:non-comm-alg}, where $(X,Y) \sim (X',Y')$ iff $|X \cap Y| = |X' \cap Y'|$. 
		This implies that every irreducible occurring in $\C[B(n,k)]$ also occurs in $\C[B(n,j)]$, and in particular, $\C[B(n,k)]$ and $\C[B(n,n-k)]$ are isomorphic as $S_n$-modules. The desideratum follows.
	\end{proof}

	For example, when $k=0$, we get only the trivial irreducible representation. When $k=1$, we get the trivial irreducible as well as that mentioned in \Cref{ex:sn-action-permutation-mod}, which is of dimension $n-1$. For $k=2$, we get these two irreducible and another irreducible, which is forced to have dimension $\binom{n}{2} - n$ (since we know the dimensions of the first two irreducibles).

	\begin{proof}[Proof of \Cref{schrijver-lem}]	
		Now, let us play with these irreducibles in order to get an SJB. If we manage to show that $U$ maps $W_{j,m(k)}$ to $W_{j+1,m(k)}$ bijectively for $k \le j < n-k$, we are done -- \nameref{schur's lemma} would imply that it then acts like some multiple of $\Id$, so if we take some orthogonal basis of $W_{k,m(k)}$, applying $U$ repeatedly maps this basis to an orthogonal basis of $W_{j,m(k)}$ for any $k \le j \le n-k$.\\
		Let us show that $U : B(n,j) \to B(n,j+1)$ is one-one for $j < n/2$. For each $X \in B(n,j)$, consider $G_X \subseteq S_n$ to be the set of all permutations that fix $X$, namely the composition of a permutation of $X$ and a permutation of $[n] \setminus X$. The only elements in $\C[B(n,j)]$ of size $j$ fixed by all such permutations (since $j < n/2$) are scalar multiples of $B(n,j)$. The desideratum follows on using \Cref{cor:glinear-11}.
		\textbf{***** INCOMPLETE *****}

		Let $(v_k,\ldots,v_{n-k})$ be an SJB in $J(n)$. To complete the proof, it remains to find $\alpha_i = \|v_{i+1}\|/\|v_i\|$. Consider the linear operator $H : \C[B(n)] \to \C[B(n)]$ defined by $X \mapsto (n-2|X|)X$. First off, observe that $UD-DU = H$. This can be proved easily using a combinatorial argument. For $X \in B(n)$ of size $k$, applying $UD$ gives $X$ back in $n-k$ ways (since we must choose one of the elements not in $X$ to add and subsequently remove), and applying $DU$ gives $X$ back in $k$ ways. Furthermore, the component for any other set $Y$ is $0$, since it can be arrived at in at most one way for $UD$ (or $DU$) -- removing the element in $X \setminus Y$ and adding the element in $Y \setminus X$.\\
		Now, recall from \Cref{prop:d-on-sjc} that $D(v_{i+1}) = \alpha_i^2 v_i$. We determine the value of $\alpha_i^2$ by induction on $i$. First off,
		\[ \alpha_k^2 v_k = D(v_{k+1}) = (DU)(v_k) = (UD-H)(v_k) = -H(v_k) = (n-2k)v_k, \]
		where we used the fact that $D(v_k) = 0$. Therefore, $\alpha_k^2 = n-2k$.\\
		For $k < i \le n-k$, we have
		\[ \alpha_i^2 v_i = (DU)(v_i) = (UD-H)(v_i) = U(\alpha_{i-1}^2 v_{i-1}) - H(v_i) = \alpha_{i-1}^2 v_i - (n-2i)v_i, \]
		and the claim follows by induction.
	\end{proof}

	\begin{lemma}[Binomial inversion]
		Let $a_0,\ldots,a_n$ and $b_0,\ldots,b_n$ be sequences. Then,
		\[ a_t = \sum_{u=0}^{n} \binom{u}{t} b_u \]
		for $t=0,\ldots,n$ iff
		\[ b_t = \sum_{u=0}^{n} (-1)^{u-t} \binom{u}{t} a_u \]
		for $t=0,\ldots,n$.
	\end{lemma}
	\begin{proof}
		Let $M$ be the $n \times n$ matrix with $tu$th entry equal to $\binom{u}{t}$, and $N$ the matrix with $tu$th entry equal to $(-1)^{u-t} \binom{u}{t}$. The question asks to show that $M = N^{-1}$. Consider the vector space spanned by $\{1,x,x^2,\ldots,x^n\}$. Another basis for this space is $\{1,(x-1),(x-1)^2,\ldots,(x-1)^n\}$. We have
		\[ x^u = \sum_{t=0}^{n} \binom{u}{t} (x-1)^t \]
		and
		\[ (x-1)^u = \sum_{t=0}^{n} (-1)^{n-t} \binom{u}{t} x^t. \]
		The desideratum follows since the two resulting change-of-basis matrices, equal to $M,N$, are inverses of each other.
	\end{proof}

	\begin{fprop}
		\label{binom-inv-2}
		It holds that
		\[ M_{i,j,t} = \sum_{u=0}^{n} (-1)^{u-t} \binom{u}{t} M_{i,u,u} M_{u,j,u}. \]
	\end{fprop}
	\begin{proof}
		Note that
		\[ M_{i,t,t} M_{t,j,t} = \sum_{u=0}^{n} \binom{u}{t} M_{i,j,t}. \]
		The $XY$th entry of the left is equal to the number of size $u$ sets $Z$ such that $Z \subseteq X,Y$, assuming $|X|=i$ and $|Y|=j$. If $X \cap Z = u$, this number is precisely $\binom{u}{t}$. This is exactly equal to the $XY$th entry of the right. \\
		To complete the proof, apply binomial inversion.
	\end{proof}

	\begin{proof}[Proof of \nameref{theo:schrijver}]
		For $i,j,k,t$, define
		\[ \beta_{i,j,k,t} = \sum_{u=0}^{n} (-1)^{u-t} \binom{u}{t} \binom{n-2k}{u-k} \binom{n-k-u}{i-u} \binom{n-k-u}{j-u}. \]
		For $0 \le k \le m$ and $k \le i,j \le n-k$, define $E_{i,j,k}$ to be the $p_k \times p_k$ matrix, with rows and columns indexed by $k,k+1,\ldots,n-k$, with the entry in row $i$ and column $j$ equal to $1$ and all other entries $0$.\\
		The block-diagonalizing unitary matrix $V$ is the change-of-basis matrix to the basis described by \Cref{schrijver-lem}. (a) follows near-immediately by \Cref{schrijver-lem1} and the proof of \Cref{schrijver-lem}, and (b) is immediate from (a).\\
		For (c), suppose that $x_{i,j,t} = 1$ and all others are $0$, so we have
		\[ \Phi(M_{i,j,t}) = (R_0,\ldots,R_m). \]
		We claim that for $0 \le k \le m$,
		\[ R_k = \begin{cases} \binom{n-2k}{i-k}^{-1/2} \binom{n-2k}{j-k}^{1/2} \beta_{i,j,k,t} E_{i,j,k}, & k \le i,j \le n-k, \\ 0, & \text{otherwise.} \end{cases} \]
		Now, the $S_n$-linear map $M_{i,j,t}$ maps homogeneous vectors at the $i$th level to some (possibly zero) vector at the $j$th level, and everything else to $0$. Since all the vectors in our basis $J(n)$ are homogeneous, all the irreducibles except those at the $i$th level are certainly mapped to $0$. In particular, the irreducible $W_{i,m(k)}$ must map to $W_{j,m(k)}$, and $W_{r,m(k)}$ maps to $0$ for any $r \ne i$.\\
		In more concrete terms, this implies that $R_k = 0$ if $i$ or $j$ is not in $k,k+1,\ldots,n-k$. So, suppose $k \le i,j \le n-k$. The above observation again implies that $R_k$ is some multiple of $E_{i,j,k}$.\\
		This is where \Cref{binom-inv-2} enters the picture. Consider the simpler case where $j=t=u$ with $i \ge u$, so
		\[ \Phi(M_{i,u,u}) = (A_0^u,\ldots,A_m^u). \]
		Again, $A_k^u$ is some multiple of $E_{i,u,k}$. Now, $M_{i,u,u}$ just takes a set $X \in B(n)$ of size $i$ to all subsets $Y \subseteq X$ of size $u$. Such a subset can be constructed by taking a ``path'' from $X$ down to $Y$, removing one element at a time. Each level of such a path is constructed precisely by $D$! Since each $Y$ is repeated by $(i-u)!$ paths, $M_{i,u,u}$ is just equal to $D^{i-u}/(i-u)!$. Recall \Cref{prop:d-on-sjc}. It follows that
		\begin{align*}
			(A_k^u)_{iu} &= \frac{1}{(i-u)!} \prod_{w=u}^{i-1} \sqrt{(w+1-k)(n-k-w)} \\
				&= \frac{1}{(i-u)!} \prod_{w=u}^{i-1} (n-k-w) \binom{n-2k}{w-k}^{1/2} \binom{n-2k}{w+1-k}^{-1/2} \\
				&= \binom{n-k-u}{i-u} \binom{n-2k}{u-k}^{1/2} \binom{n-2k}{i-k}^{-1/2}
		\end{align*}
		and therefore,
		\[ A_k^u = \begin{cases} \binom{n-k-u}{i-u} \binom{n-2k}{u-k}^{1/2} \binom{n-2k}{i-k}^{-1/2} E_{i,u,k}, & k \le u \le n-k, \\ 0, & \text{otherwise.} \end{cases} \]
		Similarly, if $\Phi(M_{u,j,u}) = (B_0^u,\ldots,B_m^u)$,
		\[ B_k^u = \begin{cases} \binom{n-k-u}{j-u} \binom{n-2k}{u-k}^{1/2} \binom{n-2k}{j-k}^{-1/2} E_{u,j,k}, & k \le u \le n-k, \\ 0, & \text{otherwise.} \end{cases} \]
		Therefore, using \Cref{binom-inv-2},
		\begin{align*}
			(R_k)_{ij} &= \sum_{u=k}^{n-k} (-1)^{u-t} \binom{u}{t} \sum_{\ell=k}^{n-k} (A_k^u)_{i\ell} (B_k^u)_{\ell j} \\
				&= \sum_{u=k}^{n-k} (-1)^{u-t} \binom{u}{t} (A_k^u)_{iu} (B_k^u)_{uj} \\
				&= \binom{n-2k}{i-k}^{-1/2} \binom{n-2k}{j-k}^{1/2} \beta_{i,j,k,t}
		\end{align*}
		as desired, proving the theorem. Here, for the final equality, we substituted the expressions for $A_k^u$ and $B_k^u$ as proved above.
	\end{proof}

\clearpage
\section{Johnson schemes}
\label{subsec:johnson-schemes}

	Following on from the previous section, let $0 \le k \le \lfloor n/2\rfloor$, and consider $\mathcal{A} = \Hom_{S_n}(B(n,k),B(n,k))$. Recall that unlike $\Hom_{S_n}(B(n),B(n))$, this $*$-algebra is commutative and of dimension $k+1$. The orbital basis of this algebra is $\{M_t\}_{0 \le t \le k}$, where $(M_t)_{XY} = 1$ if $|X \cap Y| = t$ and $0$ otherwise. Note that this is essentially just (an appropriate submatrix of) $M_{k,k,t}$. Let us determine the eigenvalues of the $M_t$ on each of its $k+1$ eigenspaces.\\
	We have already discovered these $k+1$ eigenspaces in the previous section! Indeed, we can decompose $\C[B(n,k)] = W_0 \oplus \cdots \oplus W_k$, where
	\[ W_j = \Span\{v \in J(n) : r(v)=k \text{ and the SJC on which $v$ lies starts at rank $j$} \}. \]
	As argued earlier, $\dim W_j = \binom{n}{j} - \binom{n}{j-1}$.\\
	We would like to determine the eigenvalues of $M_{k,k,t}$ on these eigenspaces.
	Recalling binomial inversion \Cref{binom-inv-2}, we have
	\[ M_{k,k,t} = \sum_{u=0}^{n} (-1)^{u-t} \binom{u}{t} M_{k,u,u} M_{u,k,u}. \]

	\textbf{***** INCOMPLETE *****}

	This implies that the eigenvalue of $M_t$ on $W_j$ is
	\[ \sum_{u=0}^n (-1)^{u-t} \binom{u}{t} \binom{k-j}{k-u} \binom{n-j-u}{k-u}. \]

	This has an interesting application in counting the number of spanning trees of the Johnson graph.

	\begin{definition}
		A \emph{rooted spanning tree} of a graph $G=(V,E)$ is a pair $(T,v)$, where $T$ is a spanning tree and $v$ is a vertex in the graph.
	\end{definition}
	Note that the number of rooted spanning trees of a graph is $|V|$ times the number of spanning trees of the graph.

	\begin{fdef}
		Given a graph $G=(V,E)$, the \emph{Laplacian} of $G$ is the $V \times V$ matrix $L = D-A$, where $D$ is the diagonal matrix with $D_{uu}$ equal to the degree of $u$, and $A$ is the adjacency matrix of the graph.
	\end{fdef}

	\begin{ftheo}[Matrix Tree Theorem]
		Let $G$ be a connected graph with Laplacian $\mathcal{L}$. Then, the number of rooted spanning trees of $G$ is equal to the product of nonzero eigenvalues of $\mathcal{L}$.
	\end{ftheo}
	It may also be shown that $0$ is an eigenvalue of $\mathcal{L}$ of multiplicity $1$. We do not prove the matrix tree theorem.

	\begin{fcor}
		The complete graph $K_n$ has $n^{n-2}$ spanning trees.
	\end{fcor}
	This is direct using the spanning tree theorem, and there are also several bijective proofs known -- see Section 1.2 of the author's \href{https://amitrajaraman.github.io/notes/ma-861/}{Combinatorics I notes} for more details.

	\begin{fcor}
		The number of spanning trees of the $n$-hypercube is $(1/n) \prod_{k=1}^{n} (2k)^{\binom{n}{k}}$.
	\end{fcor}
	We had studied the eigenvalues of the adjacency matrix of $n$-hypercube when studying the Delsarte bound. Since the graph is $n$-regular, this also gives the eigenvalues of the Laplacian.

	\begin{fcor}
		The number of spanning trees of the Johnson graph $J(n,k)$ for $k \le n/2$ is
		\[ \frac{1}{n} \prod_{j=1}^{k} (j(n-j+1))^{\binom{n}{j} - \binom{n}{j-1}}. \]
	\end{fcor}
	Again, we had studied the eigenvalues of the adjacency matrix of $J(n,k)$ earlier in this section (and indirectly in the Schrijver bound), and the graph is $k(n-k)$-regular.

	For the last two corollaries, no bijective proof is known.