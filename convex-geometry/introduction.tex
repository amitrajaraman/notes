\section{Introduction}

\begin{fdef}
A subset $S$ of a Euclidean space is said to be \textit{convex} if for any $u_1,\ldots,u_r\in S$ and non-negative $\lambda_1,\ldots,\lambda_r$ such that $\lambda_1+\cdots+\lambda_r=1$, the \textit{affine combination} $\sum_{i=1}^r \lambda_i u_i$ is in $S$ as well.
\end{fdef}

We primarily consider convex bodies, that is, compact and convex subsets of Euclidean spaces here. To put it more succinctly, a convex body is something that ``behaves a bit like a Euclidean ball".

A few simple examples of convex bodies on $\R^n$ are:
\begin{itemize}
    \item the cube $[-1,1]^n$. Here, the ratio of the radii of the circumscribed ball to the inscribed ball is $\sqrt n$, so it is not much like a Euclidean ball. We sometimes denote it by $B_\infty^n$ since it is the unit ball under the $\ell_\infty$ norm.
    \item the \textit{$n$-dimensional regular solid simplex} which is the convex hull of $n+1$ equally spaced points. Here, the ratio of the radii of the circumscribed ball to the inscribed ball is $n$. This ratio is ``maximal" in some sense.
    \item the $n$-dimensional ``octahedron" or \textit{cross-polytope} which is the convex hull of the $2n$ points $(\pm 1, 0,\ldots, 0)$, $(0,\pm 1, 0,\ldots, 0)$, $\ldots, (0,0,\ldots,0,\pm 1)$. Note that this is the unit ball on the $\ell_1$ norm on $\R^n$ so we denote it as $B_1^n$. Here, the ratio of the radii of the circumscribed ball to the inscribed ball is $\sqrt n$.
\end{itemize}

More generally, a $k$-simplex is a $k$-dimensional polytope (\Cref{def: polytope}) which is the convex hull of its $k+1$ vertices. 

\begin{definition}
A \textit{cone} in $\R^n$ is the convex hull of a single point and a convex body of dimension $n-1$. In $\R^n$, the volume of a cone of ``height" $h$ over a base of $(n-1)$-dimensional volume $B$ is $Bh/n$.
\end{definition}

Since $B_1^n$ is made up of $2^n$ pieces similar to the piece with non-negative coordinates, which is a cone of height $1$ with base analogous to the similar piece in $\R^{n-1}$, the volume of the non-negative section is $1/n!$. Therefore, the $\vol(B_1^n) = 2^n / n!$. \\

\subsection{The Euclidean Ball}
\label{section: The Euclidean Ball}

The fourth and final example is the Euclidean ball itself, namely
$$B_2^n = \left\{x\in \R^n: \sum_{i=1}^n x_i^2 \leq 1\right\}.$$

\subsubsection{Finding the Volume}

Let us now attempt to calculate $v_n = \vol(B_2^n)$. Note that we can easily get the ``surface area" of the ball from the volume by splitting it into ``thin" cones from $0$ and observing that the volume of each cone is equal to $1/n$ times its base area. Therefore, the surface area of the ball is $nv_n$. 

We perform integration in spherical polar coordinates using two variables - $r$, which denotes the distance from $0$ and $\theta$, which is a point on the unit ball that represents the direction of the point. We obviously have $x=r\theta$. The point $\theta$ carries the information of $n-1$ coordinates.\\
We can then write the integral of a general function on $\R^n$ by
\begin{equation}
\label{int of f over Rn}
    \int_{\R^n} f = \int_{r=0}^\infty \int_{S^{n-1}} f(r\theta) r^{n-1} \d\theta \d r
\end{equation}

Here, $\d\theta$ represents the area measure on the sphere. From our earlier observation, its total mass is $nv_n$. The $r^{n-1}$ factor appears because the sphere of radius $r$ has $r^{n-1}$ times that of the sphere of radius $1$.

An important thing to note about the measure corresponding to $\d\theta$ is that it is \textit{rotation-invariant}. If $A$ is a subset of the sphere and $U$ is orthogonal to $A$, then $UA$ has the same measure as $A$. Therefore, we often simplify integrals such as \ref{int of f over Rn} by pulling out the $nv_n$ factor to get

\begin{equation}
\label{find sphere volume}
    \int_{\R^n} f = nv_n \int_{r=0}^\infty \int_{S^{n-1}} f(r\theta) r^{n-1} \d\sigma_{n-1}(\theta) \d r
\end{equation}

where $\sigma_{n-1}$ is the rotation-invariant measure on $\R^{n-1}$ of total mass $1$. Now, to evaluate $v_n$, we choose a suitable $f$ such that the integrals on either side can easily be calculated, namely
\[f: x\mapsto \exp\left(-\frac{1}{2}\sum_{i=1}^n x_i^2\right).\]
Then the integral on the left of \ref{find sphere volume} is
\[
\int_{\R^n}f = \int_{\R^n}\prod_{i=1}^n \exp\left(-\frac{x_i^2}{2}\right) = \prod_{i=1}^n \int_{-\infty}^\infty \exp\left(-\frac{x_i^2}{2}\right) = \left(\sqrt{2\pi}\right)^n
\]
and the integral on the right is
\[
nv_n\int_0^\infty \int_{S^{n-1}} e^{-r^2/2}r^{n-1}\d\sigma_{n-1}\d r = nv_n\int_0^\infty e^{-r^2/2}r^{n-1}\d r = v_n 2^{n/2}\Gamma\left(\frac{n}{2}+1\right).
\]
Equating the two,
\[ v_n = \frac{\pi^{n/2}}{\Gamma\left(\frac{n}{2}+1\right)} \]
Using Stirling's Formula, we can approximate this slightly better as
\[ v_n \approx \dfrac{\pi^{n/2}}{\sqrt{2\pi} e^{-n/2}\left(\dfrac{n}{2}\right)^{(n+1)/2}} \approx \left(\frac{2\pi e}{n}\right)^{n/2}. \]
This is quite small for large $n$. The radius of a ball of volume $1$ would be approximately $\sqrt{n/2\pi e}$, which is very large!\\

\subsubsection{Some Surprising Results in Concentration}

This is possibly the first hint one should take that following your intuition is probably not a good idea when dealing with high dimensional spaces.\\

Let us now restrict ourselves to considering the ball of volume $1$.\\
What is the $(n-1)$-dimensional volume of a slice through the center of the ball? Since the slice is an $(n-1)$ dimensional ball, it is equal to
\[ v_{n-1}r^{n-1} = v_{n-1}\left(\frac{1}{v_n}\right)^{(n-1)/n}. \]
This is approximately equal to $\sqrt{e}$ (using Stirling's formula once again). More generally, the volume of the slice that is at distance $x$ from the center of the ball is equal to
\[ \sqrt{e}\left(\frac{\sqrt{r^2-x^2}}{r}\right)^{n-1} =
\sqrt{e}\left(1-\frac{x^2}{r^2}\right)^{(n-1)/2} \approx
\sqrt{e}\left(1-\frac{2\pi ex^2}{n}\right)^{(n-1)/2} \approx \sqrt{e}\exp(-\pi e x^2)\]
Note that this is normally distributed but the variance $1/2\pi e$ \textit{does not depend on $n$}! So despite the fact that the radius grows as $\sqrt{n}$, the distribution of the volume stays the same. For example, nearly all the volume (around 96\%) is concentrated in the slab with $\norm{x_1}\leq 1/2$.\\

This might lead us to believe that since the volume is concentrated around any such equator\footnote{since we could have equally well taken something other than $x_1$.} around a subspace, the volume should be concentrated around the intersection of all such equators, which seems to suggest that it should be concentrated around the center. However, for large $n$, we obviously know that most of the volume should be concentrated on the surface of the sphere\footnote{the volume of a ball of radius $dr$ ($d<1$) is $d^n\ll 1$ times that of a ball of radius $r$.}. These two points seem to be directly contradictory! However, as might be expected, this is once again because our intuition fails when dealing with high-dimensional spaces.\\
The measure of unit ball is ``concentrated" both near the surface \textit{and} around the equator, for any equator. To make more sense of this\footnote{The answers to \href{https://mathoverflow.net/questions/210291/how-to-explain-the-concentration-of-measure-phenomenon-intuitively}{this mathoverflow question} might further aid understanding}, while each $x_i$ is small, the overall distance from $0$ is quite large since the small individual coordinates are compensated by the large dimension $n$. The former leads to the point being close to the equator and the latter leads to the point being close to the surface of the ball.\\

Another fun\footnote{subject to debate} thing to think about is the following. Consider the cube $[-1,1]^n$. Construct a ball of radius $1/2$ at each of the $2^n$ vertices $(\pm 1,\ldots, \pm 1)$. Now, construct the ball with center $(\frac{1}{2},\ldots,\frac{1}{2})$ that touches each of these $2^n$ balls. Then note that for $n=4$, this ball touches (the center of each face of) the cube, and for $n\geq 5$, it actually goes \textit{outside} the cube!\\

To conclude, let us write the volume of a general convex body $K$ in spherical polar coordinates. Assume that $K$ has $0$ in its interior and for each direction $\theta\in S^{n-1}$, let $r(\theta)$ be the radius of $K$ (in that direction). Then,
\begin{equation}
\label{eqn: volume in terms of radial distance}
    \vol(K) = nv_n \int_{S^{n-1}}\int_0^{r(\theta)}s^{n-1}\d{s}\d\sigma = v_n\int_{S^{n-1}} r(\theta)^n\d\sigma.
\end{equation}

\begin{definition}
A convex body $K$ is said to be \textit{(centrally) symmetric} if $-x\in K$ whenever $x\in K$.
\end{definition}
Any symmetric body (other than the trivial $\{0\}$) is the unit ball under some $\norm{\cdot}_K$ on $\R^n$ (for example, the octahedron was the unit ball under the $\ell_1$ norm). For a general symmetric body $K$, the volume is given by
\begin{equation}
\label{eqn volume in terms of body's norm}
    \vol(K) = v_n\int_{S^{n-1}} \norm{\theta}_K^{-n} \d\sigma_{n-1}(\theta)
\end{equation}

\subsection{The Cube and other Polytopes}

So for example. since the volume of the cube $[-1,1]^n$ is $2^n$, we can use it to estimate the average radius of the cube as
\[
v_n\int_{S^{n-1}} r(\theta)^n = 2^n \implies \int_{S^{n-1}} r(\theta)^n \approx \left(\sqrt{\frac{{2n}}{\pi e}}\right)^n \\
\text{ so the average radius is approximately }\sqrt\frac{2n}{\pi e}
\]

That is, the volume of the cube is far more concentrated towards the corners (where the radius is closer to $\sqrt n$), rather than the middles of facets (where the radius is closer to $1$).

It can actually be shown\footnote{Consider the random variable $z_i=x_i^2$ where $x_i$ is drawn uniformly randomly from $[-1,1]$. Show that $\expec[z_i]=1/3$ and $\Var[z_i]=4/45$ and use the Chernoff bound to get a bound on $\Pr[\sum_i z_i\leq 1]$.} that the fraction of volume of the intersection of the cube and the ball is less than $\exp(-4n/45)$, which further emphasizes the point that nearly all the volume lies in the corners.

\begin{definition}
\label{def: polytope}
A body which is bounded by a finite number of flat facets is called a \textit{polytope}.
\end{definition}

A polytope is essentially the intersection of a finite number of half-spaces 
% (we shall later look at the Hahn-Banach separation theorem, which is a huge generalization of this to convex bodies in general).\\
Note that the cube is a polytope with $2n$ facets.\\

Earlier, we remarked that the cube is not much like a Euclidean ball. So a question that might come to mind is: If $K$ is a polytope with $m$ facets, how close can $K$ be to the Euclidean ball?

\subsubsection{Banach-Mazur Distance and Spherical Caps}

Let us define this ``closeness" more concretely.

\begin{fdef}[Banach-Mazur Distance]
The \textit{Banach-Mazur distance} $d(K,L)$ between symmetric convex bodies $K$ and $L$ is the least positive $d$ for which there is a linear image $\tilde L$ of $L$ such that $\tilde L\subseteq K\subseteq d\tilde L$.
\end{fdef}

Henceforth, we refer to the Banach-Mazur distance as just \textit{distance}.

This corresponds to the how we thought of inscribing/circumscribing a ball earlier, since the ratio of the two radii we considered is just this distance.\\
If we wanted to make this distance a metric, then we should consider $\log d$ instead of $d$ (the current distance is multiplicative and for any $K$, $d(K,K)=1$).\\

From what we mentioned earlier, we know that the distance between the cube and the Euclidean ball in $\R^n$ is at most $\sqrt{n}$. We shall prove later that it is indeed equal to $\sqrt n$.\\

As might be expected, if we want a polytope that approximates the ball very well, we would need a very large number of facets.

\begin{definition}
For a fixed unit vector $v$ and some $\varepsilon\in[0,1)$, the set
\[ C(\varepsilon,v) = \{\theta\in S^{n-1}: \langle\theta, v\rangle\geq\varepsilon\} \]
is called the \textit{$\varepsilon$-cap about $v$} or more generally, a \textit{spherical cap} (or just \textit{cap}).\\
It is often better to write a cap in terms of its radius rather than in terms of $\varepsilon$. The \textit{cap of radius $r$ about $v$} is
\[ \{\theta\in S^{n-1} : \norm{\theta-v}\leq r\} \]
\end{definition}

It is easy to see that a cap of radius $r$ is a $(1-\frac{r^2}{2})$-cap.

As we shall see in the proof of \Cref{approximating sphere to polytope}, it is useful to know some upper and lower bounds on the area of an $\varepsilon$-cap.

\begin{lemma}[Lower bound on the area of spherical caps]
\   label{spherical cap lower bound}
    For $r\in[0,2]$, a cap of radius $r$ on $S^{n-1}$ has measure (under $\sigma_{n-1}$) at least $\frac{1}{2}(r/2)^{n-1}$.
\end{lemma}
\begin{proof}
    Suppose $n\geq 2$ and let $\alpha =2\sin^{-1}(r/2)$. We can assume that $\alpha\in [0,\frac{\pi}{2}]$ since we can prove the other case similarly. Then the measure of the cap is given by
    \begin{align*}
        A(n,\alpha) &= \int_0^\alpha \frac{(n-1)v_{n-1}}{nv_n} (\sin\theta)^{n-2}\d\theta\\
            &= \frac{(n-1)\Gamma\left(\frac{n}{2}+1\right)}{n\Gamma\left(\frac{n-1}{2}+1\right)\sqrt{\pi}}\int_0^\alpha \sin^{n-2}(\theta)\d\theta \\
            &= \frac{\Gamma\left(\frac{n}{2}\right)}{\Gamma\left(\frac{n-1}{2}\right)\sqrt{\pi}}\int_0^\alpha \sin^{n-2}(\theta)\d\theta \\
            &\geq \frac{1}{\sqrt{\pi}}\int_0^\alpha \left(\frac{2\theta}{\pi}\right)^{n-2} \d\theta \\
            &= \frac{1}{\sqrt{\pi}}\cdot \left(\frac{2}{\pi}\right)^{n-2}\frac{\alpha^{n-1}}{n-1} \\
            &= \frac{4}{\sqrt{\pi}(n-1)} \left(\frac{4}{\pi}\right)^{n-2}\cdot \frac{1}{2}\left(\alpha/2\right)^{n-1} \\
            &\geq \frac{4}{\sqrt{\pi}(n-1)} \left(\frac{4}{\pi}\right)^{n-2}\cdot \frac{1}{2}(r/2)^{n-1} \\
            &= \frac{\sqrt{\pi}}{n-1} \left(\frac{4}{\pi}\right)^{n-1}\cdot \frac{1}{2}(r/2)^{n-1}
    \end{align*}
    It is easily shown that
    \[ \frac{\sqrt{\pi}}{n-1} \left(\frac{4}{\pi}\right)^{n-1} \geq 1\]
    for all $n\geq 2$, thus proving the inequality.
\end{proof}


\begin{lemma}[Upper bound on the area of spherical caps]
    \label{spherical cap upper bound}
    For $\varepsilon\in[0,1)$, the cap $C(\varepsilon,u)$ on $S^{n-1}$ has measure (under $\sigma_{n-1}$) at most $e^{-n\varepsilon^2/2}$.
\end{lemma}
\begin{proof}
    Let $\alpha=\cos^{-1}(\varepsilon)$. We may assume that $\alpha\in[0,\pi/2]$. Instead of finding the fraction of area of the spherical cap, we shall instead find the fraction of volume subtended at the center by the cap.\\
    When $\varepsilon\leq\frac{1}{\sqrt{2}}$, note that the entire volume is contained in the ball of radius $\sqrt{1-\varepsilon^2}$ centered at $\varepsilon u$. The fraction of volume of this ball is equal to $\left(\sqrt{1-\varepsilon^2}\right)^n\leq e^{-n\varepsilon^2/2}$.\\
    On the other hand, when $\varepsilon>\frac{1}{\sqrt{2}}$, this entire volume is contained in the ball of radius $\frac{1}{2\varepsilon}$ centered at $\frac{1}{2\varepsilon} u$. The fraction of volume of this ball is equal to $(2\varepsilon)^{-n} \leq e^{-n\varepsilon^2/2}$.
\end{proof}

\subsubsection{Bounds on Almost-Spherical Polytopes}

\begin{ftheo}
    \label{approximating sphere to polytope}
    Let $K$ be a symmetric polytope in $\R^n$ with $d(K,B_2^n)=d$. Then $K$ has at least $\exp(n/2d^2)$ facets. On the other hand, for each $n$, there is a polytope with $4^n$ facets whose distance from the ball is at most $2$.
\end{ftheo}
Before proving the above theorem, let us reformulate what a symmetric polytope is in another way. Suppose you have a symmetric polytope $K$ with $m$ pairs of facets. Then it is basically the intersection of $m$ slabs in $\R^n$ each of the form $\{x:|\langle x, v_i\rangle| \leq 1\}$ for some $v_i\in\R^n$. That is,
\begin{equation}
\label{eqn Kvecs}
    K = \{x : |\langle x, v_i\rangle|\leq 1 \text{ for }1\leq i\leq m\}
\end{equation}
We can then consider a linear map from $K\to \R^m$ given by
\[ T: x \mapsto (\langle x,v_1\rangle,\ldots,\langle x,v_m\rangle) \]
This maps $\R^n$ to a subspace of $\R^m$. By the formulation of $K$ given in \ref{eqn Kvecs} , the intersection of this subspace with the unit cube is just the image of $K$ under $T$! This is just an $n$-dimensional slice of $[-1,1]^m$. Even conversely, any $n$-dimensional slice of $[-1,1]^m$ is a convex body with at most $m$ pairs of facets.
\begin{proof}
For the proof, let us write what it means for each $v_i$ (following the above notation) if $B_2^n\subseteq K\subseteq dB_2^n$.
\begin{itemize}
    \item The first inclusion just says that each $v_i$ is of length at most $1$ (otherwise, one could consider $v_i/\norm{v_i}$, which would be in $B_2^n$ but not in $K$).
    \item The latter says that if $\norm{x}>d$, then there is some $i$ for which $\langle x,v_i\rangle > 1$. That is, for any unit vector $\theta$, there is some $i$ such that
\[ \langle \theta, v_i\rangle \geq \frac{1}{d}. \]
\end{itemize} 
Since we want to minimize $m$ while satisfying the above two conditions, we can clearly do no better than have $\norm{v_i}=1$ for each $i$. We want that every $\theta\in S^{n-1}$ is in one of the $m$ $(1/d)$-caps about the $(v_i)$.
\begin{itemize}
    \item Obviously, to do this, we should attempt to estimate the area of a general $\varepsilon$-cap ($\varepsilon=1/d$ here). Given \Cref{spherical cap upper bound}, we get that
    \[ m \geq \frac{1}{\exp(-n\varepsilon^2/2)} = \exp\left(\frac{n}{2d^2}\right). \]

    \item To show that there exists a polytope with the given number of facets, it is enough to find $2\cdot4^{n-1}$ points $v_1,\ldots,v_m$ such that the caps of radius $1$ centered at these points covers the sphere. Such a set is called a \textit{$1$-net}.\\
    Now, suppose we choose a set of points on the sphere such that any two of them are at least distance $1$ apart. Such a set is called a \textit{$1$-separated set}.
    
    Note that the caps of radius $1/2$ centered at each of the points in a $1$-separated set are disjoint. Since the measure of a cap of radius $1/2$ is at least $4^{-n}$ (by \Cref{spherical cap lower bound}), the number of points in a $1$-separated set is at most $4^{n}$.\\
    It is then enough to choose a ``maximal" $1$-separated set (a $1$-separated set $S$ such that $S\cup x$ is not $1$-separated for any $x\in S^{n-1}$) since it is then automatically a $1$-net!
    
    Therefore, there is a $1$-net (and thus a corresponding polytope) with at most $4^{n}$ points.
\end{itemize}
\end{proof}


\subsection{Fritz John's Theorem}

At the very beginning, we had mentioned that the distance of the cube $[-1,1]^n$ and the regular solid simplex are at distance at most $\sqrt{n}$ and $n$ from the ball respectively. However, how would one go about proving that the distances are \textit{exactly} $\sqrt{n}$ and $n$?\\

\subsubsection{The Statement of the Theorem}

Fritz John's Theorem aids us in this pursuit.\\
He considered ellipsoids inside convex bodies. If $(e_i)$ is an orthonormal basis of $\R^n$ and $(\alpha_i)$ are positive numbers, then the ellipsoid defined by
\[ \left\{x:\sum_{i=1}^n \frac{\langle x,e_j\rangle^2}{\alpha_j^2}\leq 1\right\} \]
has volume equal to $v_n\prod_i \alpha_i$. The theorem states that there is a \textit{unique} maximal ellipsoid contained in any convex body, and further, he characterized this ellipsoid! Also, if $K$ is a symmetric convex body and $\mathcal{E}$ is its maximal ellipsoid, then $K\subseteq\sqrt{n}\mathcal{E}$!\\
We can then use this characterization combined with an affine transformation to prove that the distance between the cube and the ball is $\sqrt{n}$.

We state John's Theorem after performing the affine transformation, since it is easier to understand what's going on then. Roughly, it says that there should be several points of contact between the ball and the boundary of $K$.

\begin{ftheo}[Fritz John's Theorem]
\label{fritz john's theorem}
Each convex body $K$ contains a unique ellipsoid of maximal volume. This ellipsoid is $B_2^n$ iff $B_2^n\subseteq K$ and for some $m$, there are unit vectors $(u_i)_1^m$ on the boundary of $K$ and positive numbers $(c_i)_1^m$ such that
\begin{equation}
\label{eqn fritz john lin dep}
    \sum_i c_i u_i = 0
\end{equation}
and for each $x\in\R^n$,
\begin{equation}
\label{eqn fritz john subspace contraction}
    \sum_i c_i\langle x, u_i\rangle^2 = \norm{x}^2.
\end{equation}
\end{ftheo}

\subsubsection{Some Consequences of Fritz John's Theorem}

Before proving the theorem, let us discuss some of its implications.\\

The first condition essentially says that the $(u_i)$ are not all on one side of the body\footnote{more than simple linear independence since the $c_i$ are positive.}. Intuitively, this makes sense because if the points were concentrated towards one side of the body, then we could move the ball a little bit in the opposite direction and then expand it a little to get a larger ellipsoid.\\
The second says that the $(u_i)$ are something like an orthonormal basis, in that we can resolve the norm as a weighted sum of squares of inner products.\\
\Cref{eqn fritz john subspace contraction} is equivalent to saying that for all $x\in\R^n$,
\[ x = \sum_i c_i \langle x,u_i\rangle u_i. \]
This ensures that the points do not lie close to a (proper) subspace of $\R^n$. This makes sense intuitively as well since if they did, we could contract the ellipsoid a bit in this direction and expand it orthogonally.\\

\Cref{eqn fritz john subspace contraction} is written more compactly as
\begin{equation}
\label{eqn orthogonal projection compact notation}
    \sum_i c_i u_i \otimes u_i = I_n.
\end{equation}
Here, $u_i\otimes u_i$ represents the (rank-$1$) orthogonal projection onto the span of $u_i$, the map given by $x\mapsto \langle x,u_i\rangle u_i$. Note that this map is just equal to $u_i u_i^\top$. This implies that the trace of this projection is equal to $\norm{u_i}^2 = 1$\footnote{We could have also got this more directly by using the fact that the trace of (a matrix in some basis corresponding to) a linear transformation is the sum of its eigenvalues. For an orthogonal projection, this is just equal to the rank of the target space (which is $1$ in this case).}. Equating the traces of either side of \Cref{eqn orthogonal projection compact notation}, we get
\begin{equation}
\label{eqn sum of fritz john constants is n}
    \sum_i c_i = n.
\end{equation}
Finally, note that if $K$ is a \textit{symmetric} convex body, then the first condition is obsolete since we can just find any $(u_i)$ satisfying the second condition and replacing each $u_i$ with $+u_i$ and $-u_i$ with each having half the original weight.\\

Let us now consider a couple of examples to better understand the implications of the theorem.

\begin{itemize}
    \item For the cube $[0,1]^n$, the maximal ellipsoid is $B_2^n$ as one would expect. The points of contact are the standard basis vectors $(e_i)_1^n$ of $\R^n$ with their negatives, and they do indeed satisfy
    \[ \sum_i e_i\otimes e_i = I_n. \]
    
    \item A slightly more nuanced example is that of the regular solid simplex. Unfortunately, there is no simple or standard way to represent the $n$-dimensional simplex in $n$ dimensions. It is, however, far more natural to represent it in $\R^{n+1}$ by considering the convex hull of the $n+1$ standard basis vectors $(e_i)_1^{n+1}$. We also scale it up by a factor of $\sqrt{n(n+1)}$ (so that the ball contained is $B_2^n$) such that the $n+1$ points $(p_i)$ we take the convex hull of to get the simplex are given by 
    \[ p_i = \sqrt{n(n+1)}e_i. \]
    This simplex can be parametrized as
    \[ K=\left\{x\in\R^{n+1}: \sum_{i=1}^{n+1} x_i = \sqrt{n(n+1)}\text{ and }x_i\geq 0\text{ for each }i\right\}. \]
    Similar to the cube, the contact points of the maximal ellipsoid are the centers of each of the facets. More precisely, these $n+1$ endpoints are given by
    \[ u_i = \frac{\sqrt{n(n+1)}}{n}\left(\sum_{j=1}^{n+1} e_j - e_i\right). \]
    Affinely shifting the hyperplane such that it passes through the origin (making $x_0=\frac{\sqrt{n(n+1)}}{n+1}\sum_i e_i$ the new origin) and
    setting the constants $c_i$ as $c=\frac{n}{n+1}$ for each $i$, for any $x$ in the (unshifted) body,
    \begin{align*}
        \sum_i n(n+1)c_i\langle x-x_0,u_i-x_0\rangle^2 &= c \sum_i \left( \sum_{j=1}^{n+1} \frac{x_j}{n} - \frac{x_i}{n} - \frac{2}{n+1} + \frac{1}{n+1} \right)^2 & \left(\langle x,x_0\rangle = \langle x_0,u_i\rangle = \langle x_0,x_0\rangle = \frac{1}{n+1}\right) \\
        &= n^2 \sum_i \left( \frac{x_i}{n} - \frac{1}{n(n+1)} \right)^2 \\
        &= \norm{x-x_0}^2.
    \end{align*}
    It is easily shown that $\sum_i c_i (u_i - x_0) = 0$ and that each $(u_i-x_0)$ is of unit norm, thus proving that the ball touching the centers of the facets (which is an affine shift of $B_2^n$) is the maximal ellipsoid inside the $n$-dimensional simplex.\\
\end{itemize}

Now, let us prove one of the claims that we made at the beginning of the section.

\begin{ftheo}
\label{fritz john banach mazur distance}
Suppose that $K$ is a symmetric convex body and $B_2^n$ is the maximal ellipsoid contained in $K$. Then $K\subseteq\sqrt{n} B_2^n$.\\
Suppose that $K$ is a convex body and $B_2^n$ is the maximal ellipsoid contained in $K$. Then $K\subseteq n B_2^n$.
\end{ftheo}
Note that while we have stated the above assuming that $B_2^n$ is the maximal ellipsoid, any convex body in general can be brought to this form by performing an affine shift.
\begin{proof}
\phantom{OWO}
\begin{itemize}
    \item Let $x$ be an arbitrary point in the symmetric body $K$. Our aim is to show that $\norm{x}\leq\sqrt{n}$. Let $(u_i)_1^m$ be the points as described in \nameref{fritz john's theorem}. We may assume that if $u$ is in this set, then so is $-u$.\\
    Now, note that for any $i$, the tangent plane to $K$ at $u_i$ must coincide with the tangent plane to $B_2^n$ at $u_i$ (otherwise, we would get a contradiction to $B_2^n\subseteq K$). Then, since $K$ is convex, any point in the body must be in the half-space defined by this tangent that contains $0$ -- this means that $\langle x,u_i\rangle \leq 1$ for each $i$.\\
    Then, for each $i$, we have $\langle x,u_i\rangle \leq 1$ \textit{and} $\langle x,-u_i\rangle\leq 1$ (since we've assumed that if $u$ is in the $(u_i)$, then so is $-u$). That is, $|\langle x,u_i\rangle|\leq 1$ for each $i$.\\
    Using the above along with \Cref{eqn fritz john subspace contraction} and \Cref{eqn sum of fritz john constants is n}, we now have
    \[ \norm{x}^2 = \sum_i c_i\langle x,u_i\rangle^2 \leq \sum_i c_i = n, \]
    which is exactly what we set out to prove!
    
    \item Let $x$ be an arbitary point in the convex body. From the first part, we already have that $\langle x,u_i\rangle \leq 1$ for each $i$. We also have $\langle x,u_i\rangle \geq -\norm{x}$ (since $\norm{u_i}=1$). Then,
    \begin{align*}
        0 &\leq \sum_i c_i\left(1 - \langle x,u_i\rangle\right)\left( \norm{x} + \langle x,u_i\rangle\right) \\
        \implies \sum_i c_i\langle x,u_i\rangle^2 &\leq \sum_i c_i \norm{x} + (1-\norm{x})\left\langle x,\sum_i c_i u_i \right\rangle \\
        \implies \sum_i c_i\langle x,u_i\rangle^2 &\leq \sum_i c_i \norm{x} & (\text{since }\sum_i c_i u_i = 0) \\
        \implies \norm{x} &\leq n. & (\text{by \Cref{eqn fritz john subspace contraction} and \Cref{eqn sum of fritz john constants is n}})
    \end{align*}
\end{itemize}
\end{proof}

Let us now prove Fritz John's Theorem.

\subsubsection{The Proof}

\begin{lemma}[Fritz John's Theorem Pt. 1]
\label{fritz johns theorem part 1}
Let $K$ be a convex body and for some integer $m$, let there be unit vectors $(u_i)_1^m$ in $\partial K$ and positive reals $(c_i)_1^m$ satisfying \Cref{eqn fritz john lin dep} and \Cref{eqn fritz john subspace contraction}. Then $B_2^n$ is the unique maximal ellipsoid contained in $K$.
\end{lemma}
\begin{proof}
Let
\[ \mathcal{E} = \left\{x\in\R^n:\sum_{i=1}^n \frac{\langle x,e_j\rangle^2}{\alpha_j^2}\leq 1\right\} \]
be an ellipsoid in $K$ for some orthonormal basis $(e_j)$ and positive $(\alpha_j)$. We must show that
\begin{itemize}
    \item $\prod_j \alpha_j\leq 1$ (this implies that $B_2^n$ is a maximal ellipsoid) and
    \item if $\prod_j \alpha_j = 1$, then for every $j$, $\alpha_j = 1$ (this implies that $B_2^n$ is \textit{the} maximal ellipsoid).
\end{itemize}
Now, consider the \textit{dual} of $\mathcal{E}$ given by
\[ \mathcal{E}^* = \left\{y\in\R^n:\sum_{i=1}^n \alpha_j^2\langle x,e_j\rangle^2 \leq 1\right\} \]
Observe that we can more concisely describe $\mathcal{E}^*$ as $\{y\in\R^n:\langle y,x\rangle\leq 1\text{ for all }x\in\mathcal{E}\}$ (Why?).\footnote{This becomes far easier to show when we apply a suitable linear transformation to reduce the problem to the case where $\mathcal{E}=B_2^n$.}\\
Now, note that since $\mathcal{E}\subseteq K$, for any $x\in \mathcal{E}$ and any $i$, $\langle x,u_i\rangle\leq 1$ (as proved in the first part of \Cref{fritz john banach mazur distance}). This implies that for every $i$, $u_i\in\mathcal{E}^*$! We then have
\begin{align*}
    \sum_j \alpha_j^2 &= \sum_j \alpha_j^2 \norm{e_j}^2 \\
    &= \sum_j \alpha_j^2 \sum_i c_i \langle u_i,e_j\rangle^2 \\
    &= \sum_i \left(c_i \sum_j \alpha_j^2\langle u_i,e_j\rangle^2 \right) \\
    &\leq \sum_i c_i = n & (\text{since }u_i\in\mathcal{E}^*)
\end{align*}
Then, using the AM-GM inequality,
\[ \prod_j \alpha_j \leq \left(\frac{1}{n}\sum_j \alpha_j^2\right)^{n/2} \leq 1. \]
This proves the first part. The second part follows directly as well, since if equality holds in the above equation, then every $\alpha_i^2$ must be the same (the condition for equality to hold in the AM-GM inequality).
\end{proof}

This is the easier of the two directions in Fritz John's Theorem. We now prove the harder.

\begin{lemma}[Separation Theorem]
\label{hyperplane separation theorem}
Let $X$ and $Y$ be two disjoint closed convex bodies in $\R^n$ with at least one of them bounded. Then there exists some $v\in\R^n$ such that for all $x\in X$, $\langle x,v\rangle<b$ and for all $y\in Y$, $\langle y,v\rangle>b$.
\end{lemma}
We leave the proof of the above to the reader.

\begin{lemma}[Fritz John's Theorem Pt. 2]
\label{fritz johns theorem part 2}
Let $K$ be a convex body such that $B_2^n$ is a maximal ellipsoid contained in $K$. Then, for some integer $m$, there exist unit vectors $(u_i)_1^m$ in $\partial K$ and positive reals $(c_i)_1^m$ satisfying \Cref{eqn fritz john lin dep} and \Cref{eqn orthogonal projection compact notation}.
\end{lemma}
\begin{proof}
We want to show that there exist unit vectors $(u_i)$ in $\partial{K}$ and positive constants $(c_i)$ such that
\[ \frac{1}{n}I_n = \sum_i \left(\frac{c_i}{n}\right) (u_i\otimes u_i) \]
Since $\sum_i c_i = n$, we essentially aim to show that $\frac{1}{n}I_n$ is in the convex hull of the $(u_i\otimes u_i)$ (in the space of matrices). To this end, define
\[ T = \conv\left(\{u\otimes u : u\text{ is a unit vector in }\partial K\}\right). \]
We refer to such $u$ as contact points. We want to show that $\frac{1}{n}I_n \in T$. Suppose that it is not (we shall finally show that this implies $B_2^n$ is not a maximal ellipsoid). Then \Cref{hyperplane separation theorem} implies that there exists a matrix $H=(h_{i,j})$ such that the linear map $\varphi$ from the set of matrices to $\R$ defined by
\[ (a_{i,j}) \mapsto \sum_{i,j} h_{i,j} a_{i,j}\]
satisfies
\[ \varphi\left(\frac{I_n}{n}\right) < \varphi(u\otimes u) \]
for all contact points $u$. Now, since the matrices on either side are symmetric, we may assume that $H$ is symmetric as well (Why?). And since the matrices on either side have trace equal to $1$, adding any constant to the diagonal elements of $H$ leaves the inequality unchanged. Therefore, we may suppose that the trace of $H$ is $0$. But this just says that $\varphi(I_n)=0$!\\
Therefore, we have essentially found a matrix $H$ such that for any contact point $u$,
\[ u^\top H u > 0. \quad (\text{check that }\varphi(u\otimes u) = u^\top H u) \]
Now, for $\delta>0$, consider the ellipsoid defined by
\[ \mathcal{E}_\delta = \left\{ x\in\R^n: x^\top(I_n + \delta H)x \leq 1 \right\}. \]
We claim that $\mathcal{E}_\delta$ is strictly inside $K$ for sufficiently small $\delta$. Note that for each contact point $u$,
\[ u^\top (I_n+\delta H) u = 1 + \delta\left(u^\top H u\right) > 1 \]
so no contact point (of $B_2^n$) is in $\mathcal{E}_\delta$. For each contact point $u$, consider a neighbourhood $n_u$ such that for all $x\in n_u$, $x^\top (I_n + \delta H) x > 0$ -- we know that such a neighbourhood exists due to the continuity of $x\mapsto x^\top (I_n + \delta H) x$. Let $N$ be the union of all these neighbourhoods.\\
We now want to show that for any $x\in \partial K\setminus N$, $x^\top (I_n + \delta H) x > 1$.
To this end, let $\lambda_{\min}$ be the minimum eigenvalue of $H$. For any $x\in\partial K\setminus N$, $x^\top H x \geq \lambda_{\min} \norm{x}^2$. That is, for all such $x$,
\[ x^\top (I_n + \delta H) x \geq (1+\delta\lambda_{\min})\norm{x}^2. \]
Observe that $\inf_{x\in\partial K\setminus N} \norm{x}^2 > 1$.\footnote{if it was equal to $1$, then for any $\varepsilon>0$, we would be able to find an $x$ such that $\norm{x}^2 < 1+\varepsilon$ (we trivially have that $\norm{x}^2 \geq 1$). However, this is not possible because $\partial K$ is compact, we have removed a neighbourhood around each contact point $u$, and contact points are the only points in $\partial K$ which have norm $1$.} We may also assume that $\lambda_{\min} < 0$, since the claim holds trivially otherwise (we have $\norm{x}^2 > 1$).  Then, we may set $\delta$ as a positive real which is less than $\frac{1}{|\lambda_{\min}|}\left(1 - \frac{1}{\inf_{x\in\partial K\setminus N}\norm{x}^2}\right)$. Then for all $x\in\partial K\setminus N$,
\[ (1+\delta\lambda_{\min})\norm{x}^2 > \norm{x}^2 \left(1 - \left(1 - \frac{1}{\inf_{y\in\partial K\setminus N}\norm{y}^2}\right)\right) \geq 1 \]
Therefore, $\mathcal{E}_\delta$ does not intersect $\partial K$ and is \textit{strictly} inside $K$ for sufficiently small $\delta$! \\

Now, we claim that $\mathcal{E}_\delta$ has volume at least equal to that of $B_2^n$. Indeed, its volume is given by $v_n / \prod\lambda_i$, where $(\lambda_i)$ are the eigenvalues of $(I_n+\delta H)$. Since the sum of the eigenvalues is equal to the trace of $I_n+\delta H$, which is $n$, we can use the AM-GM inequality to get
\[ \prod_i \lambda_i \leq \left(\frac{1}{n}\sum_i \lambda_i\right)^n = 1, \]
which is exactly what we want, because equality holds iff the eigenvalues are all $1$, that is, the ellipsoid is $B_2^n$ (so this leads to a contradiction).
\end{proof}

Note that we can concatenate the proofs of \Cref{fritz johns theorem part 2} and \Cref{fritz johns theorem part 1} to show that \textit{a} maximal ellipsoid is \textit{the} maximal ellipsoid (contained in a convex body).\\

There is an analogue of Fritz John's Theorem that characterizes the minimal ellipsoid that contains a given body -- this is near-direct from the notion of duality that we used in the proof of \Cref{fritz johns theorem part 1}. So for example, it follows from this analogue that the minimal ellipsoid that contains $[-1,1]^n$ is the ball of radius $\sqrt{n}$. This also enables us to say that $d([-1,1]^n, B_2^n)$ is \textit{exactly} equal to $\sqrt{n}$.\\

There are various extensions of this result. Recall how towards the beginning of these notes we had mentioned how a general convex body $K$ is essentially a unit ball under some norm. Fritz John's Theorem essentially describes linear maps from the Euclidean space to a normed space (under which the unit ball is $K$) that have largest determinant under the constraint that the Euclidean ball is mapped into $K$. There is a more general theory that (attempts to) solve this problem under different constraints.