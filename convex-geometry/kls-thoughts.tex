\documentclass{article}
\usepackage[utf8]{inputenc}
\usepackage{kls-thoughts}

\begin{document}

\thispagestyle{empty}
\titleBC

\setcounter{section}{-1}

\section{Notation}

\begin{itemize}
	\item We refer to measures by greek symbols such as $\mu$ and $\nu$ and their densities by lowercase alphabets beginning from $p$.
	\item $B$ refers to the Euclidean ball of radius $1$ in $\Rn$ (the value of $n$ is usually understandable from context).
	\item Given a measure $\mu$ on $\Rn$ and an $(n-1)$-dimensional surface(?) $S$ in $\Rn$, $\mu^+(S)$ refers to the ``surface area'' of the set $S$, that is,
	\[ \mu^+(X) = \lim_{\varepsilon \to 0} \frac{\mu(X + \varepsilon B )}{2\varepsilon}. \]
	Alternatively, if $X \subseteq \Rn$ is compact, then
	\[ \mu^+(\partial X) = \lim_{\varepsilon \to 0} \frac{\mu(X + \varepsilon B) - \mu(X)}{\varepsilon}. \]
	\item While needles in \cite{KLSConjecture} refer to one-dimensional segments with a polynomial factor ($\ell^{n-1}$ where $\ell$ is linear) in particular, we use them more generally to refer to log-concave measures with a one-dimensional support.
\end{itemize}

\section{Measure Disintegration}

	\subsection{Introduction}

		To begin with, let us describe a needle decomposition procedure given in \cite{leevempala2018klssurvey} to prove the KLS Conjecture. Suppose that we are given a log-concave measure $\mu$ with density $p$ with compact convex support $K$. Let us also fix a subset $E\subseteq K$ of measure $1/2$. We would like to bound $\mu^+(\partial E)$ below (over all such $E$).\\
		Now, suppose that we have some hyperplane $H$ that divides space into two half-spaces $H_1$ and $H_2$. Let $K_i = K \cap H_i$ and further assume that $\mu(E \cap H_i) = \frac{1}{2}\mu(K_i)$ for each $i$. Consider the measures $\mu_1$ and $\mu_2$ with densities given by
		\[ p_i(x) =
		\begin{cases}
			p(x) \frac{\mu(K)}{\mu(K_i)}, & x \in K_i, \\
			0, & \text{otherwise.}
		\end{cases} \]
		Observe that
		\begin{equation}
			\label{eqn 1}
			\begin{gathered}
				p = p_1 \frac{\mu(K_1)}{\mu(K)} + p_2 \frac{\mu(K_2)}{\mu(K)} \\
				\mu = \mu_1 \frac{\mu(K_1)}{\mu(K)} + \mu_2 \frac{\mu(K_2)}{\mu(K)}
			\end{gathered}
		\end{equation}
		More generally, suppose we have some space $\Omega$ with a probability measure $\nu$ on it such that
		\begin{equation}
			\label{eqn 2: disintegration}
			\mu = \int_\Omega \mu_{\omega} \d{\nu}(\omega),
		\end{equation}
		where the $(\mu_\omega)$ are log-concave measures on $\Rn$. In the above example, we can take $\Omega = \{1,2\}$ and $\nu(\{i\}) = \mu(K_i) / \mu(K)$ for $i\in\Omega$.\\
		Then, given any set $E$ of measure $1/2$, we have
		\begin{align}
			\mu^+(\partial E) &= \int_{\Omega} \mu_\omega^+(\partial E) \d{\nu}(\omega)  \nonumber \\
				&\geq \int_\Omega \psi_\omega \mu_\omega(E) (1-\mu_\omega(E)) \d{\nu}(\omega), \label{eqn 4}
		\end{align}
		where $\psi_\omega$ is the isoperimetric constant of $\mu_\omega$. If we manage to bound the expression on the right below by some constant independent of $E$, then the KLS conjecture follows. It is also worth noting that the decomposition we choose may be dependent on $E$ itself, we only require that the lower bound constant does not depend on this choice of $E$.

	\subsection{A proof of the \texorpdfstring{$n^{-1/2}$}{n-12} bound using needle decompositions}

		``Needle decomposition" refers to the process of performing the step we used to obtain \eqref{eqn 1} until the bodies $K_\omega$ become one-dimensional. We repeatedly split the bodies in a way that the quantity $\mu_\omega(E)$ remains constant at $1/2$. Suppose that we do so and the final limiting set of needles is $(K_\omega)_{\omega\in\Omega}$. Then, we can use one-dimensional isoperimetry to get that for any $\omega$, $\psi_\omega \gtrsim \opnorm{A_\omega}^{-1/2}$. We also have that $\mu_\omega(E) = 1/2$, so
		\begin{equation}
			\label{eqn 3}
			\mu^+(\partial E) \gtrsim \int_\Omega \opnorm{A_\omega}^{-1/2} \d{\nu}(\omega).
		\end{equation}
		We wish to bound the integral on the right below.\\

		To do so, consider \eqref{eqn 2: disintegration} (or rather, the similar expression for the density $p$). Then, we have that
		\[ \int_{\Rn} p(x) xx^\top \dif x = \int_\Omega \int_{\Rn} p_\omega(x) xx^\top \dif x \dif {\nu(\omega)}. \]
		Thus,
		\begin{equation}
			\label{eqn 5}
			A + bb^\top = \int_\Omega A_\omega + b_\omega b_\omega^\top \d{\nu(\omega)},
		\end{equation}
		where $A$ and $b$ (resp. $A_\omega$ and $b_\omega$) refer to the covariance matrix and barycenter of $\mu$ (resp. $\mu_\omega$) respectively.\\
		Assume without loss of generality that $b = 0$. Taking the trace on either side of the above expression,
		\begin{align*}
			\Tr(A) &= \int_\Omega \Tr(A_\omega) + \norm{b_\omega}^2 \d{\nu(\omega)} \\
				&\geq \int_\Omega \opnorm{A_\omega} \d{\nu(\omega)},
		\end{align*}
		where the inequality follows from the fact that $A_\omega$ is a covariance matrix so is positive semi-definite.
		One can then use H\"older's inequality to get
		\[ \left(\int_\Omega \opnorm{A_\omega} \d{\nu(\omega)} \right) \left( \int_\Omega \opnorm{A_\omega}^{-1/2} \d{\nu(\omega)} \right)^2 \geq 1 \]
		and so,
		\[ \int_\Omega \opnorm{A_\omega}^{-1/2} \d{\nu(\omega)} \gtrsim \Tr(A)^{-1/2}.  \]
		Substituting this back in \eqref{eqn 3}, we get $\psi_p \gtrsim \Tr(A)^{-1/2}$, that is, $\psi_n \gtrsim n^{-1/2}$.

	\subsection{An alternate way to look at stochastic localization}

		Let us return to \eqref{eqn 4}. In the above method of needle decomposition, we attempted to exercise control over the quantity $\mu_\omega(E) ( 1 - \mu_\omega(E) )$ for all $\omega$ by fixing $\mu_\omega(E)$ at $1/2$.\\

		How does stochastic localization fit into this? Instead of controlling $\mu_\omega(E)$, we try to control $\psi_\omega$ by defining a martingale $(p_t)$ whose isoperimetric constant is easily bounded. That is, $\expec[p_t] = p$ (this is just an integral of the form of $\eqref{eqn 1}$) and further, the isoperimetric constant of $\mu_t$ is lower bounded by $t^{1/2}$. Then, the problem comes down to estimating
		\[ \int_\Omega \mu_t(E) ( 1 - \mu_t(E) ) \d{\nu(\omega)}, \]
	which is exactly what papers such as \cite{chen2021constant} do.

	\subsection{What next?}

		Going back to needle decompositions again, we wish to show that there exists a needle decomposition conserving $\mu_\omega(E) = 1/2$ such that
		\[ \int_{\Omega} \frac{1}{\opnorm{A_\omega}^{1/2}} \d{\nu(\omega)} \gtrsim \opnorm{A}^{-1/2}. \]
		\eqref{eqn 5} for $b = 0$ gives
		\[ \opnorm{A} = \opnorm{\int_{\Omega} A_\omega + b_\omega b_\omega^\top \d{\nu(\omega)} }. \]
		Therefore, it would suffice to show that
		\[ \int_\Omega \frac{1}{\opnorm{A_\omega}^{1/2}} \d{\nu(\omega)} \gtrsim \opnorm{\int_{\Omega} A_\omega + b_\omega b_\omega^\top \d{\nu(\omega)} }^{-1/2} \]
		for some needle decomposition that conserves $\mu_\omega(E)$.\footnote{Is this inequality equivalent to the KLS Conjecture? Do there exist needle decompositions not obtained by the bisection method that conserve $\mu_\omega(E)$ and satisfy the above inequality?}\\

		Using H\"{o}lder's inequality as we did in the proof of the $n^{-1/2}$ bound, it is seen that it suffices to show
		\[ \int_\Omega \opnorm{A_\omega} \d{\nu(\omega)} \lesssim \opnorm{ \int_\Omega A_\omega + b_\omega b_\omega^\top \d{\nu(\omega)} } \]
		for some needle decomposition preserving $\mu_\omega(E)$ (it would in fact be enough to show this with some set $A\subseteq \Omega$ instead of $\Omega$ such that $\nu(A)$ is lower-bounded by a constant).\\
		Neglecting the $b_\omega b_\omega^\top$ term, it suffices to show that
		\begin{equation}
			\label{eqn: needle nearly aligned}
			\int_\Omega \opnorm{A_\omega} \d{\nu(\omega)} \lesssim \opnorm{ \int_\Omega A_\omega \d{\nu(\omega)} }.
		\end{equation}
		The above inequality essentially asks if there exists a needle decomposition where the needles are ``nearly aligned''. Indeed, if the segments of the needles are perfectly aligned, then equality holds above. We are allowing a constant factor of leeway. If the direction of the one-dimensional body $K_\omega$ is $u_\omega$, then the above is equivalent to
		\begin{equation}
			\sup_{\norm{\zeta} \leq 1} \int_\Omega \Var_{x \sim p_\omega}(x) \langle \zeta , u_\omega \rangle^2 \dif \nu(\omega) \gtrsim \int_\Omega \Var_{x \sim p_\omega}(x) \dif \nu(\omega)	
		\end{equation}

\section{More on decompositions}

	\subsection{Hyperplane bisections}

		As before, suppose we have a log-concave probability measure $\mu$ with density $p$ on the body $K$, and we fix some $E \subseteq K$ with $\mu(E) = 1/2$. Let us define the function $f_{E,K} : \Rn \setminus \{0\} \to \R$ by
		\[ f_{E,K}(x) = \left| \int_{\{ z \in \Rn : \langle z , x \rangle \geq \norm{x}^2 \}} p(y) ( \indic_E - \indic_{\Rn \setminus E} ) \dif y \right|. \]
		That is, if $H_x$ is the hyperplane defined by $x$ (orthogonal to $x$ and passing through it) and $H_x^+$ is either of the resulting halfspaces, the value of the above function at $x$ is equal to $|\mu(E \cap H_x^+) - \mu((\Rn\setminus E) \cap H_x^+)|$.\\
		This serves as a measure of how ``imbalanced'' the hyperplane corresponding to $x$ is -- $f_{E,K}(x) = 0$ iff the hyperplane corresponds to $x$ is a bisecting hyperplane (where bisecting means that $\mu(E \cap K_\omega) = \frac{1}{2} \mu(K_\omega) $, as in needle decompositions).\\
		For nice(?) $E$, $f_{E,K}$ is continuous.\\

		The primary tool used in \cite{lov-sim-on7} to prove the localization lemma was that there exists a bisecting hyperplane passing through any $(n-2)$-dimensional affine space. How would this translate in terms of the above defined function?\\
		Suppose we have an $(n-2)$-dimensional affine space orthogonal to the subspace spanned by $y,z\in\Rn$ and passing through $y$.\\
		Suppose that $x$ defines a hyperplane containing this affine space. Then $x$ is orthogonal to the plane, and so orthogonal to the space itself. That is, it must lie in the subspace spanned by $y, z$. Further, $y-x$ is orthogonal to $x$. That is, the set of all these $x$ forms a circle passing through $0$ contained in the $2$-dimensional subspace spanned by $y,z$.\\
		The conclusion of the localization method is that for any circle $S$ passing through $0$, either
		\begin{itemize}
			\item $f(w) = 0$ for some $w \in S \setminus \{0\}$ or
			\item The limit of $f(w)$ as $w$ goes to $0$ along the circle is equal to $0$ -- this corresponds to a bisecting hyperplane passing through the origin itself. It is not too difficult to check that this is well-defined and that the directional limit along either direction of the circle is the same. 
		\end{itemize}

		More generally, suppose we have some smooth curve $C$ in $\Rn$ that passes through the origin. Then, as before, either $f(w) = 0$ for some $w \in C \setminus \{0\}$ or one of the directional limits at $0$ (along $C$) is equal to $0$.\\

		A consequence of this nice property is that given any non-zero $x \in \Rn$, there exists a ``barrier'' curve $C$ in $\Rn$ separating $x$ from $0$ such that $f(y) = 0$ for any $y \in C$.

		An interesting question is to generally characterize these functions.

	\subsection{Perfectly aligned bisections need not be possible in \texorpdfstring{$(n-1)$}{(n-1)}-dimensional bodies}

		It is not too difficult to see that given an $n$-dimensional body $K$ ($n \geq 2$) and a direction $u$, it is possible to split it into $\{K_\omega\}_{\omega \in \Omega}$ such that each of the $K_\omega$ is $(n-1)$-dimensional and further, each of them contains $u$ (meaning that a shift of $\Span\{u\}$ is contained in $K_\omega$) -- this can be done by considering the set of $(n-2)$-dimensional affine spaces that contain $u$ and using an argument similar to that involved in the proof of the localization lemma.\\
		In terms of our function $f_{E,K}$, if we have $u = e_1$, we are essentially just considering
		\[ S = \{ x \in \Q^n : \langle x , u \rangle = 0 \} \]
		and arguing that given a $n$-dimensional body $K$, there exists some $x \in S$ such that $f_{E,K}(x) = 0$ and $H_x$ intersects $K$.

		However, we need not be able to go from $(n-1)$ dimensions to lower while maintaining that all the bodies contain $u$. There are simple counterexamples to show this. Our main tool so far has been to consider an appropriate curve and saying that $f_{E,K} = 0$ somewhere on the curve. However, it may be shown that any curve which would give something meaningful (the hyperplanes corresponding to the points on the curve must intersect the body) while maintaining alignedness of the bodies must pass through the projection of the origin on the body. However the value of $f_{E,K}$ at this point is $0$ anyway (the body would be contained in the corresponding hyperplane), so the method as a whole is useless.

		A natural next question is: can we give up perfect alignedness in exchange for near alignedness, which is all we would really need to show KLS?

	\subsection{A potential function}

		Let us fix $\mu$, $p$, $K$, and $E$ as usual. Also suppose we have some direction $u$. We wish to decompose the body into needles in a way that all of them are nearly in the direction of $u$. Equivalently, the hyperplanes chosen for bisection should all nearly contain $u$. That is, the set of $x$ corresponding to the hyperplanes $\{H_x\}$ must all be nearly orthogonal to $u$. So, at each step, the $x$ chosen must be such that $\langle x , u \rangle$ is small -- more precisely, $1 - \langle x , u \rangle^2 \gtrsim 1$.\\
		Also, as seen from \Cref{eqn 4}, all we really want is that $\mu_\omega(E)(1-\mu_\omega(E)) \gtrsim 1$, it might be fine to instead just minimize $f_{E,K}$ instead of ensuring that it is exactly equal to $0$. So, one may choose the $x$ corresponding to the bisecting hyperplane at each step by constructing a potential function such as
		\[ \Phi(x) = \left(1 + f_{E,K}(x)\right) \left(1 + \frac{|\langle x , u \rangle|}{\norm{x}}\right) \]
		and at each step, choosing the $x$ that minimizes $\Phi$.

\bibliographystyle{alpha}
\bibliography{references}

\end{document}