\documentclass{article}
\usepackage[utf8]{inputenc}
\usepackage{kls-thoughts}

\begin{document}

\thispagestyle{empty}
\titleBC

\setcounter{section}{-1}

\section{Notation}

\begin{itemize}
	\item We refer to measures by greek symbols such as $\mu$ and $\nu$ and their densities by lowercase alphabets beginning from $p$.
	\item $B$ refers to the Euclidean ball of radius $1$ in $\Rn$ (the value of $n$ is usually understandable from context).
	\item Given a measure $\mu$ on $\Rn$ and an $(n-1)$-dimensional surface(?) $S$ in $\Rn$, $\mu^+(S)$ refers to the ``surface area'' of the set $S$, that is,
	\[ \mu^+(X) = \lim_{\varepsilon \to 0} \frac{\mu(X + \varepsilon B )}{2\varepsilon}. \]
	Alternatively, if $X \subseteq \Rn$ is compact, then
	\[ \mu^+(\partial X) = \lim_{\varepsilon \to 0} \frac{\mu(X + \varepsilon B) - \mu(X)}{\varepsilon}. \]
	\item While needles in \cite{KLSConjecture} refer to one-dimensional segments with a polynomial factor ($\ell^{n-1}$ where $\ell$ is linear) in particular, we use them more generally to refer to log-concave measures with a one-dimensional support.
\end{itemize}

\section{Measure Disintegration}

	\subsection{Introduction}

		To begin with, let us describe a needle decomposition procedure given in \cite{leevempala2018klssurvey} to prove the KLS Conjecture. Suppose that we are given a log-concave measure $\mu$ with density $p$ with compact convex support $K$. Let us also fix a subset $E\subseteq K$ of measure $1/2$. We would like to bound $\mu^+(\partial E)$ below (over all such $E$).\\
		Now, suppose that we have some hyperplane $H$ that divides space into two half-spaces $H_1$ and $H_2$. Let $K_i = K \cap H_i$ and further assume that $\mu(E \cap H_i) = \frac{1}{2}\mu(K_i)$ for each $i$. Consider the measures $\mu_1$ and $\mu_2$ with densities given by
		\[ p_i(x) =
		\begin{cases}
			p(x) \frac{\mu(K)}{\mu(K_i)}, & x \in K_i, \\
			0, & \text{otherwise.}
		\end{cases} \]
		Observe that
		\begin{equation}
			\label{eqn 1}
			\begin{gathered}
				p = p_1 \frac{\mu(K_1)}{\mu(K)} + p_2 \frac{\mu(K_2)}{\mu(K)} \\
				\mu = \mu_1 \frac{\mu(K_1)}{\mu(K)} + \mu_2 \frac{\mu(K_2)}{\mu(K)}
			\end{gathered}
		\end{equation}
		More generally, suppose we have some space $\Omega$ with a probability measure $\nu$ on it such that
		\begin{equation}
			\label{eqn 2: disintegration}
			\mu = \int_\Omega \mu_{\omega} \d{\nu}(\omega),
		\end{equation}
		where the $(\mu_\omega)$ are log-concave measures on $\Rn$. In the above example, we can take $\Omega = \{1,2\}$ and $\nu(\{i\}) = \mu(K_i) / \mu(K)$ for $i\in\Omega$.\\
		Then, given any set $E$ of measure $1/2$, we have
		\begin{align}
			\mu^+(\partial E) &= \int_{\Omega} \mu_\omega^+(\partial E) \d{\nu}(\omega)  \nonumber \\
				&\geq \int_\Omega \psi_\omega \mu_\omega(E) (1-\mu_\omega(E)) \d{\nu}(\omega), \label{eqn 4}
		\end{align}
		where $\psi_\omega$ is the isoperimetric constant of $\mu_\omega$. If we manage to bound the expression on the right below by some constant independent of $E$, then the KLS conjecture follows. It is also worth noting that the decomposition we choose may be dependent on $E$ itself, we only require that the lower bound constant does not depend on this choice of $E$.

	\subsection{A proof of the \texorpdfstring{$n^{-1/2}$}{n-12} bound using needle decompositions}

		``Needle decomposition" refers to the process of performing the step we used to obtain \eqref{eqn 1} until the bodies $K_\omega$ become one-dimensional. We repeatedly split the bodies in a way that the quantity $\mu_\omega(E)$ remains constant at $1/2$. Suppose that we do so and the final limiting set of needles is $(K_\omega)_{\omega\in\Omega}$. Then, we can use one-dimensional isoperimetry to get that for any $\omega$, $\psi_\omega \gtrsim \opnorm{A_\omega}^{-1/2}$. We also have that $\mu_\omega(E) = 1/2$, so
		\begin{equation}
			\label{eqn 3}
			\mu^+(\partial E) \gtrsim \int_\Omega \opnorm{A_\omega}^{-1/2} \d{\nu}(\omega).
		\end{equation}
		We wish to bound the integral on the right below.\\

		To do so, consider \eqref{eqn 2: disintegration} (or rather, the similar expression for the density $p$). Then, we have that
		\[ \int_{\Rn} p(x) xx^\top \dif x = \int_\Omega \int_{\Rn} p_\omega(x) xx^\top \dif x \dif {\nu(\omega)}. \]
		Thus,
		\begin{equation}
			\label{eqn 5}
			A + bb^\top = \int_\Omega A_\omega + b_\omega b_\omega^\top \d{\nu(\omega)},
		\end{equation}
		where $A$ and $b$ (resp. $A_\omega$ and $b_\omega$) refer to the covariance matrix and barycenter of $\mu$ (resp. $\mu_\omega$) respectively.\\
		Assume without loss of generality that $b = 0$. Taking the trace on either side of the above expression,
		\begin{align*}
			\Tr(A) &= \int_\Omega \Tr(A_\omega) + \norm{b_\omega}^2 \d{\nu(\omega)} \\
				&\geq \int_\Omega \opnorm{A_\omega} \d{\nu(\omega)},
		\end{align*}
		where the inequality follows from the fact that $A_\omega$ is a covariance matrix so is positive semi-definite.
		One can then use H\"older's inequality to get
		\[ \left(\int_\Omega \opnorm{A_\omega} \d{\nu(\omega)} \right) \left( \int_\Omega \opnorm{A_\omega}^{-1/2} \d{\nu(\omega)} \right)^2 \geq 1 \]
		and so,
		\[ \int_\Omega \opnorm{A_\omega}^{-1/2} \d{\nu(\omega)} \gtrsim \Tr(A)^{-1/2}.  \]
		Substituting this back in \eqref{eqn 3}, we get $\psi_p \gtrsim \Tr(A)^{-1/2}$, that is, $\psi_n \gtrsim n^{-1/2}$.

	\subsection{An alternate way to look at stochastic localization}

		Let us return to \eqref{eqn 4}. In the above method of needle decomposition, we attempted to exercise control over the quantity $\mu_\omega(E) ( 1 - \mu_\omega(E) )$ for all $\omega$ by fixing $\mu_\omega(E)$ at $1/2$.\\

		How does stochastic localization fit into this? Instead of controlling $\mu_\omega(E)$, we try to control $\psi_\omega$ by defining a martingale $(p_t)$ whose isoperimetric constant is easily bounded. That is, $\expec[p_t] = p$ (this is just an integral of the form of $\eqref{eqn 1}$) and further, the isoperimetric constant of $\mu_t$ is lower bounded by $t^{1/2}$. Then, the problem comes down to estimating
		\[ \int_\Omega \mu_t(E) ( 1 - \mu_t(E) ) \d{\nu(\omega)}, \]
		which is exactly what papers such as \cite{chen2021constant} do.

	\subsection{What next?}

		Going back to needle decompositions again, we wish to show that there exists a needle decomposition conserving $\mu_\omega(E) = 1/2$ such that
		\[ \int_{\Omega} \opnorm{A_\omega}^{-1/2} \d{\nu(\omega)} \gtrsim \opnorm{A}^{-1/2}. \]
		\eqref{eqn 5} for $b = 0$ gives
		\[ \opnorm{A} = \opnorm{\int_{\Omega} A_\omega + b_\omega b_\omega^\top \d{\nu(\omega)} }. \]
		Therefore, it would suffice to show that
		\[ \int_\Omega \opnorm{A_\omega}^{-1/2} \d{\nu(\omega)} \gtrsim \opnorm{\int_{\Omega} A_\omega + b_\omega b_\omega^\top \d{\nu(\omega)} }^{-1/2} \]
		for some needle decomposition that conserves $\mu_\omega(E)$.\\

		Using H\"{o}lder's inequality as we did in the proof of the $n^{-1/2}$ bound, it is seen that it suffices to show
		\[ \int_\Omega \opnorm{A_\omega} \d{\nu(\omega)} \lesssim \opnorm{ \int_\Omega A_\omega + b_\omega b_\omega^\top \d{\nu(\omega)} } \]
		for some needle decomposition preserving $\mu_\omega(E)$ (it would in fact be enough to show this with some set $A\subseteq \Omega$ instead of $\Omega$ such that $\nu(A)$ is lower-bounded by a constant).\\
		Neglecting the $b_\omega b_\omega^\top$ term, it suffices to show that
		\begin{equation}
			\label{eqn: needle nearly aligned}
			\int_\Omega \opnorm{A_\omega} \d{\nu(\omega)} \lesssim \opnorm{ \int_\Omega A_\omega \d{\nu(\omega)} }.
		\end{equation}
		The above inequality essentially asks if there exists a needle decomposition where the needles are ``nearly aligned''. Indeed, if the segments of the needles are perfectly aligned, then equality holds above. We are allowing a constant factor of leeway. If the direction of the one-dimensional body $K_\omega$ is $u_\omega$, then the above is equivalent to
		\begin{equation}
			\sup_{\norm{\zeta} \leq 1} \int_\Omega \Var_{x \sim p_\omega}(x) \langle \zeta , u_\omega \rangle^2 \dif \nu(\omega) \gtrsim \int_\Omega \Var_{x \sim p_\omega}(x) \dif \nu(\omega)	
		\end{equation}
		Another way to prove the $n^{-1/2}$ bound is to take the expectation of the term on the left of the above for $\zeta$ being drawn uniformly from $S^{n-1}$.

	\subsection{Equivalence of KLS and good needle decompositions}

		In an earlier section, we showed that if for a fixed $E$ there exists a needle decomposition $\{K_\omega\}$ such that
		\[ \int_{\Omega} \opnorm{A_\omega}^{-1/2} \gtrsim 1, \]
		then the KLS conjecture holds. What about the reverse direction?\\

		Let us begin with a body $K$ with isoperimetric constant $\psi$. We claim that it is possible to split $K$ into $K_1$, $K_2$ such that $\psi_1 = \psi_2 = \psi$, where $\psi_1$ and $\psi_2$ are the isoperimetric constants of the measure restricted to $K_1$ and $K_2$ respectively.\\

		First of all, $\psi \geq \psi_i$. Indeed, if $E \subseteq K_i$ attains its isoperimetric constant, then $\psi \geq \mu^+(\partial E) / \mu(E) ( 1 - \mu(E) ) = \psi_i$.\\

		The reverse bound can be shown to be attained using a ham-sandwich type argument.\\
		Let $E\subseteq K$ attain the isoperimetric constant $\psi$. Usual bisection bisects the signed measure with density $p(y) (\indic_E - \indic_{\Rn \setminus E})$. We can use a ham-sandwich type argument to further bisect the $\mu^+(\partial E)$ part of it as well. For $\varepsilon > 0$, consider the signed measure with density $p(y) \left( \frac{1}{\varepsilon} \indic_{\partial E + \varepsilon B_2^n} - \mu^+(\partial E)\right)$. Taking the limit of the bisecting hyperplane as $\varepsilon \to 0$ ensures that $\mu_i^+(\partial E) = \mu^+(\partial E)$. The resulting bisection preserves isoperimetry because
		\[ \psi_i \geq \frac{\mu_i^+(\partial E)}{\mu_i(E) ( 1 - \mu_i(E) )} = \frac{\mu^+(\partial E)}{\mu(E) ( 1 - \mu(E) )} = \psi. \]

		In the limiting case of the bisections, we have
		\[ \int_{\Omega} \opnorm{A_\omega}^{-1/2} \gtrsim \int_\Omega \psi_\omega \mu_\omega(E) ( 1 - \mu_\omega(E)) = \int_\Omega \psi \mu(E) ( 1 - \mu(E) ) \gtrsim \psi, \]
		so if KLS holds, there always exists a good needle decomposition.

		It is conjectured in \cite{leevempala2018klssurvey} that \emph{any} needle decomposition is good.

		By a Markov argument, this integral inequality holds iff there exists a needle decomposition for each $E$ such that
		\[ \int_A \opnorm{A_\omega} \d{\nu}(\omega) \lesssim 1, \]
		where $A \subseteq \Omega$ such that $\nu(A) \gtrsim 1$.

\section{More on decompositions}

	While we have not mentioned it thus far, a useful result to know is that the KLS conjecture on the class of log-concave measures is equivalent to the KLS conjecture on the class of indicator functions on convex bodies. In particular, the class of log-concave functions is the smallest containing the class of indicator functions on convex bodies that is closed under taking marginals and weak limits, as shown in \cite{AlonsoGutirrez2015}.

	\subsection{Hyperplane bisections}

		As before, suppose we have a log-concave probability measure $\mu$ with density $p$ on the body $K$, and we fix some $E \subseteq K$ with $\mu(E) = 1/2$. Let us define the function $f_{E,K} : \Rn \setminus \{0\} \to \R$ by
		\[ f_{E,K}(x) = \left| \int_{\{ z \in \Rn : \langle z , x \rangle \geq \norm{x}^2 \}} p(y) ( \indic_E - \indic_{\Rn \setminus E} ) \dif y \right|. \]
		That is, if $H_x$ is the hyperplane defined by $x$ (orthogonal to $x$ and passing through it) and $H_x^+$ is either of the resulting halfspaces, the value of the above function at $x$ is equal to $|\mu(E \cap H_x^+) - \mu((\Rn\setminus E) \cap H_x^+)|$.\\
		This serves as a measure of how ``imbalanced'' the hyperplane corresponding to $x$ is -- $f_{E,K}(x) = 0$ iff the hyperplane corresponds to $x$ is a bisecting hyperplane (where bisecting means that $\mu(E \cap K_\omega) = \frac{1}{2} \mu(K_\omega) $, as in needle decompositions).\\
		For nice $E$, $f_{E,K}$ is continuous.\\

		The primary tool used in \cite{lov-sim-on7} to prove the localization lemma was that there exists a bisecting hyperplane passing through any $(n-2)$-dimensional affine space. How would this translate in terms of the above defined function?\\
		Suppose we have an $(n-2)$-dimensional affine space orthogonal to the subspace spanned by $y,z\in\Rn$ and passing through $y$.\\
		Suppose that $x$ defines a hyperplane containing this affine space. Then $x$ is orthogonal to the plane, and so orthogonal to the space itself. That is, it must lie in the subspace spanned by $y, z$. Further, $y-x$ is orthogonal to $x$. That is, the set of all these $x$ forms a circle passing through $0$ contained in the $2$-dimensional subspace spanned by $y,z$.\\
		The conclusion of the localization method is that for any circle $S$ passing through $0$, either
		\begin{itemize}
			\item $f_{E,K}(w) = 0$ for some $w \in S \setminus \{0\}$ or
			\item The limit of $f_{E,K}(w)$ as $w$ goes to $0$ along the circle is equal to $0$ -- this corresponds to a bisecting hyperplane passing through the origin itself. It is not too difficult to check that this is well-defined and that the directional limit along either direction of the circle is the same. 
		\end{itemize}

		More generally, suppose we have some smooth curve $C$ in $\Rn$ that passes through the origin.\footnote{Smoothness is not required, only that the curve is differentiable at the origin.} Then, as before, either $f_{E,K}(w) = 0$ for some $w \in C \setminus \{0\}$ or one of the directional limits at $0$ (along $C$) is equal to $0$.\\

		It may be shown that as a consequence, there exists a surface passing through the origin on which $f_{E,K}$ is equal to $0$.

		In general, if we start off with $E$ of some measure (not necessarily $1/2$) that we want to preserve, we can define $f_{E,K}(x)$ as
		\[ \left| \int_{\{ z \in \Rn : \langle z , x \rangle \geq \norm{x}^2 \}} p(y) ( \mu(\Rn\setminus E)\indic_E - \mu(E)\indic_{\Rn \setminus E} ) \dif y \right| \]

		% A consequence of this nice property is that given any non-zero $x \in \Rn$, there exists a ``barrier'' curve $C$ in $\Rn$ separating $x$ from $0$ such that $f(y) = 0$ for any $y \in C$.

		An interesting question is to generally characterize these functions $f_{E,K}$.

	\subsection{Aligned 2-dimensional decompositions are always possible}

		Suppose we have an $n$-dimensional body $K$ with $n>2$ along with some direction $u$ in $\Rn$. We claim that it is possible to decompose this into a set of $(n-1)$-dimensional bodies $\{K_\omega\}$ such that any of these bodies contains our specified direction $u$ (meaning that a translational shift of $\Span(\{u\})$ is contained in the minimal affine space containing any $K_\omega$).\\

		To prove this, assume without loss of generality that $u = e_n$. Consider the set of $(n-2)$-dimensional affine spaces 
		\[ S = \{ \{x \in \Rn : x_i = q_1, x_j = q_2\} : q_1, q_2 \in \Q, 1 \leq i < j \leq n-1 \}. \]
		This is similar to the argument involved in \cite{lov-sim-on7} except that we only consider the set of $(n-2)$-dimensional affine spaces that contain $u$. As the argument goes there, all the bodies must decompose into at most $(n-1)$-dimensional bodies in the limiting step -- if not, then there exists some affine space in $S$ that intersects the $n$-dimensional body, and choosing the corresponding bisecting hyperplane results in a contradiction.\\

		In fact, it turns out that we can decompose it into a set of $2$-dimensional bodies that all contain our specified direction!\\

		This is easily done using induction on $n$. Reducing the $n$-dimensional body to a set of $(n-1)$-dimensional bodies and then each of these smaller bodies to $2$-dimensional bodies gets the job done. It should be noted that this argument does not work out if the body under consideration is $2$-dimensional, since it does not make sense to have a $0$-dimensional affine space containing our direction.\\

		In general, given $k$ directions, it is possible to decompose a body $K$ into $\{K_\omega\}$ such that each of the $K_\omega$ is $(k+1)$-dimensional and each of them contains these $k$ directions.

		% It is not too difficult to see that given an $n$-dimensional body $K$ ($n \geq 2$) and a direction $u$, it is possible to split it into $\{K_\omega\}_{\omega \in \Omega}$ such that each of the $K_\omega$ is $(n-1)$-dimensional and further, each of them contains $u$  -- this can be done by considering the set of $(n-2)$-dimensional affine spaces that contain $u$ and using an argument similar to that involved in the proof of the localization lemma.\\
		% In terms of our function $f_{E,K}$, if we have $u = e_1$, we are essentially just considering
		% \[ S = \{ x \in \Q^n : \langle x , u \rangle = 0 \} \]
		% and arguing that given a $n$-dimensional body $K$, there exists some $x \in S$ such that $f_{E,K}(x) = 0$ and $H_x$ intersects $K$.

		A natural next question is: can we give up perfect alignedness in exchange for near alignedness, which is all we really need to show KLS?

	\subsection{A potential function}

		Let us fix $\mu$, $p$, $K$, and $E$ as usual. Also suppose we have some direction $u$. We wish to decompose the body into needles in a way that all of them are nearly in the direction of $u$. Equivalently, the hyperplanes chosen for bisection should all nearly contain $u$. That is, the set of $x$ corresponding to the hyperplanes $\{H_x\}$ must all be nearly orthogonal to $u$. So, at each step, the $x$ chosen must be such that $\langle x , u \rangle$ is small -- more precisely, $1 - \frac{\langle x , u \rangle^2}{\norm{x}^2} \gtrsim 1$.\\
		Also, as seen from \eqref{eqn 4}, all we really want is that $\mu_\omega(E)(1-\mu_\omega(E)) \gtrsim 1$, it might be fine to instead just minimize $f_{E,K}$ instead of ensuring that it is exactly equal to $0$. So, one may choose the $x$ corresponding to the bisecting hyperplane at each step by constructing a potential function such as
		\[ \Phi(x) = \left(1 + f_{E,K}(x)\right) \left(1 + \frac{|\langle x , u \rangle|}{\norm{x}}\right) \]
		and at each step, choosing the $x$ that minimizes $\Phi$. The reason for adding the $1$ is that otherwise, the expression would trivially be minimized if the corresponding term is $0$ irrespective of the other term.\\

		We might be able to choose our potential function more wisely.\\
		Recall that in our bound of the isoperimetric constant using needle decompositions, we don't really require that $\mu_\omega(E) = 1/2$ for each $\omega$, we only require that $\mu_\omega(E) (1 - \mu_\omega(E)) \gtrsim 1$. Suppose we have a body $K$ and we split it using the hyperplane $H_x$. Let $K_1 = H_x^+ \cap K$. Then, it is not too difficult to show that
		\[ \mu_2(E) = \mu(E) + \frac{f_{E,K}(x)}{2\mu(H_x^+)}. \]
		In general, if we start off with $K_1 = K$, and we form $K_{i+1}$ by intersecting $K_i$ with the half-space $H_{x_i}^+$ such that the limiting body is $K_\omega$, then
		\[ \mu_\omega(E) = \mu(E) + \sum_{n\in\N} \frac{f_{E,K_n}(x_n)}{2\mu_n(H_{x_n}^+)}. \]
		Our goal should be to minimize the quantity on the right, so a suitable potential might involve something along the lines of $f_{E,K}(x) / 2 \min\{ \mu(H_x^+) , \mu(H_x^-) \}$.\\

		It would be nice to have a similar handy quantity for the alignment part of it as well -- if we start off with a body $K$ and bisect it using $H_{x_1}^+, H_{x_2}^+, \cdots$ to get a limiting one-dimensional body $K_\omega$, how do we represent the ``direction'' of $K_\omega$ in terms of the $x_n$ (and perhaps some quantities involving $K$)?


		% Suppose that we obtain $K_\omega$, $\mu_\omega$ by bisecting with the hyperplanes $(H_{x_{\omega_n}})_{n \in \N}$. Let $K_{\omega_n}$, $\mu_{\omega_n}$ be the body and corresponding measure after bisecting with the first $n$ hyperplanes. Then,
		% \begin{enumerate}
		% 	\item The value of $\mu_\omega(E)$ is bounded by
		% 	\[ \mu(E) \prod_{n \in \N} \left( 1 - \frac{f_{E,K_{\omega_n}}(x_{\omega_n})}{\mu_{\omega_n}(H_{\omega_n}^\pm)} \right) \leq \mu_\omega(E) \leq \mu(E) \prod_{n \in \N} \left( 1 + \frac{f_{E,K_{\omega_n}}(x_{\omega_n})}{\mu_{\omega_n}(H_{\omega_n}^\pm)} \right), \]
		% 	where the $\pm$ depends on which of the two halfspaces we choose.
		% \end{enumerate}

	\subsection{The Poincar\'{e} Inequality}

		Given a probability measure $\mu$ on $\Rn$ with density $p$, its Poincar\'{e} constant is defined by
		\[ \zeta_p = \inf_{g\text{ smooth}} \frac{\expec_{x \sim p} \norm{\grad g(x)}_2^2}{\Var_{x \sim p} g(x)}. \]
		We also define the Cheeger constant by
		\[ h_p = \inf_{g\text{ smooth}} \frac{\expec_{x \sim p} \norm{\grad g(x)}_2}{\expec_{x \sim p} \left| g(x) - \expec_{x \sim p} g(x) \right|}. \]
		Equation (5.8) in \cite{LedouxCheegerPoincareConstant} shows that for log-concave $\mu$, $h_p^2 \sim \zeta_p$.\footnote{Even for $\mu$ that is not log-concave, it is true that $h_p^2 \lesssim \zeta_p$.} Further, more relevant to our interests, $\zeta_p \sim \psi_p^2$ (this is a consequence of Cheeger's inequality -- see \cite{CheegerInequality,Maz60}).\\
		How is the isoperimetric inequality related to these? Suppose that in the definition of the Cheeger constant, we set $g = \indic_E$ for some set $E$ (or rather, a sequence of smooth functions converging to $\indic_E$). Then, $\norm{\grad g(x)}$ behaves like a Dirac delta function on $\partial E$, and we get that $\expec_{x \sim p} \norm{\grad g(x)}$ is just $\mu^+(\partial E)$. The denominator on the other hand is the variance of a Bernoulli random variable with parameter $\mu(E)$, which is equal to $\mu(E)(1 - \mu(E))$.\\
		So, the inside expression as a whole becomes $\mu^+(\partial E) / \mu(E) ( 1 - \mu(E) )$, which is precisely the expression involved in the isoperimetric constant!

	\subsection{Arbitrary cuts through the barycenter tend to be good}

		% change to \mu(H_x^+ \cap E) - \mu(H_x^+ \cap (\Rn \setminus E))

		Suppose $\mu$ is a centered log-concave measure with density $p$. Let $E \subseteq \Rn$ with $\mu(E) = 1/2$. Consider the function $g_0 : S^{n-1} \to \R$ given by
		\[ g_0(x) = \mu(H_x^+ \cap E) - \mu(H_x^+ \cap (\Rn\setminus E)), \]
		where $H_x^+ = \{ z \in \Rn : \langle z , x \rangle \geq 0 \}$.\\
		It is not too difficult to show that $g_0$ is Lipschitz with a Lipschitz constant independent of $\mu$ and $n$.\\
		A consequence of the concentration of measure phenomenon is that given a $1$-Lipschitz function $f$ on $S^{n-1}$ with median $\med f$, if $\sigma$ is the rotation-invariant (uniform) probability measure on $S^{n-1}$,
		\[ \sigma ( \{ |f - \med f| \geq \varepsilon \} ) \leq 2e^{-n\varepsilon^2/2} \]
		and $|\med f - \expec f| \leq 12n^{-1/2}$.\\
		In our context, $\expec g_0 = 0$. Therefore,
		\[ \sigma ( \{ |g_0 - \med g_0| \geq 1/8 \} ) \leq 2e^{-nC}, \]
		where $C$ is some constant independent of $n$ and $\mu$. So, for large $n$, we see that $g_0$ is almost constant on the entirety of $S^{n-1}$, so a random hyperplane through the barycenter is good with high probability.\\

\bibliographystyle{alpha}
\bibliography{references}

\end{document}