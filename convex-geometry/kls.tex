\section{The KLS Conjecture}

\subsection{An Isoperimetric Problem}

\subsubsection{Introduction}

Suppose we have a convex body $K$ and we want to find a surface that divides $K$ into two parts, whose measure is minimum relative to that of the two parts.

\begin{fdef}
	The \textit{isoperimetric coefficient} of a convex body $K\subseteq\Rn$ is the largest number $\psi=\psi(K)$ such that for any measurable $S\subseteq K$,
	\[ \vol_{n-1}(\partial S) \geq \psi \frac{\vol(S)\vol(K\setminus S)}{\vol(K)}. \]
	Often, the above is replaced with
	\[ \vol_{n-1}(\partial S) \geq \psi \min\{\vol(S), \vol(K\setminus S)\} \]
	and this is the form we shall use throughout the rest of the section.
\end{fdef}

Since the two $\psi$s are within a factor of $2$, estimates are not influenced very drastically on making this change. This problem turns out to be very intimately related to that of volume computation we explored in the previous section.\\
\cite{lovasz-simonovits-mixing-rate-isoperimetric} bounds the isoperimetric coefficient below by $1/d$, where $d$ is the diameter of the body. Note that this is quite obvious if the separating surface $\partial S$ is a (section of a) hyperplane.\\
\cite{applegate-kannan-cube-sandwich} gives a more general result where the measure is replaced by that with density equal to any log-concave function and bounds it below by $2/d$. This is in fact as tight as we can get in terms of the diameter and indeed, the bound is attained for a thin long cylinder.\\
However, the bodies we are interested in (in say, volume computation) tend to have a certain structure to them. In particular, sandwiching makes the bodies somewhat round.\\
Throughout the rest of this section, denote the average distance of a point $K$ from its center of gravity by $M_1(K)$. The main result of this section is that for every convex body $K$,
\[ \psi(K) \geq \frac{\ln 2}{M_1(K)}. \]

\subsubsection{Needles and Localization Lemmas}

To begin, consider the following motivated by \Cref{localization lemma}.

\begin{definition}
	A \textit{needle} is a segment $[a,b]\in\Rn$ together with a non-negative linear function $\ell:I\to\R^{\geq 0}$ not identically $0$. If $N=(I,\ell)$ is a needle and $f$ is an integrable function defined on $I$, denote
	\[ \int_N f = \int_{0}^{|b-a|} f(a+tu)\ell(a+tu)^{n-1} \d{t}, \]
	where $u=(b-a)/|b-a|$.
\end{definition}

\begin{lemma}
	\label{lem: 5.1}
	Let $f_1$, $f_2$, $f_3$, $f_4$ be non-negative continuous functions defined on $\Rn$ and $\alpha,\beta>0$. The following are equivalent.
	\begin{itemize}
		\item For every convex body $K$ in $\Rn$,
			\[ \left(\int_K f_1\right)^\alpha \left(\int_K f_2\right)^\beta \leq \left(\int_K f_3\right)^\alpha \left(\int_K f_4\right)^\beta. \]
		\item For every needle $N$ in $\Rn$,
			\[ \left(\int_N f_1\right)^\alpha \left(\int_N f_2\right)^\beta \leq \left(\int_N f_3\right)^\alpha \left(\int_N f_4\right)^\beta. \]
	\end{itemize}
\end{lemma}
\begin{proof}
	The first implying the second is quite easy. For the converse, suppose that the second holds but the first does not.
	Adding a sufficiently small quantity to $f_3$ and $f_4$, we may further assume that they are (strictly) positive. We may also assume that $f_1$ and $f_2$ are positive (Why?). Choose some $A$ such that
	\[ \frac{\left(\int_K f_1\right)^\alpha}{\left(\int_K f_3\right)^\alpha} > A > \frac{\left(\int_K f_4\right)^\beta}{\left(\int_K f_2\right)^\beta}. \]
	Then,
	\[ \int_K f_1 - A^{1/\alpha}f_3 > 0 \text{ and } \int_K A^{1/\beta}f_2 - f_4 > 0. \]
	Using \Cref{localization lemma}, there is some needle $N$ such that
	\[ \int_N f_1 - A^{1/\alpha}f_3 > 0 \text{ and } \int_N A^{1/\beta}f_2 - f_4 > 0. \]
	This implies that
	\[ \frac{\left(\int_N f_1\right)^\alpha}{\left(\int_N f_3\right)^\alpha} > A > \frac{\left(\int_N f_4\right)^\beta}{\left(\int_N f_2\right)^\beta}, \]
	thus proving the claim.
\end{proof}

Observe that we can extend this more generally to the case where $f_1$ and $f_2$ are upper semicontinuous and $f_3$ and $f_4$ are lower semicontinuous. Indeed, we may consider an appropriate sequence of continuous function. In particular, this allows us to restrict ourselves from $\Rn$ to some subset $T$ of $\Rn$ by multiplying the functions with the indicator function $\indic_T$ (the functions extend to upper semicontinuous functions if $T$ is closed and lower semicontinuous functions if $T$ is open).
% *** Why?

\begin{corollary}
	\label{cor: if positive on convex set convex on some needle}
	Let $T$ be a bounded open convex set in $\Rn$, $g$ a bounded lower semicontinuous function on $T$, and $h$ a continuous function on $T$ such that
	\[ \int_T g > 0 \text{ and } \int_T h = 0. \]
	Then there is a needle $N=(I,\ell)$ with $I\subseteq T$ such that
	\[ \int_N g > 0 \text{ and } \int_N h = 0. \]
\end{corollary}
\begin{proof}
	Choose some $0<\delta<\int_T g$ and let $\varepsilon>0$. Then,
	\[ \int_T \left(g-\delta+\frac{1}{\varepsilon}h\right) > 0 \text{ and } \int_T (\varepsilon-h) > 0. \]
	Extending these functions to functions on $\Rn$ (multiplying with the indicator function) and using \Cref{localization lemma}, we get a needle $N_\varepsilon=(I_\varepsilon,\ell_\varepsilon)$ with $I_\varepsilon\subseteq T$ (Why?) such that
	\begin{equation}
		\label{eqn: 5.1}
		\int_{N_\varepsilon} \left(g-\delta+\frac{1}{\varepsilon}h\right) > 0 \text{ and } \int_{N_\varepsilon} (\varepsilon-h) > 0.
	\end{equation}
	Letting $M$ to be the supremum of $g$ on $\Rn$,
	\begin{equation}
		\label{eqn: 5.2}
		-M\varepsilon\int_{N_\varepsilon} 1 < \int_{N_\varepsilon} h < \varepsilon \int_{N_\varepsilon} 1.
	\end{equation}
	Consider these needles for $\varepsilon=1/k$ ($k\in\N$). Scaling appropriately, we may assume that the maximum of each $\ell_{1/k}$ is $1$. Using the Bolzano-Weierstrass Theorem, there is some subsequence of these needles that converges (in the sense that the endpoints of the $I_{1/k}$ and the $\ell_{1/k}$ converge)\footnote{We can think of an needle $N=([a,b],\ell)$ as an element $(a,b,\ell(a),\ell(b)-\ell(a))\in\R^{2n+2}$. In our case, this sequence is bounded because each interval is in the bounded set $T$ and each $\ell$ is between $0$ and $1$.} to some needle $N=(I,\ell)$. Combining \Cref{eqn: 5.1,eqn: 5.2} implies that $N$ satisfies the required.
\end{proof}

While these results are quite nice, exponents of a linear function are not very convenient to deal with. This motivates the following.

\subsubsection{Exponential Needles}

\begin{definition}
	An \textit{exponential needle} is a segment $[a,b]\in\Rn$ together with a real $\gamma$. If $E=(I,\gamma)$ is a needle and $f$ is an integrable function defined on $I$, denote
	\[ \int_E f = \int_{0}^{|b-a|} f(a+tu)e^{\gamma t} \d{t}, \]
	where $u=(b-a)/|b-a|$.
\end{definition}

If we manage to prove our results for an exponential needle instead, it is extremely convenient because taking exponents does not change the underlying structure of the function itself.

\begin{lemma}
	\label{localized exponential needle interconversion}
	Let $f_1$, $f_2$, $f_3$, and $f_4$ be four non-negative continuous functions defined on an interval $[a,b]$ and $\alpha,\beta>0$. Then the following are equivalent.
	\begin{itemize}
		\item For every log-concave function $F$ defined on $\R$, 
		\[ \left(\int_a^b F(t)f_1(t)\d{t}\right)^\alpha \left(\int_a^b F(t)f_2(t)\d{t}\right)^\beta \leq \left(\int_a^b F(t)f_3(t)\d{t}\right)^\alpha \left(\int_a^b F(t)f_4(t)\d{t}\right)^\beta. \]

		\item For every subinterval $[a',b']\subseteq[a,b]$ and real $\gamma$,
		\[ \left(\int_{a'}^{b'} e^{\gamma t}f_1(t)\d{t}\right)^\alpha \left(\int_{a'}^{b'} e^{\gamma t}f_2(t)\d{t}\right)^\beta \leq \left(\int_{a'}^{b'} e^{\gamma t}f_3(t)\d{t}\right)^\alpha \left(\int_{a'}^{b'} e^{\gamma t}f_4(t)\d{t}\right)^\beta. \]
	\end{itemize}
\end{lemma}

\begin{proof}
	The first implying the second is obvious (on setting $F=\indic_{[a',b']}e^{\gamma t}$).

	Note that if for some $t_0\in[a,b]$, $f_1(t_0)^\alpha f_2(t_0)^\beta > f_3(t_0)^\alpha f_4(t_0)^\beta$, then both the assertions above fail since we can consider
	\begin{itemize}
		\item the log-concave function $e^{-c(t-t_0)^2}$ for a sufficiently large $c$, or
		\item a sufficiently small interval containing $t_0$.
	\end{itemize}
	Therefore, we may assume that for all $t\in[a,b]$,
	\begin{equation*}
		\label{eqn: init observation exp needle conv body}
		\tag{$*$}
		f_1(t)^\alpha f_2(t)^\beta \leq f_3(t)^\alpha f_4(t)^\beta.
	\end{equation*}

	Suppose the second holds and the first does not for some log-concave function $F$.\\
	We may assume that $F\neq 0$ (so $F>0$) on $[a,b]$. Otherwise, we can replace it with its convolution with $e^{-ct^2}$ for a sufficiently large $c$, which is still log-concave by \Cref{convolution of log concave functions is log concave} and would still satisfy the inequality (Why?). We may also assume that $F\geq 1$ on $[a,b]$ by scaling up appropriately. Let $F=e^G$, where $G$ is a non-negative concave function on $[a,b]$.\\
	For each $n$, define $K_n\subseteq\R^{n+1}$ by
	\[ K_n = \left\{ (t,x) : t\in[a,b], x\in\Rn, \norm{x} \leq 1 + \frac{G(t)}{n} \right\}. \]
	Let $\hat{f}_i:\R^{n+1}\to\R$ by defined by $\hat{f}_i(t,x)=f_i(t)$.\\
	For sufficiently large $n$, we have $(1+G(t)/n)\approx e^{G(t)} = F(t)$, so we can write\footnote{The $F$ disappears from the integral when integrating over $\{t\}\times B_2^n$.}
	\[ \left(\int_{K_n} \hat{f}_1(t)\d{t}\right)^\alpha \left(\int_{K_n} \hat{f}_2(t)\d{t}\right)^\beta > \left(\int_{K_n} \hat{f}_3(t)\d{t}\right)^\alpha \left(\int_{K_n} \hat{f}_4(t)\d{t}\right)^\beta. \]
	Using \Cref{lem: 5.1}, let $N_n$ be a needle such that
	\[ \left(\int_{N_n} \hat{f}_1(t)\d{t}\right)^\alpha \left(\int_{N_n} \hat{f}_2(t)\d{t}\right)^\beta > \left(\int_{N_n} \hat{f}_3(t)\d{t}\right)^\alpha \left(\int_{N_n} \hat{f}_4(t)\d{t}\right)^\beta. \]

	If $N_n$ is orthogonal to the $t$-axis, then (\ref{eqn: init observation exp needle conv body}) immediately breaks so we arrive at a contradiction. Otherwise, we may project the needle onto the $t$-axis to get some $[a_n,b_n]\subseteq[a,b]$ and a linear function $\ell_n$ such that
	\begin{equation}
		\label{eqn: 5.3}
		\left(\int_{a_n}^{b_n} \ell_n(t)^n \hat{f}_1(t)\d{t}\right)^\alpha \left(\int_{a_n}^{b_n} \ell_n(t)^n \hat{f}_2(t)\d{t}\right)^\beta > \left(\int_{a_n}^{b_n} \ell_n(t)^n \hat{f}_3(t)\d{t}\right)^\alpha \left(\int_{a_n}^{b_n} \ell_n(t)^n \hat{f}_4(t)\d{t}\right)^\beta.
	\end{equation}
	By the Bolzano-Weierstrass Theorem, there is a subsequence such that $a_{n_k}$, $b_{n_k}$ converge, to say $a_0$ and $b_0$. By (\ref{eqn: init observation exp needle conv body}), $a_0 < b_0$. Suppose that $\ell_n(a_0) < \ell_n(b_0)$ for infinitely many indices -- if not, then exchange $a_0$ and $b_0$ in the following argument. Now, let each $\ell_n$ be normalized such that $\ell_n(b_0)=1$. Let $\gamma_n = \ell_n(a_0)$ for each $n$.\\
	For some subsequence, let $\gamma_n\to\gamma$ and $n(1-\gamma_n)\to\gamma'$, where $0\leq\gamma\leq 1$ and $0\leq\gamma'\leq\infty$. Henceforth, we restrict ourselves to this subsequence.
	\begin{itemize}
		\item If $\gamma\neq 1$, $\ell_n(t)^n\to 0$ for all $a_0\leq t<b_0$. Dividing \Cref{eqn: 5.3} by $\left(\int_{a_n}^{b_n}\ell_n(t)^n\d{t}\right)^{\alpha+\beta}$ and letting $n\to\infty$, we get
		\[ f_1(b_0)^\alpha f_2(b_0)^\beta \geq f_3(b_0)^\alpha f_4(b_0)^\beta. \]
		If instead of $f_3$ and $f_4$ everywhere in the proof above, we instead take $f_3+\varepsilon$ and $f_4+\varepsilon$ for a sufficiently small $\varepsilon$, we arrive at a contradiction to \Cref{eqn: init observation exp needle conv body}.

		\item Therefore, $\gamma=1$. We then have
		\[ \ell_n(t)^n = \left((1 - (1-\ell_n(t)))^{1/(1-\ell_n(t))}\right)^{n(1-\ell_n(t))}. \]
		The expression within the topmost exponent on the right goes to $1/e$. If $\gamma'=\infty$, then we again get $\ell_n(t)^n\to 0$ for $t < b_0$, so we arrive at a contradiction similar to the first case above.\\
		Otherwise, we have
		\[ \ell_n(t) \to e^{\gamma'(t-b_0)/(b_0-a_0)}. \]
		Letting $\gamma''=\gamma'/(b_0-a_0)$ and letting $n\to\infty$, we get
		\[ \left(\int_{a_0}^{b_0} e^{\gamma''(t-b_0)} f_1(t)\d{t}\right)^\alpha \left(\int_{a_0}^{b_0} e^{\gamma''(t-b_0)} f_2(t)\d{t}\right)^\beta > \left(\int_{a_0}^{b_0} e^{\gamma''(t-b_0)} f_3(t)\d{t}\right)^\alpha \left(\int_{a_0}^{b_0} e^{\gamma''(t-b_0)} f_4(t)\d{t}\right)^\beta. \]
		However, this (after multiplying by an appropriate factor on either side to remove the $b_0$ in the exponent) contradicts the original assumption that the opposite inequality holds for any exponential needle, thus completing the proof.
	\end{itemize}
\end{proof}

Next, we shall show the counterpart of \Cref{lem: 5.1} for exponential needles. This is essentially a generalized version of the above lemma, so is relatively straight-forward to prove.

\begin{ftheo}
	\label{generalized exponential needle interconversion}
	Let $f_1$, $f_2$, $f_3$, and $f_4$ be non-negative functions on $\Rn$ and $\alpha,\beta>0$. The following are equivalent.
	\begin{itemize}
		\item For every log-concave function $F$ on $\Rn$ with compact support,
			\[ \left(\int_\Rn F(t) f_1(t)\d{t}\right)^\alpha \left(\int_\Rn F(t) f_2(t)\d{t}\right)^\beta \leq \left(\int_\Rn F(t) f_3(t)\d{t}\right)^\alpha \left(\int_\Rn F(t) f_4(t)\d{t}\right)^\beta. \]
		\item For every exponential needle $E$ in $\Rn$,
			\[ \left(\int_E f_1\right)^\alpha \left(\int_E f_2\right)^\beta \leq \left(\int_E f_3\right)^\alpha \left(\int_E f_4\right)^\beta. \]
	\end{itemize}
\end{ftheo}
\begin{proof}
	Going from the first to the second isn't too difficult. Given the exponential needle over $[a,b]$ and constant $\gamma$, consider the function $F$ defined by $t\mapsto e^{\gamma \langle t-b, u\rangle}$, where $u=(b-a)/\norm{b-a}$ restricted to some $\varepsilon$-neighbourhood of $[a,b]$. Letting $\varepsilon\to 0$, we get the required.\\
	On the other hand, let the second hold but not the first for some function $F$. Then applying \Cref{lem: 5.1} on the $Ff_i$, we get some $[a,b]$ and linear function $\ell$ on $[a,b]$ such that
	\begin{multline*}
		\left(\int_0^{|b-a|} f_1(a+tu) F(a+tu)\ell(a+tu)^{n-1}\d{t}\right)^\alpha \left(\int_0^{|b-a|} f_2(a+tu) F(a+tu)\ell(a+tu)^{n-1}\d{t}\right)^\beta \\
		> \left(\int_0^{|b-a|} f_3(a+tu) F(a+tu)\ell(a+tu)^{n-1}\d{t}\right)^\alpha \left(\int_0^{|b-a|} f_4(a+tu) F(a+tu)\ell(a+tu)^{n-1}\d{t}\right)^\beta.
	\end{multline*}

	However, $F\ell^{n-1}$ is concave, so by \Cref{localized exponential needle interconversion}, there exists an exponential needle that violates the assumption.
\end{proof}

\subsubsection{An Example Using the Equivalences}

Let $K$ be a convex body $f:K\to\R$ be integrable. Define its $L_p$ norm by
\[ \norm{f}_p = \left(\frac{1}{\vol K} \int_{K} |f(x)|^p \d{x} \right)^{1/p}. \]
It is easy to see that if $0<p<q$, $\norm{f}_p \leq \norm{f}_q$.

\begin{theorem}
	Let $0<p<q$. There exists a constant $c_{p,q}$ such that for any dimension $n$, convex body $K\subseteq\Rn$ and linear function $f:K\to\R$,
	\[ \norm{f}_q \leq c_{p,q}\norm{f}_p \]
\end{theorem}
\begin{proof}
	We wish to show that for any $K$,
	\[ \left(\int_K |f|^q\right)^{1/q} \left(\int_K 1\right)^{1/p} \leq c_{p,q} \left(\int_K 1\right)^{1/q} \left(\int_K |f|^p\right)^{1/p}. \]
	Equivalently, we wish to show that for any exponential needle $E$,
	\[ \left(\int_E |f|^q\right)^{1/q} \left(\int_E 1\right)^{1/p} \leq c_{p,q} \left(\int_E 1\right)^{1/q} \left(\int_E |f|^p\right)^{1/p}. \]
	Let $E$ be an arbitrary exponential needle on interval $[a,b]$ with constant $\gamma$. That is, we wish to show that for any linear function $f$, $a,b\in\R$, and real $\gamma$,
	\[ \left( \frac{\int_a^b e^{\gamma t} |f(t)|^q\d{t}}{\int_a^b e^{\gamma t}\d{t}} \right)^{1/q} \leq c_{p,q} \left( \frac{\int_a^b e^{\gamma t} |f(t)|^p\d{t}}{\int_a^b e^{\gamma t}\d{t}}\right)^{1/p}, \]
	Since $f$ is linear, we may assume without loss of generality that $f(a+tu)=t$ on $[a,b]$ and that $\gamma=1$; for the general case where $\gamma\neq 0$, we can just substitute appropriately. The cases where $\gamma=0$ or $f$ is constant on $[a,b]$ are easily shown.\\
	\[ \varphi(a,b) = \left( \frac{\int_a^b e^{t} |f(t)|^q\d{t}}{\int_a^b e^{t}\d{t}} \right)^{1/q} \left( \frac{\int_a^b e^{t} |f(t)|^p\d{t}}{\int_a^b e^{t}\d{t}} \right)^{-1/p}. \]
	We wish to show that $c_{p,q} = \sup_{a<b} \varphi(a,b)$ is finite. Note that $\varphi$ is continuous for $a<b$. Further, for any $\alpha$, $\varphi(a,b)\to 1$ as $a,b\to\alpha$. That is, we may extend the function continuously to $a\leq b$ defining $\varphi(a,a)=1$.\\
	Now, observe that for $b\to\infty$, $\varphi(a,b)\to 1$.\footnote{$\int_a^b e^t |f(t)|^p\d{t}$ grows as $e^{-b}b^p$ and $\int_a^b e^t\d{t}$ grows as $e^b$.} On the other hand, for fixed $b$ and $a\to\infty$, $\varphi(a,b)$ remains bounded. The continuity implies that $\varphi$ is bounded and its supremum is finite.
\end{proof}

The actual calculation of the supremum above is quite tedious, however.

