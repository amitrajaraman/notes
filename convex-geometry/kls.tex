\section{The KLS Conjecture}

\subsection{An Isoperimetric Problem}

\subsubsection{Introduction}

In the context of ball-step, let us look at the mixing time of the chain for a general (not necessarily convex) body.\\
As discussed, it suffices to look at the conductance of the chain, which relates to finding a ``cut'' of the body of relatively small surface area. For example, in a dumbbell, we could have a cut down the central bottleneck, which would result in a very low conductance.\\
What about convex bodies? It is seen that hyperplanes are not bad (this is made precise in \Cref{kls conj hyperplane}). A natural next question is: could we have some convoluted cut that ends up being a bottleneck? In \cite{KLSConjecture}, Kannan, Lov\`asz, and Simonovits conjectured that this in fact cannot happen. In particular, they claim that up to a constant factor, hyperplane cuts are in fact the ``worst'' cuts.\\
We first formalize this notion of a cut to get an expression similar to that of conductance, discuss some localization lemmata similar to those discussed earlier (to reduce $n$-dimensional integrals to $1$-dimensional integrals), prove an improvement of \Cref{improvement of conductance isoperimetric inequality} in \Cref{isoperimetric coefficient bound 1,isoperimetric coefficient bound 2}, and finish off with the statement of the conjecture.

Suppose we have a convex body $K$ and we want to find a surface that divides $K$ into two parts, whose measure is minimum relative to that of the two parts.

\begin{fdef}
	\label{def: isoperimetric coefficient}
	The \textit{isoperimetric coefficient} of a convex body $K\subseteq\Rn$ is the largest number $\psi=\psi(K)$ such that for any measurable $S\subseteq K$,
	\[ \vol_{n-1}(\partial S) \geq \psi \frac{\vol(S)\vol(K\setminus S)}{\vol(K)}. \]
	Often, the above is replaced with
	\[ \vol_{n-1}(\partial S) \geq \psi \min\{\vol(S), \vol(K\setminus S)\} \]
	and this is the form we shall use throughout the rest of the section.
\end{fdef}

Since the two $\psi$s are within a factor of $2$, estimates are not influenced very drastically on making this change. This problem turns out to be very intimately related to that of volume computation we explored in the previous section.\\
\cite{lovasz-simonovits-mixing-rate-isoperimetric} bounds the isoperimetric coefficient below by $1/d$, where $d$ is the diameter of the body. Note that this is quite obvious if the separating surface $\partial S$ is a (section of a) hyperplane.\\
\cite{applegate-kannan-cube-sandwich} gives a more general result where the measure is replaced by that with density equal to any log-concave function and bounds it below by $2/d$. This is in fact as tight as we can get in terms of the diameter and indeed, the bound is attained for a thin long cylinder.\\
However, the bodies we are interested in (in say, volume computation) tend to have a certain structure to them. In particular, sandwiching makes the bodies somewhat round.\\
The main result of this section is that for every convex body $K$,
\[ \psi(K) \geq \frac{\ln 2}{M_1(K)}, \]
where $M_1(K)$ is the average distance of a point in $K$ from the center of gravity of $K$. 

\subsubsection{Needles and Localization Lemmas}
\label{sec: 5.1.2}

To begin, consider the following motivated by \Cref{localization lemma}.

\begin{definition}
	A \textit{needle} is a segment $[a,b]\in\Rn$ together with a non-negative linear function $\ell:I\to\R^{\geq 0}$ not identically $0$. If $N=(I,\ell)$ is a needle and $f$ is an integrable function defined on $I$, denote
	\[ \int_N f = \int_{0}^{|b-a|} f(a+tu)\ell(a+tu)^{n-1} \d{t}, \]
	where $u=(b-a)/|b-a|$.
\end{definition}

\begin{lemma}
	\label{lem: 5.1}
	Let $f_1$, $f_2$, $f_3$, $f_4$ be non-negative continuous functions defined on $\Rn$ and $\alpha,\beta>0$. The following are equivalent.
	\begin{itemize}
		\item For every convex body $K$ in $\Rn$,
			\[ \left(\int_K f_1\right)^\alpha \left(\int_K f_2\right)^\beta \leq \left(\int_K f_3\right)^\alpha \left(\int_K f_4\right)^\beta. \]
		\item For every needle $N$ in $\Rn$,
			\[ \left(\int_N f_1\right)^\alpha \left(\int_N f_2\right)^\beta \leq \left(\int_N f_3\right)^\alpha \left(\int_N f_4\right)^\beta. \]
	\end{itemize}
\end{lemma}
\begin{proof}
	The first implying the second is quite easy to show. For the converse, suppose that the second holds but the first does not.\\
	Adding a sufficiently small quantity to $f_3$ and $f_4$, we may further assume that they are (strictly) positive. We may also assume that $f_1$ and $f_2$ are positive (Why?). Choose some $A$ such that
	\[ \frac{\left(\int_K f_1\right)^\alpha}{\left(\int_K f_3\right)^\alpha} > A > \frac{\left(\int_K f_4\right)^\beta}{\left(\int_K f_2\right)^\beta}. \]
	Then,
	\[ \int_K f_1 - A^{1/\alpha}f_3 > 0 \text{ and } \int_K A^{1/\beta}f_2 - f_4 > 0. \]
	Using \Cref{localization lemma}, there is some needle $N$ such that
	\[ \int_N f_1 - A^{1/\alpha}f_3 > 0 \text{ and } \int_N A^{1/\beta}f_2 - f_4 > 0. \]
	This implies that
	\[ \frac{\left(\int_N f_1\right)^\alpha}{\left(\int_N f_3\right)^\alpha} > A > \frac{\left(\int_N f_4\right)^\beta}{\left(\int_N f_2\right)^\beta}, \]
	thus proving the claim.
\end{proof}

Observe that we can extend this more generally to the case where $f_1$ and $f_2$ are upper semicontinuous and $f_3$ and $f_4$ are lower semicontinuous by considering an appropriate sequence of continuous functions. In particular, this allows us to restrict ourselves from $\Rn$ to some subset $T$ of $\Rn$ by multiplying the functions with the indicator function $\indic_T$ (the functions extend to upper semicontinuous functions if $T$ is closed and lower semicontinuous functions if $T$ is open).
% *** Why?

\begin{corollary}
	\label{cor: if positive on convex set convex on some needle}
	Let $T$ be a bounded open convex set in $\Rn$, $g$ a bounded lower semicontinuous function on $T$, and $h$ a continuous function on $T$ such that
	\[ \int_T g > 0 \text{ and } \int_T h = 0. \]
	Then there is a needle $N=(I,\ell)$ with $I\subseteq T$ such that
	\[ \int_N g > 0 \text{ and } \int_N h = 0. \]
\end{corollary}
\begin{proof}
	Choose some $0<\delta<\int_T g$ and let $\varepsilon>0$. Then,
	\[ \int_T \left(g-\delta+\frac{1}{\varepsilon}h\right) > 0 \text{ and } \int_T (\varepsilon^2	-h) > 0. \]
	Extending these functions to functions on $\Rn$ (multiplying with the indicator function) and using \Cref{localization lemma}, we get a needle $N_\varepsilon=(I_\varepsilon,\ell_\varepsilon)$ with $I_\varepsilon\subseteq T$ (Why?) such that
	\begin{equation}
		\label{eqn: 5.1}
		\int_{N_\varepsilon} \left(g-\delta+\frac{1}{\varepsilon}h\right) > 0 \text{ and } \int_{N_\varepsilon} (\varepsilon^2-h) > 0.
	\end{equation}
	Observe that $\int_{N_\varepsilon} (g-\delta+\varepsilon)>0$.\\
	Taking $M$ as the supremum of $g$ on $\Rn$,
	\begin{equation}
		\label{eqn: 5.2}
		-M\varepsilon\int_{N_\varepsilon} 1 < \int_{N_\varepsilon} h < \varepsilon^2 \int_{N_\varepsilon} 1.
	\end{equation}
	Consider these needles for $\varepsilon=1/k$ ($k\in\N$). Scaling appropriately, we may assume that the maximum of each linear function $\ell_{1/k}$ is $1$. Using the Bolzano-Weierstrass Theorem, there is some subsequence of these needles that converges (in the sense that the endpoints of the $I_{1/k}$ and the $\ell_{1/k}$ converge)\footnote{We can think of an needle $N=([a,b],\ell)$ as an element $(a,b,\ell(a),\ell(b)-\ell(a))\in\R^{2n+2}$. In our case, this sequence is bounded because each interval is in the bounded set $T$ and each $\ell$ is between $0$ and $1$.} to some needle $N=(I,\ell)$. Combining \Cref{eqn: 5.1,eqn: 5.2} implies that $N$ satisfies the required (we get $\int_N (g-\delta) \geq 0$ and $\int_N h = 0$).
\end{proof}

While these results are quite nice, exponents of a linear function are not very convenient to deal with. This motivates the following.

\subsubsection{Exponential Needles}
\label{sec: 5.1.3}

\begin{definition}
	An \textit{exponential needle} is a segment $[a,b]\in\Rn$ together with a real $\gamma$. If $E=(I,\gamma)$ is a needle and $f$ is an integrable function defined on $I$, denote
	\[ \int_E f = \int_{0}^{|b-a|} f(a+tu)e^{\gamma t} \d{t}, \]
	where $u=(b-a)/\norm{b-a}$.
\end{definition}

If we manage to prove our results for an exponential needle instead, it is extremely convenient because taking exponents does not change the underlying structure of the function itself.

\begin{lemma}
	\label{localized exponential needle interconversion}
	Let $f_1$, $f_2$, $f_3$, and $f_4$ be four non-negative continuous functions defined on an interval $[a,b]$ in $\R$ and $\alpha,\beta>0$. Then the following are equivalent.
	\begin{itemize}
		\item For every log-concave function $F$ defined on $\R$, 
		\[ \left(\int_a^b F(t)f_1(t)\d{t}\right)^\alpha \left(\int_a^b F(t)f_2(t)\d{t}\right)^\beta \leq \left(\int_a^b F(t)f_3(t)\d{t}\right)^\alpha \left(\int_a^b F(t)f_4(t)\d{t}\right)^\beta. \]

		\item For every subinterval $[a',b']\subseteq[a,b]$ and real $\gamma$,
		\[ \left(\int_{a'}^{b'} e^{\gamma t}f_1(t)\d{t}\right)^\alpha \left(\int_{a'}^{b'} e^{\gamma t}f_2(t)\d{t}\right)^\beta \leq \left(\int_{a'}^{b'} e^{\gamma t}f_3(t)\d{t}\right)^\alpha \left(\int_{a'}^{b'} e^{\gamma t}f_4(t)\d{t}\right)^\beta. \]
	\end{itemize}
\end{lemma}

\begin{proof}
	The first implying the second is obvious (on setting $F=\indic_{[a',b']}e^{\gamma t}$).

	Note that if for some $t_0\in[a,b]$, $f_1(t_0)^\alpha f_2(t_0)^\beta > f_3(t_0)^\alpha f_4(t_0)^\beta$, then both the assertions above fail since we can consider
	\begin{itemize}
		\item the log-concave function $e^{-c(t-t_0)^2}$ for a sufficiently large $c$, or
		\item a sufficiently small interval containing $t_0$.
	\end{itemize}
	Therefore, we may assume that for all $t\in[a,b]$,
	\begin{equation*}
		\label{eqn: init observation exp needle conv body}
		\tag{$*$}
		f_1(t)^\alpha f_2(t)^\beta \leq f_3(t)^\alpha f_4(t)^\beta.
	\end{equation*}

	Suppose the second holds and the first does not for some log-concave function $F$.\\
	We may assume that $F\neq 0$ (so $F>0$) on $[a,b]$. Otherwise, we can replace it with its convolution with $e^{-ct^2}$ for a sufficiently large $c$, which is still log-concave by \Cref{convolution of log concave functions is log concave} and would still satisfy the inequality (Why?). We may also assume that $F\geq 1$ on $[a,b]$ by scaling up appropriately. Let $F=e^G$, where $G$ is a non-negative concave function on $[a,b]$.\\
	For each $n$, define $K_n\subseteq\R^{n+1}$ by
	\[ K_n = \left\{ (t,x) : t\in[a,b], x\in\Rn, \norm{x} \leq 1 + \frac{G(t)}{n} \right\}. \]
	Let $\hat{f}_i:\R^{n+1}\to\R$ by defined by $\hat{f}_i(t,x)=f_i(t)$.\\
	For sufficiently large $n$, we have $(1+G(t)/n)^n\approx e^{G(t)} = F(t)$, so we can write%\footnote{The $F$ disappears from the integral when integrating over the ``discs'' orthogonal to the $t$-axis.}
	\[ \left(\int_{K_n} \hat{f}_1(t)\d{t}\right)^\alpha \left(\int_{K_n} \hat{f}_2(t)\d{t}\right)^\beta > \left(\int_{K_n} \hat{f}_3(t)\d{t}\right)^\alpha \left(\int_{K_n} \hat{f}_4(t)\d{t}\right)^\beta. \]
	Using \Cref{lem: 5.1}, there exists a needle $N_n$ such that
	\[ \left(\int_{N_n} \hat{f}_1(t)\d{t}\right)^\alpha \left(\int_{N_n} \hat{f}_2(t)\d{t}\right)^\beta > \left(\int_{N_n} \hat{f}_3(t)\d{t}\right)^\alpha \left(\int_{N_n} \hat{f}_4(t)\d{t}\right)^\beta. \]

	If $N_n$ is orthogonal to the $t$-axis, then (\ref{eqn: init observation exp needle conv body}) immediately breaks so we arrive at a contradiction. Otherwise, we may project the needle onto the $t$-axis to get some $[a_n,b_n]\subseteq[a,b]$ and a linear function $\ell_n$ such that
	\begin{equation}
		\label{eqn: 5.3}
		\left(\int_{a_n}^{b_n} \ell_n(t)^n \hat{f}_1(t)\d{t}\right)^\alpha \left(\int_{a_n}^{b_n} \ell_n(t)^n \hat{f}_2(t)\d{t}\right)^\beta > \left(\int_{a_n}^{b_n} \ell_n(t)^n \hat{f}_3(t)\d{t}\right)^\alpha \left(\int_{a_n}^{b_n} \ell_n(t)^n \hat{f}_4(t)\d{t}\right)^\beta.
	\end{equation}
	By the Bolzano-Weierstrass Theorem, there is a subsequence such that $a_{n_k}$, $b_{n_k}$ converge, to say $a_0$ and $b_0$. By (\ref{eqn: init observation exp needle conv body}), $a_0 < b_0$. Suppose that $\ell_n(a_0) < \ell_n(b_0)$ for infinitely many indices -- if not, then exchange $a_0$ and $b_0$ in the following argument. Now, let each $\ell_n$ be normalized such that $\ell_n(b_0)=1$. Let $\gamma_n = \ell_n(a_0)$ for each $n$.\\
	For some subsequence, let $\gamma_n\to\gamma$ and $n(1-\gamma_n)\to\gamma'$, where $0\leq\gamma\leq 1$ and $0\leq\gamma'\leq\infty$. Henceforth, we restrict ourselves to this subsequence.
	\begin{itemize}
		\item If $\gamma\neq 1$, $\ell_n(t)^n\to 0$ for all $a_0\leq t<b_0$. Dividing \Cref{eqn: 5.3} by $\left(\int_{a_n}^{b_n}\ell_n(t)^n\d{t}\right)^{\alpha+\beta}$ and letting $n\to\infty$, we get
		\[ f_1(b_0)^\alpha f_2(b_0)^\beta \geq f_3(b_0)^\alpha f_4(b_0)^\beta. \]
		If instead of $f_3$ and $f_4$ everywhere in the proof above, we instead take $f_3+\varepsilon$ and $f_4+\varepsilon$ for a sufficiently small $\varepsilon$, we get a strict inequality above and arrive at a contradiction to (\ref{eqn: init observation exp needle conv body}).

		\item Therefore, $\gamma=1$. We then have
		\[ \ell_n(t)^n = \left((1 - (1-\ell_n(t)))^{1/(1-\ell_n(t))}\right)^{n(1-\ell_n(t))}. \]
		The inner expression goes to $1/e$. If $\gamma'=\infty$, then we again get $\ell_n(t)^n\to 0$ for $t < b_0$, so we arrive at a contradiction similar to the first case above. Otherwise, we have
		\[ \ell_n(t) \to e^{\gamma'(t-b_0)/(b_0-a_0)}. \]
		Letting $\gamma''=\gamma'/(b_0-a_0)$ and letting $n\to\infty$, we get
		\[ \left(\int_{a_0}^{b_0} e^{\gamma''(t-b_0)} f_1(t)\d{t}\right)^\alpha \left(\int_{a_0}^{b_0} e^{\gamma''(t-b_0)} f_2(t)\d{t}\right)^\beta > \left(\int_{a_0}^{b_0} e^{\gamma''(t-b_0)} f_3(t)\d{t}\right)^\alpha \left(\int_{a_0}^{b_0} e^{\gamma''(t-b_0)} f_4(t)\d{t}\right)^\beta. \]
		However, this (after multiplying by $e^{\gamma''b_0(\alpha+\beta)}$ on either side to remove the $b_0$ in the exponent) contradicts the original assumption that the opposite inequality holds for any exponential needle, thus completing the proof.
	\end{itemize}
\end{proof}

The next result is essentially a generalized version of the above lemma, so is relatively straight-forward to prove since we have various tools for localization in our repertoire at this point.

\begin{ftheo}
	\label{generalized exponential needle interconversion}
	Let $f_1$, $f_2$, $f_3$, and $f_4$ be non-negative functions on $\Rn$ and $\alpha,\beta>0$. The following are equivalent.
	\begin{itemize}
		\item For every log-concave function $F$ on $\Rn$ with compact support,
			\[ \left(\int_{\Rn} F(t) f_1(t)\d{t}\right)^\alpha \left(\int_{\Rn} F(t) f_2(t)\d{t}\right)^\beta \leq \left(\int_{\Rn} F(t) f_3(t)\d{t}\right)^\alpha \left(\int_{\Rn} F(t) f_4(t)\d{t}\right)^\beta. \]
		\item For every exponential needle $E$ in $\Rn$,
			\[ \left(\int_E f_1\right)^\alpha \left(\int_E f_2\right)^\beta \leq \left(\int_E f_3\right)^\alpha \left(\int_E f_4\right)^\beta. \]
	\end{itemize}
\end{ftheo}
\begin{proof}
	Going from the first to the second isn't too difficult. Given the exponential needle over $[a,b]$ and constant $\gamma$, consider the function $F$ defined by $t\mapsto e^{\gamma \langle t, u\rangle}$, where $u=(b-a)/\norm{b-a}$ restricted to some $\varepsilon$-neighbourhood of $[a,b]$. Letting $\varepsilon\to 0$, we get the required.\\
	On the other hand, let the second hold but not the first for some function $F$. Then applying \Cref{lem: 5.1} on the $Ff_i$, we get some $[a,b]$ and linear function $\ell$ on $[a,b]$ such that
	\begin{multline*}
		\left(\int_0^{\norm{b-a}} f_1(a+tu) F(a+tu)\ell(a+tu)^{n-1}\d{t}\right)^\alpha \left(\int_0^{\norm{b-a}} f_2(a+tu) F(a+tu)\ell(a+tu)^{n-1}\d{t}\right)^\beta \\
		> \left(\int_0^{\norm{b-a}} f_3(a+tu) F(a+tu)\ell(a+tu)^{n-1}\d{t}\right)^\alpha \left(\int_0^{\norm{b-a}} f_4(a+tu) F(a+tu)\ell(a+tu)^{n-1}\d{t}\right)^\beta,
	\end{multline*}
	where $u$ has the usual meaning of $(b-a)/\norm{b-a}$.

	However, $F\ell^{n-1}$ is log-concave, so by \Cref{localized exponential needle interconversion}, there exists an exponential needle that violates the assumption.
\end{proof}

\subsubsection{An Example Using the Equivalences}

Let $K$ be a convex body and $f:K\to\R$ be integrable. Define its $L_p$ norm by
\[ \norm{f}_p = \left(\frac{1}{\vol K} \int_{K} |f(x)|^p \d{x} \right)^{1/p}. \]
It is easy to see that if $0<p<q$, $\norm{f}_p \leq \norm{f}_q$.
% Use H\"{o}lder's
% \int (f^p)^{1/p} \leq (\int (f^p)^{q/p})^{p/q} \norm{1}_{1/(1-p/q)}, which is exactly what you want after shifting around some stuff
\begin{theorem}
	Let $0<p<q$. There exists a constant $c_{p,q}$ such that for any dimension $n$, convex body $K\subseteq\Rn$ and linear function $f:K\to\R$,
	\[ \norm{f}_q \leq c_{p,q}\norm{f}_p \]
\end{theorem}
\begin{proof}
	We wish to show that for any $K$,
	\[ \left(\int_K |f|^q\right)^{1/q} \left(\int_K 1\right)^{1/p} \leq c_{p,q} \left(\int_K 1\right)^{1/q} \left(\int_K |f|^p\right)^{1/p}. \]
	Equivalently, we wish to show that for any exponential needle $E$,
	\[ \left(\int_E |f|^q\right)^{1/q} \left(\int_E 1\right)^{1/p} \leq c_{p,q} \left(\int_E 1\right)^{1/q} \left(\int_E |f|^p\right)^{1/p}. \]
	That is, we wish to show that for any linear function $f$, $a,b\in\R$, and real $\gamma$,
	\[ \left( \frac{\int_a^b e^{\gamma t} |f(t)|^q\d{t}}{\int_a^b e^{\gamma t}\d{t}} \right)^{1/q} \leq c_{p,q} \left( \frac{\int_a^b e^{\gamma t} |f(t)|^p\d{t}}{\int_a^b e^{\gamma t}\d{t}}\right)^{1/p}, \]
	Since $f$ is linear, we may assume without loss of generality that $f(a+tu)=t$ on $[a,b]$ and that $\gamma=1$; for the general case where $\gamma\neq 0$, we can just substitute appropriately. The cases where $\gamma=0$ or $f$ is constant on $[a,b]$ are easily shown.\\
	\[ \varphi(a,b) = \left( \frac{\int_a^b e^{t} |f(t)|^q\d{t}}{\int_a^b e^{t}\d{t}} \right)^{1/q} \left( \frac{\int_a^b e^{t} |f(t)|^p\d{t}}{\int_a^b e^{t}\d{t}} \right)^{-1/p}. \]
	We wish to show that $c_{p,q} = \sup_{a<b} \varphi(a,b)$ is finite. Note that $\varphi$ is continuous for $a<b$. Further, for any $\alpha$, $\varphi(a,b)\to 1$ as $a,b\to\alpha$. That is, we may extend the function continuously to $a\leq b$ defining $\varphi(a,a)=1$.\\
	Now, observe that for fixed $a$, as $b\to\infty$, $\varphi(a,b)\to 1$.\footnote{$\int_a^b e^t |f(t)|^p\d{t}$ grows as $e^{-b}b^p$ and $\int_a^b e^t\d{t}$ grows as $e^b$.} On the other hand, for fixed $b$ and $a\to\infty$, $\varphi(a,b)$ remains bounded. The continuity implies that $\varphi$ is bounded (and its supremum is finite).
\end{proof}

The actual calculation of the supremum above is quite tedious, however.


\subsubsection{Isotropy}

The content of this section is closely related to that of volume computation, primarily \Cref{pro sandwiching}, which discussed sandwiching.

Given a convex body $K\subseteq\Rn$ and $f:K\to\R^m$, denote by $\expec_K(f)$ the ``average of $f$ over $K$''. That is,
\[ \expec_K(f) = \frac{1}{\vol(K)} \int_{K} f(x)\d{x}. \]
Denote by $b(K)=\expec_K(x)$ the center of gravity of $K$, also known as the ``baricenter'' of $K$. If $K$ is clear from context, we often denote it as just $b$. Denote by $A(K)$ the $n\times n$ matrix of inertia
\[ A(K) = \expec_K((x-b)(x-b)^\top). \]
Denote by $M_p(K)$ the $p$th moment of $K$
\[ M_p(K) = \expec_K\left(\norm{x-b}^p\right). \]
It is seen that $M_2(K)$ is the trace of $A(K)$. Further, the average squared distance between points in $K$ is
\[ \frac{1}{\vol(K)^2} \int_K \int_K \norm{x-y}^2\d{x}\d{y} = 2M_2(K). \]
It is seen that as $p\to\infty$, $M_p(K)^{1/p}$ converges to $\sup_{x\in K} \norm{x-b}$.

\begin{fdef}
	A body $K$ is said to be in \textit{isotropic position} if $b=0$ and $A(K)=I$, the identity matrix.\footnotemark
\end{fdef}
\footnotetext{Some texts use $\vol(K)=1$ and $A(K)=\lambda_K I$ for some constant $\lambda_K$. It remains an open problem as to whether the value of $\lambda_K$ across convex bodies $K\subseteq\Rn$ is bounded above.}

It may be shown the affine family of a convex body (the set of its image under affine transformations) has a unique body in isotropic position.\\
First, let us show how isotropic position is related to sandwiching.

\begin{ftheo}
	If $K$ is in isotropic position, then
	\[ \sqrt{\frac{n+2}{n}} B_2^n \subseteq K \subseteq \sqrt{n(n+2)}B_2^n. \]
\end{ftheo}

Observe that these inequalities are tight for the regular simplex and also imply the second part of \Cref{fritz john banach mazur distance}. If $K$ is in isotropic position, then for any unit $u$,
\[ \int_K \langle u,x\rangle^2 \d{x} = \vol(K). \]

\begin{proof}
	\phantom{agh}
	\begin{itemize}
		\item Suppose that $\sqrt{(n+2)/n}B_2^n\not\subseteq K$. Choosing our basis appropriately, we may assume that $K$ is contained in the half-space $x_1 > -\sqrt{(n+2)/n}$. Now, we have
		\[ \int_K x_1 = 0 \text{ and } \int_K (x_1^2 - 1) = 0. \]
		Using \Cref{cor: if positive on convex set convex on some needle} (or rather, an extension of it with a weak inequality on $\int g$), we get some needle $N=([a,b],\ell)$. We may assume that $[a,b]$ is contained in the $x_1$ axis. so that
		\[ \int_a^b x_1\ell(x_1)^{n-1} = 0 \text{ and } \int_a^b x_1^2\ell(x_1)^{n-1} \geq \int_a^b \ell(x_1)^{n-1}. \]
		We have $a > -\sqrt{\frac{n+2}{n}}$. It is easy to see\footnote{If $|b|>|a|$, then the first equality implies that $\ell$ cannot be increasing. Otherwise, we can use the second inequality to justify the assumption.} that we may assume that $\ell$ is decreasing, and thus may suppose that is of the form $t-x$ for some $\lambda\geq b$. We can then manually (and tediously) compute the integrals to arrive at a contradiction.

		\item Let $v$ be the point in $K$ furthest from $0$ (assume that $K$ is closed so this is well-defined). We wish to show that $\norm{v}\leq\sqrt{n(n+2)}$. Let $v^{\circ} = v/\norm{v}$ and for each unit $u$, let $\varphi(u) = \sup\{t\geq 0 : v+tu\in K\}$. Then,
		\[ \vol(K) = \int_{\partial B_2^n} \int_{0}^{\varphi(u)} t^{n-1}\d{t}\d{u} = \int_{\partial B_2^n} \frac{\varphi(u)^n}{n} \d{u}. \]
		We also have
		\begin{align*}
			1 &= \frac{1}{\vol(K)} \int_K \langle v^\circ, x\rangle^2 \d{x} \\
			 &=  \frac{1}{\vol(K)} \int_{\partial B_2^n} \int_0^{\varphi(u)} t^{n-1} \langle v^\circ, v+tu\rangle^2 \d{t}\d{u} \\
			 &= \frac{1}{\vol(K)} \int_{\partial B_2^n} \left( \frac{\varphi(u)^n}{n}\norm{v}^2 + 2\frac{\varphi(u)^{n+1}}{n+1} \langle v^\circ, u\rangle + \frac{\varphi(u)^{n+2}}{n+2} \langle v^\circ, u\rangle^2 \right) \d{u} \\
			 &= \frac{1}{\vol(K)} \int_{\partial B_2^n} \left( \frac{\varphi(u)^n}{n} \left(\frac{\sqrt{n(n+2)}}{n+1}\norm{v} + \sqrt{\frac{n}{n+2}}\varphi(u) \langle v^\circ, u\rangle\right)^2 + \frac{\varphi(u)^n}{n(n+1)^2}\norm{v}^2 \right) \d{u} \\
			 &= \frac{1}{\vol(K)} \left( \int_{\partial B_2^n} \frac{\varphi(u)^n}{n} \left(\frac{\sqrt{n(n+2)}}{n+1}\norm{v} + \sqrt{\frac{n}{n+2}}\varphi(u)\langle v^\circ, u\rangle\right)^2 \d{u} \right) + \frac{\norm{v}^2}{(n+1)^2}
		\end{align*}
		Note that this gives a bound of $\norm{v}\leq n+1$. To get the bound mentioned in the theorem, it remains to bound the integral by a suitable positive quantity. Now, we have
		\begin{align*}
			0 = b(K) &= \frac{1}{\vol(K)} \int_{\partial B_2^n} \int_{0}^{\varphi(u)} t^{n-1} (v+tu) \d{t}\d{u} \\
			 &= \frac{1}{\vol(K)} \int_{\partial B_2^n} \frac{\varphi(u)^n}{n}v + \frac{\varphi(u)^{n+1}}{n+1}u \d{u} \\
			 &= v + \frac{1}{\vol(K)} \int_{\partial B_2^n} \frac{\varphi(u)^{n+1}}{n+1}u\d{u}.
		\end{align*}
		Therefore,
		\begin{multline*}
			\frac{1}{\vol(K)} \int_{\partial B_2^n} \frac{\varphi(u)^n}{n} \left(\frac{\sqrt{n(n+2)}}{n+1}\norm{v} + \sqrt{\frac{n}{n+2}}\varphi(u) \langle v^\circ,u\rangle\right) \d{u} \\
			= \left(\frac{\sqrt{n(n+2)}}{n+1} - \frac{n+1}{\sqrt{n(n+2)}}\right)\norm{v} = -\frac{1}{(n+1)\sqrt{n(n+2)}}\norm{v}.
		\end{multline*}
		We can then use the Cauchy-Schwarz inequality to get
		\[ \left( \frac{1}{\vol(K)} \int_{\partial B_2^n} \frac{\varphi(u)^n}{n} \left(\frac{\sqrt{n(n+2)}}{n+1}\norm{v} + \sqrt{\frac{n}{n+2}}\varphi(u)\langle v^\circ, u\rangle\right)^2 \d{u} \right) \cdot 1 \geq \frac{1}{(n+1)^2n(n+2)}\norm{v}^2. \]
		That is,
		\[ 1 \geq \left(\frac{1}{(n+1)^2n(n+2)} + \frac{1}{(n+1)^2}\right)\norm{v}^2 = \frac{\norm{v}^2}{n(n+2)}, \]
		proving the result.
	\end{itemize}	
\end{proof}

\subsubsection{The KLS Conjecture}

Let us now move on to the main result of this section.

\begin{ftheo}
	\label{isoperimetric coefficient bound 1}
	For any convex body $K$,
	\[ \psi(K) \geq \frac{\ln 2}{M_1(K)}. \]
\end{ftheo}

In \Cref{def: isoperimetric coefficient}, let $K_3$ be the intersection of $K$ with the open $\varepsilon/2$-neighbourhood of $\partial S$. Further, let $K_1 = S\setminus K_3$ and $K_2 = (K\setminus S)\setminus K_3$. Then, it suffices to prove the following, which is yet another improvement of \Cref{conductance isoperimetric inequality,improvement of conductance isoperimetric inequality}.

\begin{ftheo}
	Let $K$ be a convex body and $K = K_1\cup K_2\cup K_3$ a decomposition of $K$ into three measurable sets such that $d(K_1,K_2)=\varepsilon>0$. Then
	\[ \vol(K_1)\vol(K_2) \leq \frac{M_1(K)}{\varepsilon\ln 2} \vol(K)\vol(K_3). \]
\end{ftheo}
\begin{proof}
	We may assume that $K_1$ and $K_2$ are closed. Assume that $b(K)=0$. Let $f_1$, $f_2$, and $f_3$ be the indicator functions on $K_1$, $K_2$, and $K_3$ respectively and $f_4(x) = \norm{x}/\varepsilon\ln 2$. We then wish to show that
	\[ \int_K f_1 \int_K f_2 \leq \int_K f_3 \int_K f_4. \]
	By \Cref{localized exponential needle interconversion}, it suffices to show that for any exponential needle $E$,
	\[ \int_E f_1 \int_E f_2 \leq \int_E f_3 \int_E f_4. \]
	Let $E$ be an arbitrary exponential defined by $[a,b]$ and $\gamma$. As before, we may assume that $\gamma=1$ by rescaling appropriately. The case $\gamma=0$ is taken care of by going to the appropriate limits.\\
	First of all, we may assume that $0\in[a,b]$. Indeed, otherwise, we can move the body such that $0$ goes to the point on $[a,b]$ closest to it initially. Then the integral of $f_4$ decreases while the others remain the same, so proving it for this case suffices.\\
	So let us restate the problem in the one-dimensional case that we have reduced it to. Let $[a,b]$ be an interval, $u\in[a,b]$ and $[a,b]=J_1\cup J_2 \cup J_3$ be a decomposition of $[a,b]$ into three measurable sets, where $d(J_1,J_2)\geq\varepsilon>0$. We wish to show that
	\[ \int_{J_1} e^t\d{t} \int_{J_2} e^t\d{t} \leq \int_{J_3} e^t\d{t} \int_a^b \frac{|t-u|}{\varepsilon\ln 2}e^t\d{t}. \]
	Here, each $J_i$ corresponds to the intersection of $K_i$ with the interval and $u$ corresponds to the position of $0$ in $[a,b]$. Let us first prove the result for the case where $J_3$ is a single interval. Let $a\leq s< s+\varepsilon\leq b$ (Why does it suffice to prove it for the case where the interval is of length $\varepsilon$?). Then we claim that
	\[ \int_a^s e^t\d{t} \int_{s+\varepsilon}^b e^t\d{t} \leq \int_s^{s+\varepsilon}e^t\d{t} \int_a^b \frac{|t-u|}{\varepsilon\ln 2}e^t\d{t}. \]
	Equivalently,
	\[ \int_a^s e^t\d{t} \int_\varepsilon^{b-s} e^t\d{t} \leq \int_0^\varepsilon e^t\d{t} \int_a^b \frac{|t-u|}{\varepsilon\ln 2}e^t\d{t}. \]
	Now, note that the expression on the left is maximized when $s=(a+b-\varepsilon)/2$ and that on the right is minimized when $u=\ln((e^a+e^b)/2)$. Substituting these values on each side and simplifying, it suffices to show that
	\[ (e^{(b-a)/2} - e^{\varepsilon/2})^2 \leq \frac{1}{\ln 2} \frac{e^\varepsilon-1}{\varepsilon} \left(-\ln\left(\frac{e^{a-b}-1}{2}\right)e^{b-a} + \ln\left(\frac{e^{b-a}-1}{2}\right)\right). \]
	On decreasing $\varepsilon$, the left increases whereas the right decreases. Therefore, it suffices to prove the above in the limit case where $\varepsilon=0$. Letting $z=e^{(b-a)/2}\geq 1$, we want to prove that
	\[ \ln 2 (z - 1)^2 + z^2\ln\left(\frac{z^{-2}-1}{2}\right) - \ln\left(\frac{z^2-1}{2}\right) \leq 0. \]
	This is a computational task and is not too difficult.\footnote{Show that the function $f$ on the left is monotone decreasing and use the fact that $f(1)=0$.}\\

	For the general case, let $[c_i,d_i]$ be maximal intervals in $J_3$ for $1\leq i\leq k$. They are each of length at least $\varepsilon$. Then we get
	\[ \sum_{i=1}^k \int_a^{c_i} e^t\d{t} \int_{d_i}^b e^t\d{t} \leq \int_{J_3}\d{t} \int_a^b \frac{|t-u|}{\varepsilon\ln 2}e^t\d{t}. \]
	We then have
	\[ \sum_{i=1}^k \int_a^{c_i} e^t\d{t} \int_{d_i}^b e^t\d{t} \geq \int_{J_1} e^t\d{t} \int_{J_2} e^t\d{t}, \]
	completing the proof.
\end{proof}


Let $K$ be an arbitrary convex body and for each $x\in K$, let $\chi_K(x)$ denote the longest segement in $K$ that has midpoint $x$. Let
\[ \chi(K) = \frac{1}{\vol(K)} \int_K \chi_K(x). \]
Note that $\chi(K)=\diam(K \cap (2x-K))$.

\begin{theorem}
	\label{isoperimetric coefficient bound 2}
	For any convex body $K$,
	\[ \psi(K) \geq \frac{1}{\chi(K)}. \]
\end{theorem}
\begin{proof}
	As before, it is equivalent to show that for any decomposition $K=K_1\cup K_2\cup K_3$, where $d(K_1,K_2)=\varepsilon>0$,
	\[ \vol(K_1)\vol(K_2) \leq \frac{1}{\varepsilon}\vol(K_3) \int_K \chi_K(x)\d{x}. \]
	The proof of this is very similar to the that of the previous theorem. It suffices to show that for any interval $[a,b]$ and any decomposition $[a,b]=J_1\cup J_2\cup J_3$ into three measurable sets such that $d(J_1,J_2)\geq \varepsilon$,
	\[ \int_{J_1} e^t\d{t} \int_{J_2} e^t\d{t} \leq \frac{1}{\varepsilon} \int_{J_3} e^t\d{t} \int_a^b \min\{t-a, b-t\}e^t\d{t}. \]
	Similar to earlier, this can be shown without too much difficulty in the case where $J_3$ is a single interval, and similarly extending it to the general case.
\end{proof}

The two bounds \Cref{isoperimetric coefficient bound 1,isoperimetric coefficient bound 2} are not comparable however. For example, \Cref{isoperimetric coefficient bound 1} gives $\psi(K)=\Omega(n^{-1/2})$ for any body in isotropic position whereas \Cref{isoperimetric coefficient bound 2} gives $\Omega(1)$ for the isotropic ball and $\Omega(n^{-1})$ for the isotropic simplex.\\

For any convex body $K$, let $\alpha(K)$ be the largest eigenvalue of $A(K)$. 

\begin{theorem}
	\label{kls conj hyperplane}
	For any convex body $K$,
	\[ \psi(K) \leq \frac{10}{\sqrt{\alpha(K)}}. \]
\end{theorem}

This is proved using the following result.

\begin{theorem}
	Let $K$ be a convex body in $\Rn$ and assume that $b(K)=0$. Let $u\in\Rn$ have unit norm and $\beta=\expec_K(\langle u,x\rangle^2)$. Then
	\[ \vol(K \cap \{x : \langle u,x\rangle < 0\}) \vol(K \cap \{x : \langle u,x\rangle > 0\}) \geq \frac{1}{10}\sqrt{\beta}\vol(K)\vol_{n-1}(K \cap \{x : \langle u,x\rangle = 0\}). \]
\end{theorem}

The above can be proved by projecting the body onto the $u$-axis and considering the resulting log-concave function (using \nameref{brunn's theorem}).

\begin{fcon}[KLS Conjecture]
	\label{con: kls conjecture}
	$\psi(K) = \Theta(\alpha(K)^{-1/2})$.
\end{fcon}

\Cref{kls conj hyperplane} and the \nameref{con: kls conjecture} together assert that up to a constant factor, a hyperplane cut is the ``worst'' cut (involved in the isoperimetric coefficient).\\

\cite{chen2021constant} has made the most progress in recent times, proving that there is a constant $c$ such that
\[ \psi(K) \geq \frac{1}{\sqrt{ n^{c\left(\log\log n/\log n\right)^{1/2}} \alpha(K)}}. \]