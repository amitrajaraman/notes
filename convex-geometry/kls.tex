\section{The KLS Conjecture}

\subsection{An Isoperimetric Problem}

\subsubsection{Introduction}

Suppose we have a convex body $K$ and we want to find a surface that divides $K$ into two parts, whose measure is minimum relative to that of the two parts.

\begin{fdef}
	The \textit{isoperimetric coefficient} of a convex body $K\subseteq\Rn$ is the largest number $\psi=\psi(K)$ such that for any measurable $S\subseteq K$,
	\[ \vol_{n-1}(\partial S) \geq \psi \frac{\vol(S)\vol(K\setminus S)}{\vol(K)}. \]
	Often, the above is replaced with
	\[ \vol_{n-1}(\partial S) \geq \psi \min\{\vol(S), \vol(K\setminus S)\} \]
	and this is the form we shall use throughout the rest of the section.
\end{fdef}

Since the two $\psi$s are within a factor of $2$, estimates are not influenced very drastically on making this change. This problem turns out to be very intimately related to that of volume computation we explored in the previous section.\\
\cite{lovasz-simonovits-mixing-rate-isoperimetric} bounds the isoperimetric coefficient below by $1/d$, where $d$ is the diameter of the body. Note that this is quite obvious if the separating surface $\partial S$ is a (section of a) hyperplane.\\
\cite{applegate-kannan-cube-sandwich} gives a more general result where the measure is replaced by that with density equal to any log-concave function and bounds it below by $2/d$. This is in fact as tight as we can get in terms of the diameter and indeed, the bound is attained for a thin long cylinder.\\
However, the bodies we are interested in (in say, volume computation) tend to have a certain structure to them. In particular, sandwiching makes the bodies somewhat round.\\
Throughout the rest of this section, denote the average distance of a point $K$ from its center of gravity by $M_1(K)$. The main result of this section is that for every convex body $K$,
\[ \psi(K) \geq \frac{\ln 2}{M_1(K)}. \]

To begin, consider the following motivated by \Cref{localization lemma}.

\begin{definition}
	A \textit{needle} is a segment $[a,b]\in\Rn$ together with a non-negative linear function $\ell:I\to\R^{\geq 0}$ not identically $0$. If $N=(I,\ell)$ is a needle and $f$ is an integrable function defined on $I$, denote
	\[ \int_N f = \int_{0}^{|b-a|} f(a+tu)\ell(a+tu)^{n-1} \d{t}, \]
	where $u=(b-a)/|b-a|$.
\end{definition}

\begin{lemma}
	\label{lem: 5.1}
	Let $f_1$, $f_2$, $f_3$, $f_4$ be non-negative continuous functions defined on $\Rn$ and $\alpha,\beta>0$. The following are equivalent.
	\begin{itemize}
		\item For every convex body $K$ in $\Rn$,
			\[ \left(\int_K f_1\right)^\alpha \left(\int_K f_2\right)^\beta \leq \left(\int_K f_3\right)^\alpha \left(\int_K f_4\right)^\beta. \]
		\item For every needle $N$ in $\Rn$,
			\[ \left(\int_N f_1\right)^\alpha \left(\int_N f_2\right)^\beta \leq \left(\int_N f_3\right)^\alpha \left(\int_N f_4\right)^\beta. \]
	\end{itemize}
\end{lemma}
\begin{proof}
	The first implying the second is quite easy. For the converse, suppose that the second holds but the first does not.
	Adding a sufficiently small quantity to $f_3$ and $f_4$, we may further assume that they are (strictly) positive. We may also assume that $f_1$ and $f_2$ are positive (Why?). Choose some $A$ such that
	\[ \frac{\left(\int_K f_1\right)^\alpha}{\left(\int_K f_3\right)^\alpha} > A > \frac{\left(\int_K f_4\right)^\beta}{\left(\int_K f_2\right)^\beta}. \]
	Then,
	\[ \int_K f_1 - A^{1/\alpha}f_3 > 0 \text{ and } \int_K A^{1/\beta}f_2 - f_4 > 0. \]
	Using \Cref{localization lemma}, there is some needle $N$ such that
	\[ \int_N f_1 - A^{1/\alpha}f_3 > 0 \text{ and } \int_N A^{1/\beta}f_2 - f_4 > 0. \]
	This implies that
	\[ \frac{\left(\int_N f_1\right)^\alpha}{\left(\int_N f_3\right)^\alpha} > A > \frac{\left(\int_N f_4\right)^\beta}{\left(\int_N f_2\right)^\beta}, \]
	thus proving the claim.
\end{proof}

Observe that we can extend this more generally to the case where $f_1$ and $f_2$ are upper semicontinuous and $f_3$ and $f_4$ are lower semicontinuous. Indeed, we may consider an appropriate sequence of continuous function. In particular, this allows us to restrict ourselves from $\Rn$ to some subset $T$ of $\Rn$ by multiplying the functions with the indicator function $\indic_T$ (the functions extend to upper semicontinuous functions if $T$ is closed and lower semicontinuous functions if $T$ is open).
% *** Why?

\begin{corollary}
	Let $T$ be a bounded open convex set in $\Rn$, $g$ a bounded lower semicontinuous function on $T$, and $h$ a continuous function on $T$ such that
	\[ \int_T g > 0 \text{ and } \int_T h = 0. \]
	Then there is a needle $N=(I,\ell)$ with $I\subseteq T$ such that
	\[ \int_N g > 0 \text{ and } \int_N h = 0. \]
\end{corollary}
\begin{proof}
	Choose some $0<\delta<\int_T g$ and let $\varepsilon>0$. Then,
	\[ \int_T \left(g-\delta+\frac{1}{\varepsilon}h\right) > 0 \text{ and } \int_T (\varepsilon-h) > 0. \]
	Extending these functions to functions on $\Rn$ (multiplying with the indicator function) and using \Cref{localization lemma}, we get a needle $N_\varepsilon=(I_\varepsilon,\ell_\varepsilon)$ with $I_\varepsilon\subseteq T$ (Why?) such that
	\begin{equation}
		\label{eqn: 5.1}
		\int_{N_\varepsilon} \left(g-\delta+\frac{1}{\varepsilon}h\right) > 0 \text{ and } \int_{N_\varepsilon} (\varepsilon-h) > 0.
	\end{equation}
	Letting $M$ to be the supremum of $g$ on $\Rn$,
	\begin{equation}
		\label{eqn: 5.2}
		-M\varepsilon\int_{N_\varepsilon} 1 < \int_{N_\varepsilon} h < \varepsilon \int_{N_\varepsilon} 1.
	\end{equation}
	Consider these needles for $\varepsilon=1/k$ ($k\in\N$). Scaling appropriately, we may assume that the maximum of each $\ell_{1/k}$ is $1$. Using the Bolzano-Weierstrass Theorem, there is some subsequence of these needles that converges (in the sense that the endpoints of the $I_{1/k}$ and the $\ell_{1/k}$ converge)\footnote{We can think of an needle $N=([a,b],\ell)$ as an element $(a,b,\ell(a),\ell(b)-\ell(a))\in\R^{2n+2}$. In our case, this sequence is bounded because each interval is in the bounded set $T$ and each $\ell$ is between $0$ and $1$.} to some needle $N=(I,\ell)$. Combining \Cref{eqn: 5.1,eqn: 5.2} implies that $N$ satisfies the required.
\end{proof}

While these results are quite nice, exponents of a linear function are not very convenient to deal with. Therefore, define the following.

\begin{definition}
	An \textit{exponential needle} is a segment $[a,b]\in\Rn$ together with a real $\gamma$. If $E=(I,\gamma)$ is a needle and $f$ is an integrable function defined on $I$, denote
	\[ \int_E f = \int_{0}^{|b-a|} f(a+tu)e^{\gamma t} \d{t}, \]
	where $u=(b-a)/|b-a|$.
\end{definition}

If we manage to prove our results for an exponential needle instead, it is extremely convenient because taking exponents does not change the underlying structure of the function itself.