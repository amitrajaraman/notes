\section{Some more randomized algorithms}

\subsection{Bipartite matching}

	\subsubsection{Lecture 20}

		The problem we shall study now is that of finding a perfect matching in a bipartite graph $G = (X,X,E)$. That is, we have two copies of a set $X$ with all edges between the two copies.

		This is a problem as old as computer science itself, and quite recently an almost-linear time algorithm for the above has been found \cite. % m polylog(m)

		We shall give an algebraic algorithm due to Mulmuley, Vazirani, Vazirani \cite{matching-mvv}. It is rather simple, and is also parallelizable. Consider the \emph{biadjacency matrix} $A_G$ of $G$, with rows and columns indexed by $X$ and $(A_G)_{uv} = 1$ if $uv$ is an edge and $0$ otherwise. Note that the $X$ used to index the rows and columns are different (choose which one is used for rows arbitrarily).

		Recall the determinant
		\[ \det(M) = \sum_{\sigma \in S_n} \sign(\pi) \prod_{i=1}^n M_{i,\sigma(i)} \]
		and permanent
		\[ \perm(M) = \sum_{\sigma \in S_n} \prod_{i=1}^n M_{i,\sigma(i)}. \]
		Suppose that the vertex sets $X$ in our bipartite graph are $[n]$.\\
		Note that any perfect matching essentially corresponds to a permutation of $[n]$ such that there is an edge between $i$ and $\sigma(i)$ for every $i \in [n]$, that is, $(A_G)_{i,\sigma(i)} = 1$ for all $i$. Due to this, we can also assign a sign to any perfect matching. \\

		Clearly, the number of perfect matchings is then just $\perm(A_G)$. On the other hand,
		\[ \det(A_G) = \sum_{M \text{ is a perfect matching}} \sign(M). \]
		Therefore, if $G$ does not have a perfect matching, $\det(A_G) = 0$. The converse is clearly not true as seen by $K_{2,2}$, which has biadjacency matrix
		\[ \begin{pmatrix} 1 & 1 \\ 1 & 1 \end{pmatrix}. \]
		Similarly, the determinant is $0$ if any two vertices have the same neighbour set.\\
		How do we change something to make the converse hold true (with high probability)? The idea is rather simple, and involves changing by biadjacency matrix by replacing each element with a random integer from $\{1,2,\ldots,2n\}$. Call this new (random) matrix $M_G$. We claim that in this new setting, $\det(M_G) \ne 0$ with probability at least $1/2$.\\
		Indeed, consider the determinant polynomial in $n^2$ variables, which is
		\[ \det\begin{pmatrix} x_{11} & x_{12} & \cdots & x_{1n} \\ x_{21} & x_{22} & \cdots & x_{2n} \\ \vdots & \vdots & \ddots & \vdots \\ x_{n1} & x_{n2} & \cdots & x_{nn}  \end{pmatrix}. \]
		Note that this is a degree $n$ polynomial. 

		\begin{flem}[Polynomial Identity Lemma]
			Let $p$ be a polynomial in $m$ variables of degree $d$. Then,
			\[  \Pr_{\alpha \sim \{1,2,\ldots,2d\}^{m}} [p(\alpha) \ne 0] \ge \frac{1}{2}. \]
		\end{flem}
		We have already seen a proof of this back in Lecture 18 (in the case where coefficients are rational), where we worked with $\F_p$ instead. Indeed, something being nonzero modulo $p$ implies nonzeroness in $\R$.\\
		An alternative proof is by induction on the number of variables.\\

		Using this lemma in our setting, we see that $\det(M_G) = 0$ with probability at most $1/2$, so we are done. Further, we can use this algorithm to actually find a perfect matching. For $i \in [n]$, assuming that the perfect matching has the edge $1i$, check if the remaining part of the graph has a perfect matching. If yes, find a perfect matching on it (recursively). Otherwise, increment $i$.\\

		Now, can we come up with a \emph{parallel} algorithm for constructing a perfect matching? Assuming we have polynomially many machines that run independently, is it possible to determine a perfect matching rapidly, say in constant or logarithmic time?\\

		In the simple case where we have a \emph{unique} perfect matching, this is quite simple by running $m$ many machines parallelly, each computing a determinant of the graph excepting the vertices in an edge $e$. If for a given $e$ the determinant is nonzero, the edge must be in the matching.\\
		The algorithm we shall see has its idea centered around the above observation.\\

		Suppose that we can assign weights to the edges $w : E \to \Z$ such that the minimum weight perfect matching is unique, where the weight of a matching $M$ is
		\[ w(M) = \sum_{e \in M} w(e). \]
		We then alter the biadjacency matrix $A_G$ so that the edge $e$'s entry is $2^{w(e)}$ instead of $1$. Then,
		\[ \det(A_G) = \sum_{\text{perfect matchings $M$}} \sign(M) 2^{w(M)}. \]
		Note that due to the uniqueness, the above determinant is nonzero! It cannot be cancelled by any sum of higher weight matchings (Why?). After that, for each edge, decrement the weight by one and see if the minimum weight has now decreased. If it has, this edge must be part of the minimum weight perfect matching.\\
		All that remains is to find a weight assignment such that there is a unique minimum weight perfect matching. It turns out that a random weight assignment does the trick. This is not immediately clear, because if we assign weights in $\{1,2,\ldots,n^2\}$, say, then despite there being possibly exponentially many matchings, the minimum weight one is unique.

		\begin{flem}[Isolation lemma]
			Assigning each edge random weights in $\{1,2,\ldots,2m\}$, the minimum weight perfect matching is unique with probability at least $1/2$.
		\end{flem}
		Further, this lemma is not particular to perfect matchings and works even for objects such as spanning trees.