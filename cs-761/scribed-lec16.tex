\documentclass{article}
\usepackage[margin=1in]{geometry}
\usepackage{amsmath,amsthm,amssymb}
\usepackage{relsize}
\newcounter{lecnum}
\usepackage{graphicx}
\graphicspath{./}
\usepackage{caption}
\usepackage{subcaption}
\newcommand{\abs}[1]{\lvert #1 \rvert}
\newcommand{\lecture}[4]{
   \newpage
   \setcounter{lecnum}{#1}
   \noindent

   \begin{center}
   \framebox{
      \vbox{\vspace{2mm}
    \hbox to 16cm { {\bf CS761 Derandomization and Pseudorandomness
                        \hfill 2022-23 Sem I} }
       \vspace{4mm}
       \hbox to 16cm { {\Large \hfill Lecture #1: #2  \hfill} }
       \vspace{2mm}
       \hbox to 16cm { {\it Scribe: #4  \hfill  Lecturer: #3} }
      \vspace{2mm}}
   }
   \end{center}
   \vspace*{4mm}
}

\newtheorem{theorem}{Theorem}[lecnum]
\newtheorem{lemma}[theorem]{Lemma}
\newtheorem{proposition}[theorem]{Proposition}
\newtheorem{claim}[theorem]{Claim}
\newtheorem{corollary}[theorem]{Corollary}
\newtheorem{definition}[theorem]{Definition}

% custom
\usepackage{enumitem}
\usepackage{hyperref}
\usepackage{cleveref}
\usepackage{mathtools}
\newcommand{\BPP}{\mathsf{BPP}}
\newcommand{\N}{\mathbb{N}}
\newcommand{\E}{\mathbb{E}}
\newcommand{\PP}{\mathsf{PP}}
\newcommand{\avg}{\text{avg}}
\newcommand{\NW}{\operatorname{NW}}

\begin{document}

\lecture{16}{06-10-2022}{Rohit Gurjar}{Amit Rajaraman}

	We had stated the following at the end of the previous lecture.

	\begin{theorem}[Yao]
		\label{yao}
		Let $D$ be a distribution on $\{0,1\}^m$. Suppose that for any $i$ and any circuit of size $2S$,
		\[ \Pr_{y \sim D} [C(y_1,\ldots,y_i) = y_{i+1}] < \frac{1}{2} + \epsilon. \]
		Then, for any circuit $B$ of size $S$,
		\[ \left| \Pr_{y \sim D} [B(y) = 1] - \Pr_{y \sim U_m}[B(y) = 1] \right| < m\epsilon. \]
	\end{theorem}
	
	\begin{proof}[Proof]
			We shall show the contrapositive of the statement. Let $B$ be a circuit of size $S$ such that
			\[ \Pr_{y \sim D} [B(y) = 1] - \Pr_{y \sim U_m}[B(y) = 1] \ge m\epsilon. \]
			We remove the modulus because if the other inequality is true, we can instead consider the probability that the output value is $0$. Define a sequence of distributions $D_0,D_1,\ldots,D_m$, where $D_i$ is obtained by drawing $x$ from $D$, and then replacing the first $i$ coordinates with draws from the uniform distribution. That is, a draw is $(y_1,\ldots,y_i,z_{i+1},\ldots,z_m)$, where $y \sim D$ and $z \sim U_m$. Note that $D_0 = U_m$ and $D_m = D$, and also that $D_i$ and $D_{i-1}$ differ only at the $i$th bit.\\
			Let
			\[ P_i = \Pr_{r \sim D_i} [B(r) = 1]. \]
			Because $P_m - P_0 \ge m\epsilon$, there is some $i$ such that $P_i - P_{i-1} \ge \epsilon$.\\
			We shall give an algorithm to predict $y_i$ given $y_1,\ldots,y_{i-1}$ (for $y \sim D$). Randomly draw $z \sim U_m$. If $B(y_1,\ldots,y_{i-1},z_{i},\ldots,z_m) = 1$, then output $z_i$, and if it is $0$ then output $1-z_i$. For the sake of succinctness, let $x = (y_1,\ldots,y_{i-1},z_i,\ldots,z_m)$. Now, the probability of success is
			\[ \frac{1}{2} \left( \underbrace{\Pr[B(x) = 1 \mid y_i = z_i]}_{P_i} + \underbrace{\Pr[B(x) = 0 \mid y_i = 1-z_i]}_{(1-\alpha)\text{, say}} \right) \]
			We have
			\[ P_{i-1} = \Pr[B(x) = 1] = \frac{1}{2} \left( \Pr[B(x) = 1 \mid y_i = z_i] + \Pr[B(x) = 1 \mid y_i = 1-z_i] \right) = \frac{1}{2} (P_i + \alpha). \]
			Therefore,
			\[ \text{probability of success} = \frac{1}{2}(P_i + 1-\alpha) = \frac{1}{2} + P_i - P_{i-1} \ge \frac{1}{2} + \epsilon. \]
			To get the final circuit $C$, note that on a random choice of $z \sim U_m$ in our algorithm, we succeed with probability at least $(1/2)+\epsilon$. That is, the expected probability of success is at least $(1/2)+\epsilon$. Therefore, there exists some specific choice which gives a probability of success at least $(1/2)+\epsilon$, which is precisely what we want.
		\end{proof}


		Let us now come to the proof of the Nisan-Wigderson result, which we restate.
		\begin{theorem}[Nisan-Wigderson]
			\label{theo: hard to prg}
			If there exists a function computable in time $2^{O(n)}$ with $H_\avg(f) \ge 2^{2n/3}$, then there exists a $(2^{\ell/45})$-PRG and in particular, $\BPP = \PP$.
		\end{theorem}
	

		The idea is as follows. Inspired by the one-bit extension in the previous lecture, we would like to consider a collection of subsets of $[\ell]$, and apply a hard function $f$ to each of them to get one extra bit to append. In all, the number of bits we append is the number of subsets we choose. If we choose all subsets to be disjoint, then the resulting new bits are completely independent of each other, but we do not get exponentially many new bits. Therefore, we allow some small amount of intersection of the subsets, and thus some small amount of correlation, without compromising the uncorrelation of the new bits by too much.

		\begin{definition}
			An \emph{$(l,k,d)$-combinatorial design} is a collection $I_1,\ldots,I_r$ of size $k$ subsets such that for distinct $i,j \in [r]$, $|I_i \cap I_j| \le d$.
		\end{definition}

		\begin{proposition}
			For $k=\ell/30, d=k/3$, there exists an $(l,k,d)$-design of size at least $2^{d/10} \ge 2^{\ell/900}$.
		\end{proposition}
		We do not prove this.

		\begin{proof}[Proof of \nameref{theo: hard to prg}]
			Set $\ell = 900 \log n$, and $k$ as from the above.\\
			Fix some combinatorial design $\mathcal{I} = \{I_1,\ldots,I_n\}$ guaranteed by the above proposition, and let $f : \{0,1\}^k \to \{0,1\}$ be a hard function. Then, given $z \in \{0,1\}^\ell$, the final pseudorandom bits we output are $f(z_{I_r})$ for each $r \in [n]$. For simplicity, denote $f(I_r) = f(z_{I_r})$. \\
			Let $f$ be computable in time $2^{O(k)}$ and $H_\avg(f) \ge 2^{2k/3}$. Denote the resulting PRG by $\NW_{\mathcal{I}}^f$. We shall show that $\NW_{\mathcal{I}}^f(U_\ell)$ is $(n^{20}/2, 1/10)$-pseudorandom. \\
			Now, we shall use \nameref{yao}, by showing unpredictability instead. That is, we are done if we show that for any circuit $C$ of size at most $n^{20}/2$,
			\[ \Pr_{z \sim U_\ell}\left[ C(f(z_{I_1}),\ldots,f(z_{I_{i-1}})) = f(z_{I_i}) \right] \le \frac{1}{2} + \frac{\epsilon}{n}, \]
			where $\epsilon = 1/10$.\\
			Suppose otherwise, and let $C$ be a circuit violating the above. Let $z' = z_{[\ell] \setminus I_i}$, and $z'' = z_{I_i}$.
			Let $f_j(z) = f(z_{I_j})$ for each $j$. Then,
			\[ \Pr_{z \sim U_\ell}\left[ C(f_1(z',z''),\ldots,f_{i-1}(z',z'')) = f(z'') \right] > \frac{1}{2} + \frac{\epsilon}{n}. \]
			Ignore $z'$ for now, fixing its value as something (this will be done in precisely the same way as in Yao's Theorem), and abuse notation to denote the new functions by $f_j$ as well. Then,
			\[ \Pr_{z \sim U_\ell}\left[ C(f_1(z''),\ldots,f_{i-1}(z'')) = f(z'') \right] > \frac{1}{2} + \frac{\epsilon}{n}. \]
			using this, we get a circuit for $f$ that succeeds with probability at least $(1/2)+\frac{\epsilon}{n}$. Note that each $f_{j}(z'')$ uses at most $d$ bits. By taking $(i-1)$ trivial circuits of $f$, which are each of size at most about $d 2^d$, we get a circuit for $f$ of size $d 2^d 2^{d/10} + 2^{2d}/2 \le 2^{2d} = n^{20}$, contradicting the hardness of $f$.
		\end{proof}

\end{document}