\section{Topological Spaces and Continuous Functions}

\setcounter{subsection}{12}
\subsection{Basis for a Topology}

\begin{exercise}
	Let $X$ be a topological space and $A\subseteq X$. Suppose that for each $x\in A$, there is an open set $U$ containing $x$ such that $U\subseteq A$. Show that $A$ is open in $X$.
\end{exercise}
\begin{solution*}
	For each $x\in A$, denote by $U_x$ an open subset of $A$ that contains $A$. Then $A = \bigcup_{x\in A} U_x$. However, an arbitrary union of open sets is open and thus, so is $A$.
\end{solution*}

\setcounter{exercise}{4}
\begin{exercise}
	Show that if $\mathcal{A}$ is a basis for a topology on $X$, the topology generated by $\mathcal{A}$ equals the intersection of all topologies that contain $\mathcal{A}$. Prove the same if $\mathcal{A}$ is a subbasis.
\end{exercise}
\begin{solution*}
	Let $\mathcal{T}$ be the topology generated by $\mathcal{A}$ and $\mathcal{T}'$ be a topology that contains $\mathcal{A}$. Let $U\in\mathcal{T}$. Then $U=\bigcup_{i\in I} B_i$ for some $(B_i)_{i\in I}$ in $\mathcal{A}$. However, each $B_i$ is also in $\mathcal{T}'$. Since an arbitrary union of open sets is open, $U\in\mathcal{T'}$ as well. Therefore, $\mathcal{T}\subseteq\mathcal{T}'$, proving the result. The solution for the case where $\mathcal{A}$ is a subbasis is very similar and so omitted.
\end{solution*}

\begin{exercise}
	Show that the collection
	\[ \mathcal{B} = \{(a,b) : a<b, a\text{ and }b\text{ are rational}\}. \]
\end{exercise}

\setcounter{subsection}{15}
\subsection{The Subspace Topology}

\begin{exercise}
	Show that if $Y$ is a subspace of $X$ and $A$ is a subset of $Y$, then the topology $A$ inherits as a subspace of $Y$ is the same as the topology it inherits as a subspace of $X$.
\end{exercise}
\begin{solution*}	
	The topology $A$ inherits as a subspace of $X$ is
	\begin{align*}
		\mathcal{T} &= \{ U\cap A : U\text{ open in }X \} \\
			&= \{ (U\cap Y) \cap A : U\text{ open in }X \} \\
			&= \{ V\cap A : V\text{ open in }Y \},
	\end{align*}
	which is the topology it inherits as a subspace of $Y$.
\end{solution*}

\begin{exercise}
	If $\mathcal{T}$ and $\mathcal{T}'$ are topologies on $X$ and $\mathcal{T}'$ is strictly finer than $\mathcal{T}$, what can you say about the corresponding subspace topologies on the subset $Y$ of $X$.
\end{exercise}
\begin{solution*}
	It is easily seen that $\mathcal{T}'_Y$ is finer than $\mathcal{T}_Y$. We further see that it need not be strictly finer by considering the example $X=\{a,b,c\}$, $Y=\{a,b\}$, $\mathcal{T}=\{\emptyset,X,\{a\},\{b\},\{a,b\}\}$, and $\mathcal{T}'$ as the discrete topology on $X$.
\end{solution*}

\begin{exercise}
	Consider $Y=[-1,1]$ as a subspace of $\R$. Which of the following is open in $Y$? Which are open in $\R$?
	\begin{align*}
		A &= \left\{ x : \frac{1}{2} < x < 1 \right\} \\
		B &= \left\{ x : \frac{1}{2} < x \leq 1 \right\} \\
		C &= \left\{ x : \frac{1}{2} \leq x < 1 \right\} \\
		D &= \left\{ x : \frac{1}{2} \leq x \leq 1 \right\} \\
		E &= \left\{ x : 0 < x < 1 \text{ and } 1/x\not\in\Zp \right\}
	\end{align*}
\end{exercise}
\begin{solution*}
	$A$ and $B$ are open in $Y$ and only $A$ is open in $\R$. This is reasonably straightforward to prove.\\
	$C$ is not open in $Y$ (and so not $\R$ either) because there is no basis element $U$ of the order topology such that $1/2 \in U \subseteq C$. A similar argument holds for $D$ as well.\\
	$E$ is open in both $\R$ and $Y$ because it can be written as a union of basis elements
	\[ E = \bigcap_{n\in\Zp} \left(\frac{1}{n+1},\frac{1}{n}\right). \]
\end{solution*}

\begin{exercise}
	A map $f:X\to Y$ is said to be an open map if for every open $U$ of $X$, $f(U)$ is open in $Y$. Show that $\pi_1:X\times Y\to X$ and $\pi_2:X\times Y\to Y$ are open. 
\end{exercise}
\begin{solution*}
	We shall only show that $\pi_1$ is open, the other case is nearly identical. Let
	\[ U = \bigcup_{i\in I} U_i \times V_i \]
	be open in $X\times Y$ for some indexing set $I$, where each $U_i$ and $V_i$ are open in $X$ and $Y$ respectively. Then,
	\[ \pi_1(U) = \pi_1 \left(\bigcup_{i\in I} U_i\times V_i = \bigcup_{i\in I} \pi_1(U_i\times V_i) = \bigcup_{i\in I} U_i \right) \]
	is open in $X$.
\end{solution*}

\setcounter{subsection}{16}
\subsection{Closed Sets and Limit Points}

\begin{exercise}
	Let $\mathcal{C}$ be a collection of subsets of set $X$. Suppose that $\emptyset$ and $X$ are in $\mathcal{C}$ and that finite unions and arbitrary intersections of elements of $\mathcal{C}$ are in $C$. Show that the collection $\mathcal{T}=\{X\setminus C:C\in\mathcal{C}\}$ is a topology on $X$.
\end{exercise}
\begin{solution*}
	Let $(U_i)_{i\in I}$ be in $\mathcal{T}$ with $U_i = X\setminus C_i$ for each $i$. Then
	\[ \bigcup_{i\in I} U_i = X \setminus \bigcap_{i\in I} C_i = X\setminus C \in \mathcal{T} \]
	for some $C\in\mathcal{C}$. Closure under finite intersections is shown similarly. We trivially have $\emptyset,X\in\mathcal{T}$ because $X,\emptyset\in\mathcal{C}$.
\end{solution*}

\begin{exercise}
	Show that if $A$ is closed in $Y$ and $Y$ is closed in $X$, then $A$ is closed in $X$.
\end{exercise}
\begin{solution*}
	Let $U$ be open in $X$ such that $Y\setminus A = U\cap Y$. Then, we can write $A$ as $X\setminus ((X\setminus Y)\cup U)$. Since $X\setminus Y$ and $U$ are open in $X$, $A$ is closed in $X$.
\end{solution*}

\begin{exercise}
	Show that if $A$ is closed in $X$ and $B$ is closed in $Y$, $A\times B$ is closed in $X\times Y$.
\end{exercise}
\begin{solution*}
	Observe that
	\[ (X\times Y) \setminus (A\times B) = ((X\setminus A)\times (Y\setminus B)) \cup ((X\setminus A)\times Y) \cup (A\times (Y\setminus B)). \]
	Since each of the sets on the right are open in $X\times Y$, $A\times B$ is closed.
\end{solution*}

\begin{exercise}
	Show that if $U$ is open in $X$ and $A$ is closed in $X$, $U\setminus A$ is open in $X$ and $A\setminus U$ is closed in $Y$.
\end{exercise}
\begin{solution*}
	This is easily seen on writing $U\setminus A = U \cap (X\setminus A)$ and $A\setminus U = A \cap (X\setminus U)$.
\end{solution*}

\setcounter{exercise}{18}
\begin{exercise}
	If $A\subseteq X$, define the boundary of $A$ by
	\[ \Bd A = \overline{A} \cap \overline{X\setminus A}.  \]
	\begin{enumerate}[(a)]
		\item Show that $A^\circ$ and $\Bd A$ are disjoint, and $\overline{A} = A^\circ \cup \Bd A$.
		\item Show that $\Bd A = \emptyset$ iff $A$ is both open and closed.
		\item Show that $U$ is open iff $\Bd U = \overline{U}\setminus U$.
		\item If $U$ is open, is it true that $U = \overline{U}^\circ$? Justify your answer.
	\end{enumerate}
\end{exercise}
\begin{solution*}
	\begin{enumerate}[(a)]
		\item Let $x\in A\setminus A^\circ$. Then for any open $U\ni x$, $U\not\subseteq A$ (otherwise, $A^\circ\cup U\supsetneq A^\circ$ is open and contained in $A$). That is, $U\cap (X\setminus A) \neq \emptyset$. However, this implies that $x\in \overline{X\setminus A}$, that is, $A\setminus A^\circ\subseteq \overline{X\setminus A}$. Therefore,
		\begin{align*}
			\overline{A}\setminus A^\circ &= (\overline{A}\setminus A) \cup (A\setminus A^\circ) \subseteq \overline{X\setminus A} \\
			\overline{A} &\subseteq A^\circ \cup \overline{X\setminus A} \\
				&= \overline{A} \cap (A^\circ \cup \overline{X\setminus A}) \\
				&= A^\circ \cup (\overline{A} \cup \overline{X\setminus A}) = A^\circ \cup \Bd A.
		\end{align*} 

		\item If $A$ is not closed, $\overline{A}\supsetneq A$ intersects $X\setminus A\subseteq \overline{X\setminus A}$, contradicting $\Bd A = \emptyset$. Similarly, $X\setminus A$ is closed as well, so $A$ is both open and closed.\\
		The other direction is similarly straightforward.

		\item If $U$ is open, $X\setminus U$ is closed so $\Bd U = \overline{U} \cap (X\setminus U) = \overline{U} \setminus U$.\\
		On the other hand, if $\overline{U}\cap (X\setminus U) = \overline{U}\cap \overline{X\setminus U}$, $X\setminus U$ must be closed. Indeed, otherwise, $\overline{X\setminus U} \setminus (X\setminus U) \subseteq U \subseteq \overline{U}$, contradicting the equality.

		\item No, this is not the case. Consider the open set $U=(1,2)\cup(2,3)\subseteq\R$. Then $\overline{U}^\circ=(1,3)$.
	\end{enumerate}
\end{solution*}

\setcounter{exercise}{20}
\begin{exercise}
	Consider the collection of all subsets $A$ of the topological space $X$. The operations of closure $A\mapsto \overline{A}$ and complementation $A\mapsto X\setminus A$ are functions from this collection to itself.
	\begin{enumerate}[(a)]
		\item Show that starting with a given set $A$, one can form no more than $14$ distinct sets by applying these two operations successively.
		\item Find a subset $A$ of $\R$ (in its usual topology) for which the maximum of $14$ is attained.
	\end{enumerate}
\end{exercise}
\begin{solution*}
	% We denote $X\setminus A$ as $A^c$.\\
	First of all, note that since $X\setminus(X\setminus A) = A$ and $\overline{\overline{A}}=\overline{A}$, it suffices to show that there are at most $14$ distinct sets where we alternately take the closure and complementation of $A$ (or rather, some set of a similar form).
	% \[ \overline{\overline{X\setminus \overline{X\setminus \overline{A}}}^{\ldots}}, X\setminus  \overline{X\setminus \overline{X\setminus \overline{X\setminus A}}}^{\ldots}, X\setminus\overline{X\setminus \overline{X\setminus \overline{A}}}^{\ldots} \text{ and } \overline{\phantom{o}^{\ldots}\overline{X\setminus \overline{X\setminus \overline{X\setminus A}}}}. \]
	We claim that for any set $A$,
	\[ \overline{X\setminus\overline{X\setminus\overline{X\setminus\overline{A}}}} = \overline{X\setminus\overline{A}}. \]
	Since $X\setminus\overline{A}$ is open, it suffices to show that for any open set $U$,
	\[ \overline{X\setminus\overline{X\setminus\overline{U}}} = \overline{U}. \]
	Let $x\in \overline{U}$ and $V\ni x$ be open. Observe that $V\cap U\neq \emptyset$. We wish to show that
	\[ V \cap \left(X\setminus\overline{X\setminus\overline{U}}\right) \neq \emptyset. \]
	Suppose otherwise. Then
	\[ V \subseteq \overline{X\setminus\overline{U}}. \]
	Let $z\in V\cap U\subseteq \overline{X\setminus\overline{U}}$. Because $V\cap U$ is an open set containing $z$, we must have
	\[ (V\cap U) \cap (X\setminus \overline{U}) \neq \emptyset. \]
	However, this is impossible because $U\cap (X\setminus \overline{U}) = \emptyset$, thus proving the claim.\\

	The required easily follows because this means that all the distinct sets are covered with at most three closures. The number of such sets is at most $2+4+4+4=14$ ($2$ for $A, X\setminus A$ and $4$ for each $1\leq i\leq 3$ corresponding to the number of closures).

	We further see that the bound is attained for $A = (-1,0)\cup(0,1)\cup\{2\}\cup([3,4]\cap\Q)$ (check this!).
\end{solution*}