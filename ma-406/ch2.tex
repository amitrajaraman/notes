\section{Topological Spaces and Continuous Functions}

\setcounter{subsection}{12}
\subsection{Basis for a Topology}

	\begin{exercise}
		Let $X$ be a topological space and $A\subseteq X$. Suppose that for each $x\in A$, there is an open set $U$ containing $x$ such that $U\subseteq A$. Show that $A$ is open in $X$.
	\end{exercise}
	\begin{solution*}
		For each $x\in A$, denote by $U_x$ an open subset of $A$ that contains $A$. Then $A = \bigcup_{x\in A} U_x$. However, an arbitrary union of open sets is open and thus, so is $A$.
	\end{solution*}

	\setcounter{exercise}{4}
	\begin{exercise}
		Show that if $\mathcal{A}$ is a basis for a topology on $X$, the topology generated by $\mathcal{A}$ equals the intersection of all topologies that contain $\mathcal{A}$. Prove the same if $\mathcal{A}$ is a subbasis.
	\end{exercise}
	\begin{solution*}
		Let $\mathcal{T}$ be the topology generated by $\mathcal{A}$ and $\mathcal{T}'$ be a topology that contains $\mathcal{A}$. Let $U\in\mathcal{T}$. Then $U=\bigcup_{i\in I} B_i$ for some $(B_i)_{i\in I}$ in $\mathcal{A}$. However, each $B_i$ is also in $\mathcal{T}'$. Since an arbitrary union of open sets is open, $U\in\mathcal{T'}$ as well. Therefore, $\mathcal{T}\subseteq\mathcal{T}'$, proving the result. The solution for the case where $\mathcal{A}$ is a subbasis is very similar and so omitted.
	\end{solution*}

	% \begin{exercise}
	% 	Show that the collection
	% 	\[ \mathcal{B} = \{(a,b) : a<b, a\text{ and }b\text{ are rational}\}. \]
	% \end{exercise}

\setcounter{subsection}{15}
\subsection{The Subspace Topology}

	\begin{exercise}
		Show that if $Y$ is a subspace of $X$ and $A$ is a subset of $Y$, then the topology $A$ inherits as a subspace of $Y$ is the same as the topology it inherits as a subspace of $X$.
	\end{exercise}
	\begin{solution*}	
		The topology $A$ inherits as a subspace of $X$ is
		\begin{align*}
			\mathcal{T} &= \{ U\cap A : U\text{ open in }X \} \\
				&= \{ (U\cap Y) \cap A : U\text{ open in }X \} \\
				&= \{ V\cap A : V\text{ open in }Y \},
		\end{align*}
		which is the topology it inherits as a subspace of $Y$.
	\end{solution*}

	\begin{exercise}
		If $\mathcal{T}$ and $\mathcal{T}'$ are topologies on $X$ and $\mathcal{T}'$ is strictly finer than $\mathcal{T}$, what can you say about the corresponding subspace topologies on the subset $Y$ of $X$.
	\end{exercise}
	\begin{solution*}
		It is easily seen that $\mathcal{T}'_Y$ is finer than $\mathcal{T}_Y$. We further see that it need not be strictly finer by considering the example $X=\{a,b,c\}$, $Y=\{a,b\}$, $\mathcal{T}=\{\emptyset,X,\{a\},\{b\},\{a,b\}\}$, and $\mathcal{T}'$ as the discrete topology on $X$.
	\end{solution*}

	\begin{exercise}
		Consider $Y=[-1,1]$ as a subspace of $\R$. Which of the following is open in $Y$? Which are open in $\R$?
		\begin{align*}
			A &= \left\{ x : \frac{1}{2} < x < 1 \right\} \\
			B &= \left\{ x : \frac{1}{2} < x \leq 1 \right\} \\
			C &= \left\{ x : \frac{1}{2} \leq x < 1 \right\} \\
			D &= \left\{ x : \frac{1}{2} \leq x \leq 1 \right\} \\
			E &= \left\{ x : 0 < x < 1 \text{ and } 1/x\not\in\Zp \right\}
		\end{align*}
	\end{exercise}
	\begin{solution*}
		$A$ and $B$ are open in $Y$ and only $A$ is open in $\R$. This is reasonably straightforward to prove.\\
		$C$ is not open in $Y$ (and so not $\R$ either) because there is no basis element $U$ of the order topology such that $1/2 \in U \subseteq C$. A similar argument holds for $D$ as well.\\
		$E$ is open in both $\R$ and $Y$ because it can be written as a union of basis elements
		\[ E = \bigcap_{n\in\Zp} \left(\frac{1}{n+1},\frac{1}{n}\right). \]
	\end{solution*}

	% 16.4
	\begin{exercise}
		\label{ex: 16.4}
		A map $f:X\to Y$ is said to be an open map if for every open $U$ of $X$, $f(U)$ is open in $Y$. Show that $\pi_1:X\times Y\to X$ and $\pi_2:X\times Y\to Y$ are open. 
	\end{exercise}
	\begin{solution*}
		We shall only show that $\pi_1$ is open, the other case is nearly identical. Let
		\[ U = \bigcup_{i\in I} U_i \times V_i \]
		be open in $X\times Y$ for some indexing set $I$, where each $U_i$ and $V_i$ are open in $X$ and $Y$ respectively. Then,
		\[ \pi_1(U) = \pi_1 \left(\bigcup_{i\in I} U_i\times V_i = \bigcup_{i\in I} \pi_1(U_i\times V_i) = \bigcup_{i\in I} U_i \right) \]
		is open in $X$.
	\end{solution*}






\setcounter{subsection}{16}
\subsection{Closed Sets and Limit Points}

	\begin{exercise}
		Let $\mathcal{C}$ be a collection of subsets of set $X$. Suppose that $\emptyset$ and $X$ are in $\mathcal{C}$ and that finite unions and arbitrary intersections of elements of $\mathcal{C}$ are in $C$. Show that the collection $\mathcal{T}=\{X\setminus C:C\in\mathcal{C}\}$ is a topology on $X$.
	\end{exercise}
	\begin{solution*}
		Let $(U_i)_{i\in I}$ be in $\mathcal{T}$ with $U_i = X\setminus C_i$ for each $i$. Then
		\[ \bigcup_{i\in I} U_i = X \setminus \bigcap_{i\in I} C_i = X\setminus C \in \mathcal{T} \]
		for some $C\in\mathcal{C}$. Closure under finite intersections is shown similarly. We trivially have $\emptyset,X\in\mathcal{T}$ because $X,\emptyset\in\mathcal{C}$.
	\end{solution*}

	\begin{exercise}
		Show that if $A$ is closed in $Y$ and $Y$ is closed in $X$, then $A$ is closed in $X$.
	\end{exercise}
	\begin{solution*}
		Let $U$ be open in $X$ such that $Y\setminus A = U\cap Y$. Then, we can write $A$ as $X\setminus ((X\setminus Y)\cup U)$. Since $X\setminus Y$ and $U$ are open in $X$, $A$ is closed in $X$.
	\end{solution*}

	\begin{exercise}
		Show that if $A$ is closed in $X$ and $B$ is closed in $Y$, $A\times B$ is closed in $X\times Y$.
	\end{exercise}
	\begin{solution*}
		Observe that
		\[ (X\times Y) \setminus (A\times B) = ((X\setminus A)\times (Y\setminus B)) \cup ((X\setminus A)\times Y) \cup (A\times (Y\setminus B)). \]
		Since each of the sets on the right are open in $X\times Y$, $A\times B$ is closed.
	\end{solution*}

	\begin{exercise}
		Show that if $U$ is open in $X$ and $A$ is closed in $X$, $U\setminus A$ is open in $X$ and $A\setminus U$ is closed in $Y$.
	\end{exercise}
	\begin{solution*}
		This is easily seen on writing $U\setminus A = U \cap (X\setminus A)$ and $A\setminus U = A \cap (X\setminus U)$.
	\end{solution*}

	\setcounter{exercise}{18}
	\begin{exercise}
		If $A\subseteq X$, define the boundary of $A$ by
		\[ \Bd A = \overline{A} \cap \overline{X\setminus A}.  \]
		\begin{enumerate}[(a)]
			\item Show that $A^\circ$ and $\Bd A$ are disjoint, and $\overline{A} = A^\circ \cup \Bd A$.
			\item Show that $\Bd A = \emptyset$ iff $A$ is both open and closed.
			\item Show that $U$ is open iff $\Bd U = \overline{U}\setminus U$.
			\item If $U$ is open, is it true that $U = \overline{U}^\circ$? Justify your answer.
		\end{enumerate}
	\end{exercise}
	\begin{solution*}
		\begin{enumerate}[(a)]
			\item Let $x\in A\setminus A^\circ$. Then for any open $U\ni x$, $U\not\subseteq A$ (otherwise, $A^\circ\cup U\supsetneq A^\circ$ is open and contained in $A$). That is, $U\cap (X\setminus A) \neq \emptyset$. However, this implies that $x\in \overline{X\setminus A}$, that is, $A\setminus A^\circ\subseteq \overline{X\setminus A}$. Therefore,
			\begin{align*}
				\overline{A}\setminus A^\circ &= (\overline{A}\setminus A) \cup (A\setminus A^\circ) \subseteq \overline{X\setminus A} \\
				\overline{A} &\subseteq A^\circ \cup \overline{X\setminus A} \\
					&= \overline{A} \cap (A^\circ \cup \overline{X\setminus A}) \\
					&= A^\circ \cup (\overline{A} \cup \overline{X\setminus A}) = A^\circ \cup \Bd A.
			\end{align*} 

			\item If $A$ is not closed, $\overline{A}\supsetneq A$ intersects $X\setminus A\subseteq \overline{X\setminus A}$, contradicting $\Bd A = \emptyset$. Similarly, $X\setminus A$ is closed as well, so $A$ is both open and closed.\\
			The other direction is similarly straightforward.

			\item If $U$ is open, $X\setminus U$ is closed so $\Bd U = \overline{U} \cap (X\setminus U) = \overline{U} \setminus U$.\\
			On the other hand, if $\overline{U}\cap (X\setminus U) = \overline{U}\cap \overline{X\setminus U}$, $X\setminus U$ must be closed. Indeed, otherwise, $\overline{X\setminus U} \setminus (X\setminus U) \subseteq U \subseteq \overline{U}$, contradicting the equality.

			\item No, this is not the case. Consider the open set $U=(1,2)\cup(2,3)\subseteq\R$. Then $\overline{U}^\circ=(1,3)$.
		\end{enumerate}
	\end{solution*}

	% \setcounter{exercise}{20}
	% \begin{exercise}
	% 	Consider the collection of all subsets $A$ of the topological space $X$. The operations of closure $A\mapsto \overline{A}$ and complementation $A\mapsto X\setminus A$ are functions from this collection to itself.
	% 	\begin{enumerate}[(a)]
	% 		\item Show that starting with a given set $A$, one can form no more than $14$ distinct sets by applying these two operations successively.
	% 		\item Find a subset $A$ of $\R$ (in its usual topology) for which the maximum of $14$ is attained.
	% 	\end{enumerate}
	% \end{exercise}
	% \begin{solution*}
	% 	% We denote $X\setminus A$ as $A^c$.\\
	% 	First of all, note that since $X\setminus(X\setminus A) = A$ and $\overline{\overline{A}}=\overline{A}$, it suffices to show that there are at most $14$ distinct sets where we alternately take the closure and complementation of $A$ (or rather, some set of a similar form).
	% 	% \[ \overline{\overline{X\setminus \overline{X\setminus \overline{A}}}^{\ldots}}, X\setminus  \overline{X\setminus \overline{X\setminus \overline{X\setminus A}}}^{\ldots}, X\setminus\overline{X\setminus \overline{X\setminus \overline{A}}}^{\ldots} \text{ and } \overline{\phantom{o}^{\ldots}\overline{X\setminus \overline{X\setminus \overline{X\setminus A}}}}. \]
	% 	We claim that for any set $A$,
	% 	\[ \overline{X\setminus\overline{X\setminus\overline{X\setminus\overline{A}}}} = \overline{X\setminus\overline{A}}. \]
	% 	Since $X\setminus\overline{A}$ is open, it suffices to show that for any open set $U$,
	% 	\[ \overline{X\setminus\overline{X\setminus\overline{U}}} = \overline{U}. \]
	% 	For one direction of containment, we have
	% 	\begin{align*}
	% 		X \setminus \overline{U} &\subseteq X \setminus U, \\
	% 		\overline{X \setminus \overline{U}} &\subseteq \overline{X \setminus U} = X\setminus U, \\
	% 		X \setminus\overline{X \setminus \overline{U}} &\supseteq U, \\
	% 		\overline{X \setminus\overline{X \setminus \overline{U}}} &\supseteq \overline{U}.
	% 	\end{align*}

	% 	Let $x\in \overline{X\setminus\overline{X\setminus \overline{U}}}$ and $V\ni x$ be open. Then
	% 	\[ V\cap \left( X\setminus\overline{X\setminus \overline{U}} \right) \neq \emptyset. \]
	% 	We wish to show that $V \cap U \neq \emptyset$.
	% 	Suppose otherwise. Then $V \subseteq X\setminus U$.
	% 	As a consequence,
	% 	\[ \left(X\setminus\overline{X\setminus \overline{U}}\right) \cap (X\setminus U) \neq \emptyset \]
	% 	Now, note that $X\setminus U$ is a \textit{closed} set containing $X\setminus \overline{U}$ and therefore, $\overline{X\setminus \overline{U}}\subseteq X\setminus U$. However, this yields a contradiction because $(X\setminus U) \cap U = \emptyset$, thus proving the claim.
	% 	% Let $z\in V\cap U\subseteq \overline{X\setminus\overline{U}}$. Because $V\cap U$ is an open set containing $z$, we must have
	% 	% \[ (V\cap U) \cap (X\setminus \overline{U}) \neq \emptyset. \]
	% 	% However, this is impossible because $U\cap (X\setminus \overline{U}) = \emptyset$, thus proving one direction of the claim.

	% 	The required easily follows because this means that all the distinct sets are covered with at most three closures. The number of such sets is at most $2+4+4+4=14$ ($2$ for $A, X\setminus A$ and $4$ for each $1\leq i\leq 3$ corresponding to the number of closures).\\

	% 	We further see that the bound is attained for $A = (-1,0)\cup(0,1)\cup\{2\}\cup([3,4]\cap\Q)$ (check this!).
	% \end{solution*}




\subsection{Continuous Functions}

	% 18.2
	\begin{exercise}
		Suppose that $f:X\to Y$ is continuous. If $x$ is a limit point of $A\subseteq X$, is $f(x)$ necessarily a limit point of $f(A)$?
	\end{exercise}
	\begin{solution*}
		No. Let $f:\R\to\R$ be the zero function, $A=\{1/n : n\in\N\}$, and $x=0$.
	\end{solution*}

	% 18.8
	\setcounter{exercise}{7}
	\begin{exercise}
		Let $Y$ be an ordered set in the order topology. Let $f,g:X\to Y$ be continuous.
		\begin{enumerate}[(a)]
			\item Show that the set $\{x : f(x)\leq g(x)\}$ is closed in $X$.
			\item Let $h:X\to Y$ be the function
			\[ h(x) = \min\{f(x),g(x)\}. \]
			Show that $h$ is continuous.
		\end{enumerate}
	\end{exercise}
	\begin{solution*}
		\begin{enumerate}[(a)]
			\item It suffices to show that $V = \{x : f(x) > g(x)\}$ is open in $X$. Let $x\in V$. Since $Y$ is Hausdorff, there are open sets $U_1, U_2$ such that $f(x)\in U_1$, $g(x)\in U_2$, and $U_1\cap U_2 = \emptyset$.\\
			We can consider a basis element $B_1\subseteq U_1$ that contains $f(x)$ in lieu of $U_1$. That is, we may suppose wlog that $U_1$ and $U_2$ are disjoint basis elements. Further, we may assume that $U_1$ is of the form $(a,\infty)$ (if it is of the form $(a,c)$ instead, we can replace it with $(a,\infty)$). Due to the disjointness assumption, this means that we can consider $U_1$ to be of the form $(a,\infty)$ and $U_2$ of the form $(-\infty,b)$ such that for any $c\in U_1, d\in U_2$, $c>d$\\
			Now, let $U=f^{-1}(U_1)\cap g^{-1}(U_2)\ni x$. $f$ and $g$ are continuous so $U$ is open. Further, for any $y\in U$, $f(y)>g(y)$ (by the above assumption), that is, $U\subseteq V$.\\
			It follows that $V$ is open (for any $x\in V$, there is an open $V\subseteq U$ such that $x\in V$).

			\item Let $U_1 = \{x : f(x) \geq g(x)\}$ and $U_2 = \{x : f(x) \leq g(x)\}$. By (a), $U_1$ and $U_2$ are both closed. Since $g$ is continuous on $U_1$, $f$ is continuous on $U_2$, and $f=g$ on $U_1\cap U_2$, we can use the pasting lemma to conclude that $h$ is continuous ($h(x)=g(x)$ on $U_1$ and $f(x)$ on $U_2$).
		\end{enumerate}
	\end{solution*}


\subsection{The Product Topology}

	% 19.4
	\setcounter{exercise}{3}
	\begin{exercise}
		Show that $(X_1\times \cdots\times X_{n-1})\times X_n$ is homeomorphic to $X_1\times\cdots\times X_n$.
	\end{exercise}
	\begin{solution*}
		Let the two topological spaces above be denoted by $X$ and $Y$ respectively. Consider the map $f:X\to Y$ with
		\[ f(x) = (\pi_1(\pi_1(x)), \ldots, \pi_{n-1}(\pi_1(x)), \pi_2(x)). \]
		We claim that $f$ is a homomorphism.\\
		If $U_i$ is open in $X_i$ for each $i$, then $f^{-1}(U_1\times\cdots\times U_n)=(U_1\times\cdots\times U_{n-1})\times U_n$ is open in $X$ ($U_1\times\cdots\times U_{n-1}$ is open in $X_1\times\cdots\times X_{n-1}$ and $U_n$ is open in $X_n$). Therefore, $f$ is continuous.\\
		On the other hand, if $U$ is open in $X_1\times\cdots\times X_{n-1}$ and $U_n$ is open in $X_n$, then $f(U\times U_n) = U_1\times\cdots\times U_n$, where each $U_i$ is open in $X_i$ by \Cref{ex: 16.4}.
	\end{solution*}

	% 19.6
	\setcounter{exercise}{5}
	\begin{exercise}
		Let $x_1,x_2,\ldots$ be a sequence of points in the product space $\prod X_\alpha$. Show that this sequence converges to the point $x$ iff the sequence $\pi_\alpha(x_1),\pi_\alpha(x_2),\ldots$ converges to $\pi_\alpha(x)$ for each $\alpha$. Is this true if we use the box topology instead of the product topology?
	\end{exercise}
	\begin{solution*}
		We first show the backward direction. Suppose $(\pi_\alpha(x_i))$ converges to $\pi_\alpha(x)$. Let $U$ be an open set containing $x$ and $B\subseteq U$ be a basis element of the product topology containing $x$. Let $B=\prod_\alpha U_\alpha$ where $U_\alpha\neq X_\alpha$ for the finite set $\{\alpha_1,\ldots,\alpha_n\}$. Since $(\pi_{\alpha_j}(x_i))$ converges to $\pi_{\alpha_j}(x)$ for each $j$, each $U_{\alpha_j}$ contains all but finitely many $\pi_\alpha(x_i)$. As a result, $B$ contains all but finitely many $x_i$ and therefore, $(x_i)$ converges to $x$.

		On the other hand, let $(x_i)$ converge to $x$. Let $U_\alpha$ be an open set in $X_\alpha$ containing $\pi_\alpha(x)$. We wish to show that it contains all but finitely many $\pi_\alpha(x_i)$. Let $U'=\prod U'_\beta$ be a basis element containing $x$ and $V=\prod V_\beta$, where $V_\beta=U'_\beta$ for $\beta\neq\alpha$ and $V_\alpha=U_\alpha\cap U'_\alpha$. Since $V$ is an open set containing $x$, it contains all but finitely many $x_i$. In particular, $V_\alpha$ contains all but finitely many $\pi_\alpha(x_i)$. The required follows.


		Observe that the forward proof works even if we use the box topology instead, but the backward direction breaks. To see that the result need not hold for the box, let the product space be $\R^\omega$, $\pi_n(x_i)=n/i$ and $\pi_n(x)=0$ for each $i,n$.
	\end{solution*}

	% 19.7
	\begin{exercise}
		Let $\R^\infty$ be the subset of $\R^\omega$ consisting of all sequences that are eventually $0$ ($x_i\neq 0$ for finitely many $i$). What is the closure of $\R^\infty$ in $\R^\omega$ in the box and product topologies?
	\end{exercise}
	\begin{solution*}
		We claim that the closure under the product topology is $\R^\omega$. Let $x\in\R^\omega$ and $U=\prod U_n$ be a basis element containing $x$. We wish to determine when $U\cap\R^\infty\neq\emptyset$. Consider $y\in\R^\infty$ such that $y_n = x_n$ if $U_n\neq\R$ and $0$ otherwise. Then $y\in U\cap\R^\infty$, thus proving the result.

		For the box topology, we claim that $\R^\infty$ is closed. Indeed, for any $x\in\R^\omega\setminus\R^\infty$, consider the open set $U=\prod U_n$, where $U_n=(x_n/2, 3x_n/2)$ if $x_n\neq 0$ and $\R$ otherwise. Then $U\cap\R^\infty$ is empty, completing the proof.
	\end{solution*}




\subsection{The Metric Topology}

	% 20.3
	\setcounter{exercise}{2}
	\begin{exercise}
		Let $X$ be a metric space with metric $d$.
		\begin{enumerate}[(a)]
			\item Show that $d:X\times X\to\R$ is continuous.
			\item Let $X'$ denote a space having the same underlying set as $X$. Show that if $d:X'\times X'\to\R$ is continuous, the topology of $X'$ is finer than the topology of $X$.
		\end{enumerate}
		This means that if $X$ has a metric $d$, the metric topology induced by it is the coarsest topology with respect to which $d$ is continuous.
	\end{exercise}
	\begin{solution*}
		\begin{enumerate}[(a)]
			\item Let $x=(x_1,x_2)\in X\times X$ and for some $\varepsilon>0$, $U=B(f(x),\varepsilon)$ be a basis element of $\R$ containing $f(x)$. Consider the open sets $U_1 = B_d(x_1,\varepsilon/4)$ and $U_2 = B_d(x_2,\varepsilon/4)$. If $y_1\in U_1$ and $y_2\in U_2$, then
			\[ d(y_1,y_2) \leq d(x_1,x_2) + d(x_1,y_1) + d(x_2,y_2) \leq f(x) + \varepsilon/2. \]
			Therefore, $d(y_1,y_2)\in U$. As a result, $U_1\times U_2$ is an open set such that $x\in f(U_1\times U_2) \subseteq U$ and $d$ is continuous.
			
			\item Consider the continuous function $g : X'\to X'\times X'$ given by $y\mapsto (x,y)$ (we proved continuity in \Cref{ex: 16.4}). Since the composition of continuous functions is continuous, so is $d\circ g : X'\to\R$. Note that for any $\varepsilon>0$, $(d\circ g)^{-1}((0,\varepsilon)) = B_d(x,\varepsilon)$. It follows that $B_d(x,\varepsilon)$ is open in $X'$ and therefore, the topology on $X'$ is finer than that on $X$.
		\end{enumerate}
	\end{solution*}	

	% 20.4
	\begin{exercise}
		Consider the product, uniform, and box topologies on $\R^\omega$. 
		\begin{enumerate}[(a)]
			\item In which topologies are the following functions from $\R$ to $\R^\omega$ continuous?
			\begin{align*}
				f(t) &= (t,2t,3t,\ldots) \\
				g(t) &= (t,t,t,\ldots) \\
				h(t) &= \left(t, \frac{1}{2}t, \frac{1}{3}t\right, \ldots)
			\end{align*}
			\item Let $X'$ denote a space having the same underlying set as $X$. Show that if $d:X'\times X'\to\R$ is continuous, the topology of $X'$ is finer than the topology of $X$.
		\end{enumerate}
		This means that if $X$ has a metric $d$, the metric topology induced by it is the coarsest topology with respect to which $d$ is continuous.
	\end{exercise}
	\begin{solution*}
		\begin{enumerate}[(a)]
			\item Let $x=(x_1,x_2)\in X\times X$ and for some $\varepsilon>0$, $U=B(f(x),\varepsilon)$ be a basis element of $\R$ containing $f(x)$. Consider the open sets $U_1 = B_d(x_1,\varepsilon/4)$ and $U_2 = B_d(x_2,\varepsilon/4)$. If $y_1\in U_1$ and $y_2\in U_2$, then
			\[ d(y_1,y_2) \leq d(x_1,x_2) + d(x_1,y_1) + d(x_2,y_2) \leq f(x) + \varepsilon/2. \]
			Therefore, $d(y_1,y_2)\in U$. As a result, $U_1\times U_2$ is an open set such that $x\in f(U_1\times U_2) \subseteq U$ and $d$ is continuous.
			
			\item Consider the continuous function $g : X'\to X'\times X'$ given by $y\mapsto (x,y)$ (we proved continuity in \Cref{ex: 16.4}). Since the composition of continuous functions is continuous, so is $d\circ g : X'\to\R$. Note that for any $\varepsilon>0$, $(d\circ g)^{-1}((0,\varepsilon)) = B_d(x,\varepsilon)$. It follows that $B_d(x,\varepsilon)$ is open in $X'$ and therefore, the topology on $X'$ is finer than that on $X$.
		\end{enumerate}
	\end{solution*}	


	% 20.5
	\setcounter{exercise}{4}
	\begin{exercise}
		Let $\R^\infty$ be the subset of $\R^\omega$ consisting of all infinite sequences that are eventually $0$. What is the closure of $\R^\infty$ in $\R^\omega$ in the uniform topology?
	\end{exercise}
	\begin{solution*}
		We claim that the closure of $\R^\infty$ is the set of all sequences that converge to $0$. Let $x\in\R^\omega$.
		\begin{itemize}
			\item Case 1. The sequence $x$ does not converge to $0$. There is then some $\varepsilon>0$ such that for infinitely many $n$, $|x_n|>\varepsilon$. We may assume that $\varepsilon<1$. Consider the open set $U = B_{\overline{\rho}}(x,\varepsilon)$. We claim that $U\cap\R^\infty = \emptyset$. Indeed, for any $y\in\R^\infty$, $|x_n-y|>\varepsilon$ for infinitely many $n$ so $y\not\in B_{\overline{d}}(x,\varepsilon)$, thus proving that $x\not\in\overline{\R^\infty}$.
			\item Case 2. The sequence $x$ converges to $x$. Let $B_{\overline{\rho}}(x,\varepsilon)$ be an arbitrary basis element containing $x$. Then there exists $N\in\N$ such that for all $n>N$, $|x_n|<\varepsilon/2$. Consider the element $y\in\R^\infty$ such that $y_n=x_n$ for $n\leq N$ and $y_n=0$ otherwise. Then $y\in B_{\overline{\rho}}(x,\varepsilon)$, thus proving that $x\in\overline{\R^\infty}$.
		\end{itemize}
	\end{solution*}

\subsection{The Metric Topology (continued)}

	% 21.2
	\setcounter{exercise}{1}
	\begin{exercise}
		Let $X$ and $Y$ be metric spaces with metrics $d_X$ and $d_Y$ respectively. Let $f:X\to Y$ such that for any $x_1,x_2\in X$,
		\[ d_Y(f(x_1), f(x_2)) = d_X(x_1,x_2). \]
		Show that $f$ is an imbedding. It is called the \textit{isometric imbedding} of $X$ in $Y$.
	\end{exercise}
	\begin{solution*}
		Let $x\in X$ and $V = B_{d_Y}(f(x),\varepsilon)$ be a basis element in $Y$ containing $f(x)$. Then $U = B_{d_X}(x,\varepsilon)$ is an open set in $X$ containing $x$ such that $f(U)\subseteq V$. Therefore, $f$ is continuous. Showing that $f$ is open is similarly straight-forward.
	\end{solution*}

	% 21.3
	\begin{exercise}
		Let $X_n$ be a metric space with metric $d_n$ for each $n\in\Zp$.
		% \begin{enumerate}[(a)]
		% 	\item Show that
		% 	\[ \rho(x,y) = \max\{d_1(x_1,y_1,\ldots,d_n(x_n,y_n))\} \]
		% 	is a metric for the product space $X_1\times\cdots\times X_n$.
		% 	\item 
			Let $\overline{d}_i=\min\{d_i,1\}$. Show that
			\[ D(x,y) = \sup \{\overline{d}_i(x_i,y_i)/i\} \]
			is a metric for the product space $\prod X_i$.
		% \end{enumerate}
	\end{exercise}
	\begin{solution*}
		% \begin{enumerate}[(a)]
		% 	\item The first two conditions are very easy to check. For the last,
		% 	\[ d(x,z) + d(z,y) = \max\{d_1(x_1,z_1),\ldots,d_n(x_n,z_n)\} + \max\{d_1(z_1,y_1),\ldots,d_n(z_n,y_n)\} \geq \max\{d_1(x_1,z_1)+d_1(z_1,y_1),\ldots,d_n(x_n,z_n)+d(z_n,y_n)\} \geq d(x,y). \]

		% 	\item
			Let $U = \prod U_i$ be a basis element of the product topology, where $U_i\neq X_i$ for $i=\alpha_1,\ldots,\alpha_n$. Let $x\in U$. For each $j$, let $B_{\overline{d}_{\alpha_j}}(x_{\alpha_j},\varepsilon_j)\subseteq U_{\alpha_j}$. Let $\varepsilon = \min\{\varepsilon_j/2\alpha_j\}$. Then $x\in B_{D}(x,\varepsilon)\subseteq U$, so the metric topology is finer than the product topology.\\
			For the other direction, let $\varepsilon>0$ and $U=B_D(x,\varepsilon)$. We want to show that there is an element of the product topology containing $x$ that is contained in $U$. Let $V=\prod V_i$, where $V_i = B_{\overline{d}_i}(x_i, i\varepsilon/2)$ for each $i$. Note that $V$ is a basis element of the product topology since $V_i=X_i$ for $i>2/\varepsilon$. Since $x\in V\subseteq U$, the product topology is finer than the metric topology, thus proving the result.
		% \end{enumerate}
	\end{solution*}


