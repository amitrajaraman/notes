\section{Connectedness and Compactness}

\setcounter{subsection}{22}
\subsection{Connected Spaces}

	% 23.1
	\begin{exercise}
		Let $\mathcal{T}$ and $\mathcal{T}'$ be two topologies on $X$. If $\mathcal{T}'\supseteq\mathcal{T}$, what does connectedness of $X$ in one topology mean about connectedness in the other?
	\end{exercise}
	\begin{solution*}
		If $\mathcal{T}'$ is connected, then so is $\mathcal{T}$. Indeed, if $\mathcal{T}$ is disconnected and has separation $X=U\cup V$, then $U\cup V$ serves as a separation of $X$ under $\mathcal{T}'$ as well. The converse need not be true, as can be seen on considering the discrete and indiscrete topology on $X$.
	\end{solution*}

	% 23.2
	\begin{exercise}
		Let $\{A_n\}$ be a sequence of connected subspaces of $X$ such that $A_n\cap A_{n+1}\neq\emptyset$ for all $n$. Show that $\bigcup A_n$ is connected.
	\end{exercise}
	\begin{solution*}
		Let $A = \bigcup A_n$. Suppose instead that $A=U\cup V$ is a separation. Since $A_1$ is connected, we may assume without loss of generality that $A_1\subseteq U$. We shall now show by induction that for any $n$, $A_n\subseteq U$. Indeed, if $A_n\subseteq U$ for some $n\geq 1$, then since $A_{n+1}$ is connected, $A_{n+1}\subseteq U$ or $A_{n+1}\subseteq V$. However, $\emptyset\neq A_n\cap A_{n+1}\subseteq U$, so $A_{n+1}\subseteq U$ as well. This contradicts the non-emptiness of $V$, completing the proof.
	\end{solution*}

	% 23.3
	\begin{exercise}
		Let $\{A_\alpha\}$ be a collection of connected subspaces of $X$ and $A$ a connected subspace of $X$. Show that if $A_\alpha\cap A\neq\emptyset$ for all $\alpha$, then $A\cup\bigcup A_\alpha$ is connected.
	\end{exercise}
	\begin{solution*}
		Denote the space by $Y$. Suppose instead that $Y=U\cup V$ is a separation. Since $A$ is connected, we may assume wlog that $A\subseteq U$. For any $\alpha$, $A_\alpha\subseteq U$ or $A_\alpha\subseteq V$. However, $A\cap A_\alpha \neq \emptyset$ so $A_\alpha\subseteq U$. This contradicts the non-emptiness of $V$, proving the result.
	\end{solution*}


	% 23.4
	\begin{exercise}
		Show that if $X$ is an infinite set, it is connected in the finite complement topology.
	\end{exercise}
	\begin{solution*}
		Suppose otherwise and let $X=U\cup V$ be a separation. Then $X\setminus U$ and $X\setminus V$ are finite, so $(X\setminus U)\cup (X\setminus V)$ is finite. Then, its complement $U\cap V$ is infinite (in particular, non-empty), contradicting the disjointedness of $U$ and $V$. 
	\end{solution*}

	% 23.8
	\setcounter{exercise}{7}
	\begin{exercise}
		Determine whether or not $\R^\omega$ is connected in the uniform topology.
	\end{exercise}
	\begin{solution*}
		Let $S$ be the set of all bounded sequences. Then it is not too difficult to show that both $S$ and $\R^\omega\setminus S$ are open, so $\R^\omega$ is disconnected.
	\end{solution*}


	% 23.9
	\begin{exercise}
		Let $A\subsetneq X$ and $B\subsetneq Y$. If $X$ and $Y$ are connected, show that $(X\times Y)\setminus(A\times B)$ is connected.
	\end{exercise}
	\begin{solution*}
		Denote the space of interest by $S$. Suppose instead that $S = U\cup V$ is a separation. Let $x\times y, z\times w\in S$. Suppose that $x\times y\in U$. Consider the connected subspaces $A = \{x\}\times Y \subseteq S$ and $B = X\times \{w\}\subseteq S$. Since their intersection is non-empty (it contains $x\times w$), their union is connected as well. Further, because $x\times y\in U$, $A\cup B\subseteq U$. In particular, $w\times z \in U$. Since our choice of $w\times z$ was arbitrary, this contradicts the non-emptiness of $V$, proving the result.
	\end{solution*}

	% 23.11
	\setcounter{exercise}{10}
	\begin{exercise}
		Let $p:X\to Y$ be a quotient map. Show that if each $p^{-1}(\{y\})$ is connected and $Y$ is connected, then $X$ is connected.
	\end{exercise}
	\begin{solution*}
		Suppose otherwise and let $X=U\cup V$ be a separation. For any $y$, either $p^{-1}(\{y\})\subseteq U$ or $p^{-1}(\{y\})\subseteq Y$ (due to connectedness). That is, $U$ and $V$ are saturated. But then, $Y = p(X) = p(U) \cup p(V)$ is a separation of $Y$, yielding a contradiction and proving the result.
	\end{solution*}

	% \begin{exercise}
	% 	Let $Y\subseteq X$; let $X$ and $Y$ be connected. Show that if $A$ and $B$ form a separation of $X\setminus Y$, then $Y\cup A$ and $Y\cup B$ are connected.
	% \end{exercise}
	% \begin{solution*}
	% 	We show that $Y\cup A$ is connected. Since $B$ is open in $X\setminus Y$, let $B = B'\cap (X\setminus Y)$ for some open $B'$.
	% \end{solution*}

\subsection{Connected Subspaces of the Real Line}

	% 24.2
	\setcounter{exercise}{1}
	\begin{exercise}
		Let $f:S^1\to\R$ be a continuous map. Show that there exist a point $x$ of $S_1$ such that $f(x)=f(-x)$.
	\end{exercise}
	\begin{solution*}
		Consider the continuous map $g:S^1\to\R$ given by $x\mapsto f(x)-f(-x)$. Choose an arbitrary $x\in S_1$. If $g(x)=0$, then we are done. Otherwise, let $y=-x$ be the diametrically opposite point to $x$. Then $g(y)=-g(x)$. Since $S^1$ is connected and $0$ lies between $g(x)$ and $g(y)$, it follows that there exists some $x_0\in S^1$ such that $g(x_0)=0$.
	\end{solution*}

	% 24.3
	\begin{exercise}
		Let $f:X\to X$ be continuous. Show that if $X=[0,1]$, there is a point $x$ such that $f(x)=x$. What happens if $X=[0,1)$ or $(0,1]$?
	\end{exercise}
	\begin{solution*}
		Consider the continuous map $g:X\to\R$ given by $x\mapsto f(x)-x$. Observe that $g(0)\geq 0$ or $g(1)\leq 0$. If equality holds at either place, we are done. Otherwise, $0$ lies between them. Since $X$ is connected, the claim follows.\\
		The claim need not hold if $X=[0,1)$ or $(0,1]$. Indeed, consider the functions $x\mapsto (x+1)/2$ and $x\mapsto x/2$.
	\end{solution*}

	% % 24.4
	% \begin{exercise}
	% 	Show that if $A$ is a countable subset of $\R^2$, $\R^2\setminus A$ is path-connected.
	% \end{exercise}
	% \begin{solution*}
	% 	Choose some $x,y\in\R^2\setminus A$.
	% \end{solution*}

\subsection{Components and Local Connectedness}

	% 25.1
	\begin{exercise}
		What are the components and path components of $\R_\ell$? What are the continuous maps $\R\to\R_\ell$?
	\end{exercise}
	\begin{solution*}
		It is obvious that any singleton in $\R_\ell$ is connected and path-connected. We claim that these are the only non-empty connected (and path-connected) subspaces. Let $A\subseteq\R_\ell$ have at least two elements. Let $x,y\in A$ with $x<y$. Then $(-\infty,y)\cap A$ and $[y,\infty)\cap A$ forms a separation of $A$, so it is not connected, and thus not a component. Therefore, the components of $\R_\ell$ are the singletons.\\
		Recall that any continuous maps connected subspaces to connected subspaces. In particular, any continuous map $\R\to\R_\ell$ maps $\R$ to a connected subspace of $\R_\ell$. However, this must be a singleton and therefore, the continuous maps $\R\to\R_\ell$ are the singletons.
	\end{solution*}

	% 25.8
	\setcounter{exercise}{7}
	\begin{exercise}
		Let $p:X\to Y$ be a quotient map. Show that if $X$ is locally connected, so is $Y$.
	\end{exercise}
	\begin{solution*}
		Let $y\in Y$ and $U$ be a neighbourhood of $Y$. Let $C$ be a component of $U$.\\
		It suffices to show that $p^{-1}(C)$ is open in $X$.
		Consider the collection of components of $p^{-1}(U)$ that intersect $p^{-1}(C)$. Observe that each of these components is open (by the local connectedness of $X$). Therefore, their union is open as well. Let $D$ be one of these components.\\
		It suffices to show that $D\subseteq p^{-1}(C)$. Let $d\in D$. Then $p(d) \in p(D)\cap C$. However, $C$ is a component, so $p(D)\subseteq C$. It follows that $D\subseteq p^{-1}(p(D)) \subseteq p^{-1}(C)$, completing the proof.
		% For each $x\in p^{-1}(y)$, let $V_x\subseteq p^{-1}(U)$ be a connected neighbourhood of $x$. Let $V = \bigcup_{x\in p^{-1}(y)} V_x$.
	\end{solution*}

\subsection{Compact Spaces}

	% 26.1
	\begin{exercise}
	\phantom{bah}
		\begin{enumerate}[(a)]
			\item Let $\mathcal{T}$ and $\mathcal{T}'$ be two topologies on the set $X$. Suppose that $\mathcal{T}'\supseteq\mathcal{T}$. What does compactness of $X$ under one of these topologies imply about compactness under the other?
			\item Show that if $X$ is compact Hausdorff under both $\mathcal{T}$ and $\mathcal{T}'$, either $\mathcal{T}$ and $\mathcal{T}'$ are equal or they are not comparable.
		\end{enumerate}
	\end{exercise}
	\begin{solution*}
		\begin{enumerate}[(a)]
			\item If $X$ is compact under $\mathcal{T}'$, then it is compact under $\mathcal{T}$. Indeed, any open cover under $\mathcal{T}$ is an open cover under $\mathcal{T}'$, and compactness implies the existence of a finite subcover.\\
			The converse need not hold, as can be seen with the example of $X=\R$, $\mathcal{T}'$ as the discrete topology, and $\mathcal{T}$ as the indiscrete topology.
			\item Suppose $\mathcal{T}\subseteq\mathcal{T}'$. Let $U\in\mathcal{T}'$. Then $X\setminus U$ is closed, and thus compact under $\mathcal{T}'$. By a method similar to part (a), $X\setminus U$ is also compact under $\mathcal{T}$, and thus closed in $\mathcal{T}$. This implies that $U$ is open in $\mathcal{T}$, and therefore $\mathcal{T}'\subseteq\mathcal{T}$, proving the claim.
		\end{enumerate}
	\end{solution*}

	% 26.2
	\begin{exercise}
	\phantom{bah}
		\begin{enumerate}[(a)]
			\item Show that in the finite complement topology on $\R$, every subspace is compact. 
			\item If $\R$ has the topology consisting of all sets $A$ such that $\R\setminus A$ is either countable or all of $\R$, is $[0,1]$ a compact space.
		\end{enumerate}
	\end{exercise}
	\begin{solution*}
		\begin{enumerate}[(a)]
			\item Let $A\subseteq\R$ and $\mathcal{A}$ be an open cover of $A$. Let $U\in\mathcal{A}$. Then $(\R\setminus U)\cap A$ is finite, suppose it is equal to $\{x_1,\ldots,x_n\}$. For each $1\leq i\leq n$, let $U_i\in\mathcal{A}$ such that $x_i\in U_i$. Then $\{U,U_1,\ldots,U_n\}$ forms a finite subcover of $A$.
			\item No, consider the open cover
			\[ \mathcal{A} = \{ \R\setminus (\Q\cap[0,1]) \cup\{x\} : x\in \Q\cap[0,1] \}. \]
			It is easy to show that $\mathcal{A}$ has no finite subcover.
		\end{enumerate}
	\end{solution*}

	% 26.3
	\begin{exercise}
		Show that a finite union of compact subspaces of $X$ is compact.
	\end{exercise}
	\begin{solution*}
		Let $U_1,\ldots,U_n$ be compact subspaces of $X$, $U = \bigcup_{1\leq i\leq n} U_i$, and $\mathcal{A}$ be an open cover of $U$. For each $1\leq i\leq n$, let $\mathcal{A}_i$ be a finite subcover of $\mathcal{A}$ of $U_i$ (such a subcover exists because $U_i$ is compact and $\mathcal{A}$ is an open cover of $U_i$). Then $\bigcup_{1\leq i\leq n}\mathcal{A}_i$ forms a finite subcover of $U$, proving the required.
	\end{solution*}

	% 26.5
	\setcounter{exercise}{4}
	\begin{exercise}
		Let $A$ and $B$ be disjoint compact subspaces of the Hausdorff subspace $X$. Show that there exist disjoint open subsets $U$ and $V$ containing $A$ and $B$ respectively.
	\end{exercise}
	\begin{solution*}
		For each $a\in A$, let $U_a$ and $V_a$ be disjoint open subsets containing $\{a\}$ and $B$ respectively. Then $\{U_a : a\in A\}$ forms an open cover of $A$, so has a finite subcover, say $\{U_{a_1},\ldots,U_{a_n}\}$. Then $\bigcup_{1\leq i\leq n} U_{a_i}$ and $\bigcap_{1\leq i\leq n} V_{a_i}$ are disjoint open subsets containing $A$ and $B$ respecitvely, completing the proof.
	\end{solution*}
	
	% 26.6
	\begin{exercise}
		Show that if $f:X\to Y$ is continuous, where $X$ is compact and $Y$ is Hausdorff, then $f$ is a closed map.
	\end{exercise}
	\begin{solution*}
		
	\end{solution*}