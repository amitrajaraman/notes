\section{Connectedness and Compactness}

\setcounter{subsection}{22}
\subsection{Connected Spaces}

	% 23.1
	\begin{exercise}
		Let $\mathcal{T}$ and $\mathcal{T}'$ be two topologies on $X$. If $\mathcal{T}'\supseteq\mathcal{T}$, what does connectedness of $X$ in one topology mean about connectedness in the other?
	\end{exercise}
	\begin{solution*}
		If $\mathcal{T}'$ is connected, then so is $\mathcal{T}$. Indeed, if $\mathcal{T}$ is disconnected and has separation $X=U\cup V$, then $U\cup V$ serves as a separation of $X$ under $\mathcal{T}'$ as well. The converse need not be true, as can be seen on considering the discrete and indiscrete topology on $X$.
	\end{solution*}

	% 23.2
	\begin{exercise}
		Let $\{A_n\}$ be a sequence of connected subspaces of $X$ such that $A_n\cap A_{n+1}\neq\emptyset$ for all $n$. Show that $\bigcup A_n$ is connected.
	\end{exercise}
	\begin{solution*}
		Let $A = \bigcup A_n$. Suppose instead that $A=U\cup V$ is a separation. Since $A_1$ is connected, we may assume without loss of generality that $A_1\subseteq U$. We shall now show by induction that for any $n$, $A_n\subseteq U$. Indeed, if $A_n\subseteq U$ for some $n\geq 1$, then since $A_{n+1}$ is connected, $A_{n+1}\subseteq U$ or $A_{n+1}\subseteq V$. However, $\emptyset\neq A_n\cap A_{n+1}\subseteq U$, so $A_{n+1}\subseteq U$ as well. This contradicts the non-emptiness of $V$, completing the proof.
	\end{solution*}

	% 23.3
	\begin{exercise}
		Let $\{A_\alpha\}$ be a collection of connected subspaces of $X$ and $A$ a connected subspace of $X$. Show that if $A_\alpha\cap A\neq\emptyset$ for all $\alpha$, then $A\cup\bigcup A_\alpha$ is connected.
	\end{exercise}
	\begin{solution*}
		Denote the space by $Y$. Suppose instead that $Y=U\cup V$ is a separation. Since $A$ is connected, we may assume wlog that $A\subseteq U$. For any $\alpha$, $A_\alpha\subseteq U$ or $A_\alpha\subseteq V$. However, $A\cap A_\alpha \neq \emptyset$ so $A_\alpha\subseteq U$. This contradicts the non-emptiness of $V$, proving the result.
	\end{solution*}


	% 23.4
	\begin{exercise}
		Show that if $X$ is an infinite set, it is connected in the finite complement topology.
	\end{exercise}
	\begin{solution*}
		Suppose otherwise and let $X=U\cup V$ be a separation. Then $X\setminus U$ and $X\setminus V$ are finite, so $(X\setminus U)\cup (X\setminus V)$ is finite. Then, its complement $U\cap V$ is infinite (in particular, non-empty), contradicting the disjointedness of $U$ and $V$. 
	\end{solution*}

	% 23.8
	\setcounter{exercise}{7}
	\begin{exercise}
		Determine whether or not $\R^\omega$ is connected in the uniform topology.
	\end{exercise}
	\begin{solution*}
		Let $S$ be the set of all bounded sequences. Then it is not too difficult to show that both $S$ and $\R^\omega\setminus S$ are open, so $\R^\omega$ is disconnected.
	\end{solution*}


	% 23.9
	\begin{exercise}
		Let $A\subsetneq X$ and $B\subsetneq Y$. If $X$ and $Y$ are connected, show that $(X\times Y)\setminus(A\times B)$ is connected.
	\end{exercise}
	\begin{solution*}
		Denote the space of interest by $S$. Suppose instead that $S = U\cup V$ is a separation. Let $x\times y, z\times w\in S$. Suppose that $x\times y\in U$. Consider the connected subspaces $A = \{x\}\times Y \subseteq S$ and $B = X\times \{w\}\subseteq S$. Since their intersection is non-empty (it contains $x\times w$), their union is connected as well. Further, because $x\times y\in U$, $A\cup B\subseteq U$. In particular, $w\times z \in U$. Since our choice of $w\times z$ was arbitrary, this contradicts the non-emptiness of $V$, proving the result.
	\end{solution*}

	% 23.11
	\setcounter{exercise}{10}
	\begin{exercise}
		Let $p:X\to Y$ be a quotient map. Show that if each $p^{-1}(\{y\})$ is connected and $Y$ is connected, then $X$ is connected.
	\end{exercise}
	\begin{solution*}
		Suppose otherwise and let $X=U\cup V$ be a separation. For any $y$, either $p^{-1}(\{y\})\subseteq U$ or $p^{-1}(\{y\})\subseteq Y$ (due to connectedness). That is, $U$ and $V$ are saturated. But then, $Y = p(X) = p(U) \cup p(V)$ is a separation of $Y$, yielding a contradiction and proving the result.
	\end{solution*}

	% \begin{exercise}
	% 	Let $Y\subseteq X$; let $X$ and $Y$ be connected. Show that if $A$ and $B$ form a separation of $X\setminus Y$, then $Y\cup A$ and $Y\cup B$ are connected.
	% \end{exercise}
	% \begin{solution*}
	% 	We show that $Y\cup A$ is connected. Since $B$ is open in $X\setminus Y$, let $B = B'\cap (X\setminus Y)$ for some open $B'$.
	% \end{solution*}

\subsection{Connected Subspaces of the Real Line}

	% 24.2
	\setcounter{exercise}{1}
	\begin{exercise}
		Let $f:S^1\to\R$ be a continuous map. Show that there exist a point $x$ of $S_1$ such that $f(x)=f(-x)$.
	\end{exercise}
	\begin{solution*}
		Consider the continuous map $g:S^1\to\R$ given by $x\mapsto f(x)-f(-x)$. Choose an arbitrary $x\in S_1$. If $g(x)=0$, then we are done. Otherwise, let $y=-x$ be the diametrically opposite point to $x$. Then $g(y)=-g(x)$. Since $S^1$ is connected and $0$ lies between $g(x)$ and $g(y)$, it follows that there exists some $x_0\in S^1$ such that $g(x_0)=0$.
	\end{solution*}

	% 24.3
	\begin{exercise}
		Let $f:X\to X$ be continuous. Show that if $X=[0,1]$, there is a point $x$ such that $f(x)=x$.
	\end{exercise}
	\begin{solution*}
		Consider the continuous map $g:X\to\R$ given by $x\mapsto f(x)-x$. Observe that $g(0)\geq 0$ or $g(1)\leq 0$. If equality holds at either place, we are done. Otherwise, $0$ lies between them. Since $X$ is connected, the claim follows.
	\end{solution*}

	% 24.3
	\begin{exercise}
		Show that if $A$ is a countable subset of $\R^2$, $\R^2\setminus A$ is path-connected.
	\end{exercise}
	\begin{solution*}
		Choose some $x,y\in\R^2\setminus A$.
	\end{solution*}