\section{Normal form games}

\subsection{Definitions}

	Normal form is a representation technique for games.
	\begin{fdef}[Normal form game]
		A \emph{normal form game} is a $3$-tuple $\langle N , (S_i)_{i \in N} , (u_i)_{i \in N} \rangle$, where
		\begin{enumerate}
			\item $N = \{1,\ldots,n\}$ is the set of players.
			\item $S_i$ is the set of \emph{strategies} of player $i$. A particular strategy is denoted $s_i \in S_i$ and the set of \emph{strategy profiles} is $S \coloneqq \bigtimes_{i \in N} S_i$, with specific elements $s = (s_1,\ldots,s_n) \in S$. A strategy profile without $i$ is $s_{-i} = (s_1,\ldots,s_{i-1},s_{i+1},\ldots,s_n)$.
			\item $u_i : S \to \R$ is the \emph{utility function} of player $i$.
		\end{enumerate}
	\end{fdef}

	If $S_i$ is finite for all $i \in N$, the game is said to be a \emph{finite} game.\\

	As mentioned earlier, a player is rational if they pick actions that maximize their utility. A player is intelligent if they know the rules of the game perfectly, and pick actions assuming that all other players are rational and intelligent.\\

	A fact is said to be \emph{common knowledge} if
	\begin{enumerate}
		\item all players know the fact,
		\item the fact that ``all players know the fact'' is also common knowledge.
	\end{enumerate}

	\begin{fex}[Propagation of common knowledge]
		There is an isolated island (with a hundred people, say) where all inhabitants have eye color either blue or black. There is no reflecting surface on the island (people cannot figure out their own eye color) and nobody can communicate with each other.\\
		One day, a truth-telling god comes to the island and declares that all blue-eyed people are bad for the island and must leave as soon as possible. He also says that there is at least one blue-eyed person on the island. The inhabitants, being deeply devout, do listen to him and leave at the end of the day if they discover that their eyes are blue. In this setting, the fact that there is at least one blue-eyed person on the island is common knowledge. \\

		If there was only one blue-eyed person, he would see that all other people have black eyes. Because the god said that there is a blue-eyed person, he infers that he must be the only blue-eyed person and leaves at the end of the first day.\\
		If there were two, then on the second day everyone would notice that all people remain on the island, so they would infer that there are at least two blue-eyed people on the island. If one of the inhabitants sees that exactly one of the other four people is blue-eyed, then he, along with the other blue-eyed person, leaves on the second day.\\
		This goes on, and it is seen that if there are exactly $n$ blue-eyed people, then all of these $n$ people leave at the end of the $n$th day.\\
	\end{fex}

	Now, we discuss the concept of \emph{domination} in NFGs. Consider the following game matrix:

	\begin{center}
	\begin{tabular}{|c||c|c|c|}
		\hline
		$1, 2$ & \textsf{L} & \textsf{M} & \textsf{R} \\
		\hline \hline
		\textsf{U} & $1, 0$ & $1,3$ & $3,2$ \\
		\hline
		\textsf{D} & $-1,6$ & $0,5$ & $3,3$ \\
		\hline
	\end{tabular}
	\end{center}

	Observe that player \textsf{2} has no reason to ever play \textsf{R}. Indeed, no matter what player $1$ chooses, they can increase their payoff by switching to \textsf{M} instead. In such a scenario, we say that \textsf{R} is dominated by \textsf{M}.\\

	\begin{fdef}
		A strategy $s_i' \in S_i$ of player $i$ is said to be \emph{strictly dominated} if there exists another strategy $s_i \in S_i$ such that for every strategy profile $s_{-i} \in S_{-i}$,
		\[ u_i(s_i,s_{-i}) > u_i(s_i',s_{-i}). \]

		A strategy $s_i' \in S_i$ of player $i$ is said to be \emph{weakly dominated} if there exists another strategy $s_i \in S_i$ such that for every strategy profile $s_{-i} \in S_{-i}$,
		\[ u_i(s_i,s_{-i}) \ge u_i(s_i',s_{-i}) \]
		and in addition, there exists some $\tilde{s}_{-i} \in S_{-i}$ such that
		\[ u_i(s_i,\tilde{s}_{-i}) > u_i(s_i',\tilde{s}_{-i}). \]
	\end{fdef}
	So, in the earlier example \textsf{R} is strictly dominated and \textsf{D} is weakly dominated.\\
	There is no reason for a rational player to ever play a (weakly or strictly) dominated strategy.

	\begin{fdef}
		A strategy $s_i \in S_i$ is \emph{strictly (weakly) dominant} if it strictly (weakly) dominates all $s_i' \in S_i \setminus \{s_i\}$.
	\end{fdef}

	Recall \Cref{ex: neighbouring kingdom dilemma}. In this, warfare strictly dominates agriculture, which is precisely what we said there, albeit in more explicit terms.\\

	Let us give another example of this.

	\begin{fex}[Second price auctions]
		Suppose there are two players having ``values'' $v_1,v_2$ respectively. Each player can choose a number in $[0,M]$, where $M \gg v_1,v_2$. The player who quotes the larger number wins the object (with ties being broken in favour of player $1$, say), and pays the losing player's chosen number. The utility of the winning player is their value $v_i$ minus their payment (the amount bid by the other player), and the utility of the losing player is $0$.\\
		As a NFG representation, $N = \{1,2\}$, $S_1 = S_2 = [0,M]$, and
		\[ u_1(s_1,s_2) = \begin{cases} v_1 - s_2, & s_1 \ge s_2, \\ 0, & \text{otherwise,} \end{cases} \text{ and } u_1(s_1,s_2) = \begin{cases} v_2 - s_1, & s_1 < s_2, \\ 0, & \text{otherwise,} \end{cases} \]
		It turns out that the strategy $s_i$ where player $i$ chooses $v_i$ is a weakly dominant strategy! Let us check this for player $1$.\\
		Suppose that $s_1' > v_1$ is in $S_1$ and $s_2 \in S_2$. We would like to show that $u_1(s_1,s_2) \ge u_1(s_1',s_2)$. If $s_2 > s_1'$, both payoffs are zero. If $s_2 \le s_1 < s_1'$, then both payoffs are equal to the same value. The remaining case is when $s_1 \le s_2 \le s_1'$. The payoff for $s_1$ is zero, but the payoff for $s_1'$ is non-positive since we are paying more than we value the item.
	\end{fex}

\subsection{Some types of equilibria}

	\begin{fdef}
		A strategy profile $(s_1^*,s_2^*,\ldots,s_n^*)$ is a \emph{strictly (weakly) dominant strategy equilibrium} if each $s_i^*$ is a strictly (weakly) dominant strategy.
	\end{fdef}
	We abbreviate the above as SDSE or WDSE.

	No rational player would play dominated strategies, so we can eliminate dominated strategies one-by-one. A point of note here is that after eliminating a strategy, we get a reduced game with fewer strategies, and this game may have dominated strategies that were not there earlier.\\
	For strictly dominated strategies, the order of elimination does not matter. For weakly dominated strategies however, some reasonable outcomes may be eliminated; this is because eliminating one player's weakly dominated strategies may introduce new dominated strategies for another.

	\begin{fex}[Order of eliminating weakly dominated strategies matters]
		\label{example: order of eliminating dominated strategies}
		Consider the following.
		\begin{center}
		\begin{tabular}{|c||c|c|c|}
			\hline
			& \textsf{L} & \textsf{C} & \textsf{R} \\
			\hline\hline
			\textsf{T} & $1,2$ & $2,3$ & $0,3$ \\
			\hline
			\textsf{M} & $2,2$ & $2,1$ & $3,2$ \\
			\hline
			\textsf{B} & $2,1$ & $0,0$ & $1,0$ \\ \hline
		\end{tabular}
		\end{center}

		% It may be checked that the dominated strategies are T, R, B, and C.

		Right off the bat, it is seen that \textsf{T}, \textsf{B}, and \textsf{C} are weakly dominated strategies. Suppose we start by imposing that player $1$ does not play \textsf{T}. If we do this, then \textsf{R} becomes weakly dominated as well, so it makes sense to eliminate it. Similarly, we can go on to eliminate \textsf{B} and \textsf{C}. Finally, the payoff of $(\mathsf{M},\mathsf{L})$ is $2,2$.\\
		On the other hand, if we eliminate strategies in the order of \textsf{B}, \textsf{L}, \textsf{C}, \textsf{T}, then the final remaining strategies are $(\mathsf{M},\mathsf{R})$, which gives a payoff of $3,2$, which is not the same as the previous $2,2$!
	\end{fex}

	Dominant strategies (and dominant strategy equilibria) need not exist in games in general.
	\begin{fex}[Dominated strategies need not exist]
		Consider the following game.
		\begin{center}
		\begin{tabular}{|c||c|c|}
			\hline
			& \textsf{L} & \textsf{R} \\
			\hline\hline
			\textsf{L} & $1, 1$ & $0,0$ \\
			\hline
			\textsf{R} & $0,0$ & $1,1$ \\
			\hline 
		\end{tabular}
		\end{center}
	\end{fex}


	As a result, dominance is not enough to give a reasonable outcome, so we must give a more refined notion.

	\begin{fdef}[Nash Equilibrium]
		A strategy profile $(s_i^*,s_{-i}^*)$ is a \emph{pure strategy Nash equilibrium} (PSNE) if for all $s_i \in S_i$,
		\[ u_i(s_i^*, s_{-i}^*) \ge u_i(s_i,s_{-i}^*). \]
	\end{fdef}

	That is, fixing the remaining players' strategies, no player can increase their payoff by moving to another strategy. More succinctly, unilateral deviation is not beneficial. \\
	In the above example, $(\mathsf{L},\mathsf{L})$ and $(\mathsf{R},\mathsf{R})$ are both PSNEs.\\
	PSNEs need not exist either!

	\begin{fex}[PSNEs need not exist]
		The following game has no PSNE.
		\begin{center}
		\begin{tabular}{|c||c|c|}
			\hline
			& \textsf{L} & \textsf{R} \\
			\hline\hline
			\textsf{L} & $-1,1$ & $1,-1$ \\
			\hline
			\textsf{R} & $1,-1$ & $1,-1$ \\ \hline
		\end{tabular}
		\end{center}
	\end{fex}

	\begin{fdef}
		A \emph{best response} of player $i$ against a strategy profile $s_{-i}$ is a strategy that gives the maximum utility. That is,
		\[ B_i(s_{-i}) = \{ s_i \in S_i : u_i(s_i,s_{-i}) \ge u_i(s_i',s_{-i}) \text{ for all $s_i' \in S_i$} \}. \]
	\end{fdef}

	So, a PSNE is a strategy profile $(s_i^*,s_{-i}^*)$ such that $s_i^* \in B_i(s_{-i}^*)$ for all $i \in N$.\\
	A PSNE gives some sort of stability. Once there, no rational player has a reason to change their strategy.\\

	One of our biggest assumptions thus far is that all players are rational and intelligent. There are, however, other types of rationality.\\
	One is risk-aversion, where each player makes pessimistic estimates of others. This worst case optimal choice is called a max-min strategy.

	\begin{fdef}
		A strategy $s_i^\text{maxmin}$ is a \emph{max-min strategy} for player $i$ if
		\[ s_i^{\text{maxmin}} \in \argmax_{s_i \in S_i} \min_{s_{-i} \in S_{-i}} u_i(s_i,s_{-i}). \]
		The \emph{max-min value} is defined by
		\[ \underline{v}_{i} = \max_{s_i \in S_i} \min_{s_{-i} \in S_{-i}} u_i(s_i,s_{-i}). \]
	\end{fdef}

	We have that for any $t_{-i} \in S_{-i}$,
	\[ u_i(s_i^\text{maxmin}, t_{-i}) = \underline{v}_i. \]

	\begin{ftheo}
		Any dominant strategy is a maxmin strategy.
	\end{ftheo}
	\begin{proof}
		Let $s_i^*$ be a dominant strategy for player $i$. We have that for any $s_{-i} \in S_{-i}$ and $s_i' \in S_i \setminus \{s_i^*\}$,
		\[ u_i(s_i^*, s_{-i}) \ge u_i(s_i', s_{-i}). \]
		In particular, when we set $s_{-i}$ as any $s_{-i}^\text{min}(s_i^*) \in \argmin_{s_{-i}' \in S_{-i}} u_i(s_i^*,s_{-i}')$, we get
		\[ \min_{s_{-i} \in S_{-i}} u_i(s_i^*, s_{-i}) = u_i(s_i^*, s_{-i}^\text{min}(s_i^*)) \ge u_i(s_i', s_{-i}^\text{min}(s_i')) \ge \min_{s_{-i} \in S_{-i}} u_i(s_i', s_{-i}) \]
		and as a result,
		\[ s_i^* \in \argmax_{s_i \in S_i} \min_{s_{-i} \in S_{-i}} u_i(s_i,s_{-i}). \qedhere \]
	\end{proof}

	\begin{ftheo}
		Every PSNE $s^* = (s_1^*,\ldots,s_n^*)$ of an NFG satisfies
		\[ u_i(s^*) \ge \underline{v}_i \]
		for all $i \in N$.
	\end{ftheo}
	\begin{proof}
		We have
		\begin{align*}
			u_i(s_i^*, s_{-i}^*) &= \max_{s_i \in S_i} u_i(s_i, s_{-i}^*) & \text{(by definition of PSNE)} \\
				&\ge \max_{s_i \in S_i} \min_{s_{-i} \in S_{-i}} u_i(s_i, s_{-i}) = \underline{v}_i. \qedhere
		\end{align*}
	\end{proof}

	What happens to stability and security when games are eliminated? Recall that depending on the order in which dominant strategies are eliminated, the final value can change.

	\begin{fex}
		The game matrix involved in this example is very similar to that in \Cref{example: order of eliminating dominated strategies}.
		\begin{center}
		\begin{tabular}{|c||c|c|c|}
			\hline
			& \textsf{L} & \textsf{C} & \textsf{R} \\
			\hline\hline
			\textsf{T} & $1,2$ & $2,3$ & $0,3$ \\
			\hline
			\textsf{M} & $2,2$ & $2,1$ & $3,2$ \\
			\hline
			\textsf{B} & $2,0$ & $0,0$ & $1,0$ \\ \hline
		\end{tabular}
		\end{center}
		The initial maxmin values for the two players are $2$ for player $1$ and $0$ for player $2$, but after we eliminate \textsf{B}, the values go to $2$ and $2$.\\
		It is not a coincidence that the maxmin value of $2$ for player $1$ is unchanged.
	\end{fex}

	\begin{ftheo}
		Consider an NFG $G = \langle N , (S_i)_{i \in N} , (u_i)_{i \in N}\rangle$. Let $\hat{s}_j \in S_j$ be a dominated strategy, and $\hat{G}$ be the residual game after removing $\hat{s}_j$. The maxmin value of player $j$ in $\hat{G}$ is equal to that in $G$.
	\end{ftheo}

	The idea is that the eliminated strategy cannot be the unique maxmin strategy in $G$ since it is dominated.

	\begin{proof}
		We are done if we show that there is a maxmin strategy in $G$ in $S_{j} \setminus \{s_{-j}\}$ (Why?).\\
		Let $\hat{s}_j$ be dominated by $t_j \in S_j \setminus \{\hat{s}_j\}$. Then, for all $s_{-j} \in S_{-j}$,
		\[ u_j(t_j,s_{-j}) \ge u_j(\hat{s}_j,s_{-j}). \]
		We have
		\[ \min_{s_{-j} \in S_{-j}} u_j(t_j,s_{-j}) \ge \min_{s_{-j} \in S_{-j}} u_j(\hat{s}_j,s_{-j}), \]
		so
		\[ \max_{s_j \in S_{j} \setminus \{\hat{s}_j\}} \min_{s_{-j} \in S_{-j}} u_j(s_j,s_{-j}) \ge \min_{s_{-j} \in S_{-j}} u_j(\hat{s}_j,s_{-j}), \]
		completing the proof -- there is another strategy whose value is at least that of $\hat{s}_j$.
	\end{proof}

	To summarize,
	\begin{itemize}
		\item eliminating strictly dominated strategies has no effect on the PSNEs,
		\item eliminating weakly dominated strategies may make the set of PSNEs smaller, but does not add new PSNEs, and
		\item the maxmin value is unaffected by eliminating (strictly or weakly) dominated strategies.
	\end{itemize}

\subsection{Matrix games}

	\begin{fdef}
		A \emph{matrix game} or \emph{two player zero-sum game} is a normal form game $\langle N , (S_i)_{i \in N} , (u_i)_{i \in N} \rangle$ with $N = \{1,2\}$ and $u_1 + u_2 \equiv 0$.
	\end{fdef}

	\begin{fex}
		\label{ex: matrix game}
		Examples of matrix games are the following.

		\begin{center}
		\begin{tabular}{|c||c|c|c|}
			\hline
			& \textsf{L} & \textsf{C} & \textsf{R} \\ \hline \hline
			\textsf{T} & $3,-3$ & $-5,5$ & $-2,2$ \\ \hline
			\textsf{M} & $1,-1$ & $4,-4$ & $1,-1$ \\ \hline
			\textsf{B} & $6,-6$ & $-3,3$ & $-5,5$ \\ \hline
		\end{tabular}
		\end{center}

		\begin{center}
		\begin{tabular}{|c||c|c|}
			\hline
			& \textsf{L} & \textsf{R} \\ \hline\hline
			\textsf{L} & $-1,1$ & $1,-1$ \\ \hline
			\textsf{R} & $1,-1$ & $-1,1$ \\ \hline
		\end{tabular}
		\end{center}

		A point of note here that we shall soon examine in more detail is that the first game has PSNEs while the second does not.
	\end{fex}

	A matrix game can be represented by a single \emph{utility matrix}, by considering the utilities of only one of the players, say player $1$. Player $2$'s utilities are then just the negative of the matrix.\\
	Given a utility matrix $u$, $u_1 \equiv u$ and $u_2 \equiv -u$.\\

	What are the PSNEs, if any, of a matrix game?

	\begin{definition}
		A \emph{saddle point} of a matrix $A$ is an index $A_{ij}$ that is the largest in the $i$th row and the smallest in the $i$th column.
	\end{definition}

	\begin{ftheo}
		In a matrix game with utility matrix $u$, $(s_1^*,s_2^*)$ is a PSNE if and only if it is a saddlepoint.
	\end{ftheo}
	\begin{proof}
		Indeed, $(s_1^*,s_2^*)$ is a saddle point iff
		\begin{align*}
			u(s_1^*,s_2^*) &\ge u(s_1,s_2*) \text{ for all $s_1 \in S_2$ and} \\
			u(s_1^*,s_2^*) &\le u(s_1^*,s_2) \text{ for all $s_2 \in S_2$,}
		\end{align*}
		which is equivalent to being a PSNE since $u_1 \equiv u$ and $u_2 \equiv -u$.
	\end{proof}

	\begin{fdef}
		Given a two player game with utility matrix $u$, define the \emph{maxmin} value $\underline{v}$ by
		\[ \underline{v} = \max_{s_1 \in S_1} \min_{s_2 \in S_2} u(s_1,s_2) \]
		and the \emph{minmax} value $\overline{v}$ by
		\[ \overline{v} = \min_{s_2 \in S_2} \max_{s_1 \in S_1} u(s_1,s_2). \]
	\end{fdef}

	The above can be slightly rephrased to say
	\begin{align*}
		\underline{v} &= -\min_{s_1 \in S_1} \max_{s_2 \in S_2} u_2(s_1,s_2) \\
		\overline{v} &= \min_{s_2 \in S_2} \max_{s_1 \in S_2} u_1(s_1,s_2).
	\end{align*}

	\begin{flem}
		\label{lemma: minmax ge maxmin}
		For matrix games, $\overline{v} \ge \underline{v}$.
	\end{flem}
	\begin{proof}
		Let $\overline{v}$ and $\underline{v}$ be attained by $(s_1,s_2)$ and $(t_1,t_2)$ respectively. Then,
		\[ \overline{v} = u(s_1,s_2) \stackrel{(1)}{\ge} u(t_1,s_2) \stackrel{(2)}{\ge} u(t_1,t_2) = \underline{v}, \]
		where $(1)$ is because $(s_1,s_2)$ is a minmax strategy and $(2)$ is because $(t_1,t_2)$ is a maxmin strategy.
	\end{proof}

	Going back to \Cref{ex: matrix game}, check that in the first game, $\overline{v} = 1 = \underline{v}$, while in the second game, $\overline{v} = 1$ and $\underline{v} = -1$.

	\begin{ftheo}
		A matrix game has a PSNE iff $\overline{v} = \underline{v}$.
	\end{ftheo}
	\begin{proof}
		We wish to show that a utility matrix $u$ has a saddle point iff its maxmin and minmax values are equal.\\
		
		Suppose that $(s_1^*,s_2^*)$ is a saddle point. Then,
		\[ u(s_1^*,s_2^*) \ge \max_{s_1 \in S_1} u(s_1,s_2^*) \ge \min_{s_2 \in S_2} \max_{s_1 \in S_1} u(s_1,s_2) = \overline{v}. \]
		An identical argument for player $2$ (keeping in mind that $u_2 \equiv -u$!) yields that $\underline{v} \ge u(s_1^*,s_2^*)$.\\
		Combining the two,
		\[ \underline{v} \ge u(s_1^*,s_2^*) \ge \overline{v}. \]
		Recalling \Cref{lemma: minmax ge maxmin}, we must have that $\overline{v} = \underline{v}$.\\

		Now, suppose that $\overline{v} = \underline{v}$. There is a strategy $(s_1^*,s_2^*)$ that is both a maxmin and minmax strategy (Why?).
		% ******************
		% Let $(s_1^*,s_2^*)$ be a maxmin strategy such that $u(s_1^*,s_2^*) = \underline{v}$. We claim that $(s_1^*,s_2^*)$ is also a minmax strategy. For any $s_1 \in S_1$, we have that
		% \[ u(s_1^*,s_2^*) \ge \min_{s_2 \in S_2} u(s_1,s_2) \]
		%  Observe that
		% \[ \min_{s_2 \in S_2} \max_{s_1 \in S_1} \overline{v} \le \max_{s_1 \in S_1} u(s_1,s_2^*) = u(s_1^*,s_2^*) \le \max_{s_1 \in S_1} u(s_1,s_2^*) \]

		% $(s_1^*,s_2^*)$ must also be a minmax strategy (Why?). % ???????
		For any $s_2 \in S_2$,
		\[ u(s_1^*,s_2) \ge \min_{t_2 \in S_2} u(s_1^*,t_2) = \max_{t_1 \in S_1} \min_{t_2 \in S_2} u(t_1,t_2) = \underline{v}. \]
		Similarly, for any $s_1 \in S_1$,
		\[ u(s_1,s_2^*) \le \max_{t_1 \in S_1} u(t_1,s_2^*) = \min_{t_2 \in S_2} \max_{t_1 \in S_1} u(t_1,t_2) = \overline{v}. \]
		The two equations above imply that $(s_1^*,s_2^*)$ is a saddle point, so we are done.
	\end{proof}

\subsection{Mixed strategies}

	So far, one issue is that PSNEs may not exist. We are also limiting ourselves to ``pure'' strategies, in the sense that there is a certain strategy that we definitely play.\\

	Given a finite set $A$, define the set of probability distributions on $A$
	\[ \Delta A = \{ p \in [0,1]^{A} \sum_{a \in A} p_a = 1 \}. \]
	A \emph{mixed strategy} of player $i$ is some $\sigma_i \in \Delta(S_i)$.\\
	Because we are looking at non-cooperative games, mixed strategies of distinct players are independent.\\
	The utility of the $i$th player for a mixed strategy profile $(\sigma_i,\sigma_{-i})$ is
	\[ u_i(\sigma_i,\sigma_{-i}) = \sum_{s_1 \in S_1} \cdots \sum_{s_n \in S_n} \left( \prod_{i \in [n]} \sigma_i(s_i) \right) u_i(s_1,\ldots,s_n). \]
	That is, the utility of a mixed strategy is the expectation of the utility.

	\begin{fex}
		\label{ex: mixed strategy}
		Consider the following game.
		\begin{center}
		\begin{tabular}{|c||c|c|}
			\hline
			& \textsf{U} & \textsf{D} \\ \hline\hline
			\textsf{L} & $-1,1$ & $1,-1$ \\ \hline
			\textsf{R} & $1,-1$ & $-1,1$ \\ \hline
		\end{tabular}
		\end{center}
		Suppose the mixed strategy $\sigma_1$ chooses $\mathsf{L}$ and $\mathsf{R}$ with probabilities $2/3$ and $1/3$ respectively, and $\sigma_2$ chooses $\mathsf{U}$ and $\mathsf{D}$ with probabilities $4/5$ and $1/5$ respectively. Then,
		\[ u_1(\sigma_1,\sigma_2) = \frac{2}{3}\cdot\frac{4}{5}\cdot(-1) + \frac{2}{3}\frac{1}{5}\cdot 1 + \frac{1}{3}\frac{4}{5}\cdot 1 + \frac{1}{3}\frac{1}{5}\cdot (-1) = -1/5. \] 
	\end{fex}

	By linearity of expectation, we have
	\begin{equation}
		\label{eqn: mixing strategies}
		u_i(\lambda \sigma_i + (1-\lambda)\sigma_i',\sigma_{-i}) = \lambda u_i(\sigma_i,\sigma_{-i}) + (1-\lambda) u_i(\sigma_i',\sigma_{-i}).
	\end{equation}
	This is referred to as mixing strategies.

	\begin{fdef}[Mixed Strategy Nash Equilibrium]
		A \emph{mixed strategy nash equilibrium} (MSNE) is a mixed strategy profile $(\sigma_i^*,\sigma_{-i}^*)$ such that for all $i \in N$,
		\[ u_i(\sigma_i^*,\sigma_{-i}^*) \ge u_i(\sigma_i',\sigma_{-i}^*) \]
		for any $\sigma_i' \in \Delta(S_i)$.
	\end{fdef}

	A PSNE is just a special case of a MSNE where all the mixed strategies are degenerate distributions.

	\begin{ftheo}
		\label{theo: for msne suffices to check pure deviations}
		A mixed strategy profile $(\sigma_i^*,\sigma_{-i}^*)$ is a MSNE iff
		\[ u_i(\sigma_i^*,\sigma_{-i}^*) \ge u_i(s_i,\sigma_{-i}^*) \]
		for all $s_i \in S_i$.
	\end{ftheo}
	The forward implication is direct on setting the $\sigma_i'$ appropriately, and the backward implication by an extension of \Cref{eqn: mixing strategies} to more than two strategies (How?).\\

	Therefore, any PSNE is a MSNE. In any MSNE, there must be some amount of ``balance'' -- in \Cref{ex: mixed strategy}, the probability distributions that assign $1/2$ to each strategy together constitute a MSNE (check this!).\\
	How do we make this notion of balance more formal?

	\begin{definition}
		Given a mixed strategy $\sigma_i$, the \emph{support} of $\sigma_i$ is
		\[ \delta(\sigma_i) = \{ s_i \in S_i : \sigma_i(s_i) > 0 \}. \]
	\end{definition}

	We now characterize MSNEs.

	\begin{ftheo}
		A mixed strategy profile $(\sigma_i^*,\sigma_{-i}^*)$ is a MSNE iff
		\begin{enumerate}[label=(\alph*)]
			\item $u_i(s_i,\sigma_{-i}^*)$ is equal for all $s_i \in \delta(\sigma_i^*)$, and
			\item $u_i(s_i,\sigma_{-i}^*) \ge u_i(s_i',\sigma_{-i}^*)$ for all $s_i \in \delta(\sigma_i^*)$, $s_i' \not\in \delta(\sigma_i^*)$.
		\end{enumerate}
	\end{ftheo}

	We encourage the reader to use the above theorem to determine MSNEs in earlier examples.

	\begin{proof}
		We have
		\begin{align*}
			u_i(\sigma_i^*,\sigma_{-i}^*) &= \max_{\sigma_i \in \Delta(S_i)} u_i(\sigma_i,\sigma_{-i}^*) \\
				&= \max_{s_i \in S_i} u_i(s_i,\sigma_{-i}^*) & \text{(similar to \Cref{theo: for msne suffices to check pure deviations})} \\
				&= \max_{s_i \in \delta(\sigma_i^*)} u_i(s_i,\sigma_{-i}^*) \\
				&= \sum_{s_i \in \delta(\sigma_i^*)} \sigma_i^*(s_i) u_i(s_i,\sigma_{-i}^*).
		\end{align*}
		It is possible for the maximum and weighted average to be equal iff all the utilities are the same, so (a) is proved.\\
		Now, suppose that there is some $s_i \in \delta(\sigma_i^*), s_i' \not\in \delta(\sigma_i^*)$ such that
		\[ u_i(s_i,\sigma_{-i}^*) < u_i(s_i',\sigma_{-i}^*). \]
		Consider the mixed strategy that is identical to $\sigma_i^*$ except that the weight that was on $s_i$ is shifted to $s_i'$. This gives a strictly larger utility, contradicting the fact that we have an MSNE and proving (b).\\

		Now, let us prove the backward direction. Let $u_i(s_i,\sigma_{-i}^*) = m_i(\sigma_{-i}^*)$ for all $s_i \in \delta(\sigma_i^*)$ (using (a)). Using (b), we have $m_i(\sigma_{-i}^*) = \max_{s_i \in S_i} u_i(s_i,\sigma_{-i}^*)$. Then,
		\begin{align*}
			u_i(\sigma_i^*,\sigma_{-i}^*) &= \sum_{s_i \in \delta(\sigma_i^*)} \sigma_i^*(s_i) u_i(s_i,\sigma_{-i}^*) \\
				&= m_i(\sigma_{-i}^*) \\
				&= \max_{s_i \in S_i} u_i(s_i,\sigma_{-i}^*),
		\end{align*}
		so we are done by \Cref{theo: for msne suffices to check pure deviations}.
	\end{proof}

	We now try to convert the above theorem to an algorithm. Suppose we have an NFG $G = \langle N , (S_i)_{i \in N} , (u_i)_{i \in N} \rangle$. The number of possible supports of $S_1 \times \cdots \times S_n$ is $K = (2^{|S_1|-1})(2^{|S_2|-1})\cdots2^{|S_n|-1}$.\\
	For every possible support profile $X_1 \times \cdots \times X_n$, solve the following feasibility program.
	\begin{align*}
		w_i &= \sum_{s_{-i} \in S_{-i}} \left( \prod_{j \ne i} \sigma_j(s_j) \right) u_i(s_i,s_{-i}) \text{ for all $s_i \in X_i$, $i \in N$ and} \\
		w_i &\ge \sum_{s_{-i} \in S_{-i}} \left( \prod_{j \ne i} \sigma_j(s_j) \right) u_i(s_i,s_{-i}) \text{ for all $s_i \in (S_i \setminus X_i)$, $i \in N$,}
	\end{align*}
	where for all $j \in N$, $\sigma_j(s_j) \ge 0$ for all $s_j \in S_j$ and $\sum_{s_j \in S_j} \sigma_j(s_j) = 1$.\\
	This program is not linear unless there are only two players.\\
	For games in general, there is no polynomial time algorithm known for this. In fact, the problem of finding an MSNE is \emph{PPAD-complete}\footnote{PPAD stands for ``Polynomial Parity Argument on Directed graphs''.}. The interested reader may see \cite{Daskalakis2009TheCO} for more details.\\

	The previous algorithm may be applied to a smaller set of strategies by removing dominated strategies. Now, we can even talk about domination by a mixed strategy. As we saw earlier, removing a weak dominated strategy can upset the equilibrium. 

	\begin{ftheo}
		If a pure strategy $s_i$ is strictly dominated by a mixed strategy $\sigma_i \in \Delta(S_i)$, any MSNE of the game chooses $s_i$ with probability zero.
	\end{ftheo}
	We can remove strictly dominated strategies without consequences.

	\begin{ftheo}[Nash, \cite{nash-noncooperative-games}]
		\label{theo: msne exists}
		Any finite game has a mixed Nash equilibrium.
	\end{ftheo}
	Above, ``finite'' means that the number of players and strategies are finite.


\subsection{Correlated equilibria}

	So far, we have worked in the setting where each agent independently picks their own strategy. Now, we look at an alternative approach with a mediating agent or device. This is merely another version of rationality. It may lead to results where the utility of all players is improved, and is computationally tractable.

	\begin{fex}
		Consider the following game, modelling the choices of cars at a crossroads.
		\begin{center}
		\begin{tabular}{|c||c|c|}
			\hline & \textsf{Wait} & \textsf{Go} \\ \hline\hline
			\textsf{Wait} & $0,0$ & $1,2$ \\ \hline
			\textsf{Go} & $2,1$ & $-10,-10$ \\ \hline
		\end{tabular}
		\end{center}
		It is clear that both $\mathsf{Wait},\mathsf{Go}$ and $\mathsf{Go},\mathsf{Wait}$ are PSNEs. It can also be seen however that a MSNE can assign some nonzero probability to the event where both players choose $\mathsf{Go}$. In practice, a traffic light guides the players. This trusted third party is called the \emph{mediator}. It randomizes over the strategy \emph{profiles} (and not just strategies like earlier) and suggests the corresponding strategies to the players. If the startegies are enforceable, then it is an equilibrium. 
	\end{fex}

	\begin{fdef}
		A \emph{correlated strategy} is a mapping $\pi \in \Delta S$.
	\end{fdef}

	In the setting of the previous example, an example of a sensible correlated strategy is that which chooses $\mathsf{Wait},\mathsf{Go}$ or $\mathsf{Go},\mathsf{Wait}$ with equal probability $1/2$.

	\begin{fdef}
		A \emph{correlated equilibrium} is a correlated strategy $\pi$ such that
		\[ \sum_{s_{-i} \in S_{-i}} \pi(s_i,s_{-i}) u_i(s_i,s_{-i}) \ge \sum_{s_{-i} \in S_{-i}} \pi(s_i,s_{-i}) u_i(s_i',s_{-i}) \]
		for all $s_i,s_i' \in S_i$ for all $i \in N$.
	\end{fdef}
	That is, no player benefits from (unilaterally) changing their strategy. This is largely similar to the definition of an MSNE, but the distribution is over strategy profiles and not strategies.\\

	One massive advantage of correlated equilibria is that they can be computed efficiently! We merely need to find a solution to the set of constraints
	\begin{align*}
		\sum_{s_{-i} \in S_{-i}} \pi(s_i,s_{-i}) u_i(s_i,s_{-i}) &\ge \sum_{s_{-i} \in S_{-i}} \pi(s_i,s_{-i}) u_i(s_i',s_{-i}) \text{ for all $s_i,s_i' \in S_i$ and $i \in N$} \\
		\pi(s) &\ge 0 \text{ for all $s \in S$} \\
		\sum_{s \in S} \pi(s) &= 1.
	\end{align*}
	If $|S_i| = m$ for all $i$, this gives $O(nm^2)$ inequalities in the first part and $O(m^n)$ inequalities for the second.\\
	All together, they represent a feasibility linear program.\\

	In MSNEs, the total number of support profiles was $O(2^{mn})$. Here, we just have $O(m^n)$ inequalities, which is exponentially smaller.\\

	\begin{ftheo}
		Given a MSNE $\sigma^*$, consider the strategy profile distribution $\pi^*$ defined by
		\[ \pi^*(s_1,\ldots,s_n) = \prod_{i=1}^n \sigma_i^*(s_i). \]
		Then, $\pi^*$ is a correlated equilibrium.
	\end{ftheo}

	We omit the proof of the above as it is straightforward.\\
	Summarizing much of our discussion thus far, we have that the set of SDSEs is contained in the set of WDSEs, which is contained in the set of PSNEs, which is contained in the set of MSNEs, which is contained in the set of CEs.

\clearpage