%!TEX root = ./main.tex

\section{The Bipartite Setting}

% LEC 1

A well-known result characterizing bipartite graphs with perfect matchings is the following.

\begin{ftheo}[Hall's Theorem]
	A bipartite graph $G(X,Y)$ has a perfect matching iff $|X|=|Y|$ and $|\Gamma(S)| \ge |S|$ for all $S \subseteq X$.
\end{ftheo}

\begin{fcor}
	For $d \ge 1$, any $d$-regular bipartite graph has a perfect matching. In particular, it is decomposable into $d$ perfect matchings.
\end{fcor}

Henceforth, assume that $G$ is a bipartite graph with $|X|=|Y|\eqdef n$. Although it is easy through Hall's Theorem to figure out if $G$ \emph{has} a perfect matching, it is harder to count the number of perfect matchings. Indeed, counting turns out to be $\#\P$-hard, and even counting modulo a prime $p$ is hard. Any perfect matching of a bipartite graph may be thought of as a permutation $\sigma \in S_n$. Recall that any permutation $\pi$ has a sign
\[ \sign(\pi) = (-1)^{\text{number of even cycles in $\pi$}}. \]
See Section 1.1 of the author's Combinatorics I notes for more details.\\

We may associate with $G$ an $n \times n$ \emph{bipartite adjacency matrix} $M$, where $M_{ij} = 1$ iff the $i$th vertex in $X$ is adjacent to the $j$th vertex in $Y$. Note that
\[ \det(M) = \sum_{\sigma \in S_n} \sign(\sigma) \prod_{i=1}^n M_{i,\sigma(i)}. \]
Note that the product is nonzero (and in this case equal to $1$) iff $\sigma$ corresponds to a perfect matching of $G$. Consequently, if every perfect matching has the same sign, we can count the number of perfect matchings by merely looking at the absolute value of the determinant of the bipartite adjacency matrix. One interesting problem is to determine which graphs are such thata all perfect matchings have the same sign. We also have
\[ \perm(M) = \sum_{\sigma \in S_n} \prod_{i=1}^n M_{i,\sigma(i)} = \text{number of perfect matchings}, \]
but unlike the determinant, the permanent is hard to compute.\\

Let $M_1,M_2$ be perfect matchings of $G$. Note that $M_1 \triangle M_2$, the set comprising of the edges in precisely one of $M_1,M_2$, is a disjoint union of alternating cycles. Here, an alternating cycle means that the edges alternate being in $M_1,M_2$. $M_2$ has the same sign as $M_1$ iff there is an even number of cycles which have lengths divisible by $4$.\\
We can use the above observation to check if all matchings have the same sign. Given a perfect matching $M$, all permutations have the same sign (as $M$) iff there exists a cycle of length divisible by $4$ with edges alternating in $M$. How do we do this? First convert the bipartite graph to a directed one by assigning to each edge a direction from $X$ to $Y$. The problem then boils down to determining if there exists an even directed cycle in the graph obtained by contracting the matching edges. This seems simple, but is in fact far from trivial. We know now due to Seymour [cite] that this is possible in polynomial time. We do not discuss the details in general. \\
Consider the specific case where $G$ is $3$-regular. The Heawood graph is known to have all matchings of the same sign.
\begin{fex}[Heawood graph]
	The Heawood graph has vertex set $\Z_{14}$, 
\end{fex}
and we can use a construction procedure called the \emph{star product} (sometimes called \emph{splicing}) or to build up larger graphs with the same property. In fact, McCaig showed that the only $3$-regular graphs with this property are obtained in this manner from the Heawood graph. There is also only one strongly $2$-connected $2$-regular directed graph without any even cycle, which is obtained by contracting the Heawood graph. %\Z_7 with edge from x to x+1,x+3. can preserve similar to star product by u1 u2 u v1 v2 and u1' u2' u' v1' v2' by u1 u2 w v1' v2' and u1' v1 and u2' v2.

% LEC 2

Now, requiring that all matchings have the same sign is rather restrictive. A better idea might be to change the non-zero entries of the bipartite adjacency matrix in a way that ensures nonzeroness of the overall determinant iff there exists a perfect matching. More specifically, Polya asked if we can we change some of the $1$ entries to $-1$ such that the resulting determinant becomes equal to the number of perfect matchings. For example, changing the bipartite adjacency matrix of $K_{2,2}$ to
\[ \begin{pmatrix} 1 & -1 \\ 1 & 1 \end{pmatrix} \]
makes the determinant equal to $2$, the number of perfect matchings. Another example is changing the bipartite adjacency matrix of $K_{3,3}$ with an edge removed to
\[ \begin{pmatrix} -1 & 1 & 1 \\ 1 & -1 & 1 \\ 1 & 1 & 0 \end{pmatrix}. \]
That is, Polya's problem asks whether some of the $1$s can be changed to $-1$s such that for any matching $\sigma$, $\sign(\sigma) = \prod_{i=1}^{n} a_{i\sigma(i)}$. We can think of the changed $\pm 1$ entries as assigning directions to the edges in the graph, with $+1$ edges being directed from $A$ to $B$ and $-1$s from $B$ to $A$ (this is called an \emph{orientation} of the graph). \\
Given a perfect matching $M$, say with positive sign, we would like to test if the matching obtained by switching the edges in any alternating cycle contributes the same way to the determinant as $M$. Let $M_1$ and $M_2$ be two matchings differing only on an alternating cycle, and let $a_1,b_1$ be the number of edges on the cycle in $M_1$ directed from $A$ to $B$ and $B$ to $A$ respectively, and $a_2,b_2$ the number in $M_2$ directed from $A$ to $B$ and $B$ to $A$ respectively. Let $\ell = a_1+b_1 = a_2+b_2$. The contribution by both matchings is the same iff
\[ (-1)^{b_1} \sign(M_1) = (-1)^{b_2} \sign(M_2) = (-1)^{b_2} (-1)^{\ell+1} \sign(M_1) = (-1)^{a_2+1} \sign(M_1), \]
that is, $(-1)^{b_1+a_2} = -1$. In other words, there is an odd number of edges oriented along the direction of traversal of the cycle (this does not depend on the direction).
\begin{fdef}
	Given an orientation of a graph, an even cycle (in the undirected graph) is said to be \emph{oddly oriented} if there is an odd number of edges oriented along the direction of traversal.
\end{fdef}

\begin{fdef}[Pfaffian orientation]
	Given a perfect matching in a bipartite graph, a \emph{Pfaffian orientation} is an orientation of the edges such that all alternating cycles are oddly oriented.
\end{fdef}
Thus, if we can find a Pfaffian orientation of a given bipartite graph, we can determine the number of perfect matchings.\\
We may assume without loss of generality that all edges in the matching we have are oriented from $A$ to $B$ -- if not, reverse the direction of all edges incident on the $A$-vertex of the ``wrongly'' oriented edge.\\
We can reduce this problem to one on directed graphs. As before, reduce the original bipartite graph to a directed graph by contracting the edges in the matching, directing an edge that was originally between $a_i,b_j$ from $v_i$ to $v_j$. Then, the existence of a Pfaffian orientation is equivalent to the existence of a $\{0,1\}$-weight assignment to the vertices $\{v_i\}$ such that no (directed) cycle has even weight. If in the Pfaffian orientation it is oriented from $a_i$ to $b_j$, we assign weight $1$ and we assign weight $0$ otherwise.\\
The above reduction can be used to easily prove that $K_{3,3}$ does \emph{not} have a Pfaffian orientation. More generally, if the directed graph has an odd (undirected) cycle with edges in both directions at all these edges (an ``odd double cycle''), the graph does not have a Pfaffian orientation. Let us now make a couple of observations.
\begin{enumerate}
	\item We may assume that no vertex has indegree and outdegree $1$. Vertices with degree $2$ (indegree and outdegree $1$) can be removed by assigning the sum of the two edges' weights to the contracted edge.
	\item We may assume that no vertex has neither indegree nor outdegree $1$. If such a vertex exists, we can split it into two vertices, one with all the incoming edges, one with all the outgoing edges, and an edge from the former to the latter. If this middle edge is assigned weight $0$ in a Pfaffian weighting, we trivially get a Pfaffian weighting for the original graph. If it is assigned weight $1$, we can get a Pfaffian weighting with weight $0$ by complimenting the weights of all edges.
\end{enumerate}

\begin{ftheo}
	A directed graph has a Pfaffian weighting iff the graph obtained by performing the above reductions does not contain a weak odd double cycle.
\end{ftheo}

% pfaffian is in \Sigma_2 ``there exists ... such that for all ...''. usual NP is \Sigma_1. In fact it's poly tho.