%!TEX root = ./main.tex

\section{Combinatorial Optimization}

	``Optimization'' is the subject where we attempt to maximize/minimize some function. ``Combinatorial'' refers to the fact that the function we are trying to optimize over a finite set.\\
	In matching theory, there is a plethora of ``min-max'' results, where the maximum of some function is associated to the minimum of another.

	\begin{fdef}
		Given a bipartite graph $G(A,B)$, define the \emph{deficiency} of $G$ by
		\[ \Def(G) = \max_{X \subseteq A} |X| - |N(X)|. \]
	\end{fdef}
	Note that $\Def(G) \ge 0$ for any graph, since $|\emptyset| - |N(\emptyset)| = 0$.

	\begin{fprop}
		The size of the largest matching in a bipartite graph $G(A,B)$ is equal to $|A| - \Def(G)$.
	\end{fprop}

	Arguably the first min-max result in matching theory was the following.

	\begin{ftheo}[K\"{o}nig's Theorem]
		\label{theo: konigs theorem}
		The size of a maximum matching in a bipartite graph is equal to the minimum number of vertices required to cover all edges in the graph (a minimum vertex cover) -- the smallest subset such that every edge has at least one endpoint in the subse.t
	\end{ftheo}
	This result is equivalent to Hall's Theorem.

	Proofs of min-max results usually go about by showing that one quantity is always larger than the other, then proving that there is some object of each type whose corresponding quantities are equal.

	Consider the problem of finding a maximum weight perfect matching -- given positive weights $(w_e)$ assigned to edges, find a matching of maximum weight. This is associated to a minimum problem using LP duality.\\
	For each edge, associate a variable $x_e \in \{0,1\}$ indicating whether an edge is in the matching or not. Then, finding a max weight matching amounts to the program
	\[
	\begin{array}{ll@{}ll}
	\text{maximize}  & \displaystyle\sum_{e \in E} w_{e} x_{e} &\\
	\text{subject to}& \displaystyle\sum\limits_{e \text{ incident on } v} x_{e} \le 1,  &\qquad v \in V\\\\
	                 & x_{e} \in \{0,1\},                                                &\qquad e \in E
	\end{array}
	\]
	This is an \emph{integer linear program}, which is \textsf{NP}-complete in general. We can get a normal linear program by relaxing the integrality constraint, say by allowing the $x_e$ variables to take any non-negative value.

	\begin{equation}
	\label{max-wt-pm-lp}
	\begin{array}{ll@{}ll}
	\text{maximize}  & \displaystyle\sum_{e \in E} w_{e} x_{e} &\\
	\text{subject to}& \displaystyle\sum\limits_{e \text{ incident on } v} x_{e} \le 1,  &\qquad v \in V\\\\
	                 & x_{e} \ge 0,                                                &\qquad e \in E
	\end{array}
	\end{equation}
	
	% Further, since this linear program must be maximized at some point on the boundary of the resulting polytope, whose vertices are integral points, there is a solution to this linear program that is also a solution to the integer linear program. It is possible to find such a solution efficiently by possibly adding additional constraints; we shall return to this later.\\
	The \emph{dual} of the above linear program \eqref{max-wt-pm-lp} is formed by taking a variable $y_v \ge 0$ for each vertex $v$, and multiplying the inequality corresponding to $v$ with $y_v$. This changes the second constraint to
	\[ \sum_{e\text{ incident on }v} y_v x_e \le y_v. \]
	Adding this up over all vertices gives
	\[ \sum_{\text{edges } e = uv} (y_u + y_v) x_e \le \sum_{v \in V} y_v. \]
	If the $y_v$ are such that for any edge $uv$, $y_u + y_v \ge w_e$, then
	\[ \sum_{e = uv} w_e x_e \le \sum_{e = uv} (y_u + y_v) x_e \le \sum_v y_v. \]
	From the perspective of \nameref{theo: konigs theorem} where all the $w_e$ are $1$, this just asks for a smallest vertex cover!

	We get the dual linear program

	\begin{equation}
	\label{dual-max-wt-pm-lp}
	\begin{array}{ll@{}ll}
	\text{minimize}  & \displaystyle\sum_{v \in V} y_v &\\
	\text{subject to} \displaystyle &y_u+y_v \ge w_e,  &\qquad uv = e \in E\\
	                  &y_v \ge 0,                      &\qquad v \in V
	\end{array}
	\end{equation}

	For example, consider the following graph.

	\begin{figure}[H]
		\centering
		\begin{tikzpicture}[every edge quotes/.style = {auto, font=\footnotesize, sloped, near start}]
		\foreach \phi in {1,...,2}{
		\node (\phi') at (1,2-2*\phi) {$\phi'$};
		\node (\phi) at (-1,2-2*\phi) {$\phi$};		
      	};
		\draw (1) edge["10"] (1'); % 10
		\draw (1) edge["5"] (2'); % 5
		\draw (2) edge["20"] (1'); % 20
		\draw (2) edge["13"] (2'); % 13
		% 4 7 \\ 13 1
	\end{tikzpicture}
	\end{figure}

	The linear program \eqref{max-wt-pm-lp} is optimized by setting $(x_{11'},x_{12'},x_{21'},x_{22'}) = (0,1,1,0)$, that is, the matching $\{12',21'\}$. On the other hand, the dual linear program \eqref{dual-max-wt-pm-lp} is optimized by setting $(y_1,y_2,y_1',y_2') = (4,13,7,1)$. Note that in either case, the optimum is equal to $25$! \\

	More generally, we have the following definition.

	\begin{fdef}
		\label{def:dual}
		Given the \emph{primal} linear program
		\[
		\begin{array}{ll@{}ll}
		\text{maximize}  & \displaystyle c^\top x \\
		\text{subject to} \displaystyle &Ax \le b,\\
		                  &x \ge 0,
		\end{array}
		\]
		its \emph{dual} is
		\[
		\begin{array}{ll@{}ll}
		\text{minimize}  & \displaystyle b^\top y \\
		\text{subject to} \displaystyle &A^\top y \ge c,\\
		                  &y \ge 0.
		\end{array}
		\]
	\end{fdef}

	Note that the dual of the dual of a primal is the primal itself.

	\begin{fprop}[Weak Duality]
		For any feasible solution $x$ of the primal and $y$ of the dual, $c^\top x \le b^\top y$.
	\end{fprop}

	% \begin{fprop}
	% 	If both the dual and primal are feasible, their optimum values are equal.
	% \end{fprop}

	% \begin{fprop}[Strong duality]
	% 	If the primal has a solution, then so does the dual. Furthermore, the resulting optimal values are equal.
	% \end{fprop}

	\Cref{max-wt-pm-lp} can more succinctly be written as
	\[
	\label{max-wt-pm-lp-2}
	\begin{array}{ll@{}ll}
	\text{maximize}  & w^\top x &\\
	\text{subject to}& \displaystyle Ax \le 1, \\
	                 & x \ge 0,
	\end{array}
	\]
	where $A$ is the $n \times m$ incidence matrix of the graph, with $A_{ij} = 1$ iff edge $e_j$ is incident on $v_i$. The dual LP (corresponding to \cref{dual-max-wt-pm-lp}) is
	\[
	\begin{array}{ll@{}ll}
	\text{minimize}  & 1^\top y &\\
	\text{subject to}& \displaystyle A^\top y \ge w, \\
	                 & y \ge 0.
	\end{array}
	\]

	% If the graph is bipartite, the matching has an integral optimal solution.\\
	Given a linear program (as in \Cref{def:dual}, say), the \emph{feasible region} of the LP is $\{x \in \R^m : Ax \le 1, x \ge 0\}$, which is a convex polytope -- the convex hull of finitely many points. A \emph{vertex} of a polytope is a point in it that is not a convex combination of two distinct points in the polytope. It may be shown that any linear program has an optimum at a vertex of the polytope. More generally, the set of points where the optimum is attained is some face of the polytope. An \emph{integral polytope} is one whose vertices have integral coordinates.\\

	\begin{fdef}
		Given a graph $G$, the polytope defined by \cref{max-wt-pm-lp-2} is integral iff $G$ is bipartite.
	\end{fdef}
	\begin{proof}
		For the forward direction, we have that if the graph has an odd cycle, the point that assigns $1/2$ to the edges in the cycle and $0$ to the remaining vertices is a vertex of the polytope. Indeed, if it were a convex combination of two distinct points in the polytope, both points are forced to have the coordinates of all edges not in the cycle as $0$, and if such a point in the polytope assigned $1/2+\epsilon$ to some edge, then the next edge in the cycle can be assigned at most $1/2-\epsilon$. Repeating this argument around the cycle yields that the other edge adjacent to the edge with weight $1/2+\epsilon$ has weight at least $1/2+\epsilon$, a contradiction. \\
		% *** FINISH THE PROOF
		% For the other direction, given a non-integral point, if the graph has an even cycle, we can cyclically alternatingly perturb the weights on some even cycle by $\pm \epsilon$ to get a convex combination. If the graph has no cycle at all, so it is a tree, a similar argument works, perturbing the weight of one edge by $\pm \epsilon$ and changing the weights of the remaining edges accordingly.
	\end{proof}

	We next give an optimality criterion for a linear program.

	\begin{flem}[Complementary slackness]
		\label{complementary slackness}
		Suppose that $x,y$ are solutions to a primal LP and its dual such that
		\begin{enumerate}[label=(\alph*)]
			\item if $y_v > 0$, the $v$th inequality in the primal is tight for $x$ and
			\item if $x_e > 0$, the $e$th inequality in the dual is tight for $y$.
		\end{enumerate}
		In this case, the value of the primal for $x$ equals the value of the dual for $y$, so both have the same optimum. 
	\end{flem}

	In the context of matchings, this means that if some edge $e$ has positive $x_e$, then the sum of $y$ values of its endpoints is equal to the weight $w_e$. Similarly, if a vertex has positive $y_v$, the sum of $x_e$ values over edges $e$ incident on $v$ is equal to $1$.

	\begin{proof}[Proof of \Cref{complementary slackness} in the setting of matchings]
		The proof is immediate since
		\[ \sum_e w_e x_e = \sum_{\substack{e = uv \\ x_e > 0}} w_e x_e = \sum_{\substack{e = uv \\ x_e > 0}} ( y_u + y_v ) x_e = \sum_u y_u \sum_{e\text{ incident on } u} x_e = \sum_{u : y_u > 0} y_u. \qedhere \]
	\end{proof}

	Let us now give an algorithm for finding a max-weight matching in the bipartite setting. Start off with the dual solution $y$, where $y_u$ is the maximum weight edge incident on $u$ if $u \in A$ and $0$ otherwise.

	\begin{figure}[H]
		\centering
		\begin{tikzpicture}[every edge quotes/.style = {auto, font=\footnotesize, sloped, very near start}]
		\foreach \phi in {1,...,3}{
		\node (\phi') at (1,4-2*\phi) {$\phi'$};
		\node (\phi) at (-1,4-2*\phi) {$\phi$};		
      	};

      	\node[left of=1] {$3$};
      	\node[left of=2] {$5$};
      	\node[left of=3] {$7$};

      	\node[right of=1'] {$0$};
      	\node[right of=2'] {$0$};
      	\node[right of=3'] {$0$};
		
		\draw (1) edge["2"] (1');
		\draw (1) edge["3"] (3');

		\draw (2) edge["5"] (1');
		\draw (2) edge["4"] (2');
		\draw (2) edge["1"] (3');
		
		\draw (3) edge["6"] (2');
		\draw (3) edge["7"] (3');
		% weights 3 5 7 on left
	\end{tikzpicture}
	\end{figure}

	We will try to maintain complementary slackness by allowing only those edges take nonzero value. Consider the set of \emph{admissible} edges whose $A$ vertex have them as the maximum weight edge. Find a maximum cardinality matching in the graph formed by admissible edges -- this can be done by any old algorithm such as that which searches for augmenting paths.
	
	\begin{figure}[H]
		\centering
		\begin{tikzpicture}[every edge quotes/.style = {auto, font=\footnotesize, sloped, very near start}]
		\foreach \phi in {1,...,3}{
		\node (\phi') at (1,4-2*\phi) {$\phi'$};
		\node (\phi) at (-1,4-2*\phi) {$\phi$};		
      	};

      	\node[left of=1] {$3$};
      	\node[left of=2] {$5$};
      	\node[left of=3] {$7$};

      	\node[right of=1'] {$0$};
      	\node[right of=2'] {$0$};
      	\node[right of=3'] {$0$};
		
		\draw (1) edge["2"] (1');
		\draw (1) edge[red,"3"] (3');

		\draw (2) edge[red,"5"] (1');
		\draw (2) edge["4"] (2');
		\draw (2) edge["1"] (3');
		
		\draw (3) edge["6"] (2');
		\draw (3) edge[red,"7"] (3');
		% weights 3 5 7 on left
	\end{tikzpicture}
	\end{figure}

	If this is is a matching from $A$ to $B$, it is a maximum weight matching. If not, there exists a subset $X$ such that $|N(X)| < |X|$. Then, increase the $y$ value of vertices in $N(X)$ by $\delta$ and decrease the $y$ value of vertices in $X$ by $\delta$, for the largest possible $\delta$. This means that we increase/decrease it until the weight of some vertex becomes $0$, or some new edge becomes admissible. In the former case, we delete that vertex from the graph.

	\begin{figure}[H]
		\centering
		\begin{tikzpicture}[every edge quotes/.style = {auto, font=\footnotesize, sloped, very near start}]
		\foreach \phi in {1,...,3}{
		\node (\phi') at (1,4-2*\phi) {$\phi'$};
		\node (\phi) at (-1,4-2*\phi) {$\phi$};		
      	};

      	\node[left of=1] {$2$};
      	\node[left of=2] {$5$};
      	\node[left of=3] {$6$};

      	\node[right of=1'] {$0$};
      	\node[right of=2'] {$0$};
      	\node[right of=3'] {$1$};
		
		\draw (1) edge["2"] (1');
		\draw (1) edge[red,"3"] (3');

		\draw (2) edge[red,"5"] (1');
		\draw (2) edge["4"] (2');
		\draw (2) edge["1"] (3');
		
		\draw (3) edge[red,"6"] (2');
		\draw (3) edge[red,"7"] (3');
		% weights 3 5 7 on left
	\end{tikzpicture}
	\end{figure}

	% While there exists a vertex in $A$ with $y_v > 0$ that is not matched, search for an augmenting path using only admissible edges. Once found, augment and reduce the number of unmatched vertices with $y > 0$. If any vertex in $A'$ (that have admissible edges) has $y$ value $0$, take an alternating path from $u$ to it.
	Now, how large can the $\delta$ mentioned earlier get? First off, none of the $y$ values can become negative, so define
	\[ \delta_1 = \min_{v \in A' \cup \{u\}} y_v. \]
	We also have the constraint due to formerly non-admissible edges $pq$ that might become tight due to the decrease in $y_p$ for some $p \in A'$. So, define
	\[ \delta_2 = \min_{\substack{pq \text{ not admissible} \\ p \in A', q \not\in B'}} (y_p + y_q - w_{pq}). \]
	Then, we set $\delta = \min\{\delta_1,\delta_2\}$, decreasing $y_u$ by $\delta$ for $u \in A'$, and increasing $y_v$ by $\delta$ for $v \in B'$. We repeat this process until all vertices with positive $y$ value are matched.\\

	Now, in the problem of finding a maximum weight \emph{perfect} matching, we instead have

	\[
	\label{max-wt-pm-lp}
	\begin{array}{ll@{}ll}
	\text{maximize}  & \displaystyle\sum_{e \in E} w_{e} x_{e} &\\
	\text{subject to}& \displaystyle\sum\limits_{e \text{ incident on } v} x_{e} = 1,  &\qquad v \in V\\\\
	                 & x_{e} \ge 0,                                                &\qquad e \in E
	\end{array}
	\]
	and in the dual,
	\[
	\label{dual-max-wt-pm-lp}
	\begin{array}{ll@{}ll}
	\text{minimize}  & \displaystyle\sum_{v \in V} y_v &\\
	\text{subject to} \displaystyle &y_u+y_v \ge w_e,  &\qquad uv = e \in E.
	\end{array}
	\]
	Note that the $y_v \ge 0$ constraint has been removed. The algorithm now is exactly the same as the earlier one for a maximum weight matching, except that the $y$ coordinates are allowed to be negative.\\

	This is an $O(mn)$ time algorithm, called the \emph{Hungarian method}. Some mild optimizations can bring it down to $O(m\sqrt{n})$. Recently, we got an almost linear time algorithm to find a maximum weight perfect matching! % sushant sachdeva

	Now, the problem of finding maximum weight matchings can be reduced to finding maximum weight perfect matchings by adding extra vertices with the edges having $0$ weight.\\

	In general graphs, we have
	\[
	\label{max-wt-pm-lp-2}
	\begin{array}{ll@{}ll}
	\text{maximize}  & w^\top x &\\
	\text{subject to}& \displaystyle\sum\limits_{e \text{ incident on } v} x_{e} = 1,  &\qquad v \in V\\\\
	                 & x_{e} \ge 0,                                                &\qquad e \in E
	\end{array}
	\]
	This is in general not an integral polytope.
	If the graph has no odd cycles, it is bipartite so integral, by an argument similar to the previous one. Even in the case where it does have odd cycles, the graph could be integral, an example being $K_4$.
	% still integral K_4

	Generally, graphs for which this polytope is integral are called \emph{Birkhoff-von Neumann} graphs. There is no characterization known for such graphs.\\
	Suppose we have a graph with two odd cycles, with the remaining vertices having a perfect matching among themselves. Then, the point assigning $1$ to the matching and $1/2$ to all vertices in the odd cycle is a vertex of the polytope.
	% 
	\begin{flem}
		\label{lem:matching-polytope-lem}
		Let $G$ be a matching-covered graph, that is, a connected graph where every edge is in a perfect matching. The polytope defined by \eqref{max-wt-pm-lp-2} is not integral iff $G$ contains evenly many disjoint odd cycles such that the remaining graph has a perfect matching.
	\end{flem}
	% cannot have an even cycle
	% no edge in more than one cycle
	% any vtx with nonzero degree has degree at least 2
	% any component has exactly one odd cycle

	This gives some sort of co-\textsf{NP} characterization of which graphs have integral perfect matching polytopes, but it is unclear how to check this property.
	% add more LP inequalities (making it exponentially large), but this can still be solved in poly time

	\begin{fdef}
		Given a graph $G = (V,E)$ with a perfect matching, identify each perfect matching $M$ of $G$ with the vector $x \in \R^E$, where $x_e = 1$ if $e \in M$ and $0$ otherwise. The \emph{perfect matching polytope} of $G$ is the convex hull of these points. The \emph{matching polytope} is defined similarly.
	\end{fdef}

	However, this polytope may have exponentially many vertices, so we need some more succinct way to describe it. It turns out that the polytope has far fewer faces than vertices, so it can easily be described by a collection of linear (in)equalities. Indeed, we just need to add appropriate inequalities to \eqref{max-wt-pm-lp-2}, using \Cref{lem:matching-polytope-lem}.
	
	\begin{ftheo}
		Given a graph $G$, its perfect matching polytope is the set $x$ of points in $\R^E$ satisfying
		\begin{align}
			x_e &\ge 0 \qquad \text{ for all } e \in E, \\
			\sum_{e \text{ incident on }v} x_e &= 1 \qquad \text{ for all } v \in V, \\
			\sum_{e \in \partial S} x_e &\ge 1 \qquad \text{ for all odd sized $S \subseteq V$ with $1 < |S| \le n/2$}.
		\end{align}
	\end{ftheo}
	Show that the third inequality discounts vertices in \Cref{lem:matching-polytope-lem}. In fact, the above constraint only needs to be added for certain $S$, not necessarily all $S$. However, which $S$ we must add depends on the graph itself. Indeed, when we take the dual, most variables are immediately set to $0$ so not many variables need to be considered. For the complete graph $K_{n}$ on evenly many vertices for instance, we do need to add every such constraint.\\
	Although this is exponentially many constraints, we can still sometimes efficiently solve a linear program. For example, the problem of finding a minimum weight spanning tree can be solved efficiently, although the \emph{spanning tree polytope}
	\begin{align*}
		\sum_{e \in G[S]} x_e &\le (|S|-1) \qquad \text{ for all $\emptyset \ne S \subseteq V$,} \\
		\sum_{e \in E} x_e &= (n-1).
	\end{align*}
	has exponentially many constraints. Above, $G[S]$ denotes the subgraph of $G$ induced by $S$. Alternatively, we could replace the first constraint with the constraint that for any partition $\mathcal{P}$ of the vertex set, the sum of $x_e$ over edges crossing the partition is at least $|\mathcal{P}|-1$. Kruskal's and Prim's algorithm to find a minimum weight spanning tree can in fact be expressed as appropriate linear programs.
	% combinatorial Optimization: Algorithms and Complexity - Papadimitrou and Steiglitz